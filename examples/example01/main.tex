%   This is the only one tex file of example01 for 
%   https://github.com/LiuGangKingston/Nestable-coordinate-system-for-Tikz-circuits.git
%            Version 1.0
%   free for non-commercial use.
%   Please send us emails for any problems/suggestions/comments.
%   Please be advised that none of us accept any responsibility
%   for any consequences arising out of the usage of this
%   software, especially for damage.
%   For usage, please refer to the README file and the following lines.
%   This code was written by
%        Gang Liu (gl.cell@outlook)
%                 (http://orcid.org/0000-0003-1575-9290)
%          and
%        Shiwei Huang (huang937@gmail.com)
%   Copyright (c) 2021
\documentclass[tikz,border=5mm]{standalone}
%\documentclass{article}
\usepackage{tikz}
\usetikzlibrary{calc,decorations.markings}
\usetikzlibrary{shapes,arrows,positioning,decorations.markings}
\usepackage{scalerel, circuitikz}
%   This is an accessory  file for 
%   https://github.com/LiuGangKingston/Nestable-coordinate-system-for-Tikz-circuits.git
%            Version 1.0
%   free for non-commercial use.
%   Please send us emails for any problems/suggestions/comments.
%   Please be advised that none of us accept any responsibility
%   for any consequences arising out of the usage of this
%   software, especially for damage.
%   For usage, please refer to the README file and the following lines.
%   This code was written by
%        Gang Liu (gl.cell@outlook)
%                 (http://orcid.org/0000-0003-1575-9290)
%          and
%        Shiwei Huang (huang937@gmail.com)
%   Copyright (c) 2021
%
%
%  The following command is to get the x-component and y-component 
%  of a coordinate. The command is
%  \getxyofcoordinate{the coordinate}{x-component}{y-component};
\newcommand{\getxyofcoordinate}[3]{%
\coordinate (tempcoord) at ($#1$);
\path (tempcoord) node {};
\pgfgetlastxy{\tempx}{\tempy};
\pgfmathsetmacro{#2}{\tempx}
\pgfmathsetmacro{#3}{\tempy}
}


%  The following command is the same as above but for given unit.
%  The command is
%  \getxyingivenunit{the unit like cm}{the coordinate}{x-component}{y-component};
\newcommand{\getxyingivenunit}[4]{%
\coordinate (tempcoord) at (1#1,1#1);
\path (tempcoord) node {};
\pgfgetlastxy{\tempxunit}{\tempyunit};
\coordinate (tempcoord) at ($#2$);
\path (tempcoord) node {};
\pgfgetlastxy{\tempx}{\tempy};
\pgfmathsetmacro{#3}{\tempx/\tempxunit}
\pgfmathsetmacro{#4}{\tempy/\tempyunit}
}


%  The following command is to print the value of a coordinate with some words at the first coordinate postion 
%  The command is
%  \printcoordinateat{the first coordinate}{the words}{the coordinate};
\newcommand{\printcoordinateat}[3]{%
\getxyingivenunit{cm}{#3}{\tempxx}{\tempyy}
\node at #1 {#2 ($\tempxx$, $\tempyy$).};
}


%  The following command is to print a keyworded coordinate system as a background.
%  The command is
%  \coordinatebackground{the KEYWORD}
%                                            {the first letter in both x and y directions}
%                                       {the second letter in both x and y directions}
%                                             {the last letter in both x and y directions};
\newcommand{\coordinatebackground}[4]{
\pgfmathsetmacro{\colourpercent}{30}
\foreach \i in {#2,#3,...,#4} 
{\node [black!\colourpercent] at (#1ppp\i\i) {\i};}
\foreach \i in {#2,#4} 
{\node [white] at (#1ppp\i\i) {\i};}
\coordinatebackgroundxy{#1}{#2}{#3}{#4}{#2}{#3}{#4};
}


%  The following command is to print a keyworded coordinate system as a background.
%  The command is
%  \coordinatebackgroundxy{the KEYWORD}
%                                                {the first letter in the x direction}
%                                           {the second letter in the x direction}
%                                                 {the last letter in the x direction}
%                                                {the first letter in the y direction}
%                                           {the second letter in the y direction}
%                                                 {the last letter in the y direction};
\newcommand{\coordinatebackgroundxy}[7]{
\pgfmathsetmacro{\bordercolourpercent}{60}
\pgfmathsetmacro{\colourpercent}{30}

\foreach \i in {#2,#3,...,#4} 
\foreach \j in {#5} 
\foreach \k in {#7} 
{\draw [dashed,black!\colourpercent] (#1ppp\i\j) -- (#1ppp\i\k);}

\foreach \i in {#5,#6,...,#7} 
\foreach \j in {#2} 
\foreach \k in {#4} 
{\draw [dashed,black!\colourpercent] (#1ppp\j\i) -- (#1ppp\k\i);}

\foreach \i in {#2,#4} 
\foreach \j in {#5} 
\foreach \k in {#7} 
{\draw [dashed,black!\bordercolourpercent] (#1ppp\i\j) -- (#1ppp\i\k);}

\foreach \i in {#5,#7} 
\foreach \j in {#2} 
\foreach \k in {#4} 
{\draw [dashed,black!\bordercolourpercent] (#1ppp\j\i) -- (#1ppp\k\i);}

\foreach \i in {#2,#3,...,#4} 
\foreach \j in {#5} 
\foreach \k in {#7} 
{
\node [black!\bordercolourpercent] at ($(#1ppp\i\j) + (0,-.2)$) {\i};
\node [black!\bordercolourpercent] at ($(#1ppp\i\k) + (0,.2)$) {\i};
}

\foreach \i in {#5,#6,...,#7} 
\foreach \j in {#2} 
\foreach \k in {#4} 
{
\node [black!\bordercolourpercent] at ($(#1ppp\k\i) + (.2,0)$) {\i};
\node [black!\bordercolourpercent] at ($(#1ppp\j\i) + (-.2,0)$) {\i};
}

}








\begin{document}


{\Huge This is to show a new coordinate system, which can be keyworded with any lower-case letters for an instance for a specific circuit or a part of it, and used for circuits to be inserted into and connected with other circuits efficiently. Whatever the coordinate system is modified, the circuit logical structure should be not changed essentially. 
\\
\\
Then many instances of such coordinate system can be used in ``one" big circuit.}

\newpage

\vspace{2cm}

{\Large Figure 1, an instance of the coordinate system (with keyword ``demobygangliu") as background with two connected devices and a few dark points with some lables.}

\begin{circuitikz}[scale=1]


% Circuits can be drawn by the following five major steps, as shown in the following example. 

% Step 1, preparations. 

% ``Install" the coordinate system with keyword ``demobygangliu".
\pgfmathsetmacro{\totaldemobygangliuxxx}{26}
\pgfmathsetmacro{\totaldemobygangliuyyy}{26}
\pgfmathsetmacro{\demobygangliuxxxspacing}{1}
\pgfmathsetmacro{\demobygangliuyyyspacing}{1}
\pgfmathsetmacro{\demobygangliuxxxa}{-8}
\pgfmathsetmacro{\demobygangliuyyya}{-8}

\pgfmathsetmacro{\demobygangliuxxxb}{\demobygangliuxxxa + \demobygangliuxxxspacing + 0.0 }
\pgfmathsetmacro{\demobygangliuxxxc}{\demobygangliuxxxb + \demobygangliuxxxspacing + 0.0 }
\pgfmathsetmacro{\demobygangliuxxxd}{\demobygangliuxxxc + \demobygangliuxxxspacing + 0.0 }
\pgfmathsetmacro{\demobygangliuxxxe}{\demobygangliuxxxd + \demobygangliuxxxspacing + 0.0 }
\pgfmathsetmacro{\demobygangliuxxxf}{\demobygangliuxxxe + \demobygangliuxxxspacing + 0.0 }
\pgfmathsetmacro{\demobygangliuxxxg}{\demobygangliuxxxf + \demobygangliuxxxspacing + 0.0 }
\pgfmathsetmacro{\demobygangliuxxxh}{\demobygangliuxxxg + \demobygangliuxxxspacing + 0.0 }
\pgfmathsetmacro{\demobygangliuxxxi}{\demobygangliuxxxh + \demobygangliuxxxspacing + 0.0 }
\pgfmathsetmacro{\demobygangliuxxxj}{\demobygangliuxxxi + \demobygangliuxxxspacing + 0.0 }
\pgfmathsetmacro{\demobygangliuxxxk}{\demobygangliuxxxj + \demobygangliuxxxspacing + 0.0 }
\pgfmathsetmacro{\demobygangliuxxxl}{\demobygangliuxxxk + \demobygangliuxxxspacing + 0.0 }
\pgfmathsetmacro{\demobygangliuxxxm}{\demobygangliuxxxl + \demobygangliuxxxspacing + 0.0 }
\pgfmathsetmacro{\demobygangliuxxxn}{\demobygangliuxxxm + \demobygangliuxxxspacing + 0.0 }
\pgfmathsetmacro{\demobygangliuxxxo}{\demobygangliuxxxn + \demobygangliuxxxspacing + 0.0 }
\pgfmathsetmacro{\demobygangliuxxxp}{\demobygangliuxxxo + \demobygangliuxxxspacing + 0.0 }
\pgfmathsetmacro{\demobygangliuxxxq}{\demobygangliuxxxp + \demobygangliuxxxspacing + 0.0 }
\pgfmathsetmacro{\demobygangliuxxxr}{\demobygangliuxxxq + \demobygangliuxxxspacing + 0.0 }
\pgfmathsetmacro{\demobygangliuxxxs}{\demobygangliuxxxr + \demobygangliuxxxspacing + 0.0 }
\pgfmathsetmacro{\demobygangliuxxxt}{\demobygangliuxxxs + \demobygangliuxxxspacing + 0.0 }
\pgfmathsetmacro{\demobygangliuxxxu}{\demobygangliuxxxt + \demobygangliuxxxspacing + 0.0 }
\pgfmathsetmacro{\demobygangliuxxxv}{\demobygangliuxxxu + \demobygangliuxxxspacing + 0.0 }
\pgfmathsetmacro{\demobygangliuxxxw}{\demobygangliuxxxv + \demobygangliuxxxspacing + 0.0 }
\pgfmathsetmacro{\demobygangliuxxxx}{\demobygangliuxxxw + \demobygangliuxxxspacing + 0.0 }
\pgfmathsetmacro{\demobygangliuxxxy}{\demobygangliuxxxx + \demobygangliuxxxspacing + 0.0 }
\pgfmathsetmacro{\demobygangliuxxxz}{\demobygangliuxxxy + \demobygangliuxxxspacing + 0.0 }

\pgfmathsetmacro{\demobygangliuyyyb}{\demobygangliuyyya + \demobygangliuyyyspacing + 0.0 }
\pgfmathsetmacro{\demobygangliuyyyc}{\demobygangliuyyyb + \demobygangliuyyyspacing + 0.0 }
\pgfmathsetmacro{\demobygangliuyyyd}{\demobygangliuyyyc + \demobygangliuyyyspacing + 0.0 }
\pgfmathsetmacro{\demobygangliuyyye}{\demobygangliuyyyd + \demobygangliuyyyspacing + 0.0 }
\pgfmathsetmacro{\demobygangliuyyyf}{\demobygangliuyyye + \demobygangliuyyyspacing + 0.0 }
\pgfmathsetmacro{\demobygangliuyyyg}{\demobygangliuyyyf + \demobygangliuyyyspacing + 0.0 }
\pgfmathsetmacro{\demobygangliuyyyh}{\demobygangliuyyyg + \demobygangliuyyyspacing + 0.0 }
\pgfmathsetmacro{\demobygangliuyyyi}{\demobygangliuyyyh + \demobygangliuyyyspacing + 0.0 }
\pgfmathsetmacro{\demobygangliuyyyj}{\demobygangliuyyyi + \demobygangliuyyyspacing + 0.0 }
\pgfmathsetmacro{\demobygangliuyyyk}{\demobygangliuyyyj + \demobygangliuyyyspacing + 0.0 }
\pgfmathsetmacro{\demobygangliuyyyl}{\demobygangliuyyyk + \demobygangliuyyyspacing + 0.0 }
\pgfmathsetmacro{\demobygangliuyyym}{\demobygangliuyyyl + \demobygangliuyyyspacing + 0.0 }
\pgfmathsetmacro{\demobygangliuyyyn}{\demobygangliuyyym + \demobygangliuyyyspacing + 0.0 }
\pgfmathsetmacro{\demobygangliuyyyo}{\demobygangliuyyyn + \demobygangliuyyyspacing + 0.0 }
\pgfmathsetmacro{\demobygangliuyyyp}{\demobygangliuyyyo + \demobygangliuyyyspacing + 0.0 }
\pgfmathsetmacro{\demobygangliuyyyq}{\demobygangliuyyyp + \demobygangliuyyyspacing + 0.0 }
\pgfmathsetmacro{\demobygangliuyyyr}{\demobygangliuyyyq + \demobygangliuyyyspacing + 0.0 }
\pgfmathsetmacro{\demobygangliuyyys}{\demobygangliuyyyr + \demobygangliuyyyspacing + 0.0 }
\pgfmathsetmacro{\demobygangliuyyyt}{\demobygangliuyyys + \demobygangliuyyyspacing + 0.0 }
\pgfmathsetmacro{\demobygangliuyyyu}{\demobygangliuyyyt + \demobygangliuyyyspacing + 0.0 }
\pgfmathsetmacro{\demobygangliuyyyv}{\demobygangliuyyyu + \demobygangliuyyyspacing + 0.0 }
\pgfmathsetmacro{\demobygangliuyyyw}{\demobygangliuyyyv + \demobygangliuyyyspacing + 0.0 }
\pgfmathsetmacro{\demobygangliuyyyx}{\demobygangliuyyyw + \demobygangliuyyyspacing + 0.0 }
\pgfmathsetmacro{\demobygangliuyyyy}{\demobygangliuyyyx + \demobygangliuyyyspacing + 0.0 }
\pgfmathsetmacro{\demobygangliuyyyz}{\demobygangliuyyyy + \demobygangliuyyyspacing + 0.0 }

\coordinate (demobygangliupppaa) at (\demobygangliuxxxa, \demobygangliuyyya);
\coordinate (demobygangliupppab) at (\demobygangliuxxxa, \demobygangliuyyyb);
\coordinate (demobygangliupppac) at (\demobygangliuxxxa, \demobygangliuyyyc);
\coordinate (demobygangliupppad) at (\demobygangliuxxxa, \demobygangliuyyyd);
\coordinate (demobygangliupppae) at (\demobygangliuxxxa, \demobygangliuyyye);
\coordinate (demobygangliupppaf) at (\demobygangliuxxxa, \demobygangliuyyyf);
\coordinate (demobygangliupppag) at (\demobygangliuxxxa, \demobygangliuyyyg);
\coordinate (demobygangliupppah) at (\demobygangliuxxxa, \demobygangliuyyyh);
\coordinate (demobygangliupppai) at (\demobygangliuxxxa, \demobygangliuyyyi);
\coordinate (demobygangliupppaj) at (\demobygangliuxxxa, \demobygangliuyyyj);
\coordinate (demobygangliupppak) at (\demobygangliuxxxa, \demobygangliuyyyk);
\coordinate (demobygangliupppal) at (\demobygangliuxxxa, \demobygangliuyyyl);
\coordinate (demobygangliupppam) at (\demobygangliuxxxa, \demobygangliuyyym);
\coordinate (demobygangliupppan) at (\demobygangliuxxxa, \demobygangliuyyyn);
\coordinate (demobygangliupppao) at (\demobygangliuxxxa, \demobygangliuyyyo);
\coordinate (demobygangliupppap) at (\demobygangliuxxxa, \demobygangliuyyyp);
\coordinate (demobygangliupppaq) at (\demobygangliuxxxa, \demobygangliuyyyq);
\coordinate (demobygangliupppar) at (\demobygangliuxxxa, \demobygangliuyyyr);
\coordinate (demobygangliupppas) at (\demobygangliuxxxa, \demobygangliuyyys);
\coordinate (demobygangliupppat) at (\demobygangliuxxxa, \demobygangliuyyyt);
\coordinate (demobygangliupppau) at (\demobygangliuxxxa, \demobygangliuyyyu);
\coordinate (demobygangliupppav) at (\demobygangliuxxxa, \demobygangliuyyyv);
\coordinate (demobygangliupppaw) at (\demobygangliuxxxa, \demobygangliuyyyw);
\coordinate (demobygangliupppax) at (\demobygangliuxxxa, \demobygangliuyyyx);
\coordinate (demobygangliupppay) at (\demobygangliuxxxa, \demobygangliuyyyy);
\coordinate (demobygangliupppaz) at (\demobygangliuxxxa, \demobygangliuyyyz);
\coordinate (demobygangliupppba) at (\demobygangliuxxxb, \demobygangliuyyya);
\coordinate (demobygangliupppbb) at (\demobygangliuxxxb, \demobygangliuyyyb);
\coordinate (demobygangliupppbc) at (\demobygangliuxxxb, \demobygangliuyyyc);
\coordinate (demobygangliupppbd) at (\demobygangliuxxxb, \demobygangliuyyyd);
\coordinate (demobygangliupppbe) at (\demobygangliuxxxb, \demobygangliuyyye);
\coordinate (demobygangliupppbf) at (\demobygangliuxxxb, \demobygangliuyyyf);
\coordinate (demobygangliupppbg) at (\demobygangliuxxxb, \demobygangliuyyyg);
\coordinate (demobygangliupppbh) at (\demobygangliuxxxb, \demobygangliuyyyh);
\coordinate (demobygangliupppbi) at (\demobygangliuxxxb, \demobygangliuyyyi);
\coordinate (demobygangliupppbj) at (\demobygangliuxxxb, \demobygangliuyyyj);
\coordinate (demobygangliupppbk) at (\demobygangliuxxxb, \demobygangliuyyyk);
\coordinate (demobygangliupppbl) at (\demobygangliuxxxb, \demobygangliuyyyl);
\coordinate (demobygangliupppbm) at (\demobygangliuxxxb, \demobygangliuyyym);
\coordinate (demobygangliupppbn) at (\demobygangliuxxxb, \demobygangliuyyyn);
\coordinate (demobygangliupppbo) at (\demobygangliuxxxb, \demobygangliuyyyo);
\coordinate (demobygangliupppbp) at (\demobygangliuxxxb, \demobygangliuyyyp);
\coordinate (demobygangliupppbq) at (\demobygangliuxxxb, \demobygangliuyyyq);
\coordinate (demobygangliupppbr) at (\demobygangliuxxxb, \demobygangliuyyyr);
\coordinate (demobygangliupppbs) at (\demobygangliuxxxb, \demobygangliuyyys);
\coordinate (demobygangliupppbt) at (\demobygangliuxxxb, \demobygangliuyyyt);
\coordinate (demobygangliupppbu) at (\demobygangliuxxxb, \demobygangliuyyyu);
\coordinate (demobygangliupppbv) at (\demobygangliuxxxb, \demobygangliuyyyv);
\coordinate (demobygangliupppbw) at (\demobygangliuxxxb, \demobygangliuyyyw);
\coordinate (demobygangliupppbx) at (\demobygangliuxxxb, \demobygangliuyyyx);
\coordinate (demobygangliupppby) at (\demobygangliuxxxb, \demobygangliuyyyy);
\coordinate (demobygangliupppbz) at (\demobygangliuxxxb, \demobygangliuyyyz);
\coordinate (demobygangliupppca) at (\demobygangliuxxxc, \demobygangliuyyya);
\coordinate (demobygangliupppcb) at (\demobygangliuxxxc, \demobygangliuyyyb);
\coordinate (demobygangliupppcc) at (\demobygangliuxxxc, \demobygangliuyyyc);
\coordinate (demobygangliupppcd) at (\demobygangliuxxxc, \demobygangliuyyyd);
\coordinate (demobygangliupppce) at (\demobygangliuxxxc, \demobygangliuyyye);
\coordinate (demobygangliupppcf) at (\demobygangliuxxxc, \demobygangliuyyyf);
\coordinate (demobygangliupppcg) at (\demobygangliuxxxc, \demobygangliuyyyg);
\coordinate (demobygangliupppch) at (\demobygangliuxxxc, \demobygangliuyyyh);
\coordinate (demobygangliupppci) at (\demobygangliuxxxc, \demobygangliuyyyi);
\coordinate (demobygangliupppcj) at (\demobygangliuxxxc, \demobygangliuyyyj);
\coordinate (demobygangliupppck) at (\demobygangliuxxxc, \demobygangliuyyyk);
\coordinate (demobygangliupppcl) at (\demobygangliuxxxc, \demobygangliuyyyl);
\coordinate (demobygangliupppcm) at (\demobygangliuxxxc, \demobygangliuyyym);
\coordinate (demobygangliupppcn) at (\demobygangliuxxxc, \demobygangliuyyyn);
\coordinate (demobygangliupppco) at (\demobygangliuxxxc, \demobygangliuyyyo);
\coordinate (demobygangliupppcp) at (\demobygangliuxxxc, \demobygangliuyyyp);
\coordinate (demobygangliupppcq) at (\demobygangliuxxxc, \demobygangliuyyyq);
\coordinate (demobygangliupppcr) at (\demobygangliuxxxc, \demobygangliuyyyr);
\coordinate (demobygangliupppcs) at (\demobygangliuxxxc, \demobygangliuyyys);
\coordinate (demobygangliupppct) at (\demobygangliuxxxc, \demobygangliuyyyt);
\coordinate (demobygangliupppcu) at (\demobygangliuxxxc, \demobygangliuyyyu);
\coordinate (demobygangliupppcv) at (\demobygangliuxxxc, \demobygangliuyyyv);
\coordinate (demobygangliupppcw) at (\demobygangliuxxxc, \demobygangliuyyyw);
\coordinate (demobygangliupppcx) at (\demobygangliuxxxc, \demobygangliuyyyx);
\coordinate (demobygangliupppcy) at (\demobygangliuxxxc, \demobygangliuyyyy);
\coordinate (demobygangliupppcz) at (\demobygangliuxxxc, \demobygangliuyyyz);
\coordinate (demobygangliupppda) at (\demobygangliuxxxd, \demobygangliuyyya);
\coordinate (demobygangliupppdb) at (\demobygangliuxxxd, \demobygangliuyyyb);
\coordinate (demobygangliupppdc) at (\demobygangliuxxxd, \demobygangliuyyyc);
\coordinate (demobygangliupppdd) at (\demobygangliuxxxd, \demobygangliuyyyd);
\coordinate (demobygangliupppde) at (\demobygangliuxxxd, \demobygangliuyyye);
\coordinate (demobygangliupppdf) at (\demobygangliuxxxd, \demobygangliuyyyf);
\coordinate (demobygangliupppdg) at (\demobygangliuxxxd, \demobygangliuyyyg);
\coordinate (demobygangliupppdh) at (\demobygangliuxxxd, \demobygangliuyyyh);
\coordinate (demobygangliupppdi) at (\demobygangliuxxxd, \demobygangliuyyyi);
\coordinate (demobygangliupppdj) at (\demobygangliuxxxd, \demobygangliuyyyj);
\coordinate (demobygangliupppdk) at (\demobygangliuxxxd, \demobygangliuyyyk);
\coordinate (demobygangliupppdl) at (\demobygangliuxxxd, \demobygangliuyyyl);
\coordinate (demobygangliupppdm) at (\demobygangliuxxxd, \demobygangliuyyym);
\coordinate (demobygangliupppdn) at (\demobygangliuxxxd, \demobygangliuyyyn);
\coordinate (demobygangliupppdo) at (\demobygangliuxxxd, \demobygangliuyyyo);
\coordinate (demobygangliupppdp) at (\demobygangliuxxxd, \demobygangliuyyyp);
\coordinate (demobygangliupppdq) at (\demobygangliuxxxd, \demobygangliuyyyq);
\coordinate (demobygangliupppdr) at (\demobygangliuxxxd, \demobygangliuyyyr);
\coordinate (demobygangliupppds) at (\demobygangliuxxxd, \demobygangliuyyys);
\coordinate (demobygangliupppdt) at (\demobygangliuxxxd, \demobygangliuyyyt);
\coordinate (demobygangliupppdu) at (\demobygangliuxxxd, \demobygangliuyyyu);
\coordinate (demobygangliupppdv) at (\demobygangliuxxxd, \demobygangliuyyyv);
\coordinate (demobygangliupppdw) at (\demobygangliuxxxd, \demobygangliuyyyw);
\coordinate (demobygangliupppdx) at (\demobygangliuxxxd, \demobygangliuyyyx);
\coordinate (demobygangliupppdy) at (\demobygangliuxxxd, \demobygangliuyyyy);
\coordinate (demobygangliupppdz) at (\demobygangliuxxxd, \demobygangliuyyyz);
\coordinate (demobygangliupppea) at (\demobygangliuxxxe, \demobygangliuyyya);
\coordinate (demobygangliupppeb) at (\demobygangliuxxxe, \demobygangliuyyyb);
\coordinate (demobygangliupppec) at (\demobygangliuxxxe, \demobygangliuyyyc);
\coordinate (demobygangliuppped) at (\demobygangliuxxxe, \demobygangliuyyyd);
\coordinate (demobygangliupppee) at (\demobygangliuxxxe, \demobygangliuyyye);
\coordinate (demobygangliupppef) at (\demobygangliuxxxe, \demobygangliuyyyf);
\coordinate (demobygangliupppeg) at (\demobygangliuxxxe, \demobygangliuyyyg);
\coordinate (demobygangliupppeh) at (\demobygangliuxxxe, \demobygangliuyyyh);
\coordinate (demobygangliupppei) at (\demobygangliuxxxe, \demobygangliuyyyi);
\coordinate (demobygangliupppej) at (\demobygangliuxxxe, \demobygangliuyyyj);
\coordinate (demobygangliupppek) at (\demobygangliuxxxe, \demobygangliuyyyk);
\coordinate (demobygangliupppel) at (\demobygangliuxxxe, \demobygangliuyyyl);
\coordinate (demobygangliupppem) at (\demobygangliuxxxe, \demobygangliuyyym);
\coordinate (demobygangliupppen) at (\demobygangliuxxxe, \demobygangliuyyyn);
\coordinate (demobygangliupppeo) at (\demobygangliuxxxe, \demobygangliuyyyo);
\coordinate (demobygangliupppep) at (\demobygangliuxxxe, \demobygangliuyyyp);
\coordinate (demobygangliupppeq) at (\demobygangliuxxxe, \demobygangliuyyyq);
\coordinate (demobygangliuppper) at (\demobygangliuxxxe, \demobygangliuyyyr);
\coordinate (demobygangliupppes) at (\demobygangliuxxxe, \demobygangliuyyys);
\coordinate (demobygangliupppet) at (\demobygangliuxxxe, \demobygangliuyyyt);
\coordinate (demobygangliupppeu) at (\demobygangliuxxxe, \demobygangliuyyyu);
\coordinate (demobygangliupppev) at (\demobygangliuxxxe, \demobygangliuyyyv);
\coordinate (demobygangliupppew) at (\demobygangliuxxxe, \demobygangliuyyyw);
\coordinate (demobygangliupppex) at (\demobygangliuxxxe, \demobygangliuyyyx);
\coordinate (demobygangliupppey) at (\demobygangliuxxxe, \demobygangliuyyyy);
\coordinate (demobygangliupppez) at (\demobygangliuxxxe, \demobygangliuyyyz);
\coordinate (demobygangliupppfa) at (\demobygangliuxxxf, \demobygangliuyyya);
\coordinate (demobygangliupppfb) at (\demobygangliuxxxf, \demobygangliuyyyb);
\coordinate (demobygangliupppfc) at (\demobygangliuxxxf, \demobygangliuyyyc);
\coordinate (demobygangliupppfd) at (\demobygangliuxxxf, \demobygangliuyyyd);
\coordinate (demobygangliupppfe) at (\demobygangliuxxxf, \demobygangliuyyye);
\coordinate (demobygangliupppff) at (\demobygangliuxxxf, \demobygangliuyyyf);
\coordinate (demobygangliupppfg) at (\demobygangliuxxxf, \demobygangliuyyyg);
\coordinate (demobygangliupppfh) at (\demobygangliuxxxf, \demobygangliuyyyh);
\coordinate (demobygangliupppfi) at (\demobygangliuxxxf, \demobygangliuyyyi);
\coordinate (demobygangliupppfj) at (\demobygangliuxxxf, \demobygangliuyyyj);
\coordinate (demobygangliupppfk) at (\demobygangliuxxxf, \demobygangliuyyyk);
\coordinate (demobygangliupppfl) at (\demobygangliuxxxf, \demobygangliuyyyl);
\coordinate (demobygangliupppfm) at (\demobygangliuxxxf, \demobygangliuyyym);
\coordinate (demobygangliupppfn) at (\demobygangliuxxxf, \demobygangliuyyyn);
\coordinate (demobygangliupppfo) at (\demobygangliuxxxf, \demobygangliuyyyo);
\coordinate (demobygangliupppfp) at (\demobygangliuxxxf, \demobygangliuyyyp);
\coordinate (demobygangliupppfq) at (\demobygangliuxxxf, \demobygangliuyyyq);
\coordinate (demobygangliupppfr) at (\demobygangliuxxxf, \demobygangliuyyyr);
\coordinate (demobygangliupppfs) at (\demobygangliuxxxf, \demobygangliuyyys);
\coordinate (demobygangliupppft) at (\demobygangliuxxxf, \demobygangliuyyyt);
\coordinate (demobygangliupppfu) at (\demobygangliuxxxf, \demobygangliuyyyu);
\coordinate (demobygangliupppfv) at (\demobygangliuxxxf, \demobygangliuyyyv);
\coordinate (demobygangliupppfw) at (\demobygangliuxxxf, \demobygangliuyyyw);
\coordinate (demobygangliupppfx) at (\demobygangliuxxxf, \demobygangliuyyyx);
\coordinate (demobygangliupppfy) at (\demobygangliuxxxf, \demobygangliuyyyy);
\coordinate (demobygangliupppfz) at (\demobygangliuxxxf, \demobygangliuyyyz);
\coordinate (demobygangliupppga) at (\demobygangliuxxxg, \demobygangliuyyya);
\coordinate (demobygangliupppgb) at (\demobygangliuxxxg, \demobygangliuyyyb);
\coordinate (demobygangliupppgc) at (\demobygangliuxxxg, \demobygangliuyyyc);
\coordinate (demobygangliupppgd) at (\demobygangliuxxxg, \demobygangliuyyyd);
\coordinate (demobygangliupppge) at (\demobygangliuxxxg, \demobygangliuyyye);
\coordinate (demobygangliupppgf) at (\demobygangliuxxxg, \demobygangliuyyyf);
\coordinate (demobygangliupppgg) at (\demobygangliuxxxg, \demobygangliuyyyg);
\coordinate (demobygangliupppgh) at (\demobygangliuxxxg, \demobygangliuyyyh);
\coordinate (demobygangliupppgi) at (\demobygangliuxxxg, \demobygangliuyyyi);
\coordinate (demobygangliupppgj) at (\demobygangliuxxxg, \demobygangliuyyyj);
\coordinate (demobygangliupppgk) at (\demobygangliuxxxg, \demobygangliuyyyk);
\coordinate (demobygangliupppgl) at (\demobygangliuxxxg, \demobygangliuyyyl);
\coordinate (demobygangliupppgm) at (\demobygangliuxxxg, \demobygangliuyyym);
\coordinate (demobygangliupppgn) at (\demobygangliuxxxg, \demobygangliuyyyn);
\coordinate (demobygangliupppgo) at (\demobygangliuxxxg, \demobygangliuyyyo);
\coordinate (demobygangliupppgp) at (\demobygangliuxxxg, \demobygangliuyyyp);
\coordinate (demobygangliupppgq) at (\demobygangliuxxxg, \demobygangliuyyyq);
\coordinate (demobygangliupppgr) at (\demobygangliuxxxg, \demobygangliuyyyr);
\coordinate (demobygangliupppgs) at (\demobygangliuxxxg, \demobygangliuyyys);
\coordinate (demobygangliupppgt) at (\demobygangliuxxxg, \demobygangliuyyyt);
\coordinate (demobygangliupppgu) at (\demobygangliuxxxg, \demobygangliuyyyu);
\coordinate (demobygangliupppgv) at (\demobygangliuxxxg, \demobygangliuyyyv);
\coordinate (demobygangliupppgw) at (\demobygangliuxxxg, \demobygangliuyyyw);
\coordinate (demobygangliupppgx) at (\demobygangliuxxxg, \demobygangliuyyyx);
\coordinate (demobygangliupppgy) at (\demobygangliuxxxg, \demobygangliuyyyy);
\coordinate (demobygangliupppgz) at (\demobygangliuxxxg, \demobygangliuyyyz);
\coordinate (demobygangliupppha) at (\demobygangliuxxxh, \demobygangliuyyya);
\coordinate (demobygangliuppphb) at (\demobygangliuxxxh, \demobygangliuyyyb);
\coordinate (demobygangliuppphc) at (\demobygangliuxxxh, \demobygangliuyyyc);
\coordinate (demobygangliuppphd) at (\demobygangliuxxxh, \demobygangliuyyyd);
\coordinate (demobygangliuppphe) at (\demobygangliuxxxh, \demobygangliuyyye);
\coordinate (demobygangliuppphf) at (\demobygangliuxxxh, \demobygangliuyyyf);
\coordinate (demobygangliuppphg) at (\demobygangliuxxxh, \demobygangliuyyyg);
\coordinate (demobygangliuppphh) at (\demobygangliuxxxh, \demobygangliuyyyh);
\coordinate (demobygangliuppphi) at (\demobygangliuxxxh, \demobygangliuyyyi);
\coordinate (demobygangliuppphj) at (\demobygangliuxxxh, \demobygangliuyyyj);
\coordinate (demobygangliuppphk) at (\demobygangliuxxxh, \demobygangliuyyyk);
\coordinate (demobygangliuppphl) at (\demobygangliuxxxh, \demobygangliuyyyl);
\coordinate (demobygangliuppphm) at (\demobygangliuxxxh, \demobygangliuyyym);
\coordinate (demobygangliuppphn) at (\demobygangliuxxxh, \demobygangliuyyyn);
\coordinate (demobygangliupppho) at (\demobygangliuxxxh, \demobygangliuyyyo);
\coordinate (demobygangliuppphp) at (\demobygangliuxxxh, \demobygangliuyyyp);
\coordinate (demobygangliuppphq) at (\demobygangliuxxxh, \demobygangliuyyyq);
\coordinate (demobygangliuppphr) at (\demobygangliuxxxh, \demobygangliuyyyr);
\coordinate (demobygangliuppphs) at (\demobygangliuxxxh, \demobygangliuyyys);
\coordinate (demobygangliupppht) at (\demobygangliuxxxh, \demobygangliuyyyt);
\coordinate (demobygangliuppphu) at (\demobygangliuxxxh, \demobygangliuyyyu);
\coordinate (demobygangliuppphv) at (\demobygangliuxxxh, \demobygangliuyyyv);
\coordinate (demobygangliuppphw) at (\demobygangliuxxxh, \demobygangliuyyyw);
\coordinate (demobygangliuppphx) at (\demobygangliuxxxh, \demobygangliuyyyx);
\coordinate (demobygangliuppphy) at (\demobygangliuxxxh, \demobygangliuyyyy);
\coordinate (demobygangliuppphz) at (\demobygangliuxxxh, \demobygangliuyyyz);
\coordinate (demobygangliupppia) at (\demobygangliuxxxi, \demobygangliuyyya);
\coordinate (demobygangliupppib) at (\demobygangliuxxxi, \demobygangliuyyyb);
\coordinate (demobygangliupppic) at (\demobygangliuxxxi, \demobygangliuyyyc);
\coordinate (demobygangliupppid) at (\demobygangliuxxxi, \demobygangliuyyyd);
\coordinate (demobygangliupppie) at (\demobygangliuxxxi, \demobygangliuyyye);
\coordinate (demobygangliupppif) at (\demobygangliuxxxi, \demobygangliuyyyf);
\coordinate (demobygangliupppig) at (\demobygangliuxxxi, \demobygangliuyyyg);
\coordinate (demobygangliupppih) at (\demobygangliuxxxi, \demobygangliuyyyh);
\coordinate (demobygangliupppii) at (\demobygangliuxxxi, \demobygangliuyyyi);
\coordinate (demobygangliupppij) at (\demobygangliuxxxi, \demobygangliuyyyj);
\coordinate (demobygangliupppik) at (\demobygangliuxxxi, \demobygangliuyyyk);
\coordinate (demobygangliupppil) at (\demobygangliuxxxi, \demobygangliuyyyl);
\coordinate (demobygangliupppim) at (\demobygangliuxxxi, \demobygangliuyyym);
\coordinate (demobygangliupppin) at (\demobygangliuxxxi, \demobygangliuyyyn);
\coordinate (demobygangliupppio) at (\demobygangliuxxxi, \demobygangliuyyyo);
\coordinate (demobygangliupppip) at (\demobygangliuxxxi, \demobygangliuyyyp);
\coordinate (demobygangliupppiq) at (\demobygangliuxxxi, \demobygangliuyyyq);
\coordinate (demobygangliupppir) at (\demobygangliuxxxi, \demobygangliuyyyr);
\coordinate (demobygangliupppis) at (\demobygangliuxxxi, \demobygangliuyyys);
\coordinate (demobygangliupppit) at (\demobygangliuxxxi, \demobygangliuyyyt);
\coordinate (demobygangliupppiu) at (\demobygangliuxxxi, \demobygangliuyyyu);
\coordinate (demobygangliupppiv) at (\demobygangliuxxxi, \demobygangliuyyyv);
\coordinate (demobygangliupppiw) at (\demobygangliuxxxi, \demobygangliuyyyw);
\coordinate (demobygangliupppix) at (\demobygangliuxxxi, \demobygangliuyyyx);
\coordinate (demobygangliupppiy) at (\demobygangliuxxxi, \demobygangliuyyyy);
\coordinate (demobygangliupppiz) at (\demobygangliuxxxi, \demobygangliuyyyz);
\coordinate (demobygangliupppja) at (\demobygangliuxxxj, \demobygangliuyyya);
\coordinate (demobygangliupppjb) at (\demobygangliuxxxj, \demobygangliuyyyb);
\coordinate (demobygangliupppjc) at (\demobygangliuxxxj, \demobygangliuyyyc);
\coordinate (demobygangliupppjd) at (\demobygangliuxxxj, \demobygangliuyyyd);
\coordinate (demobygangliupppje) at (\demobygangliuxxxj, \demobygangliuyyye);
\coordinate (demobygangliupppjf) at (\demobygangliuxxxj, \demobygangliuyyyf);
\coordinate (demobygangliupppjg) at (\demobygangliuxxxj, \demobygangliuyyyg);
\coordinate (demobygangliupppjh) at (\demobygangliuxxxj, \demobygangliuyyyh);
\coordinate (demobygangliupppji) at (\demobygangliuxxxj, \demobygangliuyyyi);
\coordinate (demobygangliupppjj) at (\demobygangliuxxxj, \demobygangliuyyyj);
\coordinate (demobygangliupppjk) at (\demobygangliuxxxj, \demobygangliuyyyk);
\coordinate (demobygangliupppjl) at (\demobygangliuxxxj, \demobygangliuyyyl);
\coordinate (demobygangliupppjm) at (\demobygangliuxxxj, \demobygangliuyyym);
\coordinate (demobygangliupppjn) at (\demobygangliuxxxj, \demobygangliuyyyn);
\coordinate (demobygangliupppjo) at (\demobygangliuxxxj, \demobygangliuyyyo);
\coordinate (demobygangliupppjp) at (\demobygangliuxxxj, \demobygangliuyyyp);
\coordinate (demobygangliupppjq) at (\demobygangliuxxxj, \demobygangliuyyyq);
\coordinate (demobygangliupppjr) at (\demobygangliuxxxj, \demobygangliuyyyr);
\coordinate (demobygangliupppjs) at (\demobygangliuxxxj, \demobygangliuyyys);
\coordinate (demobygangliupppjt) at (\demobygangliuxxxj, \demobygangliuyyyt);
\coordinate (demobygangliupppju) at (\demobygangliuxxxj, \demobygangliuyyyu);
\coordinate (demobygangliupppjv) at (\demobygangliuxxxj, \demobygangliuyyyv);
\coordinate (demobygangliupppjw) at (\demobygangliuxxxj, \demobygangliuyyyw);
\coordinate (demobygangliupppjx) at (\demobygangliuxxxj, \demobygangliuyyyx);
\coordinate (demobygangliupppjy) at (\demobygangliuxxxj, \demobygangliuyyyy);
\coordinate (demobygangliupppjz) at (\demobygangliuxxxj, \demobygangliuyyyz);
\coordinate (demobygangliupppka) at (\demobygangliuxxxk, \demobygangliuyyya);
\coordinate (demobygangliupppkb) at (\demobygangliuxxxk, \demobygangliuyyyb);
\coordinate (demobygangliupppkc) at (\demobygangliuxxxk, \demobygangliuyyyc);
\coordinate (demobygangliupppkd) at (\demobygangliuxxxk, \demobygangliuyyyd);
\coordinate (demobygangliupppke) at (\demobygangliuxxxk, \demobygangliuyyye);
\coordinate (demobygangliupppkf) at (\demobygangliuxxxk, \demobygangliuyyyf);
\coordinate (demobygangliupppkg) at (\demobygangliuxxxk, \demobygangliuyyyg);
\coordinate (demobygangliupppkh) at (\demobygangliuxxxk, \demobygangliuyyyh);
\coordinate (demobygangliupppki) at (\demobygangliuxxxk, \demobygangliuyyyi);
\coordinate (demobygangliupppkj) at (\demobygangliuxxxk, \demobygangliuyyyj);
\coordinate (demobygangliupppkk) at (\demobygangliuxxxk, \demobygangliuyyyk);
\coordinate (demobygangliupppkl) at (\demobygangliuxxxk, \demobygangliuyyyl);
\coordinate (demobygangliupppkm) at (\demobygangliuxxxk, \demobygangliuyyym);
\coordinate (demobygangliupppkn) at (\demobygangliuxxxk, \demobygangliuyyyn);
\coordinate (demobygangliupppko) at (\demobygangliuxxxk, \demobygangliuyyyo);
\coordinate (demobygangliupppkp) at (\demobygangliuxxxk, \demobygangliuyyyp);
\coordinate (demobygangliupppkq) at (\demobygangliuxxxk, \demobygangliuyyyq);
\coordinate (demobygangliupppkr) at (\demobygangliuxxxk, \demobygangliuyyyr);
\coordinate (demobygangliupppks) at (\demobygangliuxxxk, \demobygangliuyyys);
\coordinate (demobygangliupppkt) at (\demobygangliuxxxk, \demobygangliuyyyt);
\coordinate (demobygangliupppku) at (\demobygangliuxxxk, \demobygangliuyyyu);
\coordinate (demobygangliupppkv) at (\demobygangliuxxxk, \demobygangliuyyyv);
\coordinate (demobygangliupppkw) at (\demobygangliuxxxk, \demobygangliuyyyw);
\coordinate (demobygangliupppkx) at (\demobygangliuxxxk, \demobygangliuyyyx);
\coordinate (demobygangliupppky) at (\demobygangliuxxxk, \demobygangliuyyyy);
\coordinate (demobygangliupppkz) at (\demobygangliuxxxk, \demobygangliuyyyz);
\coordinate (demobygangliupppla) at (\demobygangliuxxxl, \demobygangliuyyya);
\coordinate (demobygangliuppplb) at (\demobygangliuxxxl, \demobygangliuyyyb);
\coordinate (demobygangliuppplc) at (\demobygangliuxxxl, \demobygangliuyyyc);
\coordinate (demobygangliupppld) at (\demobygangliuxxxl, \demobygangliuyyyd);
\coordinate (demobygangliuppple) at (\demobygangliuxxxl, \demobygangliuyyye);
\coordinate (demobygangliuppplf) at (\demobygangliuxxxl, \demobygangliuyyyf);
\coordinate (demobygangliuppplg) at (\demobygangliuxxxl, \demobygangliuyyyg);
\coordinate (demobygangliuppplh) at (\demobygangliuxxxl, \demobygangliuyyyh);
\coordinate (demobygangliupppli) at (\demobygangliuxxxl, \demobygangliuyyyi);
\coordinate (demobygangliuppplj) at (\demobygangliuxxxl, \demobygangliuyyyj);
\coordinate (demobygangliuppplk) at (\demobygangliuxxxl, \demobygangliuyyyk);
\coordinate (demobygangliupppll) at (\demobygangliuxxxl, \demobygangliuyyyl);
\coordinate (demobygangliuppplm) at (\demobygangliuxxxl, \demobygangliuyyym);
\coordinate (demobygangliupppln) at (\demobygangliuxxxl, \demobygangliuyyyn);
\coordinate (demobygangliuppplo) at (\demobygangliuxxxl, \demobygangliuyyyo);
\coordinate (demobygangliuppplp) at (\demobygangliuxxxl, \demobygangliuyyyp);
\coordinate (demobygangliuppplq) at (\demobygangliuxxxl, \demobygangliuyyyq);
\coordinate (demobygangliuppplr) at (\demobygangliuxxxl, \demobygangliuyyyr);
\coordinate (demobygangliupppls) at (\demobygangliuxxxl, \demobygangliuyyys);
\coordinate (demobygangliuppplt) at (\demobygangliuxxxl, \demobygangliuyyyt);
\coordinate (demobygangliuppplu) at (\demobygangliuxxxl, \demobygangliuyyyu);
\coordinate (demobygangliuppplv) at (\demobygangliuxxxl, \demobygangliuyyyv);
\coordinate (demobygangliuppplw) at (\demobygangliuxxxl, \demobygangliuyyyw);
\coordinate (demobygangliuppplx) at (\demobygangliuxxxl, \demobygangliuyyyx);
\coordinate (demobygangliuppply) at (\demobygangliuxxxl, \demobygangliuyyyy);
\coordinate (demobygangliuppplz) at (\demobygangliuxxxl, \demobygangliuyyyz);
\coordinate (demobygangliupppma) at (\demobygangliuxxxm, \demobygangliuyyya);
\coordinate (demobygangliupppmb) at (\demobygangliuxxxm, \demobygangliuyyyb);
\coordinate (demobygangliupppmc) at (\demobygangliuxxxm, \demobygangliuyyyc);
\coordinate (demobygangliupppmd) at (\demobygangliuxxxm, \demobygangliuyyyd);
\coordinate (demobygangliupppme) at (\demobygangliuxxxm, \demobygangliuyyye);
\coordinate (demobygangliupppmf) at (\demobygangliuxxxm, \demobygangliuyyyf);
\coordinate (demobygangliupppmg) at (\demobygangliuxxxm, \demobygangliuyyyg);
\coordinate (demobygangliupppmh) at (\demobygangliuxxxm, \demobygangliuyyyh);
\coordinate (demobygangliupppmi) at (\demobygangliuxxxm, \demobygangliuyyyi);
\coordinate (demobygangliupppmj) at (\demobygangliuxxxm, \demobygangliuyyyj);
\coordinate (demobygangliupppmk) at (\demobygangliuxxxm, \demobygangliuyyyk);
\coordinate (demobygangliupppml) at (\demobygangliuxxxm, \demobygangliuyyyl);
\coordinate (demobygangliupppmm) at (\demobygangliuxxxm, \demobygangliuyyym);
\coordinate (demobygangliupppmn) at (\demobygangliuxxxm, \demobygangliuyyyn);
\coordinate (demobygangliupppmo) at (\demobygangliuxxxm, \demobygangliuyyyo);
\coordinate (demobygangliupppmp) at (\demobygangliuxxxm, \demobygangliuyyyp);
\coordinate (demobygangliupppmq) at (\demobygangliuxxxm, \demobygangliuyyyq);
\coordinate (demobygangliupppmr) at (\demobygangliuxxxm, \demobygangliuyyyr);
\coordinate (demobygangliupppms) at (\demobygangliuxxxm, \demobygangliuyyys);
\coordinate (demobygangliupppmt) at (\demobygangliuxxxm, \demobygangliuyyyt);
\coordinate (demobygangliupppmu) at (\demobygangliuxxxm, \demobygangliuyyyu);
\coordinate (demobygangliupppmv) at (\demobygangliuxxxm, \demobygangliuyyyv);
\coordinate (demobygangliupppmw) at (\demobygangliuxxxm, \demobygangliuyyyw);
\coordinate (demobygangliupppmx) at (\demobygangliuxxxm, \demobygangliuyyyx);
\coordinate (demobygangliupppmy) at (\demobygangliuxxxm, \demobygangliuyyyy);
\coordinate (demobygangliupppmz) at (\demobygangliuxxxm, \demobygangliuyyyz);
\coordinate (demobygangliupppna) at (\demobygangliuxxxn, \demobygangliuyyya);
\coordinate (demobygangliupppnb) at (\demobygangliuxxxn, \demobygangliuyyyb);
\coordinate (demobygangliupppnc) at (\demobygangliuxxxn, \demobygangliuyyyc);
\coordinate (demobygangliupppnd) at (\demobygangliuxxxn, \demobygangliuyyyd);
\coordinate (demobygangliupppne) at (\demobygangliuxxxn, \demobygangliuyyye);
\coordinate (demobygangliupppnf) at (\demobygangliuxxxn, \demobygangliuyyyf);
\coordinate (demobygangliupppng) at (\demobygangliuxxxn, \demobygangliuyyyg);
\coordinate (demobygangliupppnh) at (\demobygangliuxxxn, \demobygangliuyyyh);
\coordinate (demobygangliupppni) at (\demobygangliuxxxn, \demobygangliuyyyi);
\coordinate (demobygangliupppnj) at (\demobygangliuxxxn, \demobygangliuyyyj);
\coordinate (demobygangliupppnk) at (\demobygangliuxxxn, \demobygangliuyyyk);
\coordinate (demobygangliupppnl) at (\demobygangliuxxxn, \demobygangliuyyyl);
\coordinate (demobygangliupppnm) at (\demobygangliuxxxn, \demobygangliuyyym);
\coordinate (demobygangliupppnn) at (\demobygangliuxxxn, \demobygangliuyyyn);
\coordinate (demobygangliupppno) at (\demobygangliuxxxn, \demobygangliuyyyo);
\coordinate (demobygangliupppnp) at (\demobygangliuxxxn, \demobygangliuyyyp);
\coordinate (demobygangliupppnq) at (\demobygangliuxxxn, \demobygangliuyyyq);
\coordinate (demobygangliupppnr) at (\demobygangliuxxxn, \demobygangliuyyyr);
\coordinate (demobygangliupppns) at (\demobygangliuxxxn, \demobygangliuyyys);
\coordinate (demobygangliupppnt) at (\demobygangliuxxxn, \demobygangliuyyyt);
\coordinate (demobygangliupppnu) at (\demobygangliuxxxn, \demobygangliuyyyu);
\coordinate (demobygangliupppnv) at (\demobygangliuxxxn, \demobygangliuyyyv);
\coordinate (demobygangliupppnw) at (\demobygangliuxxxn, \demobygangliuyyyw);
\coordinate (demobygangliupppnx) at (\demobygangliuxxxn, \demobygangliuyyyx);
\coordinate (demobygangliupppny) at (\demobygangliuxxxn, \demobygangliuyyyy);
\coordinate (demobygangliupppnz) at (\demobygangliuxxxn, \demobygangliuyyyz);
\coordinate (demobygangliupppoa) at (\demobygangliuxxxo, \demobygangliuyyya);
\coordinate (demobygangliupppob) at (\demobygangliuxxxo, \demobygangliuyyyb);
\coordinate (demobygangliupppoc) at (\demobygangliuxxxo, \demobygangliuyyyc);
\coordinate (demobygangliupppod) at (\demobygangliuxxxo, \demobygangliuyyyd);
\coordinate (demobygangliupppoe) at (\demobygangliuxxxo, \demobygangliuyyye);
\coordinate (demobygangliupppof) at (\demobygangliuxxxo, \demobygangliuyyyf);
\coordinate (demobygangliupppog) at (\demobygangliuxxxo, \demobygangliuyyyg);
\coordinate (demobygangliupppoh) at (\demobygangliuxxxo, \demobygangliuyyyh);
\coordinate (demobygangliupppoi) at (\demobygangliuxxxo, \demobygangliuyyyi);
\coordinate (demobygangliupppoj) at (\demobygangliuxxxo, \demobygangliuyyyj);
\coordinate (demobygangliupppok) at (\demobygangliuxxxo, \demobygangliuyyyk);
\coordinate (demobygangliupppol) at (\demobygangliuxxxo, \demobygangliuyyyl);
\coordinate (demobygangliupppom) at (\demobygangliuxxxo, \demobygangliuyyym);
\coordinate (demobygangliupppon) at (\demobygangliuxxxo, \demobygangliuyyyn);
\coordinate (demobygangliupppoo) at (\demobygangliuxxxo, \demobygangliuyyyo);
\coordinate (demobygangliupppop) at (\demobygangliuxxxo, \demobygangliuyyyp);
\coordinate (demobygangliupppoq) at (\demobygangliuxxxo, \demobygangliuyyyq);
\coordinate (demobygangliupppor) at (\demobygangliuxxxo, \demobygangliuyyyr);
\coordinate (demobygangliupppos) at (\demobygangliuxxxo, \demobygangliuyyys);
\coordinate (demobygangliupppot) at (\demobygangliuxxxo, \demobygangliuyyyt);
\coordinate (demobygangliupppou) at (\demobygangliuxxxo, \demobygangliuyyyu);
\coordinate (demobygangliupppov) at (\demobygangliuxxxo, \demobygangliuyyyv);
\coordinate (demobygangliupppow) at (\demobygangliuxxxo, \demobygangliuyyyw);
\coordinate (demobygangliupppox) at (\demobygangliuxxxo, \demobygangliuyyyx);
\coordinate (demobygangliupppoy) at (\demobygangliuxxxo, \demobygangliuyyyy);
\coordinate (demobygangliupppoz) at (\demobygangliuxxxo, \demobygangliuyyyz);
\coordinate (demobygangliuppppa) at (\demobygangliuxxxp, \demobygangliuyyya);
\coordinate (demobygangliuppppb) at (\demobygangliuxxxp, \demobygangliuyyyb);
\coordinate (demobygangliuppppc) at (\demobygangliuxxxp, \demobygangliuyyyc);
\coordinate (demobygangliuppppd) at (\demobygangliuxxxp, \demobygangliuyyyd);
\coordinate (demobygangliuppppe) at (\demobygangliuxxxp, \demobygangliuyyye);
\coordinate (demobygangliuppppf) at (\demobygangliuxxxp, \demobygangliuyyyf);
\coordinate (demobygangliuppppg) at (\demobygangliuxxxp, \demobygangliuyyyg);
\coordinate (demobygangliupppph) at (\demobygangliuxxxp, \demobygangliuyyyh);
\coordinate (demobygangliuppppi) at (\demobygangliuxxxp, \demobygangliuyyyi);
\coordinate (demobygangliuppppj) at (\demobygangliuxxxp, \demobygangliuyyyj);
\coordinate (demobygangliuppppk) at (\demobygangliuxxxp, \demobygangliuyyyk);
\coordinate (demobygangliuppppl) at (\demobygangliuxxxp, \demobygangliuyyyl);
\coordinate (demobygangliuppppm) at (\demobygangliuxxxp, \demobygangliuyyym);
\coordinate (demobygangliuppppn) at (\demobygangliuxxxp, \demobygangliuyyyn);
\coordinate (demobygangliuppppo) at (\demobygangliuxxxp, \demobygangliuyyyo);
\coordinate (demobygangliuppppp) at (\demobygangliuxxxp, \demobygangliuyyyp);
\coordinate (demobygangliuppppq) at (\demobygangliuxxxp, \demobygangliuyyyq);
\coordinate (demobygangliuppppr) at (\demobygangliuxxxp, \demobygangliuyyyr);
\coordinate (demobygangliupppps) at (\demobygangliuxxxp, \demobygangliuyyys);
\coordinate (demobygangliuppppt) at (\demobygangliuxxxp, \demobygangliuyyyt);
\coordinate (demobygangliuppppu) at (\demobygangliuxxxp, \demobygangliuyyyu);
\coordinate (demobygangliuppppv) at (\demobygangliuxxxp, \demobygangliuyyyv);
\coordinate (demobygangliuppppw) at (\demobygangliuxxxp, \demobygangliuyyyw);
\coordinate (demobygangliuppppx) at (\demobygangliuxxxp, \demobygangliuyyyx);
\coordinate (demobygangliuppppy) at (\demobygangliuxxxp, \demobygangliuyyyy);
\coordinate (demobygangliuppppz) at (\demobygangliuxxxp, \demobygangliuyyyz);
\coordinate (demobygangliupppqa) at (\demobygangliuxxxq, \demobygangliuyyya);
\coordinate (demobygangliupppqb) at (\demobygangliuxxxq, \demobygangliuyyyb);
\coordinate (demobygangliupppqc) at (\demobygangliuxxxq, \demobygangliuyyyc);
\coordinate (demobygangliupppqd) at (\demobygangliuxxxq, \demobygangliuyyyd);
\coordinate (demobygangliupppqe) at (\demobygangliuxxxq, \demobygangliuyyye);
\coordinate (demobygangliupppqf) at (\demobygangliuxxxq, \demobygangliuyyyf);
\coordinate (demobygangliupppqg) at (\demobygangliuxxxq, \demobygangliuyyyg);
\coordinate (demobygangliupppqh) at (\demobygangliuxxxq, \demobygangliuyyyh);
\coordinate (demobygangliupppqi) at (\demobygangliuxxxq, \demobygangliuyyyi);
\coordinate (demobygangliupppqj) at (\demobygangliuxxxq, \demobygangliuyyyj);
\coordinate (demobygangliupppqk) at (\demobygangliuxxxq, \demobygangliuyyyk);
\coordinate (demobygangliupppql) at (\demobygangliuxxxq, \demobygangliuyyyl);
\coordinate (demobygangliupppqm) at (\demobygangliuxxxq, \demobygangliuyyym);
\coordinate (demobygangliupppqn) at (\demobygangliuxxxq, \demobygangliuyyyn);
\coordinate (demobygangliupppqo) at (\demobygangliuxxxq, \demobygangliuyyyo);
\coordinate (demobygangliupppqp) at (\demobygangliuxxxq, \demobygangliuyyyp);
\coordinate (demobygangliupppqq) at (\demobygangliuxxxq, \demobygangliuyyyq);
\coordinate (demobygangliupppqr) at (\demobygangliuxxxq, \demobygangliuyyyr);
\coordinate (demobygangliupppqs) at (\demobygangliuxxxq, \demobygangliuyyys);
\coordinate (demobygangliupppqt) at (\demobygangliuxxxq, \demobygangliuyyyt);
\coordinate (demobygangliupppqu) at (\demobygangliuxxxq, \demobygangliuyyyu);
\coordinate (demobygangliupppqv) at (\demobygangliuxxxq, \demobygangliuyyyv);
\coordinate (demobygangliupppqw) at (\demobygangliuxxxq, \demobygangliuyyyw);
\coordinate (demobygangliupppqx) at (\demobygangliuxxxq, \demobygangliuyyyx);
\coordinate (demobygangliupppqy) at (\demobygangliuxxxq, \demobygangliuyyyy);
\coordinate (demobygangliupppqz) at (\demobygangliuxxxq, \demobygangliuyyyz);
\coordinate (demobygangliupppra) at (\demobygangliuxxxr, \demobygangliuyyya);
\coordinate (demobygangliuppprb) at (\demobygangliuxxxr, \demobygangliuyyyb);
\coordinate (demobygangliuppprc) at (\demobygangliuxxxr, \demobygangliuyyyc);
\coordinate (demobygangliuppprd) at (\demobygangliuxxxr, \demobygangliuyyyd);
\coordinate (demobygangliupppre) at (\demobygangliuxxxr, \demobygangliuyyye);
\coordinate (demobygangliuppprf) at (\demobygangliuxxxr, \demobygangliuyyyf);
\coordinate (demobygangliuppprg) at (\demobygangliuxxxr, \demobygangliuyyyg);
\coordinate (demobygangliuppprh) at (\demobygangliuxxxr, \demobygangliuyyyh);
\coordinate (demobygangliupppri) at (\demobygangliuxxxr, \demobygangliuyyyi);
\coordinate (demobygangliuppprj) at (\demobygangliuxxxr, \demobygangliuyyyj);
\coordinate (demobygangliuppprk) at (\demobygangliuxxxr, \demobygangliuyyyk);
\coordinate (demobygangliuppprl) at (\demobygangliuxxxr, \demobygangliuyyyl);
\coordinate (demobygangliuppprm) at (\demobygangliuxxxr, \demobygangliuyyym);
\coordinate (demobygangliuppprn) at (\demobygangliuxxxr, \demobygangliuyyyn);
\coordinate (demobygangliupppro) at (\demobygangliuxxxr, \demobygangliuyyyo);
\coordinate (demobygangliuppprp) at (\demobygangliuxxxr, \demobygangliuyyyp);
\coordinate (demobygangliuppprq) at (\demobygangliuxxxr, \demobygangliuyyyq);
\coordinate (demobygangliuppprr) at (\demobygangliuxxxr, \demobygangliuyyyr);
\coordinate (demobygangliuppprs) at (\demobygangliuxxxr, \demobygangliuyyys);
\coordinate (demobygangliuppprt) at (\demobygangliuxxxr, \demobygangliuyyyt);
\coordinate (demobygangliupppru) at (\demobygangliuxxxr, \demobygangliuyyyu);
\coordinate (demobygangliuppprv) at (\demobygangliuxxxr, \demobygangliuyyyv);
\coordinate (demobygangliuppprw) at (\demobygangliuxxxr, \demobygangliuyyyw);
\coordinate (demobygangliuppprx) at (\demobygangliuxxxr, \demobygangliuyyyx);
\coordinate (demobygangliupppry) at (\demobygangliuxxxr, \demobygangliuyyyy);
\coordinate (demobygangliuppprz) at (\demobygangliuxxxr, \demobygangliuyyyz);
\coordinate (demobygangliupppsa) at (\demobygangliuxxxs, \demobygangliuyyya);
\coordinate (demobygangliupppsb) at (\demobygangliuxxxs, \demobygangliuyyyb);
\coordinate (demobygangliupppsc) at (\demobygangliuxxxs, \demobygangliuyyyc);
\coordinate (demobygangliupppsd) at (\demobygangliuxxxs, \demobygangliuyyyd);
\coordinate (demobygangliupppse) at (\demobygangliuxxxs, \demobygangliuyyye);
\coordinate (demobygangliupppsf) at (\demobygangliuxxxs, \demobygangliuyyyf);
\coordinate (demobygangliupppsg) at (\demobygangliuxxxs, \demobygangliuyyyg);
\coordinate (demobygangliupppsh) at (\demobygangliuxxxs, \demobygangliuyyyh);
\coordinate (demobygangliupppsi) at (\demobygangliuxxxs, \demobygangliuyyyi);
\coordinate (demobygangliupppsj) at (\demobygangliuxxxs, \demobygangliuyyyj);
\coordinate (demobygangliupppsk) at (\demobygangliuxxxs, \demobygangliuyyyk);
\coordinate (demobygangliupppsl) at (\demobygangliuxxxs, \demobygangliuyyyl);
\coordinate (demobygangliupppsm) at (\demobygangliuxxxs, \demobygangliuyyym);
\coordinate (demobygangliupppsn) at (\demobygangliuxxxs, \demobygangliuyyyn);
\coordinate (demobygangliupppso) at (\demobygangliuxxxs, \demobygangliuyyyo);
\coordinate (demobygangliupppsp) at (\demobygangliuxxxs, \demobygangliuyyyp);
\coordinate (demobygangliupppsq) at (\demobygangliuxxxs, \demobygangliuyyyq);
\coordinate (demobygangliupppsr) at (\demobygangliuxxxs, \demobygangliuyyyr);
\coordinate (demobygangliupppss) at (\demobygangliuxxxs, \demobygangliuyyys);
\coordinate (demobygangliupppst) at (\demobygangliuxxxs, \demobygangliuyyyt);
\coordinate (demobygangliupppsu) at (\demobygangliuxxxs, \demobygangliuyyyu);
\coordinate (demobygangliupppsv) at (\demobygangliuxxxs, \demobygangliuyyyv);
\coordinate (demobygangliupppsw) at (\demobygangliuxxxs, \demobygangliuyyyw);
\coordinate (demobygangliupppsx) at (\demobygangliuxxxs, \demobygangliuyyyx);
\coordinate (demobygangliupppsy) at (\demobygangliuxxxs, \demobygangliuyyyy);
\coordinate (demobygangliupppsz) at (\demobygangliuxxxs, \demobygangliuyyyz);
\coordinate (demobygangliupppta) at (\demobygangliuxxxt, \demobygangliuyyya);
\coordinate (demobygangliuppptb) at (\demobygangliuxxxt, \demobygangliuyyyb);
\coordinate (demobygangliuppptc) at (\demobygangliuxxxt, \demobygangliuyyyc);
\coordinate (demobygangliuppptd) at (\demobygangliuxxxt, \demobygangliuyyyd);
\coordinate (demobygangliupppte) at (\demobygangliuxxxt, \demobygangliuyyye);
\coordinate (demobygangliuppptf) at (\demobygangliuxxxt, \demobygangliuyyyf);
\coordinate (demobygangliuppptg) at (\demobygangliuxxxt, \demobygangliuyyyg);
\coordinate (demobygangliupppth) at (\demobygangliuxxxt, \demobygangliuyyyh);
\coordinate (demobygangliupppti) at (\demobygangliuxxxt, \demobygangliuyyyi);
\coordinate (demobygangliuppptj) at (\demobygangliuxxxt, \demobygangliuyyyj);
\coordinate (demobygangliuppptk) at (\demobygangliuxxxt, \demobygangliuyyyk);
\coordinate (demobygangliuppptl) at (\demobygangliuxxxt, \demobygangliuyyyl);
\coordinate (demobygangliuppptm) at (\demobygangliuxxxt, \demobygangliuyyym);
\coordinate (demobygangliuppptn) at (\demobygangliuxxxt, \demobygangliuyyyn);
\coordinate (demobygangliupppto) at (\demobygangliuxxxt, \demobygangliuyyyo);
\coordinate (demobygangliuppptp) at (\demobygangliuxxxt, \demobygangliuyyyp);
\coordinate (demobygangliuppptq) at (\demobygangliuxxxt, \demobygangliuyyyq);
\coordinate (demobygangliuppptr) at (\demobygangliuxxxt, \demobygangliuyyyr);
\coordinate (demobygangliupppts) at (\demobygangliuxxxt, \demobygangliuyyys);
\coordinate (demobygangliuppptt) at (\demobygangliuxxxt, \demobygangliuyyyt);
\coordinate (demobygangliuppptu) at (\demobygangliuxxxt, \demobygangliuyyyu);
\coordinate (demobygangliuppptv) at (\demobygangliuxxxt, \demobygangliuyyyv);
\coordinate (demobygangliuppptw) at (\demobygangliuxxxt, \demobygangliuyyyw);
\coordinate (demobygangliuppptx) at (\demobygangliuxxxt, \demobygangliuyyyx);
\coordinate (demobygangliupppty) at (\demobygangliuxxxt, \demobygangliuyyyy);
\coordinate (demobygangliuppptz) at (\demobygangliuxxxt, \demobygangliuyyyz);
\coordinate (demobygangliupppua) at (\demobygangliuxxxu, \demobygangliuyyya);
\coordinate (demobygangliupppub) at (\demobygangliuxxxu, \demobygangliuyyyb);
\coordinate (demobygangliupppuc) at (\demobygangliuxxxu, \demobygangliuyyyc);
\coordinate (demobygangliupppud) at (\demobygangliuxxxu, \demobygangliuyyyd);
\coordinate (demobygangliupppue) at (\demobygangliuxxxu, \demobygangliuyyye);
\coordinate (demobygangliupppuf) at (\demobygangliuxxxu, \demobygangliuyyyf);
\coordinate (demobygangliupppug) at (\demobygangliuxxxu, \demobygangliuyyyg);
\coordinate (demobygangliupppuh) at (\demobygangliuxxxu, \demobygangliuyyyh);
\coordinate (demobygangliupppui) at (\demobygangliuxxxu, \demobygangliuyyyi);
\coordinate (demobygangliupppuj) at (\demobygangliuxxxu, \demobygangliuyyyj);
\coordinate (demobygangliupppuk) at (\demobygangliuxxxu, \demobygangliuyyyk);
\coordinate (demobygangliupppul) at (\demobygangliuxxxu, \demobygangliuyyyl);
\coordinate (demobygangliupppum) at (\demobygangliuxxxu, \demobygangliuyyym);
\coordinate (demobygangliupppun) at (\demobygangliuxxxu, \demobygangliuyyyn);
\coordinate (demobygangliupppuo) at (\demobygangliuxxxu, \demobygangliuyyyo);
\coordinate (demobygangliupppup) at (\demobygangliuxxxu, \demobygangliuyyyp);
\coordinate (demobygangliupppuq) at (\demobygangliuxxxu, \demobygangliuyyyq);
\coordinate (demobygangliupppur) at (\demobygangliuxxxu, \demobygangliuyyyr);
\coordinate (demobygangliupppus) at (\demobygangliuxxxu, \demobygangliuyyys);
\coordinate (demobygangliuppput) at (\demobygangliuxxxu, \demobygangliuyyyt);
\coordinate (demobygangliupppuu) at (\demobygangliuxxxu, \demobygangliuyyyu);
\coordinate (demobygangliupppuv) at (\demobygangliuxxxu, \demobygangliuyyyv);
\coordinate (demobygangliupppuw) at (\demobygangliuxxxu, \demobygangliuyyyw);
\coordinate (demobygangliupppux) at (\demobygangliuxxxu, \demobygangliuyyyx);
\coordinate (demobygangliupppuy) at (\demobygangliuxxxu, \demobygangliuyyyy);
\coordinate (demobygangliupppuz) at (\demobygangliuxxxu, \demobygangliuyyyz);
\coordinate (demobygangliupppva) at (\demobygangliuxxxv, \demobygangliuyyya);
\coordinate (demobygangliupppvb) at (\demobygangliuxxxv, \demobygangliuyyyb);
\coordinate (demobygangliupppvc) at (\demobygangliuxxxv, \demobygangliuyyyc);
\coordinate (demobygangliupppvd) at (\demobygangliuxxxv, \demobygangliuyyyd);
\coordinate (demobygangliupppve) at (\demobygangliuxxxv, \demobygangliuyyye);
\coordinate (demobygangliupppvf) at (\demobygangliuxxxv, \demobygangliuyyyf);
\coordinate (demobygangliupppvg) at (\demobygangliuxxxv, \demobygangliuyyyg);
\coordinate (demobygangliupppvh) at (\demobygangliuxxxv, \demobygangliuyyyh);
\coordinate (demobygangliupppvi) at (\demobygangliuxxxv, \demobygangliuyyyi);
\coordinate (demobygangliupppvj) at (\demobygangliuxxxv, \demobygangliuyyyj);
\coordinate (demobygangliupppvk) at (\demobygangliuxxxv, \demobygangliuyyyk);
\coordinate (demobygangliupppvl) at (\demobygangliuxxxv, \demobygangliuyyyl);
\coordinate (demobygangliupppvm) at (\demobygangliuxxxv, \demobygangliuyyym);
\coordinate (demobygangliupppvn) at (\demobygangliuxxxv, \demobygangliuyyyn);
\coordinate (demobygangliupppvo) at (\demobygangliuxxxv, \demobygangliuyyyo);
\coordinate (demobygangliupppvp) at (\demobygangliuxxxv, \demobygangliuyyyp);
\coordinate (demobygangliupppvq) at (\demobygangliuxxxv, \demobygangliuyyyq);
\coordinate (demobygangliupppvr) at (\demobygangliuxxxv, \demobygangliuyyyr);
\coordinate (demobygangliupppvs) at (\demobygangliuxxxv, \demobygangliuyyys);
\coordinate (demobygangliupppvt) at (\demobygangliuxxxv, \demobygangliuyyyt);
\coordinate (demobygangliupppvu) at (\demobygangliuxxxv, \demobygangliuyyyu);
\coordinate (demobygangliupppvv) at (\demobygangliuxxxv, \demobygangliuyyyv);
\coordinate (demobygangliupppvw) at (\demobygangliuxxxv, \demobygangliuyyyw);
\coordinate (demobygangliupppvx) at (\demobygangliuxxxv, \demobygangliuyyyx);
\coordinate (demobygangliupppvy) at (\demobygangliuxxxv, \demobygangliuyyyy);
\coordinate (demobygangliupppvz) at (\demobygangliuxxxv, \demobygangliuyyyz);
\coordinate (demobygangliupppwa) at (\demobygangliuxxxw, \demobygangliuyyya);
\coordinate (demobygangliupppwb) at (\demobygangliuxxxw, \demobygangliuyyyb);
\coordinate (demobygangliupppwc) at (\demobygangliuxxxw, \demobygangliuyyyc);
\coordinate (demobygangliupppwd) at (\demobygangliuxxxw, \demobygangliuyyyd);
\coordinate (demobygangliupppwe) at (\demobygangliuxxxw, \demobygangliuyyye);
\coordinate (demobygangliupppwf) at (\demobygangliuxxxw, \demobygangliuyyyf);
\coordinate (demobygangliupppwg) at (\demobygangliuxxxw, \demobygangliuyyyg);
\coordinate (demobygangliupppwh) at (\demobygangliuxxxw, \demobygangliuyyyh);
\coordinate (demobygangliupppwi) at (\demobygangliuxxxw, \demobygangliuyyyi);
\coordinate (demobygangliupppwj) at (\demobygangliuxxxw, \demobygangliuyyyj);
\coordinate (demobygangliupppwk) at (\demobygangliuxxxw, \demobygangliuyyyk);
\coordinate (demobygangliupppwl) at (\demobygangliuxxxw, \demobygangliuyyyl);
\coordinate (demobygangliupppwm) at (\demobygangliuxxxw, \demobygangliuyyym);
\coordinate (demobygangliupppwn) at (\demobygangliuxxxw, \demobygangliuyyyn);
\coordinate (demobygangliupppwo) at (\demobygangliuxxxw, \demobygangliuyyyo);
\coordinate (demobygangliupppwp) at (\demobygangliuxxxw, \demobygangliuyyyp);
\coordinate (demobygangliupppwq) at (\demobygangliuxxxw, \demobygangliuyyyq);
\coordinate (demobygangliupppwr) at (\demobygangliuxxxw, \demobygangliuyyyr);
\coordinate (demobygangliupppws) at (\demobygangliuxxxw, \demobygangliuyyys);
\coordinate (demobygangliupppwt) at (\demobygangliuxxxw, \demobygangliuyyyt);
\coordinate (demobygangliupppwu) at (\demobygangliuxxxw, \demobygangliuyyyu);
\coordinate (demobygangliupppwv) at (\demobygangliuxxxw, \demobygangliuyyyv);
\coordinate (demobygangliupppww) at (\demobygangliuxxxw, \demobygangliuyyyw);
\coordinate (demobygangliupppwx) at (\demobygangliuxxxw, \demobygangliuyyyx);
\coordinate (demobygangliupppwy) at (\demobygangliuxxxw, \demobygangliuyyyy);
\coordinate (demobygangliupppwz) at (\demobygangliuxxxw, \demobygangliuyyyz);
\coordinate (demobygangliupppxa) at (\demobygangliuxxxx, \demobygangliuyyya);
\coordinate (demobygangliupppxb) at (\demobygangliuxxxx, \demobygangliuyyyb);
\coordinate (demobygangliupppxc) at (\demobygangliuxxxx, \demobygangliuyyyc);
\coordinate (demobygangliupppxd) at (\demobygangliuxxxx, \demobygangliuyyyd);
\coordinate (demobygangliupppxe) at (\demobygangliuxxxx, \demobygangliuyyye);
\coordinate (demobygangliupppxf) at (\demobygangliuxxxx, \demobygangliuyyyf);
\coordinate (demobygangliupppxg) at (\demobygangliuxxxx, \demobygangliuyyyg);
\coordinate (demobygangliupppxh) at (\demobygangliuxxxx, \demobygangliuyyyh);
\coordinate (demobygangliupppxi) at (\demobygangliuxxxx, \demobygangliuyyyi);
\coordinate (demobygangliupppxj) at (\demobygangliuxxxx, \demobygangliuyyyj);
\coordinate (demobygangliupppxk) at (\demobygangliuxxxx, \demobygangliuyyyk);
\coordinate (demobygangliupppxl) at (\demobygangliuxxxx, \demobygangliuyyyl);
\coordinate (demobygangliupppxm) at (\demobygangliuxxxx, \demobygangliuyyym);
\coordinate (demobygangliupppxn) at (\demobygangliuxxxx, \demobygangliuyyyn);
\coordinate (demobygangliupppxo) at (\demobygangliuxxxx, \demobygangliuyyyo);
\coordinate (demobygangliupppxp) at (\demobygangliuxxxx, \demobygangliuyyyp);
\coordinate (demobygangliupppxq) at (\demobygangliuxxxx, \demobygangliuyyyq);
\coordinate (demobygangliupppxr) at (\demobygangliuxxxx, \demobygangliuyyyr);
\coordinate (demobygangliupppxs) at (\demobygangliuxxxx, \demobygangliuyyys);
\coordinate (demobygangliupppxt) at (\demobygangliuxxxx, \demobygangliuyyyt);
\coordinate (demobygangliupppxu) at (\demobygangliuxxxx, \demobygangliuyyyu);
\coordinate (demobygangliupppxv) at (\demobygangliuxxxx, \demobygangliuyyyv);
\coordinate (demobygangliupppxw) at (\demobygangliuxxxx, \demobygangliuyyyw);
\coordinate (demobygangliupppxx) at (\demobygangliuxxxx, \demobygangliuyyyx);
\coordinate (demobygangliupppxy) at (\demobygangliuxxxx, \demobygangliuyyyy);
\coordinate (demobygangliupppxz) at (\demobygangliuxxxx, \demobygangliuyyyz);
\coordinate (demobygangliupppya) at (\demobygangliuxxxy, \demobygangliuyyya);
\coordinate (demobygangliupppyb) at (\demobygangliuxxxy, \demobygangliuyyyb);
\coordinate (demobygangliupppyc) at (\demobygangliuxxxy, \demobygangliuyyyc);
\coordinate (demobygangliupppyd) at (\demobygangliuxxxy, \demobygangliuyyyd);
\coordinate (demobygangliupppye) at (\demobygangliuxxxy, \demobygangliuyyye);
\coordinate (demobygangliupppyf) at (\demobygangliuxxxy, \demobygangliuyyyf);
\coordinate (demobygangliupppyg) at (\demobygangliuxxxy, \demobygangliuyyyg);
\coordinate (demobygangliupppyh) at (\demobygangliuxxxy, \demobygangliuyyyh);
\coordinate (demobygangliupppyi) at (\demobygangliuxxxy, \demobygangliuyyyi);
\coordinate (demobygangliupppyj) at (\demobygangliuxxxy, \demobygangliuyyyj);
\coordinate (demobygangliupppyk) at (\demobygangliuxxxy, \demobygangliuyyyk);
\coordinate (demobygangliupppyl) at (\demobygangliuxxxy, \demobygangliuyyyl);
\coordinate (demobygangliupppym) at (\demobygangliuxxxy, \demobygangliuyyym);
\coordinate (demobygangliupppyn) at (\demobygangliuxxxy, \demobygangliuyyyn);
\coordinate (demobygangliupppyo) at (\demobygangliuxxxy, \demobygangliuyyyo);
\coordinate (demobygangliupppyp) at (\demobygangliuxxxy, \demobygangliuyyyp);
\coordinate (demobygangliupppyq) at (\demobygangliuxxxy, \demobygangliuyyyq);
\coordinate (demobygangliupppyr) at (\demobygangliuxxxy, \demobygangliuyyyr);
\coordinate (demobygangliupppys) at (\demobygangliuxxxy, \demobygangliuyyys);
\coordinate (demobygangliupppyt) at (\demobygangliuxxxy, \demobygangliuyyyt);
\coordinate (demobygangliupppyu) at (\demobygangliuxxxy, \demobygangliuyyyu);
\coordinate (demobygangliupppyv) at (\demobygangliuxxxy, \demobygangliuyyyv);
\coordinate (demobygangliupppyw) at (\demobygangliuxxxy, \demobygangliuyyyw);
\coordinate (demobygangliupppyx) at (\demobygangliuxxxy, \demobygangliuyyyx);
\coordinate (demobygangliupppyy) at (\demobygangliuxxxy, \demobygangliuyyyy);
\coordinate (demobygangliupppyz) at (\demobygangliuxxxy, \demobygangliuyyyz);
\coordinate (demobygangliupppza) at (\demobygangliuxxxz, \demobygangliuyyya);
\coordinate (demobygangliupppzb) at (\demobygangliuxxxz, \demobygangliuyyyb);
\coordinate (demobygangliupppzc) at (\demobygangliuxxxz, \demobygangliuyyyc);
\coordinate (demobygangliupppzd) at (\demobygangliuxxxz, \demobygangliuyyyd);
\coordinate (demobygangliupppze) at (\demobygangliuxxxz, \demobygangliuyyye);
\coordinate (demobygangliupppzf) at (\demobygangliuxxxz, \demobygangliuyyyf);
\coordinate (demobygangliupppzg) at (\demobygangliuxxxz, \demobygangliuyyyg);
\coordinate (demobygangliupppzh) at (\demobygangliuxxxz, \demobygangliuyyyh);
\coordinate (demobygangliupppzi) at (\demobygangliuxxxz, \demobygangliuyyyi);
\coordinate (demobygangliupppzj) at (\demobygangliuxxxz, \demobygangliuyyyj);
\coordinate (demobygangliupppzk) at (\demobygangliuxxxz, \demobygangliuyyyk);
\coordinate (demobygangliupppzl) at (\demobygangliuxxxz, \demobygangliuyyyl);
\coordinate (demobygangliupppzm) at (\demobygangliuxxxz, \demobygangliuyyym);
\coordinate (demobygangliupppzn) at (\demobygangliuxxxz, \demobygangliuyyyn);
\coordinate (demobygangliupppzo) at (\demobygangliuxxxz, \demobygangliuyyyo);
\coordinate (demobygangliupppzp) at (\demobygangliuxxxz, \demobygangliuyyyp);
\coordinate (demobygangliupppzq) at (\demobygangliuxxxz, \demobygangliuyyyq);
\coordinate (demobygangliupppzr) at (\demobygangliuxxxz, \demobygangliuyyyr);
\coordinate (demobygangliupppzs) at (\demobygangliuxxxz, \demobygangliuyyys);
\coordinate (demobygangliupppzt) at (\demobygangliuxxxz, \demobygangliuyyyt);
\coordinate (demobygangliupppzu) at (\demobygangliuxxxz, \demobygangliuyyyu);
\coordinate (demobygangliupppzv) at (\demobygangliuxxxz, \demobygangliuyyyv);
\coordinate (demobygangliupppzw) at (\demobygangliuxxxz, \demobygangliuyyyw);
\coordinate (demobygangliupppzx) at (\demobygangliuxxxz, \demobygangliuyyyx);
\coordinate (demobygangliupppzy) at (\demobygangliuxxxz, \demobygangliuyyyy);
\coordinate (demobygangliupppzz) at (\demobygangliuxxxz, \demobygangliuyyyz);

%\gangprintcoordinateat{(0,0)}{The last coordinate values: }{($(demobygangliupppzz)$)}; 



% Draw related part of the coordinate system with dashed helplines (centered at (demobygangliupppii)) with letters as background, which would help to determine all coordinates. 
\coordinatebackgroundxy{demobygangliu}{b}{c}{q}{a}{b}{s};

% Step 2, draw key devices, their accessories, and take related coordinates of their pins, and may define more coordinates. 

% Draw the Opamp at the coordinate (demobygangliupppii) and name it as ``swopamp".
\draw (demobygangliupppii) node [op amp, yscale=-1] (swopamp) {\ctikzflipy{Opamp}} ; 

% Its accessories and lables. 
\draw [-*](swopamp.down) -- ($(swopamp.down)+(0,1)$) node[right]{$V_+$}; 
\node at ($(swopamp.down)+(0.3,0.2)$) {7};  
\draw [-*](swopamp.up) -- ($(swopamp.up)+(0,-1)$) node[right]{$V_-$}; 
\node at ($(swopamp.up)+(0.3,-0.2)$) {4};

% Get the x- and y-components of the coordinates of the ``+" and ``-" pins. 
\getxyingivenunit{cm}{(swopamp.+)}{\swopampzx}{\swopampzy};
\getxyingivenunit{cm}{(swopamp.-)}{\swopampfx}{\swopampfy};

% Then define a few more coordinates, at least for keeping in mind.
\coordinate (plusshort) at ($(\demobygangliuxxxg,\swopampzy)$);
\fill  (plusshort) circle (2pt);  % May be commented later.
\coordinate (minusshort) at ($(\demobygangliuxxxg,\swopampfy)$);
\fill  (minusshort) circle (2pt); % May be commented later.
\coordinate (leftinter) at ($(\demobygangliuxxxe,\swopampzy)$);
\fill  (leftinter) circle (2pt);

% Draw an ``npn" at (demobygangliupppmi) and name it as "swQ".
\draw (demobygangliupppmi) node[npn](swQ){};

% Get the x- and y-components of the needed pins of it for later usage.
\getxyingivenunit{cm}{(swQ.C)}{\swQCx}{\swQCy};
\getxyingivenunit{cm}{(swQ.E)}{\swQEx}{\swQEy};

% Then define more coordinate(s).
\coordinate (Qcshort) at ($(\swQEx,\demobygangliuyyyj)$);
\fill  (Qcshort) circle (2pt); % May be commented later.
\coordinate (Qeshort) at ($(\swQEx,\demobygangliuyyyf)$);
\fill  (Qeshort) circle (2pt) node [right] {$V_0$};

% Then the rectangle by the points (demobygangliupppef) -- (demobygangliupppej) -- (Qcshort) -- (Qeshort) forms a clear area for the key devices. 

% Connect the two devices.
\draw (swopamp.out) to [short, l=$I_B$, above] (swQ.B);

% Step 3, draw other little devices. For tidiness, better to give two units in length for each new device and align them up.

% For this specific circuit, let us attach the four bi-pole devices (maybe with their accessories) to each corner of the above mentioned rectangle area for the key devices, separately. 
\end{circuitikz}

\newpage

{\Large Figure 2, the background of coordinate system with keyword ``w", where wxxxf, wxxxm, wyyyd, wyyyg, and wyyyk, are further manually increased by 1.3, 3.6 3.2, 4.5, and 2.9 respectively in the separate unique coordw.tex file, then the dashed lines are not evenly distributed, while the coordinate letters are still placed properly.}

\begin{circuitikz}[scale=1]
\pgfmathsetmacro{\totalwxxx}{26}
\pgfmathsetmacro{\totalwyyy}{26}
\pgfmathsetmacro{\wxxxspacing}{1}
\pgfmathsetmacro{\wyyyspacing}{1}
\pgfmathsetmacro{\wxxxa}{-8}
\pgfmathsetmacro{\wyyya}{-8}

\pgfmathsetmacro{\wxxxb}{\wxxxa + \wxxxspacing + 0.0 }
\pgfmathsetmacro{\wxxxc}{\wxxxb + \wxxxspacing + 0.0 }
\pgfmathsetmacro{\wxxxd}{\wxxxc + \wxxxspacing + 0.0 }
\pgfmathsetmacro{\wxxxe}{\wxxxd + \wxxxspacing + 0.0 }
\pgfmathsetmacro{\wxxxf}{\wxxxe + \wxxxspacing + 01.30 }
\pgfmathsetmacro{\wxxxg}{\wxxxf + \wxxxspacing + 0.0 }
\pgfmathsetmacro{\wxxxh}{\wxxxg + \wxxxspacing + 0.0 }
\pgfmathsetmacro{\wxxxi}{\wxxxh + \wxxxspacing + 0.0 }
\pgfmathsetmacro{\wxxxj}{\wxxxi + \wxxxspacing + 0.0 }
\pgfmathsetmacro{\wxxxk}{\wxxxj + \wxxxspacing + 0.0 }
\pgfmathsetmacro{\wxxxl}{\wxxxk + \wxxxspacing + 0.0 }
\pgfmathsetmacro{\wxxxm}{\wxxxl + \wxxxspacing + 03.60 }
\pgfmathsetmacro{\wxxxn}{\wxxxm + \wxxxspacing + 0.0 }
\pgfmathsetmacro{\wxxxo}{\wxxxn + \wxxxspacing + 0.0 }
\pgfmathsetmacro{\wxxxp}{\wxxxo + \wxxxspacing + 0.0 }
\pgfmathsetmacro{\wxxxq}{\wxxxp + \wxxxspacing + 0.0 }
\pgfmathsetmacro{\wxxxr}{\wxxxq + \wxxxspacing + 0.0 }
\pgfmathsetmacro{\wxxxs}{\wxxxr + \wxxxspacing + 0.0 }
\pgfmathsetmacro{\wxxxt}{\wxxxs + \wxxxspacing + 0.0 }
\pgfmathsetmacro{\wxxxu}{\wxxxt + \wxxxspacing + 0.0 }
\pgfmathsetmacro{\wxxxv}{\wxxxu + \wxxxspacing + 0.0 }
\pgfmathsetmacro{\wxxxw}{\wxxxv + \wxxxspacing + 0.0 }
\pgfmathsetmacro{\wxxxx}{\wxxxw + \wxxxspacing + 0.0 }
\pgfmathsetmacro{\wxxxy}{\wxxxx + \wxxxspacing + 0.0 }
\pgfmathsetmacro{\wxxxz}{\wxxxy + \wxxxspacing + 0.0 }

\pgfmathsetmacro{\wyyyb}{\wyyya + \wyyyspacing + 0.0 }
\pgfmathsetmacro{\wyyyc}{\wyyyb + \wyyyspacing + 0.0 }
\pgfmathsetmacro{\wyyyd}{\wyyyc + \wyyyspacing + 03.20 }
\pgfmathsetmacro{\wyyye}{\wyyyd + \wyyyspacing + 0.0 }
\pgfmathsetmacro{\wyyyf}{\wyyye + \wyyyspacing + 0.0 }
\pgfmathsetmacro{\wyyyg}{\wyyyf + \wyyyspacing + 04.50 }
\pgfmathsetmacro{\wyyyh}{\wyyyg + \wyyyspacing + 0.0 }
\pgfmathsetmacro{\wyyyi}{\wyyyh + \wyyyspacing + 0.0 }
\pgfmathsetmacro{\wyyyj}{\wyyyi + \wyyyspacing + 0.0 }
\pgfmathsetmacro{\wyyyk}{\wyyyj + \wyyyspacing + 02.90 }
\pgfmathsetmacro{\wyyyl}{\wyyyk + \wyyyspacing + 0.0 }
\pgfmathsetmacro{\wyyym}{\wyyyl + \wyyyspacing + 0.0 }
\pgfmathsetmacro{\wyyyn}{\wyyym + \wyyyspacing + 0.0 }
\pgfmathsetmacro{\wyyyo}{\wyyyn + \wyyyspacing + 0.0 }
\pgfmathsetmacro{\wyyyp}{\wyyyo + \wyyyspacing + 0.0 }
\pgfmathsetmacro{\wyyyq}{\wyyyp + \wyyyspacing + 0.0 }
\pgfmathsetmacro{\wyyyr}{\wyyyq + \wyyyspacing + 0.0 }
\pgfmathsetmacro{\wyyys}{\wyyyr + \wyyyspacing + 0.0 }
\pgfmathsetmacro{\wyyyt}{\wyyys + \wyyyspacing + 0.0 }
\pgfmathsetmacro{\wyyyu}{\wyyyt + \wyyyspacing + 0.0 }
\pgfmathsetmacro{\wyyyv}{\wyyyu + \wyyyspacing + 0.0 }
\pgfmathsetmacro{\wyyyw}{\wyyyv + \wyyyspacing + 0.0 }
\pgfmathsetmacro{\wyyyx}{\wyyyw + \wyyyspacing + 0.0 }
\pgfmathsetmacro{\wyyyy}{\wyyyx + \wyyyspacing + 0.0 }
\pgfmathsetmacro{\wyyyz}{\wyyyy + \wyyyspacing + 0.0 }

\coordinate (wpppaa) at (\wxxxa, \wyyya);
\coordinate (wpppab) at (\wxxxa, \wyyyb);
\coordinate (wpppac) at (\wxxxa, \wyyyc);
\coordinate (wpppad) at (\wxxxa, \wyyyd);
\coordinate (wpppae) at (\wxxxa, \wyyye);
\coordinate (wpppaf) at (\wxxxa, \wyyyf);
\coordinate (wpppag) at (\wxxxa, \wyyyg);
\coordinate (wpppah) at (\wxxxa, \wyyyh);
\coordinate (wpppai) at (\wxxxa, \wyyyi);
\coordinate (wpppaj) at (\wxxxa, \wyyyj);
\coordinate (wpppak) at (\wxxxa, \wyyyk);
\coordinate (wpppal) at (\wxxxa, \wyyyl);
\coordinate (wpppam) at (\wxxxa, \wyyym);
\coordinate (wpppan) at (\wxxxa, \wyyyn);
\coordinate (wpppao) at (\wxxxa, \wyyyo);
\coordinate (wpppap) at (\wxxxa, \wyyyp);
\coordinate (wpppaq) at (\wxxxa, \wyyyq);
\coordinate (wpppar) at (\wxxxa, \wyyyr);
\coordinate (wpppas) at (\wxxxa, \wyyys);
\coordinate (wpppat) at (\wxxxa, \wyyyt);
\coordinate (wpppau) at (\wxxxa, \wyyyu);
\coordinate (wpppav) at (\wxxxa, \wyyyv);
\coordinate (wpppaw) at (\wxxxa, \wyyyw);
\coordinate (wpppax) at (\wxxxa, \wyyyx);
\coordinate (wpppay) at (\wxxxa, \wyyyy);
\coordinate (wpppaz) at (\wxxxa, \wyyyz);
\coordinate (wpppba) at (\wxxxb, \wyyya);
\coordinate (wpppbb) at (\wxxxb, \wyyyb);
\coordinate (wpppbc) at (\wxxxb, \wyyyc);
\coordinate (wpppbd) at (\wxxxb, \wyyyd);
\coordinate (wpppbe) at (\wxxxb, \wyyye);
\coordinate (wpppbf) at (\wxxxb, \wyyyf);
\coordinate (wpppbg) at (\wxxxb, \wyyyg);
\coordinate (wpppbh) at (\wxxxb, \wyyyh);
\coordinate (wpppbi) at (\wxxxb, \wyyyi);
\coordinate (wpppbj) at (\wxxxb, \wyyyj);
\coordinate (wpppbk) at (\wxxxb, \wyyyk);
\coordinate (wpppbl) at (\wxxxb, \wyyyl);
\coordinate (wpppbm) at (\wxxxb, \wyyym);
\coordinate (wpppbn) at (\wxxxb, \wyyyn);
\coordinate (wpppbo) at (\wxxxb, \wyyyo);
\coordinate (wpppbp) at (\wxxxb, \wyyyp);
\coordinate (wpppbq) at (\wxxxb, \wyyyq);
\coordinate (wpppbr) at (\wxxxb, \wyyyr);
\coordinate (wpppbs) at (\wxxxb, \wyyys);
\coordinate (wpppbt) at (\wxxxb, \wyyyt);
\coordinate (wpppbu) at (\wxxxb, \wyyyu);
\coordinate (wpppbv) at (\wxxxb, \wyyyv);
\coordinate (wpppbw) at (\wxxxb, \wyyyw);
\coordinate (wpppbx) at (\wxxxb, \wyyyx);
\coordinate (wpppby) at (\wxxxb, \wyyyy);
\coordinate (wpppbz) at (\wxxxb, \wyyyz);
\coordinate (wpppca) at (\wxxxc, \wyyya);
\coordinate (wpppcb) at (\wxxxc, \wyyyb);
\coordinate (wpppcc) at (\wxxxc, \wyyyc);
\coordinate (wpppcd) at (\wxxxc, \wyyyd);
\coordinate (wpppce) at (\wxxxc, \wyyye);
\coordinate (wpppcf) at (\wxxxc, \wyyyf);
\coordinate (wpppcg) at (\wxxxc, \wyyyg);
\coordinate (wpppch) at (\wxxxc, \wyyyh);
\coordinate (wpppci) at (\wxxxc, \wyyyi);
\coordinate (wpppcj) at (\wxxxc, \wyyyj);
\coordinate (wpppck) at (\wxxxc, \wyyyk);
\coordinate (wpppcl) at (\wxxxc, \wyyyl);
\coordinate (wpppcm) at (\wxxxc, \wyyym);
\coordinate (wpppcn) at (\wxxxc, \wyyyn);
\coordinate (wpppco) at (\wxxxc, \wyyyo);
\coordinate (wpppcp) at (\wxxxc, \wyyyp);
\coordinate (wpppcq) at (\wxxxc, \wyyyq);
\coordinate (wpppcr) at (\wxxxc, \wyyyr);
\coordinate (wpppcs) at (\wxxxc, \wyyys);
\coordinate (wpppct) at (\wxxxc, \wyyyt);
\coordinate (wpppcu) at (\wxxxc, \wyyyu);
\coordinate (wpppcv) at (\wxxxc, \wyyyv);
\coordinate (wpppcw) at (\wxxxc, \wyyyw);
\coordinate (wpppcx) at (\wxxxc, \wyyyx);
\coordinate (wpppcy) at (\wxxxc, \wyyyy);
\coordinate (wpppcz) at (\wxxxc, \wyyyz);
\coordinate (wpppda) at (\wxxxd, \wyyya);
\coordinate (wpppdb) at (\wxxxd, \wyyyb);
\coordinate (wpppdc) at (\wxxxd, \wyyyc);
\coordinate (wpppdd) at (\wxxxd, \wyyyd);
\coordinate (wpppde) at (\wxxxd, \wyyye);
\coordinate (wpppdf) at (\wxxxd, \wyyyf);
\coordinate (wpppdg) at (\wxxxd, \wyyyg);
\coordinate (wpppdh) at (\wxxxd, \wyyyh);
\coordinate (wpppdi) at (\wxxxd, \wyyyi);
\coordinate (wpppdj) at (\wxxxd, \wyyyj);
\coordinate (wpppdk) at (\wxxxd, \wyyyk);
\coordinate (wpppdl) at (\wxxxd, \wyyyl);
\coordinate (wpppdm) at (\wxxxd, \wyyym);
\coordinate (wpppdn) at (\wxxxd, \wyyyn);
\coordinate (wpppdo) at (\wxxxd, \wyyyo);
\coordinate (wpppdp) at (\wxxxd, \wyyyp);
\coordinate (wpppdq) at (\wxxxd, \wyyyq);
\coordinate (wpppdr) at (\wxxxd, \wyyyr);
\coordinate (wpppds) at (\wxxxd, \wyyys);
\coordinate (wpppdt) at (\wxxxd, \wyyyt);
\coordinate (wpppdu) at (\wxxxd, \wyyyu);
\coordinate (wpppdv) at (\wxxxd, \wyyyv);
\coordinate (wpppdw) at (\wxxxd, \wyyyw);
\coordinate (wpppdx) at (\wxxxd, \wyyyx);
\coordinate (wpppdy) at (\wxxxd, \wyyyy);
\coordinate (wpppdz) at (\wxxxd, \wyyyz);
\coordinate (wpppea) at (\wxxxe, \wyyya);
\coordinate (wpppeb) at (\wxxxe, \wyyyb);
\coordinate (wpppec) at (\wxxxe, \wyyyc);
\coordinate (wppped) at (\wxxxe, \wyyyd);
\coordinate (wpppee) at (\wxxxe, \wyyye);
\coordinate (wpppef) at (\wxxxe, \wyyyf);
\coordinate (wpppeg) at (\wxxxe, \wyyyg);
\coordinate (wpppeh) at (\wxxxe, \wyyyh);
\coordinate (wpppei) at (\wxxxe, \wyyyi);
\coordinate (wpppej) at (\wxxxe, \wyyyj);
\coordinate (wpppek) at (\wxxxe, \wyyyk);
\coordinate (wpppel) at (\wxxxe, \wyyyl);
\coordinate (wpppem) at (\wxxxe, \wyyym);
\coordinate (wpppen) at (\wxxxe, \wyyyn);
\coordinate (wpppeo) at (\wxxxe, \wyyyo);
\coordinate (wpppep) at (\wxxxe, \wyyyp);
\coordinate (wpppeq) at (\wxxxe, \wyyyq);
\coordinate (wppper) at (\wxxxe, \wyyyr);
\coordinate (wpppes) at (\wxxxe, \wyyys);
\coordinate (wpppet) at (\wxxxe, \wyyyt);
\coordinate (wpppeu) at (\wxxxe, \wyyyu);
\coordinate (wpppev) at (\wxxxe, \wyyyv);
\coordinate (wpppew) at (\wxxxe, \wyyyw);
\coordinate (wpppex) at (\wxxxe, \wyyyx);
\coordinate (wpppey) at (\wxxxe, \wyyyy);
\coordinate (wpppez) at (\wxxxe, \wyyyz);
\coordinate (wpppfa) at (\wxxxf, \wyyya);
\coordinate (wpppfb) at (\wxxxf, \wyyyb);
\coordinate (wpppfc) at (\wxxxf, \wyyyc);
\coordinate (wpppfd) at (\wxxxf, \wyyyd);
\coordinate (wpppfe) at (\wxxxf, \wyyye);
\coordinate (wpppff) at (\wxxxf, \wyyyf);
\coordinate (wpppfg) at (\wxxxf, \wyyyg);
\coordinate (wpppfh) at (\wxxxf, \wyyyh);
\coordinate (wpppfi) at (\wxxxf, \wyyyi);
\coordinate (wpppfj) at (\wxxxf, \wyyyj);
\coordinate (wpppfk) at (\wxxxf, \wyyyk);
\coordinate (wpppfl) at (\wxxxf, \wyyyl);
\coordinate (wpppfm) at (\wxxxf, \wyyym);
\coordinate (wpppfn) at (\wxxxf, \wyyyn);
\coordinate (wpppfo) at (\wxxxf, \wyyyo);
\coordinate (wpppfp) at (\wxxxf, \wyyyp);
\coordinate (wpppfq) at (\wxxxf, \wyyyq);
\coordinate (wpppfr) at (\wxxxf, \wyyyr);
\coordinate (wpppfs) at (\wxxxf, \wyyys);
\coordinate (wpppft) at (\wxxxf, \wyyyt);
\coordinate (wpppfu) at (\wxxxf, \wyyyu);
\coordinate (wpppfv) at (\wxxxf, \wyyyv);
\coordinate (wpppfw) at (\wxxxf, \wyyyw);
\coordinate (wpppfx) at (\wxxxf, \wyyyx);
\coordinate (wpppfy) at (\wxxxf, \wyyyy);
\coordinate (wpppfz) at (\wxxxf, \wyyyz);
\coordinate (wpppga) at (\wxxxg, \wyyya);
\coordinate (wpppgb) at (\wxxxg, \wyyyb);
\coordinate (wpppgc) at (\wxxxg, \wyyyc);
\coordinate (wpppgd) at (\wxxxg, \wyyyd);
\coordinate (wpppge) at (\wxxxg, \wyyye);
\coordinate (wpppgf) at (\wxxxg, \wyyyf);
\coordinate (wpppgg) at (\wxxxg, \wyyyg);
\coordinate (wpppgh) at (\wxxxg, \wyyyh);
\coordinate (wpppgi) at (\wxxxg, \wyyyi);
\coordinate (wpppgj) at (\wxxxg, \wyyyj);
\coordinate (wpppgk) at (\wxxxg, \wyyyk);
\coordinate (wpppgl) at (\wxxxg, \wyyyl);
\coordinate (wpppgm) at (\wxxxg, \wyyym);
\coordinate (wpppgn) at (\wxxxg, \wyyyn);
\coordinate (wpppgo) at (\wxxxg, \wyyyo);
\coordinate (wpppgp) at (\wxxxg, \wyyyp);
\coordinate (wpppgq) at (\wxxxg, \wyyyq);
\coordinate (wpppgr) at (\wxxxg, \wyyyr);
\coordinate (wpppgs) at (\wxxxg, \wyyys);
\coordinate (wpppgt) at (\wxxxg, \wyyyt);
\coordinate (wpppgu) at (\wxxxg, \wyyyu);
\coordinate (wpppgv) at (\wxxxg, \wyyyv);
\coordinate (wpppgw) at (\wxxxg, \wyyyw);
\coordinate (wpppgx) at (\wxxxg, \wyyyx);
\coordinate (wpppgy) at (\wxxxg, \wyyyy);
\coordinate (wpppgz) at (\wxxxg, \wyyyz);
\coordinate (wpppha) at (\wxxxh, \wyyya);
\coordinate (wppphb) at (\wxxxh, \wyyyb);
\coordinate (wppphc) at (\wxxxh, \wyyyc);
\coordinate (wppphd) at (\wxxxh, \wyyyd);
\coordinate (wppphe) at (\wxxxh, \wyyye);
\coordinate (wppphf) at (\wxxxh, \wyyyf);
\coordinate (wppphg) at (\wxxxh, \wyyyg);
\coordinate (wppphh) at (\wxxxh, \wyyyh);
\coordinate (wppphi) at (\wxxxh, \wyyyi);
\coordinate (wppphj) at (\wxxxh, \wyyyj);
\coordinate (wppphk) at (\wxxxh, \wyyyk);
\coordinate (wppphl) at (\wxxxh, \wyyyl);
\coordinate (wppphm) at (\wxxxh, \wyyym);
\coordinate (wppphn) at (\wxxxh, \wyyyn);
\coordinate (wpppho) at (\wxxxh, \wyyyo);
\coordinate (wppphp) at (\wxxxh, \wyyyp);
\coordinate (wppphq) at (\wxxxh, \wyyyq);
\coordinate (wppphr) at (\wxxxh, \wyyyr);
\coordinate (wppphs) at (\wxxxh, \wyyys);
\coordinate (wpppht) at (\wxxxh, \wyyyt);
\coordinate (wppphu) at (\wxxxh, \wyyyu);
\coordinate (wppphv) at (\wxxxh, \wyyyv);
\coordinate (wppphw) at (\wxxxh, \wyyyw);
\coordinate (wppphx) at (\wxxxh, \wyyyx);
\coordinate (wppphy) at (\wxxxh, \wyyyy);
\coordinate (wppphz) at (\wxxxh, \wyyyz);
\coordinate (wpppia) at (\wxxxi, \wyyya);
\coordinate (wpppib) at (\wxxxi, \wyyyb);
\coordinate (wpppic) at (\wxxxi, \wyyyc);
\coordinate (wpppid) at (\wxxxi, \wyyyd);
\coordinate (wpppie) at (\wxxxi, \wyyye);
\coordinate (wpppif) at (\wxxxi, \wyyyf);
\coordinate (wpppig) at (\wxxxi, \wyyyg);
\coordinate (wpppih) at (\wxxxi, \wyyyh);
\coordinate (wpppii) at (\wxxxi, \wyyyi);
\coordinate (wpppij) at (\wxxxi, \wyyyj);
\coordinate (wpppik) at (\wxxxi, \wyyyk);
\coordinate (wpppil) at (\wxxxi, \wyyyl);
\coordinate (wpppim) at (\wxxxi, \wyyym);
\coordinate (wpppin) at (\wxxxi, \wyyyn);
\coordinate (wpppio) at (\wxxxi, \wyyyo);
\coordinate (wpppip) at (\wxxxi, \wyyyp);
\coordinate (wpppiq) at (\wxxxi, \wyyyq);
\coordinate (wpppir) at (\wxxxi, \wyyyr);
\coordinate (wpppis) at (\wxxxi, \wyyys);
\coordinate (wpppit) at (\wxxxi, \wyyyt);
\coordinate (wpppiu) at (\wxxxi, \wyyyu);
\coordinate (wpppiv) at (\wxxxi, \wyyyv);
\coordinate (wpppiw) at (\wxxxi, \wyyyw);
\coordinate (wpppix) at (\wxxxi, \wyyyx);
\coordinate (wpppiy) at (\wxxxi, \wyyyy);
\coordinate (wpppiz) at (\wxxxi, \wyyyz);
\coordinate (wpppja) at (\wxxxj, \wyyya);
\coordinate (wpppjb) at (\wxxxj, \wyyyb);
\coordinate (wpppjc) at (\wxxxj, \wyyyc);
\coordinate (wpppjd) at (\wxxxj, \wyyyd);
\coordinate (wpppje) at (\wxxxj, \wyyye);
\coordinate (wpppjf) at (\wxxxj, \wyyyf);
\coordinate (wpppjg) at (\wxxxj, \wyyyg);
\coordinate (wpppjh) at (\wxxxj, \wyyyh);
\coordinate (wpppji) at (\wxxxj, \wyyyi);
\coordinate (wpppjj) at (\wxxxj, \wyyyj);
\coordinate (wpppjk) at (\wxxxj, \wyyyk);
\coordinate (wpppjl) at (\wxxxj, \wyyyl);
\coordinate (wpppjm) at (\wxxxj, \wyyym);
\coordinate (wpppjn) at (\wxxxj, \wyyyn);
\coordinate (wpppjo) at (\wxxxj, \wyyyo);
\coordinate (wpppjp) at (\wxxxj, \wyyyp);
\coordinate (wpppjq) at (\wxxxj, \wyyyq);
\coordinate (wpppjr) at (\wxxxj, \wyyyr);
\coordinate (wpppjs) at (\wxxxj, \wyyys);
\coordinate (wpppjt) at (\wxxxj, \wyyyt);
\coordinate (wpppju) at (\wxxxj, \wyyyu);
\coordinate (wpppjv) at (\wxxxj, \wyyyv);
\coordinate (wpppjw) at (\wxxxj, \wyyyw);
\coordinate (wpppjx) at (\wxxxj, \wyyyx);
\coordinate (wpppjy) at (\wxxxj, \wyyyy);
\coordinate (wpppjz) at (\wxxxj, \wyyyz);
\coordinate (wpppka) at (\wxxxk, \wyyya);
\coordinate (wpppkb) at (\wxxxk, \wyyyb);
\coordinate (wpppkc) at (\wxxxk, \wyyyc);
\coordinate (wpppkd) at (\wxxxk, \wyyyd);
\coordinate (wpppke) at (\wxxxk, \wyyye);
\coordinate (wpppkf) at (\wxxxk, \wyyyf);
\coordinate (wpppkg) at (\wxxxk, \wyyyg);
\coordinate (wpppkh) at (\wxxxk, \wyyyh);
\coordinate (wpppki) at (\wxxxk, \wyyyi);
\coordinate (wpppkj) at (\wxxxk, \wyyyj);
\coordinate (wpppkk) at (\wxxxk, \wyyyk);
\coordinate (wpppkl) at (\wxxxk, \wyyyl);
\coordinate (wpppkm) at (\wxxxk, \wyyym);
\coordinate (wpppkn) at (\wxxxk, \wyyyn);
\coordinate (wpppko) at (\wxxxk, \wyyyo);
\coordinate (wpppkp) at (\wxxxk, \wyyyp);
\coordinate (wpppkq) at (\wxxxk, \wyyyq);
\coordinate (wpppkr) at (\wxxxk, \wyyyr);
\coordinate (wpppks) at (\wxxxk, \wyyys);
\coordinate (wpppkt) at (\wxxxk, \wyyyt);
\coordinate (wpppku) at (\wxxxk, \wyyyu);
\coordinate (wpppkv) at (\wxxxk, \wyyyv);
\coordinate (wpppkw) at (\wxxxk, \wyyyw);
\coordinate (wpppkx) at (\wxxxk, \wyyyx);
\coordinate (wpppky) at (\wxxxk, \wyyyy);
\coordinate (wpppkz) at (\wxxxk, \wyyyz);
\coordinate (wpppla) at (\wxxxl, \wyyya);
\coordinate (wppplb) at (\wxxxl, \wyyyb);
\coordinate (wppplc) at (\wxxxl, \wyyyc);
\coordinate (wpppld) at (\wxxxl, \wyyyd);
\coordinate (wppple) at (\wxxxl, \wyyye);
\coordinate (wppplf) at (\wxxxl, \wyyyf);
\coordinate (wppplg) at (\wxxxl, \wyyyg);
\coordinate (wppplh) at (\wxxxl, \wyyyh);
\coordinate (wpppli) at (\wxxxl, \wyyyi);
\coordinate (wppplj) at (\wxxxl, \wyyyj);
\coordinate (wppplk) at (\wxxxl, \wyyyk);
\coordinate (wpppll) at (\wxxxl, \wyyyl);
\coordinate (wppplm) at (\wxxxl, \wyyym);
\coordinate (wpppln) at (\wxxxl, \wyyyn);
\coordinate (wppplo) at (\wxxxl, \wyyyo);
\coordinate (wppplp) at (\wxxxl, \wyyyp);
\coordinate (wppplq) at (\wxxxl, \wyyyq);
\coordinate (wppplr) at (\wxxxl, \wyyyr);
\coordinate (wpppls) at (\wxxxl, \wyyys);
\coordinate (wppplt) at (\wxxxl, \wyyyt);
\coordinate (wppplu) at (\wxxxl, \wyyyu);
\coordinate (wppplv) at (\wxxxl, \wyyyv);
\coordinate (wppplw) at (\wxxxl, \wyyyw);
\coordinate (wppplx) at (\wxxxl, \wyyyx);
\coordinate (wppply) at (\wxxxl, \wyyyy);
\coordinate (wppplz) at (\wxxxl, \wyyyz);
\coordinate (wpppma) at (\wxxxm, \wyyya);
\coordinate (wpppmb) at (\wxxxm, \wyyyb);
\coordinate (wpppmc) at (\wxxxm, \wyyyc);
\coordinate (wpppmd) at (\wxxxm, \wyyyd);
\coordinate (wpppme) at (\wxxxm, \wyyye);
\coordinate (wpppmf) at (\wxxxm, \wyyyf);
\coordinate (wpppmg) at (\wxxxm, \wyyyg);
\coordinate (wpppmh) at (\wxxxm, \wyyyh);
\coordinate (wpppmi) at (\wxxxm, \wyyyi);
\coordinate (wpppmj) at (\wxxxm, \wyyyj);
\coordinate (wpppmk) at (\wxxxm, \wyyyk);
\coordinate (wpppml) at (\wxxxm, \wyyyl);
\coordinate (wpppmm) at (\wxxxm, \wyyym);
\coordinate (wpppmn) at (\wxxxm, \wyyyn);
\coordinate (wpppmo) at (\wxxxm, \wyyyo);
\coordinate (wpppmp) at (\wxxxm, \wyyyp);
\coordinate (wpppmq) at (\wxxxm, \wyyyq);
\coordinate (wpppmr) at (\wxxxm, \wyyyr);
\coordinate (wpppms) at (\wxxxm, \wyyys);
\coordinate (wpppmt) at (\wxxxm, \wyyyt);
\coordinate (wpppmu) at (\wxxxm, \wyyyu);
\coordinate (wpppmv) at (\wxxxm, \wyyyv);
\coordinate (wpppmw) at (\wxxxm, \wyyyw);
\coordinate (wpppmx) at (\wxxxm, \wyyyx);
\coordinate (wpppmy) at (\wxxxm, \wyyyy);
\coordinate (wpppmz) at (\wxxxm, \wyyyz);
\coordinate (wpppna) at (\wxxxn, \wyyya);
\coordinate (wpppnb) at (\wxxxn, \wyyyb);
\coordinate (wpppnc) at (\wxxxn, \wyyyc);
\coordinate (wpppnd) at (\wxxxn, \wyyyd);
\coordinate (wpppne) at (\wxxxn, \wyyye);
\coordinate (wpppnf) at (\wxxxn, \wyyyf);
\coordinate (wpppng) at (\wxxxn, \wyyyg);
\coordinate (wpppnh) at (\wxxxn, \wyyyh);
\coordinate (wpppni) at (\wxxxn, \wyyyi);
\coordinate (wpppnj) at (\wxxxn, \wyyyj);
\coordinate (wpppnk) at (\wxxxn, \wyyyk);
\coordinate (wpppnl) at (\wxxxn, \wyyyl);
\coordinate (wpppnm) at (\wxxxn, \wyyym);
\coordinate (wpppnn) at (\wxxxn, \wyyyn);
\coordinate (wpppno) at (\wxxxn, \wyyyo);
\coordinate (wpppnp) at (\wxxxn, \wyyyp);
\coordinate (wpppnq) at (\wxxxn, \wyyyq);
\coordinate (wpppnr) at (\wxxxn, \wyyyr);
\coordinate (wpppns) at (\wxxxn, \wyyys);
\coordinate (wpppnt) at (\wxxxn, \wyyyt);
\coordinate (wpppnu) at (\wxxxn, \wyyyu);
\coordinate (wpppnv) at (\wxxxn, \wyyyv);
\coordinate (wpppnw) at (\wxxxn, \wyyyw);
\coordinate (wpppnx) at (\wxxxn, \wyyyx);
\coordinate (wpppny) at (\wxxxn, \wyyyy);
\coordinate (wpppnz) at (\wxxxn, \wyyyz);
\coordinate (wpppoa) at (\wxxxo, \wyyya);
\coordinate (wpppob) at (\wxxxo, \wyyyb);
\coordinate (wpppoc) at (\wxxxo, \wyyyc);
\coordinate (wpppod) at (\wxxxo, \wyyyd);
\coordinate (wpppoe) at (\wxxxo, \wyyye);
\coordinate (wpppof) at (\wxxxo, \wyyyf);
\coordinate (wpppog) at (\wxxxo, \wyyyg);
\coordinate (wpppoh) at (\wxxxo, \wyyyh);
\coordinate (wpppoi) at (\wxxxo, \wyyyi);
\coordinate (wpppoj) at (\wxxxo, \wyyyj);
\coordinate (wpppok) at (\wxxxo, \wyyyk);
\coordinate (wpppol) at (\wxxxo, \wyyyl);
\coordinate (wpppom) at (\wxxxo, \wyyym);
\coordinate (wpppon) at (\wxxxo, \wyyyn);
\coordinate (wpppoo) at (\wxxxo, \wyyyo);
\coordinate (wpppop) at (\wxxxo, \wyyyp);
\coordinate (wpppoq) at (\wxxxo, \wyyyq);
\coordinate (wpppor) at (\wxxxo, \wyyyr);
\coordinate (wpppos) at (\wxxxo, \wyyys);
\coordinate (wpppot) at (\wxxxo, \wyyyt);
\coordinate (wpppou) at (\wxxxo, \wyyyu);
\coordinate (wpppov) at (\wxxxo, \wyyyv);
\coordinate (wpppow) at (\wxxxo, \wyyyw);
\coordinate (wpppox) at (\wxxxo, \wyyyx);
\coordinate (wpppoy) at (\wxxxo, \wyyyy);
\coordinate (wpppoz) at (\wxxxo, \wyyyz);
\coordinate (wppppa) at (\wxxxp, \wyyya);
\coordinate (wppppb) at (\wxxxp, \wyyyb);
\coordinate (wppppc) at (\wxxxp, \wyyyc);
\coordinate (wppppd) at (\wxxxp, \wyyyd);
\coordinate (wppppe) at (\wxxxp, \wyyye);
\coordinate (wppppf) at (\wxxxp, \wyyyf);
\coordinate (wppppg) at (\wxxxp, \wyyyg);
\coordinate (wpppph) at (\wxxxp, \wyyyh);
\coordinate (wppppi) at (\wxxxp, \wyyyi);
\coordinate (wppppj) at (\wxxxp, \wyyyj);
\coordinate (wppppk) at (\wxxxp, \wyyyk);
\coordinate (wppppl) at (\wxxxp, \wyyyl);
\coordinate (wppppm) at (\wxxxp, \wyyym);
\coordinate (wppppn) at (\wxxxp, \wyyyn);
\coordinate (wppppo) at (\wxxxp, \wyyyo);
\coordinate (wppppp) at (\wxxxp, \wyyyp);
\coordinate (wppppq) at (\wxxxp, \wyyyq);
\coordinate (wppppr) at (\wxxxp, \wyyyr);
\coordinate (wpppps) at (\wxxxp, \wyyys);
\coordinate (wppppt) at (\wxxxp, \wyyyt);
\coordinate (wppppu) at (\wxxxp, \wyyyu);
\coordinate (wppppv) at (\wxxxp, \wyyyv);
\coordinate (wppppw) at (\wxxxp, \wyyyw);
\coordinate (wppppx) at (\wxxxp, \wyyyx);
\coordinate (wppppy) at (\wxxxp, \wyyyy);
\coordinate (wppppz) at (\wxxxp, \wyyyz);
\coordinate (wpppqa) at (\wxxxq, \wyyya);
\coordinate (wpppqb) at (\wxxxq, \wyyyb);
\coordinate (wpppqc) at (\wxxxq, \wyyyc);
\coordinate (wpppqd) at (\wxxxq, \wyyyd);
\coordinate (wpppqe) at (\wxxxq, \wyyye);
\coordinate (wpppqf) at (\wxxxq, \wyyyf);
\coordinate (wpppqg) at (\wxxxq, \wyyyg);
\coordinate (wpppqh) at (\wxxxq, \wyyyh);
\coordinate (wpppqi) at (\wxxxq, \wyyyi);
\coordinate (wpppqj) at (\wxxxq, \wyyyj);
\coordinate (wpppqk) at (\wxxxq, \wyyyk);
\coordinate (wpppql) at (\wxxxq, \wyyyl);
\coordinate (wpppqm) at (\wxxxq, \wyyym);
\coordinate (wpppqn) at (\wxxxq, \wyyyn);
\coordinate (wpppqo) at (\wxxxq, \wyyyo);
\coordinate (wpppqp) at (\wxxxq, \wyyyp);
\coordinate (wpppqq) at (\wxxxq, \wyyyq);
\coordinate (wpppqr) at (\wxxxq, \wyyyr);
\coordinate (wpppqs) at (\wxxxq, \wyyys);
\coordinate (wpppqt) at (\wxxxq, \wyyyt);
\coordinate (wpppqu) at (\wxxxq, \wyyyu);
\coordinate (wpppqv) at (\wxxxq, \wyyyv);
\coordinate (wpppqw) at (\wxxxq, \wyyyw);
\coordinate (wpppqx) at (\wxxxq, \wyyyx);
\coordinate (wpppqy) at (\wxxxq, \wyyyy);
\coordinate (wpppqz) at (\wxxxq, \wyyyz);
\coordinate (wpppra) at (\wxxxr, \wyyya);
\coordinate (wppprb) at (\wxxxr, \wyyyb);
\coordinate (wppprc) at (\wxxxr, \wyyyc);
\coordinate (wppprd) at (\wxxxr, \wyyyd);
\coordinate (wpppre) at (\wxxxr, \wyyye);
\coordinate (wppprf) at (\wxxxr, \wyyyf);
\coordinate (wppprg) at (\wxxxr, \wyyyg);
\coordinate (wppprh) at (\wxxxr, \wyyyh);
\coordinate (wpppri) at (\wxxxr, \wyyyi);
\coordinate (wppprj) at (\wxxxr, \wyyyj);
\coordinate (wppprk) at (\wxxxr, \wyyyk);
\coordinate (wppprl) at (\wxxxr, \wyyyl);
\coordinate (wppprm) at (\wxxxr, \wyyym);
\coordinate (wppprn) at (\wxxxr, \wyyyn);
\coordinate (wpppro) at (\wxxxr, \wyyyo);
\coordinate (wppprp) at (\wxxxr, \wyyyp);
\coordinate (wppprq) at (\wxxxr, \wyyyq);
\coordinate (wppprr) at (\wxxxr, \wyyyr);
\coordinate (wppprs) at (\wxxxr, \wyyys);
\coordinate (wppprt) at (\wxxxr, \wyyyt);
\coordinate (wpppru) at (\wxxxr, \wyyyu);
\coordinate (wppprv) at (\wxxxr, \wyyyv);
\coordinate (wppprw) at (\wxxxr, \wyyyw);
\coordinate (wppprx) at (\wxxxr, \wyyyx);
\coordinate (wpppry) at (\wxxxr, \wyyyy);
\coordinate (wppprz) at (\wxxxr, \wyyyz);
\coordinate (wpppsa) at (\wxxxs, \wyyya);
\coordinate (wpppsb) at (\wxxxs, \wyyyb);
\coordinate (wpppsc) at (\wxxxs, \wyyyc);
\coordinate (wpppsd) at (\wxxxs, \wyyyd);
\coordinate (wpppse) at (\wxxxs, \wyyye);
\coordinate (wpppsf) at (\wxxxs, \wyyyf);
\coordinate (wpppsg) at (\wxxxs, \wyyyg);
\coordinate (wpppsh) at (\wxxxs, \wyyyh);
\coordinate (wpppsi) at (\wxxxs, \wyyyi);
\coordinate (wpppsj) at (\wxxxs, \wyyyj);
\coordinate (wpppsk) at (\wxxxs, \wyyyk);
\coordinate (wpppsl) at (\wxxxs, \wyyyl);
\coordinate (wpppsm) at (\wxxxs, \wyyym);
\coordinate (wpppsn) at (\wxxxs, \wyyyn);
\coordinate (wpppso) at (\wxxxs, \wyyyo);
\coordinate (wpppsp) at (\wxxxs, \wyyyp);
\coordinate (wpppsq) at (\wxxxs, \wyyyq);
\coordinate (wpppsr) at (\wxxxs, \wyyyr);
\coordinate (wpppss) at (\wxxxs, \wyyys);
\coordinate (wpppst) at (\wxxxs, \wyyyt);
\coordinate (wpppsu) at (\wxxxs, \wyyyu);
\coordinate (wpppsv) at (\wxxxs, \wyyyv);
\coordinate (wpppsw) at (\wxxxs, \wyyyw);
\coordinate (wpppsx) at (\wxxxs, \wyyyx);
\coordinate (wpppsy) at (\wxxxs, \wyyyy);
\coordinate (wpppsz) at (\wxxxs, \wyyyz);
\coordinate (wpppta) at (\wxxxt, \wyyya);
\coordinate (wppptb) at (\wxxxt, \wyyyb);
\coordinate (wppptc) at (\wxxxt, \wyyyc);
\coordinate (wppptd) at (\wxxxt, \wyyyd);
\coordinate (wpppte) at (\wxxxt, \wyyye);
\coordinate (wppptf) at (\wxxxt, \wyyyf);
\coordinate (wppptg) at (\wxxxt, \wyyyg);
\coordinate (wpppth) at (\wxxxt, \wyyyh);
\coordinate (wpppti) at (\wxxxt, \wyyyi);
\coordinate (wppptj) at (\wxxxt, \wyyyj);
\coordinate (wppptk) at (\wxxxt, \wyyyk);
\coordinate (wppptl) at (\wxxxt, \wyyyl);
\coordinate (wppptm) at (\wxxxt, \wyyym);
\coordinate (wppptn) at (\wxxxt, \wyyyn);
\coordinate (wpppto) at (\wxxxt, \wyyyo);
\coordinate (wppptp) at (\wxxxt, \wyyyp);
\coordinate (wppptq) at (\wxxxt, \wyyyq);
\coordinate (wppptr) at (\wxxxt, \wyyyr);
\coordinate (wpppts) at (\wxxxt, \wyyys);
\coordinate (wppptt) at (\wxxxt, \wyyyt);
\coordinate (wppptu) at (\wxxxt, \wyyyu);
\coordinate (wppptv) at (\wxxxt, \wyyyv);
\coordinate (wppptw) at (\wxxxt, \wyyyw);
\coordinate (wppptx) at (\wxxxt, \wyyyx);
\coordinate (wpppty) at (\wxxxt, \wyyyy);
\coordinate (wppptz) at (\wxxxt, \wyyyz);
\coordinate (wpppua) at (\wxxxu, \wyyya);
\coordinate (wpppub) at (\wxxxu, \wyyyb);
\coordinate (wpppuc) at (\wxxxu, \wyyyc);
\coordinate (wpppud) at (\wxxxu, \wyyyd);
\coordinate (wpppue) at (\wxxxu, \wyyye);
\coordinate (wpppuf) at (\wxxxu, \wyyyf);
\coordinate (wpppug) at (\wxxxu, \wyyyg);
\coordinate (wpppuh) at (\wxxxu, \wyyyh);
\coordinate (wpppui) at (\wxxxu, \wyyyi);
\coordinate (wpppuj) at (\wxxxu, \wyyyj);
\coordinate (wpppuk) at (\wxxxu, \wyyyk);
\coordinate (wpppul) at (\wxxxu, \wyyyl);
\coordinate (wpppum) at (\wxxxu, \wyyym);
\coordinate (wpppun) at (\wxxxu, \wyyyn);
\coordinate (wpppuo) at (\wxxxu, \wyyyo);
\coordinate (wpppup) at (\wxxxu, \wyyyp);
\coordinate (wpppuq) at (\wxxxu, \wyyyq);
\coordinate (wpppur) at (\wxxxu, \wyyyr);
\coordinate (wpppus) at (\wxxxu, \wyyys);
\coordinate (wppput) at (\wxxxu, \wyyyt);
\coordinate (wpppuu) at (\wxxxu, \wyyyu);
\coordinate (wpppuv) at (\wxxxu, \wyyyv);
\coordinate (wpppuw) at (\wxxxu, \wyyyw);
\coordinate (wpppux) at (\wxxxu, \wyyyx);
\coordinate (wpppuy) at (\wxxxu, \wyyyy);
\coordinate (wpppuz) at (\wxxxu, \wyyyz);
\coordinate (wpppva) at (\wxxxv, \wyyya);
\coordinate (wpppvb) at (\wxxxv, \wyyyb);
\coordinate (wpppvc) at (\wxxxv, \wyyyc);
\coordinate (wpppvd) at (\wxxxv, \wyyyd);
\coordinate (wpppve) at (\wxxxv, \wyyye);
\coordinate (wpppvf) at (\wxxxv, \wyyyf);
\coordinate (wpppvg) at (\wxxxv, \wyyyg);
\coordinate (wpppvh) at (\wxxxv, \wyyyh);
\coordinate (wpppvi) at (\wxxxv, \wyyyi);
\coordinate (wpppvj) at (\wxxxv, \wyyyj);
\coordinate (wpppvk) at (\wxxxv, \wyyyk);
\coordinate (wpppvl) at (\wxxxv, \wyyyl);
\coordinate (wpppvm) at (\wxxxv, \wyyym);
\coordinate (wpppvn) at (\wxxxv, \wyyyn);
\coordinate (wpppvo) at (\wxxxv, \wyyyo);
\coordinate (wpppvp) at (\wxxxv, \wyyyp);
\coordinate (wpppvq) at (\wxxxv, \wyyyq);
\coordinate (wpppvr) at (\wxxxv, \wyyyr);
\coordinate (wpppvs) at (\wxxxv, \wyyys);
\coordinate (wpppvt) at (\wxxxv, \wyyyt);
\coordinate (wpppvu) at (\wxxxv, \wyyyu);
\coordinate (wpppvv) at (\wxxxv, \wyyyv);
\coordinate (wpppvw) at (\wxxxv, \wyyyw);
\coordinate (wpppvx) at (\wxxxv, \wyyyx);
\coordinate (wpppvy) at (\wxxxv, \wyyyy);
\coordinate (wpppvz) at (\wxxxv, \wyyyz);
\coordinate (wpppwa) at (\wxxxw, \wyyya);
\coordinate (wpppwb) at (\wxxxw, \wyyyb);
\coordinate (wpppwc) at (\wxxxw, \wyyyc);
\coordinate (wpppwd) at (\wxxxw, \wyyyd);
\coordinate (wpppwe) at (\wxxxw, \wyyye);
\coordinate (wpppwf) at (\wxxxw, \wyyyf);
\coordinate (wpppwg) at (\wxxxw, \wyyyg);
\coordinate (wpppwh) at (\wxxxw, \wyyyh);
\coordinate (wpppwi) at (\wxxxw, \wyyyi);
\coordinate (wpppwj) at (\wxxxw, \wyyyj);
\coordinate (wpppwk) at (\wxxxw, \wyyyk);
\coordinate (wpppwl) at (\wxxxw, \wyyyl);
\coordinate (wpppwm) at (\wxxxw, \wyyym);
\coordinate (wpppwn) at (\wxxxw, \wyyyn);
\coordinate (wpppwo) at (\wxxxw, \wyyyo);
\coordinate (wpppwp) at (\wxxxw, \wyyyp);
\coordinate (wpppwq) at (\wxxxw, \wyyyq);
\coordinate (wpppwr) at (\wxxxw, \wyyyr);
\coordinate (wpppws) at (\wxxxw, \wyyys);
\coordinate (wpppwt) at (\wxxxw, \wyyyt);
\coordinate (wpppwu) at (\wxxxw, \wyyyu);
\coordinate (wpppwv) at (\wxxxw, \wyyyv);
\coordinate (wpppww) at (\wxxxw, \wyyyw);
\coordinate (wpppwx) at (\wxxxw, \wyyyx);
\coordinate (wpppwy) at (\wxxxw, \wyyyy);
\coordinate (wpppwz) at (\wxxxw, \wyyyz);
\coordinate (wpppxa) at (\wxxxx, \wyyya);
\coordinate (wpppxb) at (\wxxxx, \wyyyb);
\coordinate (wpppxc) at (\wxxxx, \wyyyc);
\coordinate (wpppxd) at (\wxxxx, \wyyyd);
\coordinate (wpppxe) at (\wxxxx, \wyyye);
\coordinate (wpppxf) at (\wxxxx, \wyyyf);
\coordinate (wpppxg) at (\wxxxx, \wyyyg);
\coordinate (wpppxh) at (\wxxxx, \wyyyh);
\coordinate (wpppxi) at (\wxxxx, \wyyyi);
\coordinate (wpppxj) at (\wxxxx, \wyyyj);
\coordinate (wpppxk) at (\wxxxx, \wyyyk);
\coordinate (wpppxl) at (\wxxxx, \wyyyl);
\coordinate (wpppxm) at (\wxxxx, \wyyym);
\coordinate (wpppxn) at (\wxxxx, \wyyyn);
\coordinate (wpppxo) at (\wxxxx, \wyyyo);
\coordinate (wpppxp) at (\wxxxx, \wyyyp);
\coordinate (wpppxq) at (\wxxxx, \wyyyq);
\coordinate (wpppxr) at (\wxxxx, \wyyyr);
\coordinate (wpppxs) at (\wxxxx, \wyyys);
\coordinate (wpppxt) at (\wxxxx, \wyyyt);
\coordinate (wpppxu) at (\wxxxx, \wyyyu);
\coordinate (wpppxv) at (\wxxxx, \wyyyv);
\coordinate (wpppxw) at (\wxxxx, \wyyyw);
\coordinate (wpppxx) at (\wxxxx, \wyyyx);
\coordinate (wpppxy) at (\wxxxx, \wyyyy);
\coordinate (wpppxz) at (\wxxxx, \wyyyz);
\coordinate (wpppya) at (\wxxxy, \wyyya);
\coordinate (wpppyb) at (\wxxxy, \wyyyb);
\coordinate (wpppyc) at (\wxxxy, \wyyyc);
\coordinate (wpppyd) at (\wxxxy, \wyyyd);
\coordinate (wpppye) at (\wxxxy, \wyyye);
\coordinate (wpppyf) at (\wxxxy, \wyyyf);
\coordinate (wpppyg) at (\wxxxy, \wyyyg);
\coordinate (wpppyh) at (\wxxxy, \wyyyh);
\coordinate (wpppyi) at (\wxxxy, \wyyyi);
\coordinate (wpppyj) at (\wxxxy, \wyyyj);
\coordinate (wpppyk) at (\wxxxy, \wyyyk);
\coordinate (wpppyl) at (\wxxxy, \wyyyl);
\coordinate (wpppym) at (\wxxxy, \wyyym);
\coordinate (wpppyn) at (\wxxxy, \wyyyn);
\coordinate (wpppyo) at (\wxxxy, \wyyyo);
\coordinate (wpppyp) at (\wxxxy, \wyyyp);
\coordinate (wpppyq) at (\wxxxy, \wyyyq);
\coordinate (wpppyr) at (\wxxxy, \wyyyr);
\coordinate (wpppys) at (\wxxxy, \wyyys);
\coordinate (wpppyt) at (\wxxxy, \wyyyt);
\coordinate (wpppyu) at (\wxxxy, \wyyyu);
\coordinate (wpppyv) at (\wxxxy, \wyyyv);
\coordinate (wpppyw) at (\wxxxy, \wyyyw);
\coordinate (wpppyx) at (\wxxxy, \wyyyx);
\coordinate (wpppyy) at (\wxxxy, \wyyyy);
\coordinate (wpppyz) at (\wxxxy, \wyyyz);
\coordinate (wpppza) at (\wxxxz, \wyyya);
\coordinate (wpppzb) at (\wxxxz, \wyyyb);
\coordinate (wpppzc) at (\wxxxz, \wyyyc);
\coordinate (wpppzd) at (\wxxxz, \wyyyd);
\coordinate (wpppze) at (\wxxxz, \wyyye);
\coordinate (wpppzf) at (\wxxxz, \wyyyf);
\coordinate (wpppzg) at (\wxxxz, \wyyyg);
\coordinate (wpppzh) at (\wxxxz, \wyyyh);
\coordinate (wpppzi) at (\wxxxz, \wyyyi);
\coordinate (wpppzj) at (\wxxxz, \wyyyj);
\coordinate (wpppzk) at (\wxxxz, \wyyyk);
\coordinate (wpppzl) at (\wxxxz, \wyyyl);
\coordinate (wpppzm) at (\wxxxz, \wyyym);
\coordinate (wpppzn) at (\wxxxz, \wyyyn);
\coordinate (wpppzo) at (\wxxxz, \wyyyo);
\coordinate (wpppzp) at (\wxxxz, \wyyyp);
\coordinate (wpppzq) at (\wxxxz, \wyyyq);
\coordinate (wpppzr) at (\wxxxz, \wyyyr);
\coordinate (wpppzs) at (\wxxxz, \wyyys);
\coordinate (wpppzt) at (\wxxxz, \wyyyt);
\coordinate (wpppzu) at (\wxxxz, \wyyyu);
\coordinate (wpppzv) at (\wxxxz, \wyyyv);
\coordinate (wpppzw) at (\wxxxz, \wyyyw);
\coordinate (wpppzx) at (\wxxxz, \wyyyx);
\coordinate (wpppzy) at (\wxxxz, \wyyyy);
\coordinate (wpppzz) at (\wxxxz, \wyyyz);

%\gangprintcoordinateat{(0,0)}{The last coordinate values: }{($(wpppzz)$)}; 

 
\coordinatebackground{w}{c}{d}{o};
\end{circuitikz}
















\newpage

\vspace{2cm}

{\Large Figure 3, a circuit is drawn (in five steps actually) with the help of the coordinate system ``demobygangliu".}

\begin{circuitikz}[scale=1]


% Circuits can be drawn by the following five major steps, as shown in the following example. 

% Step 1, preparations. 

% "Install" the coordinate system with keyword ``demobygangliu".
\pgfmathsetmacro{\totaldemobygangliuxxx}{26}
\pgfmathsetmacro{\totaldemobygangliuyyy}{26}
\pgfmathsetmacro{\demobygangliuxxxspacing}{1}
\pgfmathsetmacro{\demobygangliuyyyspacing}{1}
\pgfmathsetmacro{\demobygangliuxxxa}{-8}
\pgfmathsetmacro{\demobygangliuyyya}{-8}

\pgfmathsetmacro{\demobygangliuxxxb}{\demobygangliuxxxa + \demobygangliuxxxspacing + 0.0 }
\pgfmathsetmacro{\demobygangliuxxxc}{\demobygangliuxxxb + \demobygangliuxxxspacing + 0.0 }
\pgfmathsetmacro{\demobygangliuxxxd}{\demobygangliuxxxc + \demobygangliuxxxspacing + 0.0 }
\pgfmathsetmacro{\demobygangliuxxxe}{\demobygangliuxxxd + \demobygangliuxxxspacing + 0.0 }
\pgfmathsetmacro{\demobygangliuxxxf}{\demobygangliuxxxe + \demobygangliuxxxspacing + 0.0 }
\pgfmathsetmacro{\demobygangliuxxxg}{\demobygangliuxxxf + \demobygangliuxxxspacing + 0.0 }
\pgfmathsetmacro{\demobygangliuxxxh}{\demobygangliuxxxg + \demobygangliuxxxspacing + 0.0 }
\pgfmathsetmacro{\demobygangliuxxxi}{\demobygangliuxxxh + \demobygangliuxxxspacing + 0.0 }
\pgfmathsetmacro{\demobygangliuxxxj}{\demobygangliuxxxi + \demobygangliuxxxspacing + 0.0 }
\pgfmathsetmacro{\demobygangliuxxxk}{\demobygangliuxxxj + \demobygangliuxxxspacing + 0.0 }
\pgfmathsetmacro{\demobygangliuxxxl}{\demobygangliuxxxk + \demobygangliuxxxspacing + 0.0 }
\pgfmathsetmacro{\demobygangliuxxxm}{\demobygangliuxxxl + \demobygangliuxxxspacing + 0.0 }
\pgfmathsetmacro{\demobygangliuxxxn}{\demobygangliuxxxm + \demobygangliuxxxspacing + 0.0 }
\pgfmathsetmacro{\demobygangliuxxxo}{\demobygangliuxxxn + \demobygangliuxxxspacing + 0.0 }
\pgfmathsetmacro{\demobygangliuxxxp}{\demobygangliuxxxo + \demobygangliuxxxspacing + 0.0 }
\pgfmathsetmacro{\demobygangliuxxxq}{\demobygangliuxxxp + \demobygangliuxxxspacing + 0.0 }
\pgfmathsetmacro{\demobygangliuxxxr}{\demobygangliuxxxq + \demobygangliuxxxspacing + 0.0 }
\pgfmathsetmacro{\demobygangliuxxxs}{\demobygangliuxxxr + \demobygangliuxxxspacing + 0.0 }
\pgfmathsetmacro{\demobygangliuxxxt}{\demobygangliuxxxs + \demobygangliuxxxspacing + 0.0 }
\pgfmathsetmacro{\demobygangliuxxxu}{\demobygangliuxxxt + \demobygangliuxxxspacing + 0.0 }
\pgfmathsetmacro{\demobygangliuxxxv}{\demobygangliuxxxu + \demobygangliuxxxspacing + 0.0 }
\pgfmathsetmacro{\demobygangliuxxxw}{\demobygangliuxxxv + \demobygangliuxxxspacing + 0.0 }
\pgfmathsetmacro{\demobygangliuxxxx}{\demobygangliuxxxw + \demobygangliuxxxspacing + 0.0 }
\pgfmathsetmacro{\demobygangliuxxxy}{\demobygangliuxxxx + \demobygangliuxxxspacing + 0.0 }
\pgfmathsetmacro{\demobygangliuxxxz}{\demobygangliuxxxy + \demobygangliuxxxspacing + 0.0 }

\pgfmathsetmacro{\demobygangliuyyyb}{\demobygangliuyyya + \demobygangliuyyyspacing + 0.0 }
\pgfmathsetmacro{\demobygangliuyyyc}{\demobygangliuyyyb + \demobygangliuyyyspacing + 0.0 }
\pgfmathsetmacro{\demobygangliuyyyd}{\demobygangliuyyyc + \demobygangliuyyyspacing + 0.0 }
\pgfmathsetmacro{\demobygangliuyyye}{\demobygangliuyyyd + \demobygangliuyyyspacing + 0.0 }
\pgfmathsetmacro{\demobygangliuyyyf}{\demobygangliuyyye + \demobygangliuyyyspacing + 0.0 }
\pgfmathsetmacro{\demobygangliuyyyg}{\demobygangliuyyyf + \demobygangliuyyyspacing + 0.0 }
\pgfmathsetmacro{\demobygangliuyyyh}{\demobygangliuyyyg + \demobygangliuyyyspacing + 0.0 }
\pgfmathsetmacro{\demobygangliuyyyi}{\demobygangliuyyyh + \demobygangliuyyyspacing + 0.0 }
\pgfmathsetmacro{\demobygangliuyyyj}{\demobygangliuyyyi + \demobygangliuyyyspacing + 0.0 }
\pgfmathsetmacro{\demobygangliuyyyk}{\demobygangliuyyyj + \demobygangliuyyyspacing + 0.0 }
\pgfmathsetmacro{\demobygangliuyyyl}{\demobygangliuyyyk + \demobygangliuyyyspacing + 0.0 }
\pgfmathsetmacro{\demobygangliuyyym}{\demobygangliuyyyl + \demobygangliuyyyspacing + 0.0 }
\pgfmathsetmacro{\demobygangliuyyyn}{\demobygangliuyyym + \demobygangliuyyyspacing + 0.0 }
\pgfmathsetmacro{\demobygangliuyyyo}{\demobygangliuyyyn + \demobygangliuyyyspacing + 0.0 }
\pgfmathsetmacro{\demobygangliuyyyp}{\demobygangliuyyyo + \demobygangliuyyyspacing + 0.0 }
\pgfmathsetmacro{\demobygangliuyyyq}{\demobygangliuyyyp + \demobygangliuyyyspacing + 0.0 }
\pgfmathsetmacro{\demobygangliuyyyr}{\demobygangliuyyyq + \demobygangliuyyyspacing + 0.0 }
\pgfmathsetmacro{\demobygangliuyyys}{\demobygangliuyyyr + \demobygangliuyyyspacing + 0.0 }
\pgfmathsetmacro{\demobygangliuyyyt}{\demobygangliuyyys + \demobygangliuyyyspacing + 0.0 }
\pgfmathsetmacro{\demobygangliuyyyu}{\demobygangliuyyyt + \demobygangliuyyyspacing + 0.0 }
\pgfmathsetmacro{\demobygangliuyyyv}{\demobygangliuyyyu + \demobygangliuyyyspacing + 0.0 }
\pgfmathsetmacro{\demobygangliuyyyw}{\demobygangliuyyyv + \demobygangliuyyyspacing + 0.0 }
\pgfmathsetmacro{\demobygangliuyyyx}{\demobygangliuyyyw + \demobygangliuyyyspacing + 0.0 }
\pgfmathsetmacro{\demobygangliuyyyy}{\demobygangliuyyyx + \demobygangliuyyyspacing + 0.0 }
\pgfmathsetmacro{\demobygangliuyyyz}{\demobygangliuyyyy + \demobygangliuyyyspacing + 0.0 }

\coordinate (demobygangliupppaa) at (\demobygangliuxxxa, \demobygangliuyyya);
\coordinate (demobygangliupppab) at (\demobygangliuxxxa, \demobygangliuyyyb);
\coordinate (demobygangliupppac) at (\demobygangliuxxxa, \demobygangliuyyyc);
\coordinate (demobygangliupppad) at (\demobygangliuxxxa, \demobygangliuyyyd);
\coordinate (demobygangliupppae) at (\demobygangliuxxxa, \demobygangliuyyye);
\coordinate (demobygangliupppaf) at (\demobygangliuxxxa, \demobygangliuyyyf);
\coordinate (demobygangliupppag) at (\demobygangliuxxxa, \demobygangliuyyyg);
\coordinate (demobygangliupppah) at (\demobygangliuxxxa, \demobygangliuyyyh);
\coordinate (demobygangliupppai) at (\demobygangliuxxxa, \demobygangliuyyyi);
\coordinate (demobygangliupppaj) at (\demobygangliuxxxa, \demobygangliuyyyj);
\coordinate (demobygangliupppak) at (\demobygangliuxxxa, \demobygangliuyyyk);
\coordinate (demobygangliupppal) at (\demobygangliuxxxa, \demobygangliuyyyl);
\coordinate (demobygangliupppam) at (\demobygangliuxxxa, \demobygangliuyyym);
\coordinate (demobygangliupppan) at (\demobygangliuxxxa, \demobygangliuyyyn);
\coordinate (demobygangliupppao) at (\demobygangliuxxxa, \demobygangliuyyyo);
\coordinate (demobygangliupppap) at (\demobygangliuxxxa, \demobygangliuyyyp);
\coordinate (demobygangliupppaq) at (\demobygangliuxxxa, \demobygangliuyyyq);
\coordinate (demobygangliupppar) at (\demobygangliuxxxa, \demobygangliuyyyr);
\coordinate (demobygangliupppas) at (\demobygangliuxxxa, \demobygangliuyyys);
\coordinate (demobygangliupppat) at (\demobygangliuxxxa, \demobygangliuyyyt);
\coordinate (demobygangliupppau) at (\demobygangliuxxxa, \demobygangliuyyyu);
\coordinate (demobygangliupppav) at (\demobygangliuxxxa, \demobygangliuyyyv);
\coordinate (demobygangliupppaw) at (\demobygangliuxxxa, \demobygangliuyyyw);
\coordinate (demobygangliupppax) at (\demobygangliuxxxa, \demobygangliuyyyx);
\coordinate (demobygangliupppay) at (\demobygangliuxxxa, \demobygangliuyyyy);
\coordinate (demobygangliupppaz) at (\demobygangliuxxxa, \demobygangliuyyyz);
\coordinate (demobygangliupppba) at (\demobygangliuxxxb, \demobygangliuyyya);
\coordinate (demobygangliupppbb) at (\demobygangliuxxxb, \demobygangliuyyyb);
\coordinate (demobygangliupppbc) at (\demobygangliuxxxb, \demobygangliuyyyc);
\coordinate (demobygangliupppbd) at (\demobygangliuxxxb, \demobygangliuyyyd);
\coordinate (demobygangliupppbe) at (\demobygangliuxxxb, \demobygangliuyyye);
\coordinate (demobygangliupppbf) at (\demobygangliuxxxb, \demobygangliuyyyf);
\coordinate (demobygangliupppbg) at (\demobygangliuxxxb, \demobygangliuyyyg);
\coordinate (demobygangliupppbh) at (\demobygangliuxxxb, \demobygangliuyyyh);
\coordinate (demobygangliupppbi) at (\demobygangliuxxxb, \demobygangliuyyyi);
\coordinate (demobygangliupppbj) at (\demobygangliuxxxb, \demobygangliuyyyj);
\coordinate (demobygangliupppbk) at (\demobygangliuxxxb, \demobygangliuyyyk);
\coordinate (demobygangliupppbl) at (\demobygangliuxxxb, \demobygangliuyyyl);
\coordinate (demobygangliupppbm) at (\demobygangliuxxxb, \demobygangliuyyym);
\coordinate (demobygangliupppbn) at (\demobygangliuxxxb, \demobygangliuyyyn);
\coordinate (demobygangliupppbo) at (\demobygangliuxxxb, \demobygangliuyyyo);
\coordinate (demobygangliupppbp) at (\demobygangliuxxxb, \demobygangliuyyyp);
\coordinate (demobygangliupppbq) at (\demobygangliuxxxb, \demobygangliuyyyq);
\coordinate (demobygangliupppbr) at (\demobygangliuxxxb, \demobygangliuyyyr);
\coordinate (demobygangliupppbs) at (\demobygangliuxxxb, \demobygangliuyyys);
\coordinate (demobygangliupppbt) at (\demobygangliuxxxb, \demobygangliuyyyt);
\coordinate (demobygangliupppbu) at (\demobygangliuxxxb, \demobygangliuyyyu);
\coordinate (demobygangliupppbv) at (\demobygangliuxxxb, \demobygangliuyyyv);
\coordinate (demobygangliupppbw) at (\demobygangliuxxxb, \demobygangliuyyyw);
\coordinate (demobygangliupppbx) at (\demobygangliuxxxb, \demobygangliuyyyx);
\coordinate (demobygangliupppby) at (\demobygangliuxxxb, \demobygangliuyyyy);
\coordinate (demobygangliupppbz) at (\demobygangliuxxxb, \demobygangliuyyyz);
\coordinate (demobygangliupppca) at (\demobygangliuxxxc, \demobygangliuyyya);
\coordinate (demobygangliupppcb) at (\demobygangliuxxxc, \demobygangliuyyyb);
\coordinate (demobygangliupppcc) at (\demobygangliuxxxc, \demobygangliuyyyc);
\coordinate (demobygangliupppcd) at (\demobygangliuxxxc, \demobygangliuyyyd);
\coordinate (demobygangliupppce) at (\demobygangliuxxxc, \demobygangliuyyye);
\coordinate (demobygangliupppcf) at (\demobygangliuxxxc, \demobygangliuyyyf);
\coordinate (demobygangliupppcg) at (\demobygangliuxxxc, \demobygangliuyyyg);
\coordinate (demobygangliupppch) at (\demobygangliuxxxc, \demobygangliuyyyh);
\coordinate (demobygangliupppci) at (\demobygangliuxxxc, \demobygangliuyyyi);
\coordinate (demobygangliupppcj) at (\demobygangliuxxxc, \demobygangliuyyyj);
\coordinate (demobygangliupppck) at (\demobygangliuxxxc, \demobygangliuyyyk);
\coordinate (demobygangliupppcl) at (\demobygangliuxxxc, \demobygangliuyyyl);
\coordinate (demobygangliupppcm) at (\demobygangliuxxxc, \demobygangliuyyym);
\coordinate (demobygangliupppcn) at (\demobygangliuxxxc, \demobygangliuyyyn);
\coordinate (demobygangliupppco) at (\demobygangliuxxxc, \demobygangliuyyyo);
\coordinate (demobygangliupppcp) at (\demobygangliuxxxc, \demobygangliuyyyp);
\coordinate (demobygangliupppcq) at (\demobygangliuxxxc, \demobygangliuyyyq);
\coordinate (demobygangliupppcr) at (\demobygangliuxxxc, \demobygangliuyyyr);
\coordinate (demobygangliupppcs) at (\demobygangliuxxxc, \demobygangliuyyys);
\coordinate (demobygangliupppct) at (\demobygangliuxxxc, \demobygangliuyyyt);
\coordinate (demobygangliupppcu) at (\demobygangliuxxxc, \demobygangliuyyyu);
\coordinate (demobygangliupppcv) at (\demobygangliuxxxc, \demobygangliuyyyv);
\coordinate (demobygangliupppcw) at (\demobygangliuxxxc, \demobygangliuyyyw);
\coordinate (demobygangliupppcx) at (\demobygangliuxxxc, \demobygangliuyyyx);
\coordinate (demobygangliupppcy) at (\demobygangliuxxxc, \demobygangliuyyyy);
\coordinate (demobygangliupppcz) at (\demobygangliuxxxc, \demobygangliuyyyz);
\coordinate (demobygangliupppda) at (\demobygangliuxxxd, \demobygangliuyyya);
\coordinate (demobygangliupppdb) at (\demobygangliuxxxd, \demobygangliuyyyb);
\coordinate (demobygangliupppdc) at (\demobygangliuxxxd, \demobygangliuyyyc);
\coordinate (demobygangliupppdd) at (\demobygangliuxxxd, \demobygangliuyyyd);
\coordinate (demobygangliupppde) at (\demobygangliuxxxd, \demobygangliuyyye);
\coordinate (demobygangliupppdf) at (\demobygangliuxxxd, \demobygangliuyyyf);
\coordinate (demobygangliupppdg) at (\demobygangliuxxxd, \demobygangliuyyyg);
\coordinate (demobygangliupppdh) at (\demobygangliuxxxd, \demobygangliuyyyh);
\coordinate (demobygangliupppdi) at (\demobygangliuxxxd, \demobygangliuyyyi);
\coordinate (demobygangliupppdj) at (\demobygangliuxxxd, \demobygangliuyyyj);
\coordinate (demobygangliupppdk) at (\demobygangliuxxxd, \demobygangliuyyyk);
\coordinate (demobygangliupppdl) at (\demobygangliuxxxd, \demobygangliuyyyl);
\coordinate (demobygangliupppdm) at (\demobygangliuxxxd, \demobygangliuyyym);
\coordinate (demobygangliupppdn) at (\demobygangliuxxxd, \demobygangliuyyyn);
\coordinate (demobygangliupppdo) at (\demobygangliuxxxd, \demobygangliuyyyo);
\coordinate (demobygangliupppdp) at (\demobygangliuxxxd, \demobygangliuyyyp);
\coordinate (demobygangliupppdq) at (\demobygangliuxxxd, \demobygangliuyyyq);
\coordinate (demobygangliupppdr) at (\demobygangliuxxxd, \demobygangliuyyyr);
\coordinate (demobygangliupppds) at (\demobygangliuxxxd, \demobygangliuyyys);
\coordinate (demobygangliupppdt) at (\demobygangliuxxxd, \demobygangliuyyyt);
\coordinate (demobygangliupppdu) at (\demobygangliuxxxd, \demobygangliuyyyu);
\coordinate (demobygangliupppdv) at (\demobygangliuxxxd, \demobygangliuyyyv);
\coordinate (demobygangliupppdw) at (\demobygangliuxxxd, \demobygangliuyyyw);
\coordinate (demobygangliupppdx) at (\demobygangliuxxxd, \demobygangliuyyyx);
\coordinate (demobygangliupppdy) at (\demobygangliuxxxd, \demobygangliuyyyy);
\coordinate (demobygangliupppdz) at (\demobygangliuxxxd, \demobygangliuyyyz);
\coordinate (demobygangliupppea) at (\demobygangliuxxxe, \demobygangliuyyya);
\coordinate (demobygangliupppeb) at (\demobygangliuxxxe, \demobygangliuyyyb);
\coordinate (demobygangliupppec) at (\demobygangliuxxxe, \demobygangliuyyyc);
\coordinate (demobygangliuppped) at (\demobygangliuxxxe, \demobygangliuyyyd);
\coordinate (demobygangliupppee) at (\demobygangliuxxxe, \demobygangliuyyye);
\coordinate (demobygangliupppef) at (\demobygangliuxxxe, \demobygangliuyyyf);
\coordinate (demobygangliupppeg) at (\demobygangliuxxxe, \demobygangliuyyyg);
\coordinate (demobygangliupppeh) at (\demobygangliuxxxe, \demobygangliuyyyh);
\coordinate (demobygangliupppei) at (\demobygangliuxxxe, \demobygangliuyyyi);
\coordinate (demobygangliupppej) at (\demobygangliuxxxe, \demobygangliuyyyj);
\coordinate (demobygangliupppek) at (\demobygangliuxxxe, \demobygangliuyyyk);
\coordinate (demobygangliupppel) at (\demobygangliuxxxe, \demobygangliuyyyl);
\coordinate (demobygangliupppem) at (\demobygangliuxxxe, \demobygangliuyyym);
\coordinate (demobygangliupppen) at (\demobygangliuxxxe, \demobygangliuyyyn);
\coordinate (demobygangliupppeo) at (\demobygangliuxxxe, \demobygangliuyyyo);
\coordinate (demobygangliupppep) at (\demobygangliuxxxe, \demobygangliuyyyp);
\coordinate (demobygangliupppeq) at (\demobygangliuxxxe, \demobygangliuyyyq);
\coordinate (demobygangliuppper) at (\demobygangliuxxxe, \demobygangliuyyyr);
\coordinate (demobygangliupppes) at (\demobygangliuxxxe, \demobygangliuyyys);
\coordinate (demobygangliupppet) at (\demobygangliuxxxe, \demobygangliuyyyt);
\coordinate (demobygangliupppeu) at (\demobygangliuxxxe, \demobygangliuyyyu);
\coordinate (demobygangliupppev) at (\demobygangliuxxxe, \demobygangliuyyyv);
\coordinate (demobygangliupppew) at (\demobygangliuxxxe, \demobygangliuyyyw);
\coordinate (demobygangliupppex) at (\demobygangliuxxxe, \demobygangliuyyyx);
\coordinate (demobygangliupppey) at (\demobygangliuxxxe, \demobygangliuyyyy);
\coordinate (demobygangliupppez) at (\demobygangliuxxxe, \demobygangliuyyyz);
\coordinate (demobygangliupppfa) at (\demobygangliuxxxf, \demobygangliuyyya);
\coordinate (demobygangliupppfb) at (\demobygangliuxxxf, \demobygangliuyyyb);
\coordinate (demobygangliupppfc) at (\demobygangliuxxxf, \demobygangliuyyyc);
\coordinate (demobygangliupppfd) at (\demobygangliuxxxf, \demobygangliuyyyd);
\coordinate (demobygangliupppfe) at (\demobygangliuxxxf, \demobygangliuyyye);
\coordinate (demobygangliupppff) at (\demobygangliuxxxf, \demobygangliuyyyf);
\coordinate (demobygangliupppfg) at (\demobygangliuxxxf, \demobygangliuyyyg);
\coordinate (demobygangliupppfh) at (\demobygangliuxxxf, \demobygangliuyyyh);
\coordinate (demobygangliupppfi) at (\demobygangliuxxxf, \demobygangliuyyyi);
\coordinate (demobygangliupppfj) at (\demobygangliuxxxf, \demobygangliuyyyj);
\coordinate (demobygangliupppfk) at (\demobygangliuxxxf, \demobygangliuyyyk);
\coordinate (demobygangliupppfl) at (\demobygangliuxxxf, \demobygangliuyyyl);
\coordinate (demobygangliupppfm) at (\demobygangliuxxxf, \demobygangliuyyym);
\coordinate (demobygangliupppfn) at (\demobygangliuxxxf, \demobygangliuyyyn);
\coordinate (demobygangliupppfo) at (\demobygangliuxxxf, \demobygangliuyyyo);
\coordinate (demobygangliupppfp) at (\demobygangliuxxxf, \demobygangliuyyyp);
\coordinate (demobygangliupppfq) at (\demobygangliuxxxf, \demobygangliuyyyq);
\coordinate (demobygangliupppfr) at (\demobygangliuxxxf, \demobygangliuyyyr);
\coordinate (demobygangliupppfs) at (\demobygangliuxxxf, \demobygangliuyyys);
\coordinate (demobygangliupppft) at (\demobygangliuxxxf, \demobygangliuyyyt);
\coordinate (demobygangliupppfu) at (\demobygangliuxxxf, \demobygangliuyyyu);
\coordinate (demobygangliupppfv) at (\demobygangliuxxxf, \demobygangliuyyyv);
\coordinate (demobygangliupppfw) at (\demobygangliuxxxf, \demobygangliuyyyw);
\coordinate (demobygangliupppfx) at (\demobygangliuxxxf, \demobygangliuyyyx);
\coordinate (demobygangliupppfy) at (\demobygangliuxxxf, \demobygangliuyyyy);
\coordinate (demobygangliupppfz) at (\demobygangliuxxxf, \demobygangliuyyyz);
\coordinate (demobygangliupppga) at (\demobygangliuxxxg, \demobygangliuyyya);
\coordinate (demobygangliupppgb) at (\demobygangliuxxxg, \demobygangliuyyyb);
\coordinate (demobygangliupppgc) at (\demobygangliuxxxg, \demobygangliuyyyc);
\coordinate (demobygangliupppgd) at (\demobygangliuxxxg, \demobygangliuyyyd);
\coordinate (demobygangliupppge) at (\demobygangliuxxxg, \demobygangliuyyye);
\coordinate (demobygangliupppgf) at (\demobygangliuxxxg, \demobygangliuyyyf);
\coordinate (demobygangliupppgg) at (\demobygangliuxxxg, \demobygangliuyyyg);
\coordinate (demobygangliupppgh) at (\demobygangliuxxxg, \demobygangliuyyyh);
\coordinate (demobygangliupppgi) at (\demobygangliuxxxg, \demobygangliuyyyi);
\coordinate (demobygangliupppgj) at (\demobygangliuxxxg, \demobygangliuyyyj);
\coordinate (demobygangliupppgk) at (\demobygangliuxxxg, \demobygangliuyyyk);
\coordinate (demobygangliupppgl) at (\demobygangliuxxxg, \demobygangliuyyyl);
\coordinate (demobygangliupppgm) at (\demobygangliuxxxg, \demobygangliuyyym);
\coordinate (demobygangliupppgn) at (\demobygangliuxxxg, \demobygangliuyyyn);
\coordinate (demobygangliupppgo) at (\demobygangliuxxxg, \demobygangliuyyyo);
\coordinate (demobygangliupppgp) at (\demobygangliuxxxg, \demobygangliuyyyp);
\coordinate (demobygangliupppgq) at (\demobygangliuxxxg, \demobygangliuyyyq);
\coordinate (demobygangliupppgr) at (\demobygangliuxxxg, \demobygangliuyyyr);
\coordinate (demobygangliupppgs) at (\demobygangliuxxxg, \demobygangliuyyys);
\coordinate (demobygangliupppgt) at (\demobygangliuxxxg, \demobygangliuyyyt);
\coordinate (demobygangliupppgu) at (\demobygangliuxxxg, \demobygangliuyyyu);
\coordinate (demobygangliupppgv) at (\demobygangliuxxxg, \demobygangliuyyyv);
\coordinate (demobygangliupppgw) at (\demobygangliuxxxg, \demobygangliuyyyw);
\coordinate (demobygangliupppgx) at (\demobygangliuxxxg, \demobygangliuyyyx);
\coordinate (demobygangliupppgy) at (\demobygangliuxxxg, \demobygangliuyyyy);
\coordinate (demobygangliupppgz) at (\demobygangliuxxxg, \demobygangliuyyyz);
\coordinate (demobygangliupppha) at (\demobygangliuxxxh, \demobygangliuyyya);
\coordinate (demobygangliuppphb) at (\demobygangliuxxxh, \demobygangliuyyyb);
\coordinate (demobygangliuppphc) at (\demobygangliuxxxh, \demobygangliuyyyc);
\coordinate (demobygangliuppphd) at (\demobygangliuxxxh, \demobygangliuyyyd);
\coordinate (demobygangliuppphe) at (\demobygangliuxxxh, \demobygangliuyyye);
\coordinate (demobygangliuppphf) at (\demobygangliuxxxh, \demobygangliuyyyf);
\coordinate (demobygangliuppphg) at (\demobygangliuxxxh, \demobygangliuyyyg);
\coordinate (demobygangliuppphh) at (\demobygangliuxxxh, \demobygangliuyyyh);
\coordinate (demobygangliuppphi) at (\demobygangliuxxxh, \demobygangliuyyyi);
\coordinate (demobygangliuppphj) at (\demobygangliuxxxh, \demobygangliuyyyj);
\coordinate (demobygangliuppphk) at (\demobygangliuxxxh, \demobygangliuyyyk);
\coordinate (demobygangliuppphl) at (\demobygangliuxxxh, \demobygangliuyyyl);
\coordinate (demobygangliuppphm) at (\demobygangliuxxxh, \demobygangliuyyym);
\coordinate (demobygangliuppphn) at (\demobygangliuxxxh, \demobygangliuyyyn);
\coordinate (demobygangliupppho) at (\demobygangliuxxxh, \demobygangliuyyyo);
\coordinate (demobygangliuppphp) at (\demobygangliuxxxh, \demobygangliuyyyp);
\coordinate (demobygangliuppphq) at (\demobygangliuxxxh, \demobygangliuyyyq);
\coordinate (demobygangliuppphr) at (\demobygangliuxxxh, \demobygangliuyyyr);
\coordinate (demobygangliuppphs) at (\demobygangliuxxxh, \demobygangliuyyys);
\coordinate (demobygangliupppht) at (\demobygangliuxxxh, \demobygangliuyyyt);
\coordinate (demobygangliuppphu) at (\demobygangliuxxxh, \demobygangliuyyyu);
\coordinate (demobygangliuppphv) at (\demobygangliuxxxh, \demobygangliuyyyv);
\coordinate (demobygangliuppphw) at (\demobygangliuxxxh, \demobygangliuyyyw);
\coordinate (demobygangliuppphx) at (\demobygangliuxxxh, \demobygangliuyyyx);
\coordinate (demobygangliuppphy) at (\demobygangliuxxxh, \demobygangliuyyyy);
\coordinate (demobygangliuppphz) at (\demobygangliuxxxh, \demobygangliuyyyz);
\coordinate (demobygangliupppia) at (\demobygangliuxxxi, \demobygangliuyyya);
\coordinate (demobygangliupppib) at (\demobygangliuxxxi, \demobygangliuyyyb);
\coordinate (demobygangliupppic) at (\demobygangliuxxxi, \demobygangliuyyyc);
\coordinate (demobygangliupppid) at (\demobygangliuxxxi, \demobygangliuyyyd);
\coordinate (demobygangliupppie) at (\demobygangliuxxxi, \demobygangliuyyye);
\coordinate (demobygangliupppif) at (\demobygangliuxxxi, \demobygangliuyyyf);
\coordinate (demobygangliupppig) at (\demobygangliuxxxi, \demobygangliuyyyg);
\coordinate (demobygangliupppih) at (\demobygangliuxxxi, \demobygangliuyyyh);
\coordinate (demobygangliupppii) at (\demobygangliuxxxi, \demobygangliuyyyi);
\coordinate (demobygangliupppij) at (\demobygangliuxxxi, \demobygangliuyyyj);
\coordinate (demobygangliupppik) at (\demobygangliuxxxi, \demobygangliuyyyk);
\coordinate (demobygangliupppil) at (\demobygangliuxxxi, \demobygangliuyyyl);
\coordinate (demobygangliupppim) at (\demobygangliuxxxi, \demobygangliuyyym);
\coordinate (demobygangliupppin) at (\demobygangliuxxxi, \demobygangliuyyyn);
\coordinate (demobygangliupppio) at (\demobygangliuxxxi, \demobygangliuyyyo);
\coordinate (demobygangliupppip) at (\demobygangliuxxxi, \demobygangliuyyyp);
\coordinate (demobygangliupppiq) at (\demobygangliuxxxi, \demobygangliuyyyq);
\coordinate (demobygangliupppir) at (\demobygangliuxxxi, \demobygangliuyyyr);
\coordinate (demobygangliupppis) at (\demobygangliuxxxi, \demobygangliuyyys);
\coordinate (demobygangliupppit) at (\demobygangliuxxxi, \demobygangliuyyyt);
\coordinate (demobygangliupppiu) at (\demobygangliuxxxi, \demobygangliuyyyu);
\coordinate (demobygangliupppiv) at (\demobygangliuxxxi, \demobygangliuyyyv);
\coordinate (demobygangliupppiw) at (\demobygangliuxxxi, \demobygangliuyyyw);
\coordinate (demobygangliupppix) at (\demobygangliuxxxi, \demobygangliuyyyx);
\coordinate (demobygangliupppiy) at (\demobygangliuxxxi, \demobygangliuyyyy);
\coordinate (demobygangliupppiz) at (\demobygangliuxxxi, \demobygangliuyyyz);
\coordinate (demobygangliupppja) at (\demobygangliuxxxj, \demobygangliuyyya);
\coordinate (demobygangliupppjb) at (\demobygangliuxxxj, \demobygangliuyyyb);
\coordinate (demobygangliupppjc) at (\demobygangliuxxxj, \demobygangliuyyyc);
\coordinate (demobygangliupppjd) at (\demobygangliuxxxj, \demobygangliuyyyd);
\coordinate (demobygangliupppje) at (\demobygangliuxxxj, \demobygangliuyyye);
\coordinate (demobygangliupppjf) at (\demobygangliuxxxj, \demobygangliuyyyf);
\coordinate (demobygangliupppjg) at (\demobygangliuxxxj, \demobygangliuyyyg);
\coordinate (demobygangliupppjh) at (\demobygangliuxxxj, \demobygangliuyyyh);
\coordinate (demobygangliupppji) at (\demobygangliuxxxj, \demobygangliuyyyi);
\coordinate (demobygangliupppjj) at (\demobygangliuxxxj, \demobygangliuyyyj);
\coordinate (demobygangliupppjk) at (\demobygangliuxxxj, \demobygangliuyyyk);
\coordinate (demobygangliupppjl) at (\demobygangliuxxxj, \demobygangliuyyyl);
\coordinate (demobygangliupppjm) at (\demobygangliuxxxj, \demobygangliuyyym);
\coordinate (demobygangliupppjn) at (\demobygangliuxxxj, \demobygangliuyyyn);
\coordinate (demobygangliupppjo) at (\demobygangliuxxxj, \demobygangliuyyyo);
\coordinate (demobygangliupppjp) at (\demobygangliuxxxj, \demobygangliuyyyp);
\coordinate (demobygangliupppjq) at (\demobygangliuxxxj, \demobygangliuyyyq);
\coordinate (demobygangliupppjr) at (\demobygangliuxxxj, \demobygangliuyyyr);
\coordinate (demobygangliupppjs) at (\demobygangliuxxxj, \demobygangliuyyys);
\coordinate (demobygangliupppjt) at (\demobygangliuxxxj, \demobygangliuyyyt);
\coordinate (demobygangliupppju) at (\demobygangliuxxxj, \demobygangliuyyyu);
\coordinate (demobygangliupppjv) at (\demobygangliuxxxj, \demobygangliuyyyv);
\coordinate (demobygangliupppjw) at (\demobygangliuxxxj, \demobygangliuyyyw);
\coordinate (demobygangliupppjx) at (\demobygangliuxxxj, \demobygangliuyyyx);
\coordinate (demobygangliupppjy) at (\demobygangliuxxxj, \demobygangliuyyyy);
\coordinate (demobygangliupppjz) at (\demobygangliuxxxj, \demobygangliuyyyz);
\coordinate (demobygangliupppka) at (\demobygangliuxxxk, \demobygangliuyyya);
\coordinate (demobygangliupppkb) at (\demobygangliuxxxk, \demobygangliuyyyb);
\coordinate (demobygangliupppkc) at (\demobygangliuxxxk, \demobygangliuyyyc);
\coordinate (demobygangliupppkd) at (\demobygangliuxxxk, \demobygangliuyyyd);
\coordinate (demobygangliupppke) at (\demobygangliuxxxk, \demobygangliuyyye);
\coordinate (demobygangliupppkf) at (\demobygangliuxxxk, \demobygangliuyyyf);
\coordinate (demobygangliupppkg) at (\demobygangliuxxxk, \demobygangliuyyyg);
\coordinate (demobygangliupppkh) at (\demobygangliuxxxk, \demobygangliuyyyh);
\coordinate (demobygangliupppki) at (\demobygangliuxxxk, \demobygangliuyyyi);
\coordinate (demobygangliupppkj) at (\demobygangliuxxxk, \demobygangliuyyyj);
\coordinate (demobygangliupppkk) at (\demobygangliuxxxk, \demobygangliuyyyk);
\coordinate (demobygangliupppkl) at (\demobygangliuxxxk, \demobygangliuyyyl);
\coordinate (demobygangliupppkm) at (\demobygangliuxxxk, \demobygangliuyyym);
\coordinate (demobygangliupppkn) at (\demobygangliuxxxk, \demobygangliuyyyn);
\coordinate (demobygangliupppko) at (\demobygangliuxxxk, \demobygangliuyyyo);
\coordinate (demobygangliupppkp) at (\demobygangliuxxxk, \demobygangliuyyyp);
\coordinate (demobygangliupppkq) at (\demobygangliuxxxk, \demobygangliuyyyq);
\coordinate (demobygangliupppkr) at (\demobygangliuxxxk, \demobygangliuyyyr);
\coordinate (demobygangliupppks) at (\demobygangliuxxxk, \demobygangliuyyys);
\coordinate (demobygangliupppkt) at (\demobygangliuxxxk, \demobygangliuyyyt);
\coordinate (demobygangliupppku) at (\demobygangliuxxxk, \demobygangliuyyyu);
\coordinate (demobygangliupppkv) at (\demobygangliuxxxk, \demobygangliuyyyv);
\coordinate (demobygangliupppkw) at (\demobygangliuxxxk, \demobygangliuyyyw);
\coordinate (demobygangliupppkx) at (\demobygangliuxxxk, \demobygangliuyyyx);
\coordinate (demobygangliupppky) at (\demobygangliuxxxk, \demobygangliuyyyy);
\coordinate (demobygangliupppkz) at (\demobygangliuxxxk, \demobygangliuyyyz);
\coordinate (demobygangliupppla) at (\demobygangliuxxxl, \demobygangliuyyya);
\coordinate (demobygangliuppplb) at (\demobygangliuxxxl, \demobygangliuyyyb);
\coordinate (demobygangliuppplc) at (\demobygangliuxxxl, \demobygangliuyyyc);
\coordinate (demobygangliupppld) at (\demobygangliuxxxl, \demobygangliuyyyd);
\coordinate (demobygangliuppple) at (\demobygangliuxxxl, \demobygangliuyyye);
\coordinate (demobygangliuppplf) at (\demobygangliuxxxl, \demobygangliuyyyf);
\coordinate (demobygangliuppplg) at (\demobygangliuxxxl, \demobygangliuyyyg);
\coordinate (demobygangliuppplh) at (\demobygangliuxxxl, \demobygangliuyyyh);
\coordinate (demobygangliupppli) at (\demobygangliuxxxl, \demobygangliuyyyi);
\coordinate (demobygangliuppplj) at (\demobygangliuxxxl, \demobygangliuyyyj);
\coordinate (demobygangliuppplk) at (\demobygangliuxxxl, \demobygangliuyyyk);
\coordinate (demobygangliupppll) at (\demobygangliuxxxl, \demobygangliuyyyl);
\coordinate (demobygangliuppplm) at (\demobygangliuxxxl, \demobygangliuyyym);
\coordinate (demobygangliupppln) at (\demobygangliuxxxl, \demobygangliuyyyn);
\coordinate (demobygangliuppplo) at (\demobygangliuxxxl, \demobygangliuyyyo);
\coordinate (demobygangliuppplp) at (\demobygangliuxxxl, \demobygangliuyyyp);
\coordinate (demobygangliuppplq) at (\demobygangliuxxxl, \demobygangliuyyyq);
\coordinate (demobygangliuppplr) at (\demobygangliuxxxl, \demobygangliuyyyr);
\coordinate (demobygangliupppls) at (\demobygangliuxxxl, \demobygangliuyyys);
\coordinate (demobygangliuppplt) at (\demobygangliuxxxl, \demobygangliuyyyt);
\coordinate (demobygangliuppplu) at (\demobygangliuxxxl, \demobygangliuyyyu);
\coordinate (demobygangliuppplv) at (\demobygangliuxxxl, \demobygangliuyyyv);
\coordinate (demobygangliuppplw) at (\demobygangliuxxxl, \demobygangliuyyyw);
\coordinate (demobygangliuppplx) at (\demobygangliuxxxl, \demobygangliuyyyx);
\coordinate (demobygangliuppply) at (\demobygangliuxxxl, \demobygangliuyyyy);
\coordinate (demobygangliuppplz) at (\demobygangliuxxxl, \demobygangliuyyyz);
\coordinate (demobygangliupppma) at (\demobygangliuxxxm, \demobygangliuyyya);
\coordinate (demobygangliupppmb) at (\demobygangliuxxxm, \demobygangliuyyyb);
\coordinate (demobygangliupppmc) at (\demobygangliuxxxm, \demobygangliuyyyc);
\coordinate (demobygangliupppmd) at (\demobygangliuxxxm, \demobygangliuyyyd);
\coordinate (demobygangliupppme) at (\demobygangliuxxxm, \demobygangliuyyye);
\coordinate (demobygangliupppmf) at (\demobygangliuxxxm, \demobygangliuyyyf);
\coordinate (demobygangliupppmg) at (\demobygangliuxxxm, \demobygangliuyyyg);
\coordinate (demobygangliupppmh) at (\demobygangliuxxxm, \demobygangliuyyyh);
\coordinate (demobygangliupppmi) at (\demobygangliuxxxm, \demobygangliuyyyi);
\coordinate (demobygangliupppmj) at (\demobygangliuxxxm, \demobygangliuyyyj);
\coordinate (demobygangliupppmk) at (\demobygangliuxxxm, \demobygangliuyyyk);
\coordinate (demobygangliupppml) at (\demobygangliuxxxm, \demobygangliuyyyl);
\coordinate (demobygangliupppmm) at (\demobygangliuxxxm, \demobygangliuyyym);
\coordinate (demobygangliupppmn) at (\demobygangliuxxxm, \demobygangliuyyyn);
\coordinate (demobygangliupppmo) at (\demobygangliuxxxm, \demobygangliuyyyo);
\coordinate (demobygangliupppmp) at (\demobygangliuxxxm, \demobygangliuyyyp);
\coordinate (demobygangliupppmq) at (\demobygangliuxxxm, \demobygangliuyyyq);
\coordinate (demobygangliupppmr) at (\demobygangliuxxxm, \demobygangliuyyyr);
\coordinate (demobygangliupppms) at (\demobygangliuxxxm, \demobygangliuyyys);
\coordinate (demobygangliupppmt) at (\demobygangliuxxxm, \demobygangliuyyyt);
\coordinate (demobygangliupppmu) at (\demobygangliuxxxm, \demobygangliuyyyu);
\coordinate (demobygangliupppmv) at (\demobygangliuxxxm, \demobygangliuyyyv);
\coordinate (demobygangliupppmw) at (\demobygangliuxxxm, \demobygangliuyyyw);
\coordinate (demobygangliupppmx) at (\demobygangliuxxxm, \demobygangliuyyyx);
\coordinate (demobygangliupppmy) at (\demobygangliuxxxm, \demobygangliuyyyy);
\coordinate (demobygangliupppmz) at (\demobygangliuxxxm, \demobygangliuyyyz);
\coordinate (demobygangliupppna) at (\demobygangliuxxxn, \demobygangliuyyya);
\coordinate (demobygangliupppnb) at (\demobygangliuxxxn, \demobygangliuyyyb);
\coordinate (demobygangliupppnc) at (\demobygangliuxxxn, \demobygangliuyyyc);
\coordinate (demobygangliupppnd) at (\demobygangliuxxxn, \demobygangliuyyyd);
\coordinate (demobygangliupppne) at (\demobygangliuxxxn, \demobygangliuyyye);
\coordinate (demobygangliupppnf) at (\demobygangliuxxxn, \demobygangliuyyyf);
\coordinate (demobygangliupppng) at (\demobygangliuxxxn, \demobygangliuyyyg);
\coordinate (demobygangliupppnh) at (\demobygangliuxxxn, \demobygangliuyyyh);
\coordinate (demobygangliupppni) at (\demobygangliuxxxn, \demobygangliuyyyi);
\coordinate (demobygangliupppnj) at (\demobygangliuxxxn, \demobygangliuyyyj);
\coordinate (demobygangliupppnk) at (\demobygangliuxxxn, \demobygangliuyyyk);
\coordinate (demobygangliupppnl) at (\demobygangliuxxxn, \demobygangliuyyyl);
\coordinate (demobygangliupppnm) at (\demobygangliuxxxn, \demobygangliuyyym);
\coordinate (demobygangliupppnn) at (\demobygangliuxxxn, \demobygangliuyyyn);
\coordinate (demobygangliupppno) at (\demobygangliuxxxn, \demobygangliuyyyo);
\coordinate (demobygangliupppnp) at (\demobygangliuxxxn, \demobygangliuyyyp);
\coordinate (demobygangliupppnq) at (\demobygangliuxxxn, \demobygangliuyyyq);
\coordinate (demobygangliupppnr) at (\demobygangliuxxxn, \demobygangliuyyyr);
\coordinate (demobygangliupppns) at (\demobygangliuxxxn, \demobygangliuyyys);
\coordinate (demobygangliupppnt) at (\demobygangliuxxxn, \demobygangliuyyyt);
\coordinate (demobygangliupppnu) at (\demobygangliuxxxn, \demobygangliuyyyu);
\coordinate (demobygangliupppnv) at (\demobygangliuxxxn, \demobygangliuyyyv);
\coordinate (demobygangliupppnw) at (\demobygangliuxxxn, \demobygangliuyyyw);
\coordinate (demobygangliupppnx) at (\demobygangliuxxxn, \demobygangliuyyyx);
\coordinate (demobygangliupppny) at (\demobygangliuxxxn, \demobygangliuyyyy);
\coordinate (demobygangliupppnz) at (\demobygangliuxxxn, \demobygangliuyyyz);
\coordinate (demobygangliupppoa) at (\demobygangliuxxxo, \demobygangliuyyya);
\coordinate (demobygangliupppob) at (\demobygangliuxxxo, \demobygangliuyyyb);
\coordinate (demobygangliupppoc) at (\demobygangliuxxxo, \demobygangliuyyyc);
\coordinate (demobygangliupppod) at (\demobygangliuxxxo, \demobygangliuyyyd);
\coordinate (demobygangliupppoe) at (\demobygangliuxxxo, \demobygangliuyyye);
\coordinate (demobygangliupppof) at (\demobygangliuxxxo, \demobygangliuyyyf);
\coordinate (demobygangliupppog) at (\demobygangliuxxxo, \demobygangliuyyyg);
\coordinate (demobygangliupppoh) at (\demobygangliuxxxo, \demobygangliuyyyh);
\coordinate (demobygangliupppoi) at (\demobygangliuxxxo, \demobygangliuyyyi);
\coordinate (demobygangliupppoj) at (\demobygangliuxxxo, \demobygangliuyyyj);
\coordinate (demobygangliupppok) at (\demobygangliuxxxo, \demobygangliuyyyk);
\coordinate (demobygangliupppol) at (\demobygangliuxxxo, \demobygangliuyyyl);
\coordinate (demobygangliupppom) at (\demobygangliuxxxo, \demobygangliuyyym);
\coordinate (demobygangliupppon) at (\demobygangliuxxxo, \demobygangliuyyyn);
\coordinate (demobygangliupppoo) at (\demobygangliuxxxo, \demobygangliuyyyo);
\coordinate (demobygangliupppop) at (\demobygangliuxxxo, \demobygangliuyyyp);
\coordinate (demobygangliupppoq) at (\demobygangliuxxxo, \demobygangliuyyyq);
\coordinate (demobygangliupppor) at (\demobygangliuxxxo, \demobygangliuyyyr);
\coordinate (demobygangliupppos) at (\demobygangliuxxxo, \demobygangliuyyys);
\coordinate (demobygangliupppot) at (\demobygangliuxxxo, \demobygangliuyyyt);
\coordinate (demobygangliupppou) at (\demobygangliuxxxo, \demobygangliuyyyu);
\coordinate (demobygangliupppov) at (\demobygangliuxxxo, \demobygangliuyyyv);
\coordinate (demobygangliupppow) at (\demobygangliuxxxo, \demobygangliuyyyw);
\coordinate (demobygangliupppox) at (\demobygangliuxxxo, \demobygangliuyyyx);
\coordinate (demobygangliupppoy) at (\demobygangliuxxxo, \demobygangliuyyyy);
\coordinate (demobygangliupppoz) at (\demobygangliuxxxo, \demobygangliuyyyz);
\coordinate (demobygangliuppppa) at (\demobygangliuxxxp, \demobygangliuyyya);
\coordinate (demobygangliuppppb) at (\demobygangliuxxxp, \demobygangliuyyyb);
\coordinate (demobygangliuppppc) at (\demobygangliuxxxp, \demobygangliuyyyc);
\coordinate (demobygangliuppppd) at (\demobygangliuxxxp, \demobygangliuyyyd);
\coordinate (demobygangliuppppe) at (\demobygangliuxxxp, \demobygangliuyyye);
\coordinate (demobygangliuppppf) at (\demobygangliuxxxp, \demobygangliuyyyf);
\coordinate (demobygangliuppppg) at (\demobygangliuxxxp, \demobygangliuyyyg);
\coordinate (demobygangliupppph) at (\demobygangliuxxxp, \demobygangliuyyyh);
\coordinate (demobygangliuppppi) at (\demobygangliuxxxp, \demobygangliuyyyi);
\coordinate (demobygangliuppppj) at (\demobygangliuxxxp, \demobygangliuyyyj);
\coordinate (demobygangliuppppk) at (\demobygangliuxxxp, \demobygangliuyyyk);
\coordinate (demobygangliuppppl) at (\demobygangliuxxxp, \demobygangliuyyyl);
\coordinate (demobygangliuppppm) at (\demobygangliuxxxp, \demobygangliuyyym);
\coordinate (demobygangliuppppn) at (\demobygangliuxxxp, \demobygangliuyyyn);
\coordinate (demobygangliuppppo) at (\demobygangliuxxxp, \demobygangliuyyyo);
\coordinate (demobygangliuppppp) at (\demobygangliuxxxp, \demobygangliuyyyp);
\coordinate (demobygangliuppppq) at (\demobygangliuxxxp, \demobygangliuyyyq);
\coordinate (demobygangliuppppr) at (\demobygangliuxxxp, \demobygangliuyyyr);
\coordinate (demobygangliupppps) at (\demobygangliuxxxp, \demobygangliuyyys);
\coordinate (demobygangliuppppt) at (\demobygangliuxxxp, \demobygangliuyyyt);
\coordinate (demobygangliuppppu) at (\demobygangliuxxxp, \demobygangliuyyyu);
\coordinate (demobygangliuppppv) at (\demobygangliuxxxp, \demobygangliuyyyv);
\coordinate (demobygangliuppppw) at (\demobygangliuxxxp, \demobygangliuyyyw);
\coordinate (demobygangliuppppx) at (\demobygangliuxxxp, \demobygangliuyyyx);
\coordinate (demobygangliuppppy) at (\demobygangliuxxxp, \demobygangliuyyyy);
\coordinate (demobygangliuppppz) at (\demobygangliuxxxp, \demobygangliuyyyz);
\coordinate (demobygangliupppqa) at (\demobygangliuxxxq, \demobygangliuyyya);
\coordinate (demobygangliupppqb) at (\demobygangliuxxxq, \demobygangliuyyyb);
\coordinate (demobygangliupppqc) at (\demobygangliuxxxq, \demobygangliuyyyc);
\coordinate (demobygangliupppqd) at (\demobygangliuxxxq, \demobygangliuyyyd);
\coordinate (demobygangliupppqe) at (\demobygangliuxxxq, \demobygangliuyyye);
\coordinate (demobygangliupppqf) at (\demobygangliuxxxq, \demobygangliuyyyf);
\coordinate (demobygangliupppqg) at (\demobygangliuxxxq, \demobygangliuyyyg);
\coordinate (demobygangliupppqh) at (\demobygangliuxxxq, \demobygangliuyyyh);
\coordinate (demobygangliupppqi) at (\demobygangliuxxxq, \demobygangliuyyyi);
\coordinate (demobygangliupppqj) at (\demobygangliuxxxq, \demobygangliuyyyj);
\coordinate (demobygangliupppqk) at (\demobygangliuxxxq, \demobygangliuyyyk);
\coordinate (demobygangliupppql) at (\demobygangliuxxxq, \demobygangliuyyyl);
\coordinate (demobygangliupppqm) at (\demobygangliuxxxq, \demobygangliuyyym);
\coordinate (demobygangliupppqn) at (\demobygangliuxxxq, \demobygangliuyyyn);
\coordinate (demobygangliupppqo) at (\demobygangliuxxxq, \demobygangliuyyyo);
\coordinate (demobygangliupppqp) at (\demobygangliuxxxq, \demobygangliuyyyp);
\coordinate (demobygangliupppqq) at (\demobygangliuxxxq, \demobygangliuyyyq);
\coordinate (demobygangliupppqr) at (\demobygangliuxxxq, \demobygangliuyyyr);
\coordinate (demobygangliupppqs) at (\demobygangliuxxxq, \demobygangliuyyys);
\coordinate (demobygangliupppqt) at (\demobygangliuxxxq, \demobygangliuyyyt);
\coordinate (demobygangliupppqu) at (\demobygangliuxxxq, \demobygangliuyyyu);
\coordinate (demobygangliupppqv) at (\demobygangliuxxxq, \demobygangliuyyyv);
\coordinate (demobygangliupppqw) at (\demobygangliuxxxq, \demobygangliuyyyw);
\coordinate (demobygangliupppqx) at (\demobygangliuxxxq, \demobygangliuyyyx);
\coordinate (demobygangliupppqy) at (\demobygangliuxxxq, \demobygangliuyyyy);
\coordinate (demobygangliupppqz) at (\demobygangliuxxxq, \demobygangliuyyyz);
\coordinate (demobygangliupppra) at (\demobygangliuxxxr, \demobygangliuyyya);
\coordinate (demobygangliuppprb) at (\demobygangliuxxxr, \demobygangliuyyyb);
\coordinate (demobygangliuppprc) at (\demobygangliuxxxr, \demobygangliuyyyc);
\coordinate (demobygangliuppprd) at (\demobygangliuxxxr, \demobygangliuyyyd);
\coordinate (demobygangliupppre) at (\demobygangliuxxxr, \demobygangliuyyye);
\coordinate (demobygangliuppprf) at (\demobygangliuxxxr, \demobygangliuyyyf);
\coordinate (demobygangliuppprg) at (\demobygangliuxxxr, \demobygangliuyyyg);
\coordinate (demobygangliuppprh) at (\demobygangliuxxxr, \demobygangliuyyyh);
\coordinate (demobygangliupppri) at (\demobygangliuxxxr, \demobygangliuyyyi);
\coordinate (demobygangliuppprj) at (\demobygangliuxxxr, \demobygangliuyyyj);
\coordinate (demobygangliuppprk) at (\demobygangliuxxxr, \demobygangliuyyyk);
\coordinate (demobygangliuppprl) at (\demobygangliuxxxr, \demobygangliuyyyl);
\coordinate (demobygangliuppprm) at (\demobygangliuxxxr, \demobygangliuyyym);
\coordinate (demobygangliuppprn) at (\demobygangliuxxxr, \demobygangliuyyyn);
\coordinate (demobygangliupppro) at (\demobygangliuxxxr, \demobygangliuyyyo);
\coordinate (demobygangliuppprp) at (\demobygangliuxxxr, \demobygangliuyyyp);
\coordinate (demobygangliuppprq) at (\demobygangliuxxxr, \demobygangliuyyyq);
\coordinate (demobygangliuppprr) at (\demobygangliuxxxr, \demobygangliuyyyr);
\coordinate (demobygangliuppprs) at (\demobygangliuxxxr, \demobygangliuyyys);
\coordinate (demobygangliuppprt) at (\demobygangliuxxxr, \demobygangliuyyyt);
\coordinate (demobygangliupppru) at (\demobygangliuxxxr, \demobygangliuyyyu);
\coordinate (demobygangliuppprv) at (\demobygangliuxxxr, \demobygangliuyyyv);
\coordinate (demobygangliuppprw) at (\demobygangliuxxxr, \demobygangliuyyyw);
\coordinate (demobygangliuppprx) at (\demobygangliuxxxr, \demobygangliuyyyx);
\coordinate (demobygangliupppry) at (\demobygangliuxxxr, \demobygangliuyyyy);
\coordinate (demobygangliuppprz) at (\demobygangliuxxxr, \demobygangliuyyyz);
\coordinate (demobygangliupppsa) at (\demobygangliuxxxs, \demobygangliuyyya);
\coordinate (demobygangliupppsb) at (\demobygangliuxxxs, \demobygangliuyyyb);
\coordinate (demobygangliupppsc) at (\demobygangliuxxxs, \demobygangliuyyyc);
\coordinate (demobygangliupppsd) at (\demobygangliuxxxs, \demobygangliuyyyd);
\coordinate (demobygangliupppse) at (\demobygangliuxxxs, \demobygangliuyyye);
\coordinate (demobygangliupppsf) at (\demobygangliuxxxs, \demobygangliuyyyf);
\coordinate (demobygangliupppsg) at (\demobygangliuxxxs, \demobygangliuyyyg);
\coordinate (demobygangliupppsh) at (\demobygangliuxxxs, \demobygangliuyyyh);
\coordinate (demobygangliupppsi) at (\demobygangliuxxxs, \demobygangliuyyyi);
\coordinate (demobygangliupppsj) at (\demobygangliuxxxs, \demobygangliuyyyj);
\coordinate (demobygangliupppsk) at (\demobygangliuxxxs, \demobygangliuyyyk);
\coordinate (demobygangliupppsl) at (\demobygangliuxxxs, \demobygangliuyyyl);
\coordinate (demobygangliupppsm) at (\demobygangliuxxxs, \demobygangliuyyym);
\coordinate (demobygangliupppsn) at (\demobygangliuxxxs, \demobygangliuyyyn);
\coordinate (demobygangliupppso) at (\demobygangliuxxxs, \demobygangliuyyyo);
\coordinate (demobygangliupppsp) at (\demobygangliuxxxs, \demobygangliuyyyp);
\coordinate (demobygangliupppsq) at (\demobygangliuxxxs, \demobygangliuyyyq);
\coordinate (demobygangliupppsr) at (\demobygangliuxxxs, \demobygangliuyyyr);
\coordinate (demobygangliupppss) at (\demobygangliuxxxs, \demobygangliuyyys);
\coordinate (demobygangliupppst) at (\demobygangliuxxxs, \demobygangliuyyyt);
\coordinate (demobygangliupppsu) at (\demobygangliuxxxs, \demobygangliuyyyu);
\coordinate (demobygangliupppsv) at (\demobygangliuxxxs, \demobygangliuyyyv);
\coordinate (demobygangliupppsw) at (\demobygangliuxxxs, \demobygangliuyyyw);
\coordinate (demobygangliupppsx) at (\demobygangliuxxxs, \demobygangliuyyyx);
\coordinate (demobygangliupppsy) at (\demobygangliuxxxs, \demobygangliuyyyy);
\coordinate (demobygangliupppsz) at (\demobygangliuxxxs, \demobygangliuyyyz);
\coordinate (demobygangliupppta) at (\demobygangliuxxxt, \demobygangliuyyya);
\coordinate (demobygangliuppptb) at (\demobygangliuxxxt, \demobygangliuyyyb);
\coordinate (demobygangliuppptc) at (\demobygangliuxxxt, \demobygangliuyyyc);
\coordinate (demobygangliuppptd) at (\demobygangliuxxxt, \demobygangliuyyyd);
\coordinate (demobygangliupppte) at (\demobygangliuxxxt, \demobygangliuyyye);
\coordinate (demobygangliuppptf) at (\demobygangliuxxxt, \demobygangliuyyyf);
\coordinate (demobygangliuppptg) at (\demobygangliuxxxt, \demobygangliuyyyg);
\coordinate (demobygangliupppth) at (\demobygangliuxxxt, \demobygangliuyyyh);
\coordinate (demobygangliupppti) at (\demobygangliuxxxt, \demobygangliuyyyi);
\coordinate (demobygangliuppptj) at (\demobygangliuxxxt, \demobygangliuyyyj);
\coordinate (demobygangliuppptk) at (\demobygangliuxxxt, \demobygangliuyyyk);
\coordinate (demobygangliuppptl) at (\demobygangliuxxxt, \demobygangliuyyyl);
\coordinate (demobygangliuppptm) at (\demobygangliuxxxt, \demobygangliuyyym);
\coordinate (demobygangliuppptn) at (\demobygangliuxxxt, \demobygangliuyyyn);
\coordinate (demobygangliupppto) at (\demobygangliuxxxt, \demobygangliuyyyo);
\coordinate (demobygangliuppptp) at (\demobygangliuxxxt, \demobygangliuyyyp);
\coordinate (demobygangliuppptq) at (\demobygangliuxxxt, \demobygangliuyyyq);
\coordinate (demobygangliuppptr) at (\demobygangliuxxxt, \demobygangliuyyyr);
\coordinate (demobygangliupppts) at (\demobygangliuxxxt, \demobygangliuyyys);
\coordinate (demobygangliuppptt) at (\demobygangliuxxxt, \demobygangliuyyyt);
\coordinate (demobygangliuppptu) at (\demobygangliuxxxt, \demobygangliuyyyu);
\coordinate (demobygangliuppptv) at (\demobygangliuxxxt, \demobygangliuyyyv);
\coordinate (demobygangliuppptw) at (\demobygangliuxxxt, \demobygangliuyyyw);
\coordinate (demobygangliuppptx) at (\demobygangliuxxxt, \demobygangliuyyyx);
\coordinate (demobygangliupppty) at (\demobygangliuxxxt, \demobygangliuyyyy);
\coordinate (demobygangliuppptz) at (\demobygangliuxxxt, \demobygangliuyyyz);
\coordinate (demobygangliupppua) at (\demobygangliuxxxu, \demobygangliuyyya);
\coordinate (demobygangliupppub) at (\demobygangliuxxxu, \demobygangliuyyyb);
\coordinate (demobygangliupppuc) at (\demobygangliuxxxu, \demobygangliuyyyc);
\coordinate (demobygangliupppud) at (\demobygangliuxxxu, \demobygangliuyyyd);
\coordinate (demobygangliupppue) at (\demobygangliuxxxu, \demobygangliuyyye);
\coordinate (demobygangliupppuf) at (\demobygangliuxxxu, \demobygangliuyyyf);
\coordinate (demobygangliupppug) at (\demobygangliuxxxu, \demobygangliuyyyg);
\coordinate (demobygangliupppuh) at (\demobygangliuxxxu, \demobygangliuyyyh);
\coordinate (demobygangliupppui) at (\demobygangliuxxxu, \demobygangliuyyyi);
\coordinate (demobygangliupppuj) at (\demobygangliuxxxu, \demobygangliuyyyj);
\coordinate (demobygangliupppuk) at (\demobygangliuxxxu, \demobygangliuyyyk);
\coordinate (demobygangliupppul) at (\demobygangliuxxxu, \demobygangliuyyyl);
\coordinate (demobygangliupppum) at (\demobygangliuxxxu, \demobygangliuyyym);
\coordinate (demobygangliupppun) at (\demobygangliuxxxu, \demobygangliuyyyn);
\coordinate (demobygangliupppuo) at (\demobygangliuxxxu, \demobygangliuyyyo);
\coordinate (demobygangliupppup) at (\demobygangliuxxxu, \demobygangliuyyyp);
\coordinate (demobygangliupppuq) at (\demobygangliuxxxu, \demobygangliuyyyq);
\coordinate (demobygangliupppur) at (\demobygangliuxxxu, \demobygangliuyyyr);
\coordinate (demobygangliupppus) at (\demobygangliuxxxu, \demobygangliuyyys);
\coordinate (demobygangliuppput) at (\demobygangliuxxxu, \demobygangliuyyyt);
\coordinate (demobygangliupppuu) at (\demobygangliuxxxu, \demobygangliuyyyu);
\coordinate (demobygangliupppuv) at (\demobygangliuxxxu, \demobygangliuyyyv);
\coordinate (demobygangliupppuw) at (\demobygangliuxxxu, \demobygangliuyyyw);
\coordinate (demobygangliupppux) at (\demobygangliuxxxu, \demobygangliuyyyx);
\coordinate (demobygangliupppuy) at (\demobygangliuxxxu, \demobygangliuyyyy);
\coordinate (demobygangliupppuz) at (\demobygangliuxxxu, \demobygangliuyyyz);
\coordinate (demobygangliupppva) at (\demobygangliuxxxv, \demobygangliuyyya);
\coordinate (demobygangliupppvb) at (\demobygangliuxxxv, \demobygangliuyyyb);
\coordinate (demobygangliupppvc) at (\demobygangliuxxxv, \demobygangliuyyyc);
\coordinate (demobygangliupppvd) at (\demobygangliuxxxv, \demobygangliuyyyd);
\coordinate (demobygangliupppve) at (\demobygangliuxxxv, \demobygangliuyyye);
\coordinate (demobygangliupppvf) at (\demobygangliuxxxv, \demobygangliuyyyf);
\coordinate (demobygangliupppvg) at (\demobygangliuxxxv, \demobygangliuyyyg);
\coordinate (demobygangliupppvh) at (\demobygangliuxxxv, \demobygangliuyyyh);
\coordinate (demobygangliupppvi) at (\demobygangliuxxxv, \demobygangliuyyyi);
\coordinate (demobygangliupppvj) at (\demobygangliuxxxv, \demobygangliuyyyj);
\coordinate (demobygangliupppvk) at (\demobygangliuxxxv, \demobygangliuyyyk);
\coordinate (demobygangliupppvl) at (\demobygangliuxxxv, \demobygangliuyyyl);
\coordinate (demobygangliupppvm) at (\demobygangliuxxxv, \demobygangliuyyym);
\coordinate (demobygangliupppvn) at (\demobygangliuxxxv, \demobygangliuyyyn);
\coordinate (demobygangliupppvo) at (\demobygangliuxxxv, \demobygangliuyyyo);
\coordinate (demobygangliupppvp) at (\demobygangliuxxxv, \demobygangliuyyyp);
\coordinate (demobygangliupppvq) at (\demobygangliuxxxv, \demobygangliuyyyq);
\coordinate (demobygangliupppvr) at (\demobygangliuxxxv, \demobygangliuyyyr);
\coordinate (demobygangliupppvs) at (\demobygangliuxxxv, \demobygangliuyyys);
\coordinate (demobygangliupppvt) at (\demobygangliuxxxv, \demobygangliuyyyt);
\coordinate (demobygangliupppvu) at (\demobygangliuxxxv, \demobygangliuyyyu);
\coordinate (demobygangliupppvv) at (\demobygangliuxxxv, \demobygangliuyyyv);
\coordinate (demobygangliupppvw) at (\demobygangliuxxxv, \demobygangliuyyyw);
\coordinate (demobygangliupppvx) at (\demobygangliuxxxv, \demobygangliuyyyx);
\coordinate (demobygangliupppvy) at (\demobygangliuxxxv, \demobygangliuyyyy);
\coordinate (demobygangliupppvz) at (\demobygangliuxxxv, \demobygangliuyyyz);
\coordinate (demobygangliupppwa) at (\demobygangliuxxxw, \demobygangliuyyya);
\coordinate (demobygangliupppwb) at (\demobygangliuxxxw, \demobygangliuyyyb);
\coordinate (demobygangliupppwc) at (\demobygangliuxxxw, \demobygangliuyyyc);
\coordinate (demobygangliupppwd) at (\demobygangliuxxxw, \demobygangliuyyyd);
\coordinate (demobygangliupppwe) at (\demobygangliuxxxw, \demobygangliuyyye);
\coordinate (demobygangliupppwf) at (\demobygangliuxxxw, \demobygangliuyyyf);
\coordinate (demobygangliupppwg) at (\demobygangliuxxxw, \demobygangliuyyyg);
\coordinate (demobygangliupppwh) at (\demobygangliuxxxw, \demobygangliuyyyh);
\coordinate (demobygangliupppwi) at (\demobygangliuxxxw, \demobygangliuyyyi);
\coordinate (demobygangliupppwj) at (\demobygangliuxxxw, \demobygangliuyyyj);
\coordinate (demobygangliupppwk) at (\demobygangliuxxxw, \demobygangliuyyyk);
\coordinate (demobygangliupppwl) at (\demobygangliuxxxw, \demobygangliuyyyl);
\coordinate (demobygangliupppwm) at (\demobygangliuxxxw, \demobygangliuyyym);
\coordinate (demobygangliupppwn) at (\demobygangliuxxxw, \demobygangliuyyyn);
\coordinate (demobygangliupppwo) at (\demobygangliuxxxw, \demobygangliuyyyo);
\coordinate (demobygangliupppwp) at (\demobygangliuxxxw, \demobygangliuyyyp);
\coordinate (demobygangliupppwq) at (\demobygangliuxxxw, \demobygangliuyyyq);
\coordinate (demobygangliupppwr) at (\demobygangliuxxxw, \demobygangliuyyyr);
\coordinate (demobygangliupppws) at (\demobygangliuxxxw, \demobygangliuyyys);
\coordinate (demobygangliupppwt) at (\demobygangliuxxxw, \demobygangliuyyyt);
\coordinate (demobygangliupppwu) at (\demobygangliuxxxw, \demobygangliuyyyu);
\coordinate (demobygangliupppwv) at (\demobygangliuxxxw, \demobygangliuyyyv);
\coordinate (demobygangliupppww) at (\demobygangliuxxxw, \demobygangliuyyyw);
\coordinate (demobygangliupppwx) at (\demobygangliuxxxw, \demobygangliuyyyx);
\coordinate (demobygangliupppwy) at (\demobygangliuxxxw, \demobygangliuyyyy);
\coordinate (demobygangliupppwz) at (\demobygangliuxxxw, \demobygangliuyyyz);
\coordinate (demobygangliupppxa) at (\demobygangliuxxxx, \demobygangliuyyya);
\coordinate (demobygangliupppxb) at (\demobygangliuxxxx, \demobygangliuyyyb);
\coordinate (demobygangliupppxc) at (\demobygangliuxxxx, \demobygangliuyyyc);
\coordinate (demobygangliupppxd) at (\demobygangliuxxxx, \demobygangliuyyyd);
\coordinate (demobygangliupppxe) at (\demobygangliuxxxx, \demobygangliuyyye);
\coordinate (demobygangliupppxf) at (\demobygangliuxxxx, \demobygangliuyyyf);
\coordinate (demobygangliupppxg) at (\demobygangliuxxxx, \demobygangliuyyyg);
\coordinate (demobygangliupppxh) at (\demobygangliuxxxx, \demobygangliuyyyh);
\coordinate (demobygangliupppxi) at (\demobygangliuxxxx, \demobygangliuyyyi);
\coordinate (demobygangliupppxj) at (\demobygangliuxxxx, \demobygangliuyyyj);
\coordinate (demobygangliupppxk) at (\demobygangliuxxxx, \demobygangliuyyyk);
\coordinate (demobygangliupppxl) at (\demobygangliuxxxx, \demobygangliuyyyl);
\coordinate (demobygangliupppxm) at (\demobygangliuxxxx, \demobygangliuyyym);
\coordinate (demobygangliupppxn) at (\demobygangliuxxxx, \demobygangliuyyyn);
\coordinate (demobygangliupppxo) at (\demobygangliuxxxx, \demobygangliuyyyo);
\coordinate (demobygangliupppxp) at (\demobygangliuxxxx, \demobygangliuyyyp);
\coordinate (demobygangliupppxq) at (\demobygangliuxxxx, \demobygangliuyyyq);
\coordinate (demobygangliupppxr) at (\demobygangliuxxxx, \demobygangliuyyyr);
\coordinate (demobygangliupppxs) at (\demobygangliuxxxx, \demobygangliuyyys);
\coordinate (demobygangliupppxt) at (\demobygangliuxxxx, \demobygangliuyyyt);
\coordinate (demobygangliupppxu) at (\demobygangliuxxxx, \demobygangliuyyyu);
\coordinate (demobygangliupppxv) at (\demobygangliuxxxx, \demobygangliuyyyv);
\coordinate (demobygangliupppxw) at (\demobygangliuxxxx, \demobygangliuyyyw);
\coordinate (demobygangliupppxx) at (\demobygangliuxxxx, \demobygangliuyyyx);
\coordinate (demobygangliupppxy) at (\demobygangliuxxxx, \demobygangliuyyyy);
\coordinate (demobygangliupppxz) at (\demobygangliuxxxx, \demobygangliuyyyz);
\coordinate (demobygangliupppya) at (\demobygangliuxxxy, \demobygangliuyyya);
\coordinate (demobygangliupppyb) at (\demobygangliuxxxy, \demobygangliuyyyb);
\coordinate (demobygangliupppyc) at (\demobygangliuxxxy, \demobygangliuyyyc);
\coordinate (demobygangliupppyd) at (\demobygangliuxxxy, \demobygangliuyyyd);
\coordinate (demobygangliupppye) at (\demobygangliuxxxy, \demobygangliuyyye);
\coordinate (demobygangliupppyf) at (\demobygangliuxxxy, \demobygangliuyyyf);
\coordinate (demobygangliupppyg) at (\demobygangliuxxxy, \demobygangliuyyyg);
\coordinate (demobygangliupppyh) at (\demobygangliuxxxy, \demobygangliuyyyh);
\coordinate (demobygangliupppyi) at (\demobygangliuxxxy, \demobygangliuyyyi);
\coordinate (demobygangliupppyj) at (\demobygangliuxxxy, \demobygangliuyyyj);
\coordinate (demobygangliupppyk) at (\demobygangliuxxxy, \demobygangliuyyyk);
\coordinate (demobygangliupppyl) at (\demobygangliuxxxy, \demobygangliuyyyl);
\coordinate (demobygangliupppym) at (\demobygangliuxxxy, \demobygangliuyyym);
\coordinate (demobygangliupppyn) at (\demobygangliuxxxy, \demobygangliuyyyn);
\coordinate (demobygangliupppyo) at (\demobygangliuxxxy, \demobygangliuyyyo);
\coordinate (demobygangliupppyp) at (\demobygangliuxxxy, \demobygangliuyyyp);
\coordinate (demobygangliupppyq) at (\demobygangliuxxxy, \demobygangliuyyyq);
\coordinate (demobygangliupppyr) at (\demobygangliuxxxy, \demobygangliuyyyr);
\coordinate (demobygangliupppys) at (\demobygangliuxxxy, \demobygangliuyyys);
\coordinate (demobygangliupppyt) at (\demobygangliuxxxy, \demobygangliuyyyt);
\coordinate (demobygangliupppyu) at (\demobygangliuxxxy, \demobygangliuyyyu);
\coordinate (demobygangliupppyv) at (\demobygangliuxxxy, \demobygangliuyyyv);
\coordinate (demobygangliupppyw) at (\demobygangliuxxxy, \demobygangliuyyyw);
\coordinate (demobygangliupppyx) at (\demobygangliuxxxy, \demobygangliuyyyx);
\coordinate (demobygangliupppyy) at (\demobygangliuxxxy, \demobygangliuyyyy);
\coordinate (demobygangliupppyz) at (\demobygangliuxxxy, \demobygangliuyyyz);
\coordinate (demobygangliupppza) at (\demobygangliuxxxz, \demobygangliuyyya);
\coordinate (demobygangliupppzb) at (\demobygangliuxxxz, \demobygangliuyyyb);
\coordinate (demobygangliupppzc) at (\demobygangliuxxxz, \demobygangliuyyyc);
\coordinate (demobygangliupppzd) at (\demobygangliuxxxz, \demobygangliuyyyd);
\coordinate (demobygangliupppze) at (\demobygangliuxxxz, \demobygangliuyyye);
\coordinate (demobygangliupppzf) at (\demobygangliuxxxz, \demobygangliuyyyf);
\coordinate (demobygangliupppzg) at (\demobygangliuxxxz, \demobygangliuyyyg);
\coordinate (demobygangliupppzh) at (\demobygangliuxxxz, \demobygangliuyyyh);
\coordinate (demobygangliupppzi) at (\demobygangliuxxxz, \demobygangliuyyyi);
\coordinate (demobygangliupppzj) at (\demobygangliuxxxz, \demobygangliuyyyj);
\coordinate (demobygangliupppzk) at (\demobygangliuxxxz, \demobygangliuyyyk);
\coordinate (demobygangliupppzl) at (\demobygangliuxxxz, \demobygangliuyyyl);
\coordinate (demobygangliupppzm) at (\demobygangliuxxxz, \demobygangliuyyym);
\coordinate (demobygangliupppzn) at (\demobygangliuxxxz, \demobygangliuyyyn);
\coordinate (demobygangliupppzo) at (\demobygangliuxxxz, \demobygangliuyyyo);
\coordinate (demobygangliupppzp) at (\demobygangliuxxxz, \demobygangliuyyyp);
\coordinate (demobygangliupppzq) at (\demobygangliuxxxz, \demobygangliuyyyq);
\coordinate (demobygangliupppzr) at (\demobygangliuxxxz, \demobygangliuyyyr);
\coordinate (demobygangliupppzs) at (\demobygangliuxxxz, \demobygangliuyyys);
\coordinate (demobygangliupppzt) at (\demobygangliuxxxz, \demobygangliuyyyt);
\coordinate (demobygangliupppzu) at (\demobygangliuxxxz, \demobygangliuyyyu);
\coordinate (demobygangliupppzv) at (\demobygangliuxxxz, \demobygangliuyyyv);
\coordinate (demobygangliupppzw) at (\demobygangliuxxxz, \demobygangliuyyyw);
\coordinate (demobygangliupppzx) at (\demobygangliuxxxz, \demobygangliuyyyx);
\coordinate (demobygangliupppzy) at (\demobygangliuxxxz, \demobygangliuyyyy);
\coordinate (demobygangliupppzz) at (\demobygangliuxxxz, \demobygangliuyyyz);

%\gangprintcoordinateat{(0,0)}{The last coordinate values: }{($(demobygangliupppzz)$)}; 



% Draw related part of the coordinate system with dashed helplines (centered at (demobygangliupppii)) with letters as background, which would help to determine all coordinates. 
\coordinatebackground{demobygangliu}{c}{d}{o};

% Step 2, draw key devices, their accessories, and take related coordinates of their pins, and may define more coordinates. 

% Draw the Opamp at the coordinate (demobygangliupppii) and name it as "swopamp".
\draw (demobygangliupppii) node [op amp, yscale=-1] (swopamp) {\ctikzflipy{Opamp}} ; 

% Its accessories and lables. 
\draw [-*](swopamp.down) -- ($(swopamp.down)+(0,1)$) node[right]{$V_+$}; 
\node at ($(swopamp.down)+(0.3,0.2)$) {7};  
\draw [-*](swopamp.up) -- ($(swopamp.up)+(0,-1)$) node[right]{$V_-$}; 
\node at ($(swopamp.up)+(0.3,-0.2)$) {4};

% Get the x- and y-components of the coordinates of the "+" and "-" pins. 
\getxyingivenunit{cm}{(swopamp.+)}{\swopampzx}{\swopampzy};
\getxyingivenunit{cm}{(swopamp.-)}{\swopampfx}{\swopampfy};

% Then define a few more coordinates, at least for keeping in mind.
\coordinate (plusshort) at ($(\demobygangliuxxxg,\swopampzy)$);
\fill  (plusshort) circle (2pt);  % May be commented later.
\coordinate (minusshort) at ($(\demobygangliuxxxg,\swopampfy)$);
\fill  (minusshort) circle (2pt); % May be commented later.
\coordinate (leftinter) at ($(\demobygangliuxxxe,\swopampzy)$);
\fill  (leftinter) circle (2pt);

% Draw an "npn" at (demobygangliupppmi) and name it as "swQ".
\draw (demobygangliupppmi) node[npn](swQ){};

% Get the x- and y-components of the needed pins of it for later usage.
\getxyingivenunit{cm}{(swQ.C)}{\swQCx}{\swQCy};
\getxyingivenunit{cm}{(swQ.E)}{\swQEx}{\swQEy};

% Then define more coordinate(s).
\coordinate (Qcshort) at ($(\swQEx,\demobygangliuyyyj)$);
\fill  (Qcshort) circle (2pt); % May be commented later.
\coordinate (Qeshort) at ($(\swQEx,\demobygangliuyyyf)$);
\fill  (Qeshort) circle (2pt) node [right] {$V_0$};

% Then the rectangle by the points (demobygangliupppef) -- (demobygangliupppej) -- (Qcshort) -- (Qeshort) forms a clear area for the key devices. 

% Connect the two devices.
\draw (swopamp.out) to [short, l=$I_B$, above] (swQ.B);

% Step 3, draw other little devices. For tidiness, better to give two units in length for each new device and align them up.

% For this specific circuit, let us attach the four bi-pole devices (maybe with their accessories) to each corner of the above mentioned rectangle area for the key devices, separately. 

\draw  (demobygangliuppped) node [ground] {} to [empty ZZener diode] (demobygangliupppef) -- (leftinter);
% The Latex system can not work properly when I put the following label into the above "[empty ZZener diode]" in the form of an additional "l= ..." option, then I have to employ the following "\node ..." command. Then the label should be aligned with the "ZZener" as much as possible even the coordinate system is modified later. Since the "\demobygangliuyyyd" and "\demobygangliuyyyf" micros are used to position the "ZZener", then better to use the center between them to locate the label, rather than using "\demobygangliuyyye" directly. This idea is also applied in later "\node ..." commands. 
\node at ($(\demobygangliuxxxe-1.3, \demobygangliuyyyd*0.5+\demobygangliuyyyf*0.5)$) {$V_Z = 5\textnormal{V}$};

\draw (demobygangliupppej) to [generic] (demobygangliupppel) -| (swQ.C);
\node at ($(\demobygangliuxxxe-1.1, \demobygangliuyyyj*0.5+\demobygangliuyyyl*0.5)$) {$R_{1}=47k\Omega$};
\node [right] at (\swQCx,\demobygangliuyyyj*0.5+\demobygangliuyyyl*0.5) {$I_C \approx \beta I_B$};

\draw  (\swQEx, \demobygangliuyyyd) node [ground] {} to [generic] (\swQEx, \demobygangliuyyyf) -- (swQ.E);
\node at ($(\swQEx+1.4, \demobygangliuyyyd*0.5+\demobygangliuyyyf*0.5)$) {$R_{E}=100k\Omega$};

% Draw the top area. 
\draw  (\demobygangliuxxxi-0.2,\demobygangliuyyyn) --  (\demobygangliuxxxi+0.2,\demobygangliuyyyn) node [right] {$V_{cc}=15\textnormal{V}$} ;
\draw [->] (demobygangliupppin) -- (demobygangliupppim) node [right] {$I$};  
\draw  (demobygangliupppim) -- (demobygangliupppil);
\fill  (demobygangliupppil) circle (2pt);

% Step 4, other shorts.
\draw  (swopamp.+)  to [short, l_=$I_+ \approx 0 $, above] (plusshort) -- (leftinter) -- (demobygangliupppej);

\draw  (swopamp.-)  to [short, l_=$I_- \approx 0 $, above] (minusshort) |- (Qeshort);

%Step 5, all the rest, especially labels. May also clean unnecessary staff, like the system background and dark points for showing newly defined coordinates previously. 
\draw [->] ($(\swQEx-0.4, \demobygangliuyyyf - 0.4)$) -- node [left] {$I_E$} ($(\swQEx-0.4, \demobygangliuyyyd + 0.4)$);

\draw [->] ($(\demobygangliuxxxe+0.4, \demobygangliuyyyl*0.5+\demobygangliuyyyj*0.5 + 0.6)$) -- node [right] {$I_1$} ($(\demobygangliuxxxe+0.4, \demobygangliuyyyl*0.5+\demobygangliuyyyj*0.5 - 0.6)$);


\end{circuitikz}















\newpage


\vspace{2cm}

{\Large Figure 4, the same as the above but erasing the coordinate background by commenting out the one command 
``\textbackslash coordinatebackground ..." line.}

\begin{circuitikz}[scale=1]


% Circuits can be drawn by the following five major steps, as shown in the following example. 

% Step 1, preparations. 

% "Install" the coordinate system with keyword ``demobygangliu".
\pgfmathsetmacro{\totaldemobygangliuxxx}{26}
\pgfmathsetmacro{\totaldemobygangliuyyy}{26}
\pgfmathsetmacro{\demobygangliuxxxspacing}{1}
\pgfmathsetmacro{\demobygangliuyyyspacing}{1}
\pgfmathsetmacro{\demobygangliuxxxa}{-8}
\pgfmathsetmacro{\demobygangliuyyya}{-8}

\pgfmathsetmacro{\demobygangliuxxxb}{\demobygangliuxxxa + \demobygangliuxxxspacing + 0.0 }
\pgfmathsetmacro{\demobygangliuxxxc}{\demobygangliuxxxb + \demobygangliuxxxspacing + 0.0 }
\pgfmathsetmacro{\demobygangliuxxxd}{\demobygangliuxxxc + \demobygangliuxxxspacing + 0.0 }
\pgfmathsetmacro{\demobygangliuxxxe}{\demobygangliuxxxd + \demobygangliuxxxspacing + 0.0 }
\pgfmathsetmacro{\demobygangliuxxxf}{\demobygangliuxxxe + \demobygangliuxxxspacing + 0.0 }
\pgfmathsetmacro{\demobygangliuxxxg}{\demobygangliuxxxf + \demobygangliuxxxspacing + 0.0 }
\pgfmathsetmacro{\demobygangliuxxxh}{\demobygangliuxxxg + \demobygangliuxxxspacing + 0.0 }
\pgfmathsetmacro{\demobygangliuxxxi}{\demobygangliuxxxh + \demobygangliuxxxspacing + 0.0 }
\pgfmathsetmacro{\demobygangliuxxxj}{\demobygangliuxxxi + \demobygangliuxxxspacing + 0.0 }
\pgfmathsetmacro{\demobygangliuxxxk}{\demobygangliuxxxj + \demobygangliuxxxspacing + 0.0 }
\pgfmathsetmacro{\demobygangliuxxxl}{\demobygangliuxxxk + \demobygangliuxxxspacing + 0.0 }
\pgfmathsetmacro{\demobygangliuxxxm}{\demobygangliuxxxl + \demobygangliuxxxspacing + 0.0 }
\pgfmathsetmacro{\demobygangliuxxxn}{\demobygangliuxxxm + \demobygangliuxxxspacing + 0.0 }
\pgfmathsetmacro{\demobygangliuxxxo}{\demobygangliuxxxn + \demobygangliuxxxspacing + 0.0 }
\pgfmathsetmacro{\demobygangliuxxxp}{\demobygangliuxxxo + \demobygangliuxxxspacing + 0.0 }
\pgfmathsetmacro{\demobygangliuxxxq}{\demobygangliuxxxp + \demobygangliuxxxspacing + 0.0 }
\pgfmathsetmacro{\demobygangliuxxxr}{\demobygangliuxxxq + \demobygangliuxxxspacing + 0.0 }
\pgfmathsetmacro{\demobygangliuxxxs}{\demobygangliuxxxr + \demobygangliuxxxspacing + 0.0 }
\pgfmathsetmacro{\demobygangliuxxxt}{\demobygangliuxxxs + \demobygangliuxxxspacing + 0.0 }
\pgfmathsetmacro{\demobygangliuxxxu}{\demobygangliuxxxt + \demobygangliuxxxspacing + 0.0 }
\pgfmathsetmacro{\demobygangliuxxxv}{\demobygangliuxxxu + \demobygangliuxxxspacing + 0.0 }
\pgfmathsetmacro{\demobygangliuxxxw}{\demobygangliuxxxv + \demobygangliuxxxspacing + 0.0 }
\pgfmathsetmacro{\demobygangliuxxxx}{\demobygangliuxxxw + \demobygangliuxxxspacing + 0.0 }
\pgfmathsetmacro{\demobygangliuxxxy}{\demobygangliuxxxx + \demobygangliuxxxspacing + 0.0 }
\pgfmathsetmacro{\demobygangliuxxxz}{\demobygangliuxxxy + \demobygangliuxxxspacing + 0.0 }

\pgfmathsetmacro{\demobygangliuyyyb}{\demobygangliuyyya + \demobygangliuyyyspacing + 0.0 }
\pgfmathsetmacro{\demobygangliuyyyc}{\demobygangliuyyyb + \demobygangliuyyyspacing + 0.0 }
\pgfmathsetmacro{\demobygangliuyyyd}{\demobygangliuyyyc + \demobygangliuyyyspacing + 0.0 }
\pgfmathsetmacro{\demobygangliuyyye}{\demobygangliuyyyd + \demobygangliuyyyspacing + 0.0 }
\pgfmathsetmacro{\demobygangliuyyyf}{\demobygangliuyyye + \demobygangliuyyyspacing + 0.0 }
\pgfmathsetmacro{\demobygangliuyyyg}{\demobygangliuyyyf + \demobygangliuyyyspacing + 0.0 }
\pgfmathsetmacro{\demobygangliuyyyh}{\demobygangliuyyyg + \demobygangliuyyyspacing + 0.0 }
\pgfmathsetmacro{\demobygangliuyyyi}{\demobygangliuyyyh + \demobygangliuyyyspacing + 0.0 }
\pgfmathsetmacro{\demobygangliuyyyj}{\demobygangliuyyyi + \demobygangliuyyyspacing + 0.0 }
\pgfmathsetmacro{\demobygangliuyyyk}{\demobygangliuyyyj + \demobygangliuyyyspacing + 0.0 }
\pgfmathsetmacro{\demobygangliuyyyl}{\demobygangliuyyyk + \demobygangliuyyyspacing + 0.0 }
\pgfmathsetmacro{\demobygangliuyyym}{\demobygangliuyyyl + \demobygangliuyyyspacing + 0.0 }
\pgfmathsetmacro{\demobygangliuyyyn}{\demobygangliuyyym + \demobygangliuyyyspacing + 0.0 }
\pgfmathsetmacro{\demobygangliuyyyo}{\demobygangliuyyyn + \demobygangliuyyyspacing + 0.0 }
\pgfmathsetmacro{\demobygangliuyyyp}{\demobygangliuyyyo + \demobygangliuyyyspacing + 0.0 }
\pgfmathsetmacro{\demobygangliuyyyq}{\demobygangliuyyyp + \demobygangliuyyyspacing + 0.0 }
\pgfmathsetmacro{\demobygangliuyyyr}{\demobygangliuyyyq + \demobygangliuyyyspacing + 0.0 }
\pgfmathsetmacro{\demobygangliuyyys}{\demobygangliuyyyr + \demobygangliuyyyspacing + 0.0 }
\pgfmathsetmacro{\demobygangliuyyyt}{\demobygangliuyyys + \demobygangliuyyyspacing + 0.0 }
\pgfmathsetmacro{\demobygangliuyyyu}{\demobygangliuyyyt + \demobygangliuyyyspacing + 0.0 }
\pgfmathsetmacro{\demobygangliuyyyv}{\demobygangliuyyyu + \demobygangliuyyyspacing + 0.0 }
\pgfmathsetmacro{\demobygangliuyyyw}{\demobygangliuyyyv + \demobygangliuyyyspacing + 0.0 }
\pgfmathsetmacro{\demobygangliuyyyx}{\demobygangliuyyyw + \demobygangliuyyyspacing + 0.0 }
\pgfmathsetmacro{\demobygangliuyyyy}{\demobygangliuyyyx + \demobygangliuyyyspacing + 0.0 }
\pgfmathsetmacro{\demobygangliuyyyz}{\demobygangliuyyyy + \demobygangliuyyyspacing + 0.0 }

\coordinate (demobygangliupppaa) at (\demobygangliuxxxa, \demobygangliuyyya);
\coordinate (demobygangliupppab) at (\demobygangliuxxxa, \demobygangliuyyyb);
\coordinate (demobygangliupppac) at (\demobygangliuxxxa, \demobygangliuyyyc);
\coordinate (demobygangliupppad) at (\demobygangliuxxxa, \demobygangliuyyyd);
\coordinate (demobygangliupppae) at (\demobygangliuxxxa, \demobygangliuyyye);
\coordinate (demobygangliupppaf) at (\demobygangliuxxxa, \demobygangliuyyyf);
\coordinate (demobygangliupppag) at (\demobygangliuxxxa, \demobygangliuyyyg);
\coordinate (demobygangliupppah) at (\demobygangliuxxxa, \demobygangliuyyyh);
\coordinate (demobygangliupppai) at (\demobygangliuxxxa, \demobygangliuyyyi);
\coordinate (demobygangliupppaj) at (\demobygangliuxxxa, \demobygangliuyyyj);
\coordinate (demobygangliupppak) at (\demobygangliuxxxa, \demobygangliuyyyk);
\coordinate (demobygangliupppal) at (\demobygangliuxxxa, \demobygangliuyyyl);
\coordinate (demobygangliupppam) at (\demobygangliuxxxa, \demobygangliuyyym);
\coordinate (demobygangliupppan) at (\demobygangliuxxxa, \demobygangliuyyyn);
\coordinate (demobygangliupppao) at (\demobygangliuxxxa, \demobygangliuyyyo);
\coordinate (demobygangliupppap) at (\demobygangliuxxxa, \demobygangliuyyyp);
\coordinate (demobygangliupppaq) at (\demobygangliuxxxa, \demobygangliuyyyq);
\coordinate (demobygangliupppar) at (\demobygangliuxxxa, \demobygangliuyyyr);
\coordinate (demobygangliupppas) at (\demobygangliuxxxa, \demobygangliuyyys);
\coordinate (demobygangliupppat) at (\demobygangliuxxxa, \demobygangliuyyyt);
\coordinate (demobygangliupppau) at (\demobygangliuxxxa, \demobygangliuyyyu);
\coordinate (demobygangliupppav) at (\demobygangliuxxxa, \demobygangliuyyyv);
\coordinate (demobygangliupppaw) at (\demobygangliuxxxa, \demobygangliuyyyw);
\coordinate (demobygangliupppax) at (\demobygangliuxxxa, \demobygangliuyyyx);
\coordinate (demobygangliupppay) at (\demobygangliuxxxa, \demobygangliuyyyy);
\coordinate (demobygangliupppaz) at (\demobygangliuxxxa, \demobygangliuyyyz);
\coordinate (demobygangliupppba) at (\demobygangliuxxxb, \demobygangliuyyya);
\coordinate (demobygangliupppbb) at (\demobygangliuxxxb, \demobygangliuyyyb);
\coordinate (demobygangliupppbc) at (\demobygangliuxxxb, \demobygangliuyyyc);
\coordinate (demobygangliupppbd) at (\demobygangliuxxxb, \demobygangliuyyyd);
\coordinate (demobygangliupppbe) at (\demobygangliuxxxb, \demobygangliuyyye);
\coordinate (demobygangliupppbf) at (\demobygangliuxxxb, \demobygangliuyyyf);
\coordinate (demobygangliupppbg) at (\demobygangliuxxxb, \demobygangliuyyyg);
\coordinate (demobygangliupppbh) at (\demobygangliuxxxb, \demobygangliuyyyh);
\coordinate (demobygangliupppbi) at (\demobygangliuxxxb, \demobygangliuyyyi);
\coordinate (demobygangliupppbj) at (\demobygangliuxxxb, \demobygangliuyyyj);
\coordinate (demobygangliupppbk) at (\demobygangliuxxxb, \demobygangliuyyyk);
\coordinate (demobygangliupppbl) at (\demobygangliuxxxb, \demobygangliuyyyl);
\coordinate (demobygangliupppbm) at (\demobygangliuxxxb, \demobygangliuyyym);
\coordinate (demobygangliupppbn) at (\demobygangliuxxxb, \demobygangliuyyyn);
\coordinate (demobygangliupppbo) at (\demobygangliuxxxb, \demobygangliuyyyo);
\coordinate (demobygangliupppbp) at (\demobygangliuxxxb, \demobygangliuyyyp);
\coordinate (demobygangliupppbq) at (\demobygangliuxxxb, \demobygangliuyyyq);
\coordinate (demobygangliupppbr) at (\demobygangliuxxxb, \demobygangliuyyyr);
\coordinate (demobygangliupppbs) at (\demobygangliuxxxb, \demobygangliuyyys);
\coordinate (demobygangliupppbt) at (\demobygangliuxxxb, \demobygangliuyyyt);
\coordinate (demobygangliupppbu) at (\demobygangliuxxxb, \demobygangliuyyyu);
\coordinate (demobygangliupppbv) at (\demobygangliuxxxb, \demobygangliuyyyv);
\coordinate (demobygangliupppbw) at (\demobygangliuxxxb, \demobygangliuyyyw);
\coordinate (demobygangliupppbx) at (\demobygangliuxxxb, \demobygangliuyyyx);
\coordinate (demobygangliupppby) at (\demobygangliuxxxb, \demobygangliuyyyy);
\coordinate (demobygangliupppbz) at (\demobygangliuxxxb, \demobygangliuyyyz);
\coordinate (demobygangliupppca) at (\demobygangliuxxxc, \demobygangliuyyya);
\coordinate (demobygangliupppcb) at (\demobygangliuxxxc, \demobygangliuyyyb);
\coordinate (demobygangliupppcc) at (\demobygangliuxxxc, \demobygangliuyyyc);
\coordinate (demobygangliupppcd) at (\demobygangliuxxxc, \demobygangliuyyyd);
\coordinate (demobygangliupppce) at (\demobygangliuxxxc, \demobygangliuyyye);
\coordinate (demobygangliupppcf) at (\demobygangliuxxxc, \demobygangliuyyyf);
\coordinate (demobygangliupppcg) at (\demobygangliuxxxc, \demobygangliuyyyg);
\coordinate (demobygangliupppch) at (\demobygangliuxxxc, \demobygangliuyyyh);
\coordinate (demobygangliupppci) at (\demobygangliuxxxc, \demobygangliuyyyi);
\coordinate (demobygangliupppcj) at (\demobygangliuxxxc, \demobygangliuyyyj);
\coordinate (demobygangliupppck) at (\demobygangliuxxxc, \demobygangliuyyyk);
\coordinate (demobygangliupppcl) at (\demobygangliuxxxc, \demobygangliuyyyl);
\coordinate (demobygangliupppcm) at (\demobygangliuxxxc, \demobygangliuyyym);
\coordinate (demobygangliupppcn) at (\demobygangliuxxxc, \demobygangliuyyyn);
\coordinate (demobygangliupppco) at (\demobygangliuxxxc, \demobygangliuyyyo);
\coordinate (demobygangliupppcp) at (\demobygangliuxxxc, \demobygangliuyyyp);
\coordinate (demobygangliupppcq) at (\demobygangliuxxxc, \demobygangliuyyyq);
\coordinate (demobygangliupppcr) at (\demobygangliuxxxc, \demobygangliuyyyr);
\coordinate (demobygangliupppcs) at (\demobygangliuxxxc, \demobygangliuyyys);
\coordinate (demobygangliupppct) at (\demobygangliuxxxc, \demobygangliuyyyt);
\coordinate (demobygangliupppcu) at (\demobygangliuxxxc, \demobygangliuyyyu);
\coordinate (demobygangliupppcv) at (\demobygangliuxxxc, \demobygangliuyyyv);
\coordinate (demobygangliupppcw) at (\demobygangliuxxxc, \demobygangliuyyyw);
\coordinate (demobygangliupppcx) at (\demobygangliuxxxc, \demobygangliuyyyx);
\coordinate (demobygangliupppcy) at (\demobygangliuxxxc, \demobygangliuyyyy);
\coordinate (demobygangliupppcz) at (\demobygangliuxxxc, \demobygangliuyyyz);
\coordinate (demobygangliupppda) at (\demobygangliuxxxd, \demobygangliuyyya);
\coordinate (demobygangliupppdb) at (\demobygangliuxxxd, \demobygangliuyyyb);
\coordinate (demobygangliupppdc) at (\demobygangliuxxxd, \demobygangliuyyyc);
\coordinate (demobygangliupppdd) at (\demobygangliuxxxd, \demobygangliuyyyd);
\coordinate (demobygangliupppde) at (\demobygangliuxxxd, \demobygangliuyyye);
\coordinate (demobygangliupppdf) at (\demobygangliuxxxd, \demobygangliuyyyf);
\coordinate (demobygangliupppdg) at (\demobygangliuxxxd, \demobygangliuyyyg);
\coordinate (demobygangliupppdh) at (\demobygangliuxxxd, \demobygangliuyyyh);
\coordinate (demobygangliupppdi) at (\demobygangliuxxxd, \demobygangliuyyyi);
\coordinate (demobygangliupppdj) at (\demobygangliuxxxd, \demobygangliuyyyj);
\coordinate (demobygangliupppdk) at (\demobygangliuxxxd, \demobygangliuyyyk);
\coordinate (demobygangliupppdl) at (\demobygangliuxxxd, \demobygangliuyyyl);
\coordinate (demobygangliupppdm) at (\demobygangliuxxxd, \demobygangliuyyym);
\coordinate (demobygangliupppdn) at (\demobygangliuxxxd, \demobygangliuyyyn);
\coordinate (demobygangliupppdo) at (\demobygangliuxxxd, \demobygangliuyyyo);
\coordinate (demobygangliupppdp) at (\demobygangliuxxxd, \demobygangliuyyyp);
\coordinate (demobygangliupppdq) at (\demobygangliuxxxd, \demobygangliuyyyq);
\coordinate (demobygangliupppdr) at (\demobygangliuxxxd, \demobygangliuyyyr);
\coordinate (demobygangliupppds) at (\demobygangliuxxxd, \demobygangliuyyys);
\coordinate (demobygangliupppdt) at (\demobygangliuxxxd, \demobygangliuyyyt);
\coordinate (demobygangliupppdu) at (\demobygangliuxxxd, \demobygangliuyyyu);
\coordinate (demobygangliupppdv) at (\demobygangliuxxxd, \demobygangliuyyyv);
\coordinate (demobygangliupppdw) at (\demobygangliuxxxd, \demobygangliuyyyw);
\coordinate (demobygangliupppdx) at (\demobygangliuxxxd, \demobygangliuyyyx);
\coordinate (demobygangliupppdy) at (\demobygangliuxxxd, \demobygangliuyyyy);
\coordinate (demobygangliupppdz) at (\demobygangliuxxxd, \demobygangliuyyyz);
\coordinate (demobygangliupppea) at (\demobygangliuxxxe, \demobygangliuyyya);
\coordinate (demobygangliupppeb) at (\demobygangliuxxxe, \demobygangliuyyyb);
\coordinate (demobygangliupppec) at (\demobygangliuxxxe, \demobygangliuyyyc);
\coordinate (demobygangliuppped) at (\demobygangliuxxxe, \demobygangliuyyyd);
\coordinate (demobygangliupppee) at (\demobygangliuxxxe, \demobygangliuyyye);
\coordinate (demobygangliupppef) at (\demobygangliuxxxe, \demobygangliuyyyf);
\coordinate (demobygangliupppeg) at (\demobygangliuxxxe, \demobygangliuyyyg);
\coordinate (demobygangliupppeh) at (\demobygangliuxxxe, \demobygangliuyyyh);
\coordinate (demobygangliupppei) at (\demobygangliuxxxe, \demobygangliuyyyi);
\coordinate (demobygangliupppej) at (\demobygangliuxxxe, \demobygangliuyyyj);
\coordinate (demobygangliupppek) at (\demobygangliuxxxe, \demobygangliuyyyk);
\coordinate (demobygangliupppel) at (\demobygangliuxxxe, \demobygangliuyyyl);
\coordinate (demobygangliupppem) at (\demobygangliuxxxe, \demobygangliuyyym);
\coordinate (demobygangliupppen) at (\demobygangliuxxxe, \demobygangliuyyyn);
\coordinate (demobygangliupppeo) at (\demobygangliuxxxe, \demobygangliuyyyo);
\coordinate (demobygangliupppep) at (\demobygangliuxxxe, \demobygangliuyyyp);
\coordinate (demobygangliupppeq) at (\demobygangliuxxxe, \demobygangliuyyyq);
\coordinate (demobygangliuppper) at (\demobygangliuxxxe, \demobygangliuyyyr);
\coordinate (demobygangliupppes) at (\demobygangliuxxxe, \demobygangliuyyys);
\coordinate (demobygangliupppet) at (\demobygangliuxxxe, \demobygangliuyyyt);
\coordinate (demobygangliupppeu) at (\demobygangliuxxxe, \demobygangliuyyyu);
\coordinate (demobygangliupppev) at (\demobygangliuxxxe, \demobygangliuyyyv);
\coordinate (demobygangliupppew) at (\demobygangliuxxxe, \demobygangliuyyyw);
\coordinate (demobygangliupppex) at (\demobygangliuxxxe, \demobygangliuyyyx);
\coordinate (demobygangliupppey) at (\demobygangliuxxxe, \demobygangliuyyyy);
\coordinate (demobygangliupppez) at (\demobygangliuxxxe, \demobygangliuyyyz);
\coordinate (demobygangliupppfa) at (\demobygangliuxxxf, \demobygangliuyyya);
\coordinate (demobygangliupppfb) at (\demobygangliuxxxf, \demobygangliuyyyb);
\coordinate (demobygangliupppfc) at (\demobygangliuxxxf, \demobygangliuyyyc);
\coordinate (demobygangliupppfd) at (\demobygangliuxxxf, \demobygangliuyyyd);
\coordinate (demobygangliupppfe) at (\demobygangliuxxxf, \demobygangliuyyye);
\coordinate (demobygangliupppff) at (\demobygangliuxxxf, \demobygangliuyyyf);
\coordinate (demobygangliupppfg) at (\demobygangliuxxxf, \demobygangliuyyyg);
\coordinate (demobygangliupppfh) at (\demobygangliuxxxf, \demobygangliuyyyh);
\coordinate (demobygangliupppfi) at (\demobygangliuxxxf, \demobygangliuyyyi);
\coordinate (demobygangliupppfj) at (\demobygangliuxxxf, \demobygangliuyyyj);
\coordinate (demobygangliupppfk) at (\demobygangliuxxxf, \demobygangliuyyyk);
\coordinate (demobygangliupppfl) at (\demobygangliuxxxf, \demobygangliuyyyl);
\coordinate (demobygangliupppfm) at (\demobygangliuxxxf, \demobygangliuyyym);
\coordinate (demobygangliupppfn) at (\demobygangliuxxxf, \demobygangliuyyyn);
\coordinate (demobygangliupppfo) at (\demobygangliuxxxf, \demobygangliuyyyo);
\coordinate (demobygangliupppfp) at (\demobygangliuxxxf, \demobygangliuyyyp);
\coordinate (demobygangliupppfq) at (\demobygangliuxxxf, \demobygangliuyyyq);
\coordinate (demobygangliupppfr) at (\demobygangliuxxxf, \demobygangliuyyyr);
\coordinate (demobygangliupppfs) at (\demobygangliuxxxf, \demobygangliuyyys);
\coordinate (demobygangliupppft) at (\demobygangliuxxxf, \demobygangliuyyyt);
\coordinate (demobygangliupppfu) at (\demobygangliuxxxf, \demobygangliuyyyu);
\coordinate (demobygangliupppfv) at (\demobygangliuxxxf, \demobygangliuyyyv);
\coordinate (demobygangliupppfw) at (\demobygangliuxxxf, \demobygangliuyyyw);
\coordinate (demobygangliupppfx) at (\demobygangliuxxxf, \demobygangliuyyyx);
\coordinate (demobygangliupppfy) at (\demobygangliuxxxf, \demobygangliuyyyy);
\coordinate (demobygangliupppfz) at (\demobygangliuxxxf, \demobygangliuyyyz);
\coordinate (demobygangliupppga) at (\demobygangliuxxxg, \demobygangliuyyya);
\coordinate (demobygangliupppgb) at (\demobygangliuxxxg, \demobygangliuyyyb);
\coordinate (demobygangliupppgc) at (\demobygangliuxxxg, \demobygangliuyyyc);
\coordinate (demobygangliupppgd) at (\demobygangliuxxxg, \demobygangliuyyyd);
\coordinate (demobygangliupppge) at (\demobygangliuxxxg, \demobygangliuyyye);
\coordinate (demobygangliupppgf) at (\demobygangliuxxxg, \demobygangliuyyyf);
\coordinate (demobygangliupppgg) at (\demobygangliuxxxg, \demobygangliuyyyg);
\coordinate (demobygangliupppgh) at (\demobygangliuxxxg, \demobygangliuyyyh);
\coordinate (demobygangliupppgi) at (\demobygangliuxxxg, \demobygangliuyyyi);
\coordinate (demobygangliupppgj) at (\demobygangliuxxxg, \demobygangliuyyyj);
\coordinate (demobygangliupppgk) at (\demobygangliuxxxg, \demobygangliuyyyk);
\coordinate (demobygangliupppgl) at (\demobygangliuxxxg, \demobygangliuyyyl);
\coordinate (demobygangliupppgm) at (\demobygangliuxxxg, \demobygangliuyyym);
\coordinate (demobygangliupppgn) at (\demobygangliuxxxg, \demobygangliuyyyn);
\coordinate (demobygangliupppgo) at (\demobygangliuxxxg, \demobygangliuyyyo);
\coordinate (demobygangliupppgp) at (\demobygangliuxxxg, \demobygangliuyyyp);
\coordinate (demobygangliupppgq) at (\demobygangliuxxxg, \demobygangliuyyyq);
\coordinate (demobygangliupppgr) at (\demobygangliuxxxg, \demobygangliuyyyr);
\coordinate (demobygangliupppgs) at (\demobygangliuxxxg, \demobygangliuyyys);
\coordinate (demobygangliupppgt) at (\demobygangliuxxxg, \demobygangliuyyyt);
\coordinate (demobygangliupppgu) at (\demobygangliuxxxg, \demobygangliuyyyu);
\coordinate (demobygangliupppgv) at (\demobygangliuxxxg, \demobygangliuyyyv);
\coordinate (demobygangliupppgw) at (\demobygangliuxxxg, \demobygangliuyyyw);
\coordinate (demobygangliupppgx) at (\demobygangliuxxxg, \demobygangliuyyyx);
\coordinate (demobygangliupppgy) at (\demobygangliuxxxg, \demobygangliuyyyy);
\coordinate (demobygangliupppgz) at (\demobygangliuxxxg, \demobygangliuyyyz);
\coordinate (demobygangliupppha) at (\demobygangliuxxxh, \demobygangliuyyya);
\coordinate (demobygangliuppphb) at (\demobygangliuxxxh, \demobygangliuyyyb);
\coordinate (demobygangliuppphc) at (\demobygangliuxxxh, \demobygangliuyyyc);
\coordinate (demobygangliuppphd) at (\demobygangliuxxxh, \demobygangliuyyyd);
\coordinate (demobygangliuppphe) at (\demobygangliuxxxh, \demobygangliuyyye);
\coordinate (demobygangliuppphf) at (\demobygangliuxxxh, \demobygangliuyyyf);
\coordinate (demobygangliuppphg) at (\demobygangliuxxxh, \demobygangliuyyyg);
\coordinate (demobygangliuppphh) at (\demobygangliuxxxh, \demobygangliuyyyh);
\coordinate (demobygangliuppphi) at (\demobygangliuxxxh, \demobygangliuyyyi);
\coordinate (demobygangliuppphj) at (\demobygangliuxxxh, \demobygangliuyyyj);
\coordinate (demobygangliuppphk) at (\demobygangliuxxxh, \demobygangliuyyyk);
\coordinate (demobygangliuppphl) at (\demobygangliuxxxh, \demobygangliuyyyl);
\coordinate (demobygangliuppphm) at (\demobygangliuxxxh, \demobygangliuyyym);
\coordinate (demobygangliuppphn) at (\demobygangliuxxxh, \demobygangliuyyyn);
\coordinate (demobygangliupppho) at (\demobygangliuxxxh, \demobygangliuyyyo);
\coordinate (demobygangliuppphp) at (\demobygangliuxxxh, \demobygangliuyyyp);
\coordinate (demobygangliuppphq) at (\demobygangliuxxxh, \demobygangliuyyyq);
\coordinate (demobygangliuppphr) at (\demobygangliuxxxh, \demobygangliuyyyr);
\coordinate (demobygangliuppphs) at (\demobygangliuxxxh, \demobygangliuyyys);
\coordinate (demobygangliupppht) at (\demobygangliuxxxh, \demobygangliuyyyt);
\coordinate (demobygangliuppphu) at (\demobygangliuxxxh, \demobygangliuyyyu);
\coordinate (demobygangliuppphv) at (\demobygangliuxxxh, \demobygangliuyyyv);
\coordinate (demobygangliuppphw) at (\demobygangliuxxxh, \demobygangliuyyyw);
\coordinate (demobygangliuppphx) at (\demobygangliuxxxh, \demobygangliuyyyx);
\coordinate (demobygangliuppphy) at (\demobygangliuxxxh, \demobygangliuyyyy);
\coordinate (demobygangliuppphz) at (\demobygangliuxxxh, \demobygangliuyyyz);
\coordinate (demobygangliupppia) at (\demobygangliuxxxi, \demobygangliuyyya);
\coordinate (demobygangliupppib) at (\demobygangliuxxxi, \demobygangliuyyyb);
\coordinate (demobygangliupppic) at (\demobygangliuxxxi, \demobygangliuyyyc);
\coordinate (demobygangliupppid) at (\demobygangliuxxxi, \demobygangliuyyyd);
\coordinate (demobygangliupppie) at (\demobygangliuxxxi, \demobygangliuyyye);
\coordinate (demobygangliupppif) at (\demobygangliuxxxi, \demobygangliuyyyf);
\coordinate (demobygangliupppig) at (\demobygangliuxxxi, \demobygangliuyyyg);
\coordinate (demobygangliupppih) at (\demobygangliuxxxi, \demobygangliuyyyh);
\coordinate (demobygangliupppii) at (\demobygangliuxxxi, \demobygangliuyyyi);
\coordinate (demobygangliupppij) at (\demobygangliuxxxi, \demobygangliuyyyj);
\coordinate (demobygangliupppik) at (\demobygangliuxxxi, \demobygangliuyyyk);
\coordinate (demobygangliupppil) at (\demobygangliuxxxi, \demobygangliuyyyl);
\coordinate (demobygangliupppim) at (\demobygangliuxxxi, \demobygangliuyyym);
\coordinate (demobygangliupppin) at (\demobygangliuxxxi, \demobygangliuyyyn);
\coordinate (demobygangliupppio) at (\demobygangliuxxxi, \demobygangliuyyyo);
\coordinate (demobygangliupppip) at (\demobygangliuxxxi, \demobygangliuyyyp);
\coordinate (demobygangliupppiq) at (\demobygangliuxxxi, \demobygangliuyyyq);
\coordinate (demobygangliupppir) at (\demobygangliuxxxi, \demobygangliuyyyr);
\coordinate (demobygangliupppis) at (\demobygangliuxxxi, \demobygangliuyyys);
\coordinate (demobygangliupppit) at (\demobygangliuxxxi, \demobygangliuyyyt);
\coordinate (demobygangliupppiu) at (\demobygangliuxxxi, \demobygangliuyyyu);
\coordinate (demobygangliupppiv) at (\demobygangliuxxxi, \demobygangliuyyyv);
\coordinate (demobygangliupppiw) at (\demobygangliuxxxi, \demobygangliuyyyw);
\coordinate (demobygangliupppix) at (\demobygangliuxxxi, \demobygangliuyyyx);
\coordinate (demobygangliupppiy) at (\demobygangliuxxxi, \demobygangliuyyyy);
\coordinate (demobygangliupppiz) at (\demobygangliuxxxi, \demobygangliuyyyz);
\coordinate (demobygangliupppja) at (\demobygangliuxxxj, \demobygangliuyyya);
\coordinate (demobygangliupppjb) at (\demobygangliuxxxj, \demobygangliuyyyb);
\coordinate (demobygangliupppjc) at (\demobygangliuxxxj, \demobygangliuyyyc);
\coordinate (demobygangliupppjd) at (\demobygangliuxxxj, \demobygangliuyyyd);
\coordinate (demobygangliupppje) at (\demobygangliuxxxj, \demobygangliuyyye);
\coordinate (demobygangliupppjf) at (\demobygangliuxxxj, \demobygangliuyyyf);
\coordinate (demobygangliupppjg) at (\demobygangliuxxxj, \demobygangliuyyyg);
\coordinate (demobygangliupppjh) at (\demobygangliuxxxj, \demobygangliuyyyh);
\coordinate (demobygangliupppji) at (\demobygangliuxxxj, \demobygangliuyyyi);
\coordinate (demobygangliupppjj) at (\demobygangliuxxxj, \demobygangliuyyyj);
\coordinate (demobygangliupppjk) at (\demobygangliuxxxj, \demobygangliuyyyk);
\coordinate (demobygangliupppjl) at (\demobygangliuxxxj, \demobygangliuyyyl);
\coordinate (demobygangliupppjm) at (\demobygangliuxxxj, \demobygangliuyyym);
\coordinate (demobygangliupppjn) at (\demobygangliuxxxj, \demobygangliuyyyn);
\coordinate (demobygangliupppjo) at (\demobygangliuxxxj, \demobygangliuyyyo);
\coordinate (demobygangliupppjp) at (\demobygangliuxxxj, \demobygangliuyyyp);
\coordinate (demobygangliupppjq) at (\demobygangliuxxxj, \demobygangliuyyyq);
\coordinate (demobygangliupppjr) at (\demobygangliuxxxj, \demobygangliuyyyr);
\coordinate (demobygangliupppjs) at (\demobygangliuxxxj, \demobygangliuyyys);
\coordinate (demobygangliupppjt) at (\demobygangliuxxxj, \demobygangliuyyyt);
\coordinate (demobygangliupppju) at (\demobygangliuxxxj, \demobygangliuyyyu);
\coordinate (demobygangliupppjv) at (\demobygangliuxxxj, \demobygangliuyyyv);
\coordinate (demobygangliupppjw) at (\demobygangliuxxxj, \demobygangliuyyyw);
\coordinate (demobygangliupppjx) at (\demobygangliuxxxj, \demobygangliuyyyx);
\coordinate (demobygangliupppjy) at (\demobygangliuxxxj, \demobygangliuyyyy);
\coordinate (demobygangliupppjz) at (\demobygangliuxxxj, \demobygangliuyyyz);
\coordinate (demobygangliupppka) at (\demobygangliuxxxk, \demobygangliuyyya);
\coordinate (demobygangliupppkb) at (\demobygangliuxxxk, \demobygangliuyyyb);
\coordinate (demobygangliupppkc) at (\demobygangliuxxxk, \demobygangliuyyyc);
\coordinate (demobygangliupppkd) at (\demobygangliuxxxk, \demobygangliuyyyd);
\coordinate (demobygangliupppke) at (\demobygangliuxxxk, \demobygangliuyyye);
\coordinate (demobygangliupppkf) at (\demobygangliuxxxk, \demobygangliuyyyf);
\coordinate (demobygangliupppkg) at (\demobygangliuxxxk, \demobygangliuyyyg);
\coordinate (demobygangliupppkh) at (\demobygangliuxxxk, \demobygangliuyyyh);
\coordinate (demobygangliupppki) at (\demobygangliuxxxk, \demobygangliuyyyi);
\coordinate (demobygangliupppkj) at (\demobygangliuxxxk, \demobygangliuyyyj);
\coordinate (demobygangliupppkk) at (\demobygangliuxxxk, \demobygangliuyyyk);
\coordinate (demobygangliupppkl) at (\demobygangliuxxxk, \demobygangliuyyyl);
\coordinate (demobygangliupppkm) at (\demobygangliuxxxk, \demobygangliuyyym);
\coordinate (demobygangliupppkn) at (\demobygangliuxxxk, \demobygangliuyyyn);
\coordinate (demobygangliupppko) at (\demobygangliuxxxk, \demobygangliuyyyo);
\coordinate (demobygangliupppkp) at (\demobygangliuxxxk, \demobygangliuyyyp);
\coordinate (demobygangliupppkq) at (\demobygangliuxxxk, \demobygangliuyyyq);
\coordinate (demobygangliupppkr) at (\demobygangliuxxxk, \demobygangliuyyyr);
\coordinate (demobygangliupppks) at (\demobygangliuxxxk, \demobygangliuyyys);
\coordinate (demobygangliupppkt) at (\demobygangliuxxxk, \demobygangliuyyyt);
\coordinate (demobygangliupppku) at (\demobygangliuxxxk, \demobygangliuyyyu);
\coordinate (demobygangliupppkv) at (\demobygangliuxxxk, \demobygangliuyyyv);
\coordinate (demobygangliupppkw) at (\demobygangliuxxxk, \demobygangliuyyyw);
\coordinate (demobygangliupppkx) at (\demobygangliuxxxk, \demobygangliuyyyx);
\coordinate (demobygangliupppky) at (\demobygangliuxxxk, \demobygangliuyyyy);
\coordinate (demobygangliupppkz) at (\demobygangliuxxxk, \demobygangliuyyyz);
\coordinate (demobygangliupppla) at (\demobygangliuxxxl, \demobygangliuyyya);
\coordinate (demobygangliuppplb) at (\demobygangliuxxxl, \demobygangliuyyyb);
\coordinate (demobygangliuppplc) at (\demobygangliuxxxl, \demobygangliuyyyc);
\coordinate (demobygangliupppld) at (\demobygangliuxxxl, \demobygangliuyyyd);
\coordinate (demobygangliuppple) at (\demobygangliuxxxl, \demobygangliuyyye);
\coordinate (demobygangliuppplf) at (\demobygangliuxxxl, \demobygangliuyyyf);
\coordinate (demobygangliuppplg) at (\demobygangliuxxxl, \demobygangliuyyyg);
\coordinate (demobygangliuppplh) at (\demobygangliuxxxl, \demobygangliuyyyh);
\coordinate (demobygangliupppli) at (\demobygangliuxxxl, \demobygangliuyyyi);
\coordinate (demobygangliuppplj) at (\demobygangliuxxxl, \demobygangliuyyyj);
\coordinate (demobygangliuppplk) at (\demobygangliuxxxl, \demobygangliuyyyk);
\coordinate (demobygangliupppll) at (\demobygangliuxxxl, \demobygangliuyyyl);
\coordinate (demobygangliuppplm) at (\demobygangliuxxxl, \demobygangliuyyym);
\coordinate (demobygangliupppln) at (\demobygangliuxxxl, \demobygangliuyyyn);
\coordinate (demobygangliuppplo) at (\demobygangliuxxxl, \demobygangliuyyyo);
\coordinate (demobygangliuppplp) at (\demobygangliuxxxl, \demobygangliuyyyp);
\coordinate (demobygangliuppplq) at (\demobygangliuxxxl, \demobygangliuyyyq);
\coordinate (demobygangliuppplr) at (\demobygangliuxxxl, \demobygangliuyyyr);
\coordinate (demobygangliupppls) at (\demobygangliuxxxl, \demobygangliuyyys);
\coordinate (demobygangliuppplt) at (\demobygangliuxxxl, \demobygangliuyyyt);
\coordinate (demobygangliuppplu) at (\demobygangliuxxxl, \demobygangliuyyyu);
\coordinate (demobygangliuppplv) at (\demobygangliuxxxl, \demobygangliuyyyv);
\coordinate (demobygangliuppplw) at (\demobygangliuxxxl, \demobygangliuyyyw);
\coordinate (demobygangliuppplx) at (\demobygangliuxxxl, \demobygangliuyyyx);
\coordinate (demobygangliuppply) at (\demobygangliuxxxl, \demobygangliuyyyy);
\coordinate (demobygangliuppplz) at (\demobygangliuxxxl, \demobygangliuyyyz);
\coordinate (demobygangliupppma) at (\demobygangliuxxxm, \demobygangliuyyya);
\coordinate (demobygangliupppmb) at (\demobygangliuxxxm, \demobygangliuyyyb);
\coordinate (demobygangliupppmc) at (\demobygangliuxxxm, \demobygangliuyyyc);
\coordinate (demobygangliupppmd) at (\demobygangliuxxxm, \demobygangliuyyyd);
\coordinate (demobygangliupppme) at (\demobygangliuxxxm, \demobygangliuyyye);
\coordinate (demobygangliupppmf) at (\demobygangliuxxxm, \demobygangliuyyyf);
\coordinate (demobygangliupppmg) at (\demobygangliuxxxm, \demobygangliuyyyg);
\coordinate (demobygangliupppmh) at (\demobygangliuxxxm, \demobygangliuyyyh);
\coordinate (demobygangliupppmi) at (\demobygangliuxxxm, \demobygangliuyyyi);
\coordinate (demobygangliupppmj) at (\demobygangliuxxxm, \demobygangliuyyyj);
\coordinate (demobygangliupppmk) at (\demobygangliuxxxm, \demobygangliuyyyk);
\coordinate (demobygangliupppml) at (\demobygangliuxxxm, \demobygangliuyyyl);
\coordinate (demobygangliupppmm) at (\demobygangliuxxxm, \demobygangliuyyym);
\coordinate (demobygangliupppmn) at (\demobygangliuxxxm, \demobygangliuyyyn);
\coordinate (demobygangliupppmo) at (\demobygangliuxxxm, \demobygangliuyyyo);
\coordinate (demobygangliupppmp) at (\demobygangliuxxxm, \demobygangliuyyyp);
\coordinate (demobygangliupppmq) at (\demobygangliuxxxm, \demobygangliuyyyq);
\coordinate (demobygangliupppmr) at (\demobygangliuxxxm, \demobygangliuyyyr);
\coordinate (demobygangliupppms) at (\demobygangliuxxxm, \demobygangliuyyys);
\coordinate (demobygangliupppmt) at (\demobygangliuxxxm, \demobygangliuyyyt);
\coordinate (demobygangliupppmu) at (\demobygangliuxxxm, \demobygangliuyyyu);
\coordinate (demobygangliupppmv) at (\demobygangliuxxxm, \demobygangliuyyyv);
\coordinate (demobygangliupppmw) at (\demobygangliuxxxm, \demobygangliuyyyw);
\coordinate (demobygangliupppmx) at (\demobygangliuxxxm, \demobygangliuyyyx);
\coordinate (demobygangliupppmy) at (\demobygangliuxxxm, \demobygangliuyyyy);
\coordinate (demobygangliupppmz) at (\demobygangliuxxxm, \demobygangliuyyyz);
\coordinate (demobygangliupppna) at (\demobygangliuxxxn, \demobygangliuyyya);
\coordinate (demobygangliupppnb) at (\demobygangliuxxxn, \demobygangliuyyyb);
\coordinate (demobygangliupppnc) at (\demobygangliuxxxn, \demobygangliuyyyc);
\coordinate (demobygangliupppnd) at (\demobygangliuxxxn, \demobygangliuyyyd);
\coordinate (demobygangliupppne) at (\demobygangliuxxxn, \demobygangliuyyye);
\coordinate (demobygangliupppnf) at (\demobygangliuxxxn, \demobygangliuyyyf);
\coordinate (demobygangliupppng) at (\demobygangliuxxxn, \demobygangliuyyyg);
\coordinate (demobygangliupppnh) at (\demobygangliuxxxn, \demobygangliuyyyh);
\coordinate (demobygangliupppni) at (\demobygangliuxxxn, \demobygangliuyyyi);
\coordinate (demobygangliupppnj) at (\demobygangliuxxxn, \demobygangliuyyyj);
\coordinate (demobygangliupppnk) at (\demobygangliuxxxn, \demobygangliuyyyk);
\coordinate (demobygangliupppnl) at (\demobygangliuxxxn, \demobygangliuyyyl);
\coordinate (demobygangliupppnm) at (\demobygangliuxxxn, \demobygangliuyyym);
\coordinate (demobygangliupppnn) at (\demobygangliuxxxn, \demobygangliuyyyn);
\coordinate (demobygangliupppno) at (\demobygangliuxxxn, \demobygangliuyyyo);
\coordinate (demobygangliupppnp) at (\demobygangliuxxxn, \demobygangliuyyyp);
\coordinate (demobygangliupppnq) at (\demobygangliuxxxn, \demobygangliuyyyq);
\coordinate (demobygangliupppnr) at (\demobygangliuxxxn, \demobygangliuyyyr);
\coordinate (demobygangliupppns) at (\demobygangliuxxxn, \demobygangliuyyys);
\coordinate (demobygangliupppnt) at (\demobygangliuxxxn, \demobygangliuyyyt);
\coordinate (demobygangliupppnu) at (\demobygangliuxxxn, \demobygangliuyyyu);
\coordinate (demobygangliupppnv) at (\demobygangliuxxxn, \demobygangliuyyyv);
\coordinate (demobygangliupppnw) at (\demobygangliuxxxn, \demobygangliuyyyw);
\coordinate (demobygangliupppnx) at (\demobygangliuxxxn, \demobygangliuyyyx);
\coordinate (demobygangliupppny) at (\demobygangliuxxxn, \demobygangliuyyyy);
\coordinate (demobygangliupppnz) at (\demobygangliuxxxn, \demobygangliuyyyz);
\coordinate (demobygangliupppoa) at (\demobygangliuxxxo, \demobygangliuyyya);
\coordinate (demobygangliupppob) at (\demobygangliuxxxo, \demobygangliuyyyb);
\coordinate (demobygangliupppoc) at (\demobygangliuxxxo, \demobygangliuyyyc);
\coordinate (demobygangliupppod) at (\demobygangliuxxxo, \demobygangliuyyyd);
\coordinate (demobygangliupppoe) at (\demobygangliuxxxo, \demobygangliuyyye);
\coordinate (demobygangliupppof) at (\demobygangliuxxxo, \demobygangliuyyyf);
\coordinate (demobygangliupppog) at (\demobygangliuxxxo, \demobygangliuyyyg);
\coordinate (demobygangliupppoh) at (\demobygangliuxxxo, \demobygangliuyyyh);
\coordinate (demobygangliupppoi) at (\demobygangliuxxxo, \demobygangliuyyyi);
\coordinate (demobygangliupppoj) at (\demobygangliuxxxo, \demobygangliuyyyj);
\coordinate (demobygangliupppok) at (\demobygangliuxxxo, \demobygangliuyyyk);
\coordinate (demobygangliupppol) at (\demobygangliuxxxo, \demobygangliuyyyl);
\coordinate (demobygangliupppom) at (\demobygangliuxxxo, \demobygangliuyyym);
\coordinate (demobygangliupppon) at (\demobygangliuxxxo, \demobygangliuyyyn);
\coordinate (demobygangliupppoo) at (\demobygangliuxxxo, \demobygangliuyyyo);
\coordinate (demobygangliupppop) at (\demobygangliuxxxo, \demobygangliuyyyp);
\coordinate (demobygangliupppoq) at (\demobygangliuxxxo, \demobygangliuyyyq);
\coordinate (demobygangliupppor) at (\demobygangliuxxxo, \demobygangliuyyyr);
\coordinate (demobygangliupppos) at (\demobygangliuxxxo, \demobygangliuyyys);
\coordinate (demobygangliupppot) at (\demobygangliuxxxo, \demobygangliuyyyt);
\coordinate (demobygangliupppou) at (\demobygangliuxxxo, \demobygangliuyyyu);
\coordinate (demobygangliupppov) at (\demobygangliuxxxo, \demobygangliuyyyv);
\coordinate (demobygangliupppow) at (\demobygangliuxxxo, \demobygangliuyyyw);
\coordinate (demobygangliupppox) at (\demobygangliuxxxo, \demobygangliuyyyx);
\coordinate (demobygangliupppoy) at (\demobygangliuxxxo, \demobygangliuyyyy);
\coordinate (demobygangliupppoz) at (\demobygangliuxxxo, \demobygangliuyyyz);
\coordinate (demobygangliuppppa) at (\demobygangliuxxxp, \demobygangliuyyya);
\coordinate (demobygangliuppppb) at (\demobygangliuxxxp, \demobygangliuyyyb);
\coordinate (demobygangliuppppc) at (\demobygangliuxxxp, \demobygangliuyyyc);
\coordinate (demobygangliuppppd) at (\demobygangliuxxxp, \demobygangliuyyyd);
\coordinate (demobygangliuppppe) at (\demobygangliuxxxp, \demobygangliuyyye);
\coordinate (demobygangliuppppf) at (\demobygangliuxxxp, \demobygangliuyyyf);
\coordinate (demobygangliuppppg) at (\demobygangliuxxxp, \demobygangliuyyyg);
\coordinate (demobygangliupppph) at (\demobygangliuxxxp, \demobygangliuyyyh);
\coordinate (demobygangliuppppi) at (\demobygangliuxxxp, \demobygangliuyyyi);
\coordinate (demobygangliuppppj) at (\demobygangliuxxxp, \demobygangliuyyyj);
\coordinate (demobygangliuppppk) at (\demobygangliuxxxp, \demobygangliuyyyk);
\coordinate (demobygangliuppppl) at (\demobygangliuxxxp, \demobygangliuyyyl);
\coordinate (demobygangliuppppm) at (\demobygangliuxxxp, \demobygangliuyyym);
\coordinate (demobygangliuppppn) at (\demobygangliuxxxp, \demobygangliuyyyn);
\coordinate (demobygangliuppppo) at (\demobygangliuxxxp, \demobygangliuyyyo);
\coordinate (demobygangliuppppp) at (\demobygangliuxxxp, \demobygangliuyyyp);
\coordinate (demobygangliuppppq) at (\demobygangliuxxxp, \demobygangliuyyyq);
\coordinate (demobygangliuppppr) at (\demobygangliuxxxp, \demobygangliuyyyr);
\coordinate (demobygangliupppps) at (\demobygangliuxxxp, \demobygangliuyyys);
\coordinate (demobygangliuppppt) at (\demobygangliuxxxp, \demobygangliuyyyt);
\coordinate (demobygangliuppppu) at (\demobygangliuxxxp, \demobygangliuyyyu);
\coordinate (demobygangliuppppv) at (\demobygangliuxxxp, \demobygangliuyyyv);
\coordinate (demobygangliuppppw) at (\demobygangliuxxxp, \demobygangliuyyyw);
\coordinate (demobygangliuppppx) at (\demobygangliuxxxp, \demobygangliuyyyx);
\coordinate (demobygangliuppppy) at (\demobygangliuxxxp, \demobygangliuyyyy);
\coordinate (demobygangliuppppz) at (\demobygangliuxxxp, \demobygangliuyyyz);
\coordinate (demobygangliupppqa) at (\demobygangliuxxxq, \demobygangliuyyya);
\coordinate (demobygangliupppqb) at (\demobygangliuxxxq, \demobygangliuyyyb);
\coordinate (demobygangliupppqc) at (\demobygangliuxxxq, \demobygangliuyyyc);
\coordinate (demobygangliupppqd) at (\demobygangliuxxxq, \demobygangliuyyyd);
\coordinate (demobygangliupppqe) at (\demobygangliuxxxq, \demobygangliuyyye);
\coordinate (demobygangliupppqf) at (\demobygangliuxxxq, \demobygangliuyyyf);
\coordinate (demobygangliupppqg) at (\demobygangliuxxxq, \demobygangliuyyyg);
\coordinate (demobygangliupppqh) at (\demobygangliuxxxq, \demobygangliuyyyh);
\coordinate (demobygangliupppqi) at (\demobygangliuxxxq, \demobygangliuyyyi);
\coordinate (demobygangliupppqj) at (\demobygangliuxxxq, \demobygangliuyyyj);
\coordinate (demobygangliupppqk) at (\demobygangliuxxxq, \demobygangliuyyyk);
\coordinate (demobygangliupppql) at (\demobygangliuxxxq, \demobygangliuyyyl);
\coordinate (demobygangliupppqm) at (\demobygangliuxxxq, \demobygangliuyyym);
\coordinate (demobygangliupppqn) at (\demobygangliuxxxq, \demobygangliuyyyn);
\coordinate (demobygangliupppqo) at (\demobygangliuxxxq, \demobygangliuyyyo);
\coordinate (demobygangliupppqp) at (\demobygangliuxxxq, \demobygangliuyyyp);
\coordinate (demobygangliupppqq) at (\demobygangliuxxxq, \demobygangliuyyyq);
\coordinate (demobygangliupppqr) at (\demobygangliuxxxq, \demobygangliuyyyr);
\coordinate (demobygangliupppqs) at (\demobygangliuxxxq, \demobygangliuyyys);
\coordinate (demobygangliupppqt) at (\demobygangliuxxxq, \demobygangliuyyyt);
\coordinate (demobygangliupppqu) at (\demobygangliuxxxq, \demobygangliuyyyu);
\coordinate (demobygangliupppqv) at (\demobygangliuxxxq, \demobygangliuyyyv);
\coordinate (demobygangliupppqw) at (\demobygangliuxxxq, \demobygangliuyyyw);
\coordinate (demobygangliupppqx) at (\demobygangliuxxxq, \demobygangliuyyyx);
\coordinate (demobygangliupppqy) at (\demobygangliuxxxq, \demobygangliuyyyy);
\coordinate (demobygangliupppqz) at (\demobygangliuxxxq, \demobygangliuyyyz);
\coordinate (demobygangliupppra) at (\demobygangliuxxxr, \demobygangliuyyya);
\coordinate (demobygangliuppprb) at (\demobygangliuxxxr, \demobygangliuyyyb);
\coordinate (demobygangliuppprc) at (\demobygangliuxxxr, \demobygangliuyyyc);
\coordinate (demobygangliuppprd) at (\demobygangliuxxxr, \demobygangliuyyyd);
\coordinate (demobygangliupppre) at (\demobygangliuxxxr, \demobygangliuyyye);
\coordinate (demobygangliuppprf) at (\demobygangliuxxxr, \demobygangliuyyyf);
\coordinate (demobygangliuppprg) at (\demobygangliuxxxr, \demobygangliuyyyg);
\coordinate (demobygangliuppprh) at (\demobygangliuxxxr, \demobygangliuyyyh);
\coordinate (demobygangliupppri) at (\demobygangliuxxxr, \demobygangliuyyyi);
\coordinate (demobygangliuppprj) at (\demobygangliuxxxr, \demobygangliuyyyj);
\coordinate (demobygangliuppprk) at (\demobygangliuxxxr, \demobygangliuyyyk);
\coordinate (demobygangliuppprl) at (\demobygangliuxxxr, \demobygangliuyyyl);
\coordinate (demobygangliuppprm) at (\demobygangliuxxxr, \demobygangliuyyym);
\coordinate (demobygangliuppprn) at (\demobygangliuxxxr, \demobygangliuyyyn);
\coordinate (demobygangliupppro) at (\demobygangliuxxxr, \demobygangliuyyyo);
\coordinate (demobygangliuppprp) at (\demobygangliuxxxr, \demobygangliuyyyp);
\coordinate (demobygangliuppprq) at (\demobygangliuxxxr, \demobygangliuyyyq);
\coordinate (demobygangliuppprr) at (\demobygangliuxxxr, \demobygangliuyyyr);
\coordinate (demobygangliuppprs) at (\demobygangliuxxxr, \demobygangliuyyys);
\coordinate (demobygangliuppprt) at (\demobygangliuxxxr, \demobygangliuyyyt);
\coordinate (demobygangliupppru) at (\demobygangliuxxxr, \demobygangliuyyyu);
\coordinate (demobygangliuppprv) at (\demobygangliuxxxr, \demobygangliuyyyv);
\coordinate (demobygangliuppprw) at (\demobygangliuxxxr, \demobygangliuyyyw);
\coordinate (demobygangliuppprx) at (\demobygangliuxxxr, \demobygangliuyyyx);
\coordinate (demobygangliupppry) at (\demobygangliuxxxr, \demobygangliuyyyy);
\coordinate (demobygangliuppprz) at (\demobygangliuxxxr, \demobygangliuyyyz);
\coordinate (demobygangliupppsa) at (\demobygangliuxxxs, \demobygangliuyyya);
\coordinate (demobygangliupppsb) at (\demobygangliuxxxs, \demobygangliuyyyb);
\coordinate (demobygangliupppsc) at (\demobygangliuxxxs, \demobygangliuyyyc);
\coordinate (demobygangliupppsd) at (\demobygangliuxxxs, \demobygangliuyyyd);
\coordinate (demobygangliupppse) at (\demobygangliuxxxs, \demobygangliuyyye);
\coordinate (demobygangliupppsf) at (\demobygangliuxxxs, \demobygangliuyyyf);
\coordinate (demobygangliupppsg) at (\demobygangliuxxxs, \demobygangliuyyyg);
\coordinate (demobygangliupppsh) at (\demobygangliuxxxs, \demobygangliuyyyh);
\coordinate (demobygangliupppsi) at (\demobygangliuxxxs, \demobygangliuyyyi);
\coordinate (demobygangliupppsj) at (\demobygangliuxxxs, \demobygangliuyyyj);
\coordinate (demobygangliupppsk) at (\demobygangliuxxxs, \demobygangliuyyyk);
\coordinate (demobygangliupppsl) at (\demobygangliuxxxs, \demobygangliuyyyl);
\coordinate (demobygangliupppsm) at (\demobygangliuxxxs, \demobygangliuyyym);
\coordinate (demobygangliupppsn) at (\demobygangliuxxxs, \demobygangliuyyyn);
\coordinate (demobygangliupppso) at (\demobygangliuxxxs, \demobygangliuyyyo);
\coordinate (demobygangliupppsp) at (\demobygangliuxxxs, \demobygangliuyyyp);
\coordinate (demobygangliupppsq) at (\demobygangliuxxxs, \demobygangliuyyyq);
\coordinate (demobygangliupppsr) at (\demobygangliuxxxs, \demobygangliuyyyr);
\coordinate (demobygangliupppss) at (\demobygangliuxxxs, \demobygangliuyyys);
\coordinate (demobygangliupppst) at (\demobygangliuxxxs, \demobygangliuyyyt);
\coordinate (demobygangliupppsu) at (\demobygangliuxxxs, \demobygangliuyyyu);
\coordinate (demobygangliupppsv) at (\demobygangliuxxxs, \demobygangliuyyyv);
\coordinate (demobygangliupppsw) at (\demobygangliuxxxs, \demobygangliuyyyw);
\coordinate (demobygangliupppsx) at (\demobygangliuxxxs, \demobygangliuyyyx);
\coordinate (demobygangliupppsy) at (\demobygangliuxxxs, \demobygangliuyyyy);
\coordinate (demobygangliupppsz) at (\demobygangliuxxxs, \demobygangliuyyyz);
\coordinate (demobygangliupppta) at (\demobygangliuxxxt, \demobygangliuyyya);
\coordinate (demobygangliuppptb) at (\demobygangliuxxxt, \demobygangliuyyyb);
\coordinate (demobygangliuppptc) at (\demobygangliuxxxt, \demobygangliuyyyc);
\coordinate (demobygangliuppptd) at (\demobygangliuxxxt, \demobygangliuyyyd);
\coordinate (demobygangliupppte) at (\demobygangliuxxxt, \demobygangliuyyye);
\coordinate (demobygangliuppptf) at (\demobygangliuxxxt, \demobygangliuyyyf);
\coordinate (demobygangliuppptg) at (\demobygangliuxxxt, \demobygangliuyyyg);
\coordinate (demobygangliupppth) at (\demobygangliuxxxt, \demobygangliuyyyh);
\coordinate (demobygangliupppti) at (\demobygangliuxxxt, \demobygangliuyyyi);
\coordinate (demobygangliuppptj) at (\demobygangliuxxxt, \demobygangliuyyyj);
\coordinate (demobygangliuppptk) at (\demobygangliuxxxt, \demobygangliuyyyk);
\coordinate (demobygangliuppptl) at (\demobygangliuxxxt, \demobygangliuyyyl);
\coordinate (demobygangliuppptm) at (\demobygangliuxxxt, \demobygangliuyyym);
\coordinate (demobygangliuppptn) at (\demobygangliuxxxt, \demobygangliuyyyn);
\coordinate (demobygangliupppto) at (\demobygangliuxxxt, \demobygangliuyyyo);
\coordinate (demobygangliuppptp) at (\demobygangliuxxxt, \demobygangliuyyyp);
\coordinate (demobygangliuppptq) at (\demobygangliuxxxt, \demobygangliuyyyq);
\coordinate (demobygangliuppptr) at (\demobygangliuxxxt, \demobygangliuyyyr);
\coordinate (demobygangliupppts) at (\demobygangliuxxxt, \demobygangliuyyys);
\coordinate (demobygangliuppptt) at (\demobygangliuxxxt, \demobygangliuyyyt);
\coordinate (demobygangliuppptu) at (\demobygangliuxxxt, \demobygangliuyyyu);
\coordinate (demobygangliuppptv) at (\demobygangliuxxxt, \demobygangliuyyyv);
\coordinate (demobygangliuppptw) at (\demobygangliuxxxt, \demobygangliuyyyw);
\coordinate (demobygangliuppptx) at (\demobygangliuxxxt, \demobygangliuyyyx);
\coordinate (demobygangliupppty) at (\demobygangliuxxxt, \demobygangliuyyyy);
\coordinate (demobygangliuppptz) at (\demobygangliuxxxt, \demobygangliuyyyz);
\coordinate (demobygangliupppua) at (\demobygangliuxxxu, \demobygangliuyyya);
\coordinate (demobygangliupppub) at (\demobygangliuxxxu, \demobygangliuyyyb);
\coordinate (demobygangliupppuc) at (\demobygangliuxxxu, \demobygangliuyyyc);
\coordinate (demobygangliupppud) at (\demobygangliuxxxu, \demobygangliuyyyd);
\coordinate (demobygangliupppue) at (\demobygangliuxxxu, \demobygangliuyyye);
\coordinate (demobygangliupppuf) at (\demobygangliuxxxu, \demobygangliuyyyf);
\coordinate (demobygangliupppug) at (\demobygangliuxxxu, \demobygangliuyyyg);
\coordinate (demobygangliupppuh) at (\demobygangliuxxxu, \demobygangliuyyyh);
\coordinate (demobygangliupppui) at (\demobygangliuxxxu, \demobygangliuyyyi);
\coordinate (demobygangliupppuj) at (\demobygangliuxxxu, \demobygangliuyyyj);
\coordinate (demobygangliupppuk) at (\demobygangliuxxxu, \demobygangliuyyyk);
\coordinate (demobygangliupppul) at (\demobygangliuxxxu, \demobygangliuyyyl);
\coordinate (demobygangliupppum) at (\demobygangliuxxxu, \demobygangliuyyym);
\coordinate (demobygangliupppun) at (\demobygangliuxxxu, \demobygangliuyyyn);
\coordinate (demobygangliupppuo) at (\demobygangliuxxxu, \demobygangliuyyyo);
\coordinate (demobygangliupppup) at (\demobygangliuxxxu, \demobygangliuyyyp);
\coordinate (demobygangliupppuq) at (\demobygangliuxxxu, \demobygangliuyyyq);
\coordinate (demobygangliupppur) at (\demobygangliuxxxu, \demobygangliuyyyr);
\coordinate (demobygangliupppus) at (\demobygangliuxxxu, \demobygangliuyyys);
\coordinate (demobygangliuppput) at (\demobygangliuxxxu, \demobygangliuyyyt);
\coordinate (demobygangliupppuu) at (\demobygangliuxxxu, \demobygangliuyyyu);
\coordinate (demobygangliupppuv) at (\demobygangliuxxxu, \demobygangliuyyyv);
\coordinate (demobygangliupppuw) at (\demobygangliuxxxu, \demobygangliuyyyw);
\coordinate (demobygangliupppux) at (\demobygangliuxxxu, \demobygangliuyyyx);
\coordinate (demobygangliupppuy) at (\demobygangliuxxxu, \demobygangliuyyyy);
\coordinate (demobygangliupppuz) at (\demobygangliuxxxu, \demobygangliuyyyz);
\coordinate (demobygangliupppva) at (\demobygangliuxxxv, \demobygangliuyyya);
\coordinate (demobygangliupppvb) at (\demobygangliuxxxv, \demobygangliuyyyb);
\coordinate (demobygangliupppvc) at (\demobygangliuxxxv, \demobygangliuyyyc);
\coordinate (demobygangliupppvd) at (\demobygangliuxxxv, \demobygangliuyyyd);
\coordinate (demobygangliupppve) at (\demobygangliuxxxv, \demobygangliuyyye);
\coordinate (demobygangliupppvf) at (\demobygangliuxxxv, \demobygangliuyyyf);
\coordinate (demobygangliupppvg) at (\demobygangliuxxxv, \demobygangliuyyyg);
\coordinate (demobygangliupppvh) at (\demobygangliuxxxv, \demobygangliuyyyh);
\coordinate (demobygangliupppvi) at (\demobygangliuxxxv, \demobygangliuyyyi);
\coordinate (demobygangliupppvj) at (\demobygangliuxxxv, \demobygangliuyyyj);
\coordinate (demobygangliupppvk) at (\demobygangliuxxxv, \demobygangliuyyyk);
\coordinate (demobygangliupppvl) at (\demobygangliuxxxv, \demobygangliuyyyl);
\coordinate (demobygangliupppvm) at (\demobygangliuxxxv, \demobygangliuyyym);
\coordinate (demobygangliupppvn) at (\demobygangliuxxxv, \demobygangliuyyyn);
\coordinate (demobygangliupppvo) at (\demobygangliuxxxv, \demobygangliuyyyo);
\coordinate (demobygangliupppvp) at (\demobygangliuxxxv, \demobygangliuyyyp);
\coordinate (demobygangliupppvq) at (\demobygangliuxxxv, \demobygangliuyyyq);
\coordinate (demobygangliupppvr) at (\demobygangliuxxxv, \demobygangliuyyyr);
\coordinate (demobygangliupppvs) at (\demobygangliuxxxv, \demobygangliuyyys);
\coordinate (demobygangliupppvt) at (\demobygangliuxxxv, \demobygangliuyyyt);
\coordinate (demobygangliupppvu) at (\demobygangliuxxxv, \demobygangliuyyyu);
\coordinate (demobygangliupppvv) at (\demobygangliuxxxv, \demobygangliuyyyv);
\coordinate (demobygangliupppvw) at (\demobygangliuxxxv, \demobygangliuyyyw);
\coordinate (demobygangliupppvx) at (\demobygangliuxxxv, \demobygangliuyyyx);
\coordinate (demobygangliupppvy) at (\demobygangliuxxxv, \demobygangliuyyyy);
\coordinate (demobygangliupppvz) at (\demobygangliuxxxv, \demobygangliuyyyz);
\coordinate (demobygangliupppwa) at (\demobygangliuxxxw, \demobygangliuyyya);
\coordinate (demobygangliupppwb) at (\demobygangliuxxxw, \demobygangliuyyyb);
\coordinate (demobygangliupppwc) at (\demobygangliuxxxw, \demobygangliuyyyc);
\coordinate (demobygangliupppwd) at (\demobygangliuxxxw, \demobygangliuyyyd);
\coordinate (demobygangliupppwe) at (\demobygangliuxxxw, \demobygangliuyyye);
\coordinate (demobygangliupppwf) at (\demobygangliuxxxw, \demobygangliuyyyf);
\coordinate (demobygangliupppwg) at (\demobygangliuxxxw, \demobygangliuyyyg);
\coordinate (demobygangliupppwh) at (\demobygangliuxxxw, \demobygangliuyyyh);
\coordinate (demobygangliupppwi) at (\demobygangliuxxxw, \demobygangliuyyyi);
\coordinate (demobygangliupppwj) at (\demobygangliuxxxw, \demobygangliuyyyj);
\coordinate (demobygangliupppwk) at (\demobygangliuxxxw, \demobygangliuyyyk);
\coordinate (demobygangliupppwl) at (\demobygangliuxxxw, \demobygangliuyyyl);
\coordinate (demobygangliupppwm) at (\demobygangliuxxxw, \demobygangliuyyym);
\coordinate (demobygangliupppwn) at (\demobygangliuxxxw, \demobygangliuyyyn);
\coordinate (demobygangliupppwo) at (\demobygangliuxxxw, \demobygangliuyyyo);
\coordinate (demobygangliupppwp) at (\demobygangliuxxxw, \demobygangliuyyyp);
\coordinate (demobygangliupppwq) at (\demobygangliuxxxw, \demobygangliuyyyq);
\coordinate (demobygangliupppwr) at (\demobygangliuxxxw, \demobygangliuyyyr);
\coordinate (demobygangliupppws) at (\demobygangliuxxxw, \demobygangliuyyys);
\coordinate (demobygangliupppwt) at (\demobygangliuxxxw, \demobygangliuyyyt);
\coordinate (demobygangliupppwu) at (\demobygangliuxxxw, \demobygangliuyyyu);
\coordinate (demobygangliupppwv) at (\demobygangliuxxxw, \demobygangliuyyyv);
\coordinate (demobygangliupppww) at (\demobygangliuxxxw, \demobygangliuyyyw);
\coordinate (demobygangliupppwx) at (\demobygangliuxxxw, \demobygangliuyyyx);
\coordinate (demobygangliupppwy) at (\demobygangliuxxxw, \demobygangliuyyyy);
\coordinate (demobygangliupppwz) at (\demobygangliuxxxw, \demobygangliuyyyz);
\coordinate (demobygangliupppxa) at (\demobygangliuxxxx, \demobygangliuyyya);
\coordinate (demobygangliupppxb) at (\demobygangliuxxxx, \demobygangliuyyyb);
\coordinate (demobygangliupppxc) at (\demobygangliuxxxx, \demobygangliuyyyc);
\coordinate (demobygangliupppxd) at (\demobygangliuxxxx, \demobygangliuyyyd);
\coordinate (demobygangliupppxe) at (\demobygangliuxxxx, \demobygangliuyyye);
\coordinate (demobygangliupppxf) at (\demobygangliuxxxx, \demobygangliuyyyf);
\coordinate (demobygangliupppxg) at (\demobygangliuxxxx, \demobygangliuyyyg);
\coordinate (demobygangliupppxh) at (\demobygangliuxxxx, \demobygangliuyyyh);
\coordinate (demobygangliupppxi) at (\demobygangliuxxxx, \demobygangliuyyyi);
\coordinate (demobygangliupppxj) at (\demobygangliuxxxx, \demobygangliuyyyj);
\coordinate (demobygangliupppxk) at (\demobygangliuxxxx, \demobygangliuyyyk);
\coordinate (demobygangliupppxl) at (\demobygangliuxxxx, \demobygangliuyyyl);
\coordinate (demobygangliupppxm) at (\demobygangliuxxxx, \demobygangliuyyym);
\coordinate (demobygangliupppxn) at (\demobygangliuxxxx, \demobygangliuyyyn);
\coordinate (demobygangliupppxo) at (\demobygangliuxxxx, \demobygangliuyyyo);
\coordinate (demobygangliupppxp) at (\demobygangliuxxxx, \demobygangliuyyyp);
\coordinate (demobygangliupppxq) at (\demobygangliuxxxx, \demobygangliuyyyq);
\coordinate (demobygangliupppxr) at (\demobygangliuxxxx, \demobygangliuyyyr);
\coordinate (demobygangliupppxs) at (\demobygangliuxxxx, \demobygangliuyyys);
\coordinate (demobygangliupppxt) at (\demobygangliuxxxx, \demobygangliuyyyt);
\coordinate (demobygangliupppxu) at (\demobygangliuxxxx, \demobygangliuyyyu);
\coordinate (demobygangliupppxv) at (\demobygangliuxxxx, \demobygangliuyyyv);
\coordinate (demobygangliupppxw) at (\demobygangliuxxxx, \demobygangliuyyyw);
\coordinate (demobygangliupppxx) at (\demobygangliuxxxx, \demobygangliuyyyx);
\coordinate (demobygangliupppxy) at (\demobygangliuxxxx, \demobygangliuyyyy);
\coordinate (demobygangliupppxz) at (\demobygangliuxxxx, \demobygangliuyyyz);
\coordinate (demobygangliupppya) at (\demobygangliuxxxy, \demobygangliuyyya);
\coordinate (demobygangliupppyb) at (\demobygangliuxxxy, \demobygangliuyyyb);
\coordinate (demobygangliupppyc) at (\demobygangliuxxxy, \demobygangliuyyyc);
\coordinate (demobygangliupppyd) at (\demobygangliuxxxy, \demobygangliuyyyd);
\coordinate (demobygangliupppye) at (\demobygangliuxxxy, \demobygangliuyyye);
\coordinate (demobygangliupppyf) at (\demobygangliuxxxy, \demobygangliuyyyf);
\coordinate (demobygangliupppyg) at (\demobygangliuxxxy, \demobygangliuyyyg);
\coordinate (demobygangliupppyh) at (\demobygangliuxxxy, \demobygangliuyyyh);
\coordinate (demobygangliupppyi) at (\demobygangliuxxxy, \demobygangliuyyyi);
\coordinate (demobygangliupppyj) at (\demobygangliuxxxy, \demobygangliuyyyj);
\coordinate (demobygangliupppyk) at (\demobygangliuxxxy, \demobygangliuyyyk);
\coordinate (demobygangliupppyl) at (\demobygangliuxxxy, \demobygangliuyyyl);
\coordinate (demobygangliupppym) at (\demobygangliuxxxy, \demobygangliuyyym);
\coordinate (demobygangliupppyn) at (\demobygangliuxxxy, \demobygangliuyyyn);
\coordinate (demobygangliupppyo) at (\demobygangliuxxxy, \demobygangliuyyyo);
\coordinate (demobygangliupppyp) at (\demobygangliuxxxy, \demobygangliuyyyp);
\coordinate (demobygangliupppyq) at (\demobygangliuxxxy, \demobygangliuyyyq);
\coordinate (demobygangliupppyr) at (\demobygangliuxxxy, \demobygangliuyyyr);
\coordinate (demobygangliupppys) at (\demobygangliuxxxy, \demobygangliuyyys);
\coordinate (demobygangliupppyt) at (\demobygangliuxxxy, \demobygangliuyyyt);
\coordinate (demobygangliupppyu) at (\demobygangliuxxxy, \demobygangliuyyyu);
\coordinate (demobygangliupppyv) at (\demobygangliuxxxy, \demobygangliuyyyv);
\coordinate (demobygangliupppyw) at (\demobygangliuxxxy, \demobygangliuyyyw);
\coordinate (demobygangliupppyx) at (\demobygangliuxxxy, \demobygangliuyyyx);
\coordinate (demobygangliupppyy) at (\demobygangliuxxxy, \demobygangliuyyyy);
\coordinate (demobygangliupppyz) at (\demobygangliuxxxy, \demobygangliuyyyz);
\coordinate (demobygangliupppza) at (\demobygangliuxxxz, \demobygangliuyyya);
\coordinate (demobygangliupppzb) at (\demobygangliuxxxz, \demobygangliuyyyb);
\coordinate (demobygangliupppzc) at (\demobygangliuxxxz, \demobygangliuyyyc);
\coordinate (demobygangliupppzd) at (\demobygangliuxxxz, \demobygangliuyyyd);
\coordinate (demobygangliupppze) at (\demobygangliuxxxz, \demobygangliuyyye);
\coordinate (demobygangliupppzf) at (\demobygangliuxxxz, \demobygangliuyyyf);
\coordinate (demobygangliupppzg) at (\demobygangliuxxxz, \demobygangliuyyyg);
\coordinate (demobygangliupppzh) at (\demobygangliuxxxz, \demobygangliuyyyh);
\coordinate (demobygangliupppzi) at (\demobygangliuxxxz, \demobygangliuyyyi);
\coordinate (demobygangliupppzj) at (\demobygangliuxxxz, \demobygangliuyyyj);
\coordinate (demobygangliupppzk) at (\demobygangliuxxxz, \demobygangliuyyyk);
\coordinate (demobygangliupppzl) at (\demobygangliuxxxz, \demobygangliuyyyl);
\coordinate (demobygangliupppzm) at (\demobygangliuxxxz, \demobygangliuyyym);
\coordinate (demobygangliupppzn) at (\demobygangliuxxxz, \demobygangliuyyyn);
\coordinate (demobygangliupppzo) at (\demobygangliuxxxz, \demobygangliuyyyo);
\coordinate (demobygangliupppzp) at (\demobygangliuxxxz, \demobygangliuyyyp);
\coordinate (demobygangliupppzq) at (\demobygangliuxxxz, \demobygangliuyyyq);
\coordinate (demobygangliupppzr) at (\demobygangliuxxxz, \demobygangliuyyyr);
\coordinate (demobygangliupppzs) at (\demobygangliuxxxz, \demobygangliuyyys);
\coordinate (demobygangliupppzt) at (\demobygangliuxxxz, \demobygangliuyyyt);
\coordinate (demobygangliupppzu) at (\demobygangliuxxxz, \demobygangliuyyyu);
\coordinate (demobygangliupppzv) at (\demobygangliuxxxz, \demobygangliuyyyv);
\coordinate (demobygangliupppzw) at (\demobygangliuxxxz, \demobygangliuyyyw);
\coordinate (demobygangliupppzx) at (\demobygangliuxxxz, \demobygangliuyyyx);
\coordinate (demobygangliupppzy) at (\demobygangliuxxxz, \demobygangliuyyyy);
\coordinate (demobygangliupppzz) at (\demobygangliuxxxz, \demobygangliuyyyz);

%\gangprintcoordinateat{(0,0)}{The last coordinate values: }{($(demobygangliupppzz)$)}; 



% Draw related part of the coordinate system with dashed helplines (centered at (demobygangliupppii)) with letters as background, which would help to determine all coordinates. 
%\coordinatebackground{demobygangliu}{c}{d}{o};

% Step 2, draw key devices, their accessories, and take related coordinates of their pins, and may define more coordinates. 

% Draw the Opamp at the coordinate (demobygangliupppii) and name it as "swopamp".
\draw (demobygangliupppii) node [op amp, yscale=-1] (swopamp) {\ctikzflipy{Opamp}} ; 

% Its accessories and lables. 
\draw [-*](swopamp.down) -- ($(swopamp.down)+(0,1)$) node[right]{$V_+$}; 
\node at ($(swopamp.down)+(0.3,0.2)$) {7};  
\draw [-*](swopamp.up) -- ($(swopamp.up)+(0,-1)$) node[right]{$V_-$}; 
\node at ($(swopamp.up)+(0.3,-0.2)$) {4};

% Get the x- and y-components of the coordinates of the "+" and "-" pins. 
\getxyingivenunit{cm}{(swopamp.+)}{\swopampzx}{\swopampzy};
\getxyingivenunit{cm}{(swopamp.-)}{\swopampfx}{\swopampfy};

% Then define a few more coordinates, at least for keeping in mind.
\coordinate (plusshort) at ($(\demobygangliuxxxg,\swopampzy)$);
%\fill  (plusshort) circle (2pt);  % May be commented later.
\coordinate (minusshort) at ($(\demobygangliuxxxg,\swopampfy)$);
%\fill  (minusshort) circle (2pt); % May be commented later.
\coordinate (leftinter) at ($(\demobygangliuxxxe,\swopampzy)$);
\fill  (leftinter) circle (2pt);

% Draw an "npn" at (demobygangliupppmi) and name it as "swQ".
\draw (demobygangliupppmi) node[npn](swQ){};

% Get the x- and y-components of the needed pins of it for later usage.
\getxyingivenunit{cm}{(swQ.C)}{\swQCx}{\swQCy};
\getxyingivenunit{cm}{(swQ.E)}{\swQEx}{\swQEy};

% Then define more coordinate(s).
\coordinate (Qcshort) at ($(\swQEx,\demobygangliuyyyj)$);
%\fill  (Qcshort) circle (2pt); % May be commented later.
\coordinate (Qeshort) at ($(\swQEx,\demobygangliuyyyf)$);
\fill  (Qeshort) circle (2pt) node [right] {$V_0$};

% Then the rectangle by the points (demobygangliupppef) -- (demobygangliupppej) -- (Qcshort) -- (Qeshort) forms a clear area for the key devices. 

% Connect the two devices.
\draw (swopamp.out) to [short, l=$I_B$, above] (swQ.B);

% Step 3, draw other little devices. For tidiness, better to give two units in length for each new device and align them up.

% For this specific circuit, let us attach the four bi-pole devices (maybe with their accessories) to each corner of the above mentioned rectangle area for the key devices, separately. 

\draw  (demobygangliuppped) node [ground] {} to [empty ZZener diode] (demobygangliupppef) -- (leftinter);
% The Latex system can not work properly when I put the following label into the above "[empty ZZener diode]" in the form of an additional "l= ..." option, then I have to employ the following "\node ..." command. Then the label should be aligned with the "ZZener" as much as possible even the coordinate system is modified later. Since the "\demobygangliuyyyd" and "\demobygangliuyyyf" micros are used to position the "ZZener", then better to use the center between them to locate the label, rather than using "\demobygangliuyyye" directly. This idea is also applied in later "\node ..." commands. 
\node at ($(\demobygangliuxxxe-1.3, \demobygangliuyyyd*0.5+\demobygangliuyyyf*0.5)$) {$V_Z = 5\textnormal{V}$};

\draw (demobygangliupppej) to [generic] (demobygangliupppel) -| (swQ.C);
\node at ($(\demobygangliuxxxe-1.1, \demobygangliuyyyj*0.5+\demobygangliuyyyl*0.5)$) {$R_{1}=47k\Omega$};
\node [right] at (\swQCx,\demobygangliuyyyj*0.5+\demobygangliuyyyl*0.5) {$I_C \approx \beta I_B$};

\draw  (\swQEx, \demobygangliuyyyd) node [ground] {} to [generic] (\swQEx, \demobygangliuyyyf) -- (swQ.E);
\node at ($(\swQEx+1.4, \demobygangliuyyyd*0.5+\demobygangliuyyyf*0.5)$) {$R_{E}=100k\Omega$};

% Draw the top area. 
\draw  (\demobygangliuxxxi-0.2,\demobygangliuyyyn) --  (\demobygangliuxxxi+0.2,\demobygangliuyyyn) node [right] {$V_{cc}=15\textnormal{V}$} ;
\draw [->] (demobygangliupppin) -- (demobygangliupppim) node [right] {$I$};  
\draw  (demobygangliupppim) -- (demobygangliupppil);
\fill  (demobygangliupppil) circle (2pt);

% Step 4, other shorts.
\draw  (swopamp.+)  to [short, l_=$I_+ \approx 0 $, above] (plusshort) -- (leftinter) -- (demobygangliupppej);

\draw  (swopamp.-)  to [short, l_=$I_- \approx 0 $, above] (minusshort) |- (Qeshort);

%Step 5, all the rest, especially labels. May also clean unnecessary staff, like the system background and dark points for showing newly defined coordinates previously. 
\draw [->] ($(\swQEx-0.4, \demobygangliuyyyf - 0.4)$) -- node [left] {$I_E$} ($(\swQEx-0.4, \demobygangliuyyyd + 0.4)$);

\draw [->] ($(\demobygangliuxxxe+0.4, \demobygangliuyyyl*0.5+\demobygangliuyyyj*0.5 + 0.6)$) -- node [right] {$I_1$} ($(\demobygangliuxxxe+0.4, \demobygangliuyyyl*0.5+\demobygangliuyyyj*0.5 - 0.6)$);


\end{circuitikz}



































































\newpage

{\Huge Supposing the following two additional circuits will be inserted into the above one at proper positions, then connected.}

\vspace{3cm}

{\Large Figure 5, the first circuit to be inserted.}

\vspace{2cm}

\begin{circuitikz}
% The following is the first circuit, which uses "h" coordinate system, to be inserted into and connected to the above circuit.
% It is modified based on Figure 2.

% Circuits can be drawn by the following five major steps, as shown in the following example. 

% Step 1, preparations. 

% "Install" the coordinate system with keyword "h".
\pgfmathsetmacro{\totalhxxx}{26}
\pgfmathsetmacro{\totalhyyy}{26}
\pgfmathsetmacro{\hxxxspacing}{1}
\pgfmathsetmacro{\hyyyspacing}{1}
\pgfmathsetmacro{\hxxxa}{-2.5}
\pgfmathsetmacro{\hyyya}{06.5}

\pgfmathsetmacro{\hxxxb}{\hxxxa + \hxxxspacing + 0.0 }
\pgfmathsetmacro{\hxxxc}{\hxxxb + \hxxxspacing + 0.0 }
\pgfmathsetmacro{\hxxxd}{\hxxxc + \hxxxspacing + 0.0 }
\pgfmathsetmacro{\hxxxe}{\hxxxd + \hxxxspacing + 0.0 }
\pgfmathsetmacro{\hxxxf}{\hxxxe + \hxxxspacing + 0.0 }
\pgfmathsetmacro{\hxxxg}{\hxxxf + \hxxxspacing + 0.0 }
\pgfmathsetmacro{\hxxxh}{\hxxxg + \hxxxspacing + 0.0 }
\pgfmathsetmacro{\hxxxi}{\hxxxh + \hxxxspacing + 0.0 }
\pgfmathsetmacro{\hxxxj}{\hxxxi + \hxxxspacing + 0.0 }
\pgfmathsetmacro{\hxxxk}{\hxxxj + \hxxxspacing + 0.0 }
\pgfmathsetmacro{\hxxxl}{\hxxxk + \hxxxspacing + 0.0 }
\pgfmathsetmacro{\hxxxm}{\hxxxl + \hxxxspacing + 0.0 }
\pgfmathsetmacro{\hxxxn}{\hxxxm + \hxxxspacing + 0.0 }
\pgfmathsetmacro{\hxxxo}{\hxxxn + \hxxxspacing + 0.0 }
\pgfmathsetmacro{\hxxxp}{\hxxxo + \hxxxspacing + 0.0 }
\pgfmathsetmacro{\hxxxq}{\hxxxp + \hxxxspacing + 0.0 }
\pgfmathsetmacro{\hxxxr}{\hxxxq + \hxxxspacing + 0.0 }
\pgfmathsetmacro{\hxxxs}{\hxxxr + \hxxxspacing + 0.0 }
\pgfmathsetmacro{\hxxxt}{\hxxxs + \hxxxspacing + 0.0 }
\pgfmathsetmacro{\hxxxu}{\hxxxt + \hxxxspacing + 0.0 }
\pgfmathsetmacro{\hxxxv}{\hxxxu + \hxxxspacing + 0.0 }
\pgfmathsetmacro{\hxxxw}{\hxxxv + \hxxxspacing + 0.0 }
\pgfmathsetmacro{\hxxxx}{\hxxxw + \hxxxspacing + 0.0 }
\pgfmathsetmacro{\hxxxy}{\hxxxx + \hxxxspacing + 0.0 }
\pgfmathsetmacro{\hxxxz}{\hxxxy + \hxxxspacing + 0.0 }

\pgfmathsetmacro{\hyyyb}{\hyyya + \hyyyspacing + 0.0 }
\pgfmathsetmacro{\hyyyc}{\hyyyb + \hyyyspacing + 0.0 }
\pgfmathsetmacro{\hyyyd}{\hyyyc + \hyyyspacing + 0.0 }
\pgfmathsetmacro{\hyyye}{\hyyyd + \hyyyspacing + 0.0 }
\pgfmathsetmacro{\hyyyf}{\hyyye + \hyyyspacing + 0.0 }
\pgfmathsetmacro{\hyyyg}{\hyyyf + \hyyyspacing + 0.0 }
\pgfmathsetmacro{\hyyyh}{\hyyyg + \hyyyspacing + 0.0 }
\pgfmathsetmacro{\hyyyi}{\hyyyh + \hyyyspacing  -1.0 }
\pgfmathsetmacro{\hyyyj}{\hyyyi + \hyyyspacing + 0.0 }
\pgfmathsetmacro{\hyyyk}{\hyyyj + \hyyyspacing + 0.0 }
\pgfmathsetmacro{\hyyyl}{\hyyyk + \hyyyspacing + 0.0 }
\pgfmathsetmacro{\hyyym}{\hyyyl + \hyyyspacing + 0.0 }
\pgfmathsetmacro{\hyyyn}{\hyyym + \hyyyspacing + 0.0 }
\pgfmathsetmacro{\hyyyo}{\hyyyn + \hyyyspacing + 0.0 }
\pgfmathsetmacro{\hyyyp}{\hyyyo + \hyyyspacing + 0.0 }
\pgfmathsetmacro{\hyyyq}{\hyyyp + \hyyyspacing + 0.0 }
\pgfmathsetmacro{\hyyyr}{\hyyyq + \hyyyspacing + 0.0 }
\pgfmathsetmacro{\hyyys}{\hyyyr + \hyyyspacing + 0.0 }
\pgfmathsetmacro{\hyyyt}{\hyyys + \hyyyspacing + 0.0 }
\pgfmathsetmacro{\hyyyu}{\hyyyt + \hyyyspacing + 0.0 }
\pgfmathsetmacro{\hyyyv}{\hyyyu + \hyyyspacing + 0.0 }
\pgfmathsetmacro{\hyyyw}{\hyyyv + \hyyyspacing + 0.0 }
\pgfmathsetmacro{\hyyyx}{\hyyyw + \hyyyspacing + 0.0 }
\pgfmathsetmacro{\hyyyy}{\hyyyx + \hyyyspacing + 0.0 }
\pgfmathsetmacro{\hyyyz}{\hyyyy + \hyyyspacing + 0.0 }

\coordinate (hpppaa) at (\hxxxa, \hyyya);
\coordinate (hpppab) at (\hxxxa, \hyyyb);
\coordinate (hpppac) at (\hxxxa, \hyyyc);
\coordinate (hpppad) at (\hxxxa, \hyyyd);
\coordinate (hpppae) at (\hxxxa, \hyyye);
\coordinate (hpppaf) at (\hxxxa, \hyyyf);
\coordinate (hpppag) at (\hxxxa, \hyyyg);
\coordinate (hpppah) at (\hxxxa, \hyyyh);
\coordinate (hpppai) at (\hxxxa, \hyyyi);
\coordinate (hpppaj) at (\hxxxa, \hyyyj);
\coordinate (hpppak) at (\hxxxa, \hyyyk);
\coordinate (hpppal) at (\hxxxa, \hyyyl);
\coordinate (hpppam) at (\hxxxa, \hyyym);
\coordinate (hpppan) at (\hxxxa, \hyyyn);
\coordinate (hpppao) at (\hxxxa, \hyyyo);
\coordinate (hpppap) at (\hxxxa, \hyyyp);
\coordinate (hpppaq) at (\hxxxa, \hyyyq);
\coordinate (hpppar) at (\hxxxa, \hyyyr);
\coordinate (hpppas) at (\hxxxa, \hyyys);
\coordinate (hpppat) at (\hxxxa, \hyyyt);
\coordinate (hpppau) at (\hxxxa, \hyyyu);
\coordinate (hpppav) at (\hxxxa, \hyyyv);
\coordinate (hpppaw) at (\hxxxa, \hyyyw);
\coordinate (hpppax) at (\hxxxa, \hyyyx);
\coordinate (hpppay) at (\hxxxa, \hyyyy);
\coordinate (hpppaz) at (\hxxxa, \hyyyz);
\coordinate (hpppba) at (\hxxxb, \hyyya);
\coordinate (hpppbb) at (\hxxxb, \hyyyb);
\coordinate (hpppbc) at (\hxxxb, \hyyyc);
\coordinate (hpppbd) at (\hxxxb, \hyyyd);
\coordinate (hpppbe) at (\hxxxb, \hyyye);
\coordinate (hpppbf) at (\hxxxb, \hyyyf);
\coordinate (hpppbg) at (\hxxxb, \hyyyg);
\coordinate (hpppbh) at (\hxxxb, \hyyyh);
\coordinate (hpppbi) at (\hxxxb, \hyyyi);
\coordinate (hpppbj) at (\hxxxb, \hyyyj);
\coordinate (hpppbk) at (\hxxxb, \hyyyk);
\coordinate (hpppbl) at (\hxxxb, \hyyyl);
\coordinate (hpppbm) at (\hxxxb, \hyyym);
\coordinate (hpppbn) at (\hxxxb, \hyyyn);
\coordinate (hpppbo) at (\hxxxb, \hyyyo);
\coordinate (hpppbp) at (\hxxxb, \hyyyp);
\coordinate (hpppbq) at (\hxxxb, \hyyyq);
\coordinate (hpppbr) at (\hxxxb, \hyyyr);
\coordinate (hpppbs) at (\hxxxb, \hyyys);
\coordinate (hpppbt) at (\hxxxb, \hyyyt);
\coordinate (hpppbu) at (\hxxxb, \hyyyu);
\coordinate (hpppbv) at (\hxxxb, \hyyyv);
\coordinate (hpppbw) at (\hxxxb, \hyyyw);
\coordinate (hpppbx) at (\hxxxb, \hyyyx);
\coordinate (hpppby) at (\hxxxb, \hyyyy);
\coordinate (hpppbz) at (\hxxxb, \hyyyz);
\coordinate (hpppca) at (\hxxxc, \hyyya);
\coordinate (hpppcb) at (\hxxxc, \hyyyb);
\coordinate (hpppcc) at (\hxxxc, \hyyyc);
\coordinate (hpppcd) at (\hxxxc, \hyyyd);
\coordinate (hpppce) at (\hxxxc, \hyyye);
\coordinate (hpppcf) at (\hxxxc, \hyyyf);
\coordinate (hpppcg) at (\hxxxc, \hyyyg);
\coordinate (hpppch) at (\hxxxc, \hyyyh);
\coordinate (hpppci) at (\hxxxc, \hyyyi);
\coordinate (hpppcj) at (\hxxxc, \hyyyj);
\coordinate (hpppck) at (\hxxxc, \hyyyk);
\coordinate (hpppcl) at (\hxxxc, \hyyyl);
\coordinate (hpppcm) at (\hxxxc, \hyyym);
\coordinate (hpppcn) at (\hxxxc, \hyyyn);
\coordinate (hpppco) at (\hxxxc, \hyyyo);
\coordinate (hpppcp) at (\hxxxc, \hyyyp);
\coordinate (hpppcq) at (\hxxxc, \hyyyq);
\coordinate (hpppcr) at (\hxxxc, \hyyyr);
\coordinate (hpppcs) at (\hxxxc, \hyyys);
\coordinate (hpppct) at (\hxxxc, \hyyyt);
\coordinate (hpppcu) at (\hxxxc, \hyyyu);
\coordinate (hpppcv) at (\hxxxc, \hyyyv);
\coordinate (hpppcw) at (\hxxxc, \hyyyw);
\coordinate (hpppcx) at (\hxxxc, \hyyyx);
\coordinate (hpppcy) at (\hxxxc, \hyyyy);
\coordinate (hpppcz) at (\hxxxc, \hyyyz);
\coordinate (hpppda) at (\hxxxd, \hyyya);
\coordinate (hpppdb) at (\hxxxd, \hyyyb);
\coordinate (hpppdc) at (\hxxxd, \hyyyc);
\coordinate (hpppdd) at (\hxxxd, \hyyyd);
\coordinate (hpppde) at (\hxxxd, \hyyye);
\coordinate (hpppdf) at (\hxxxd, \hyyyf);
\coordinate (hpppdg) at (\hxxxd, \hyyyg);
\coordinate (hpppdh) at (\hxxxd, \hyyyh);
\coordinate (hpppdi) at (\hxxxd, \hyyyi);
\coordinate (hpppdj) at (\hxxxd, \hyyyj);
\coordinate (hpppdk) at (\hxxxd, \hyyyk);
\coordinate (hpppdl) at (\hxxxd, \hyyyl);
\coordinate (hpppdm) at (\hxxxd, \hyyym);
\coordinate (hpppdn) at (\hxxxd, \hyyyn);
\coordinate (hpppdo) at (\hxxxd, \hyyyo);
\coordinate (hpppdp) at (\hxxxd, \hyyyp);
\coordinate (hpppdq) at (\hxxxd, \hyyyq);
\coordinate (hpppdr) at (\hxxxd, \hyyyr);
\coordinate (hpppds) at (\hxxxd, \hyyys);
\coordinate (hpppdt) at (\hxxxd, \hyyyt);
\coordinate (hpppdu) at (\hxxxd, \hyyyu);
\coordinate (hpppdv) at (\hxxxd, \hyyyv);
\coordinate (hpppdw) at (\hxxxd, \hyyyw);
\coordinate (hpppdx) at (\hxxxd, \hyyyx);
\coordinate (hpppdy) at (\hxxxd, \hyyyy);
\coordinate (hpppdz) at (\hxxxd, \hyyyz);
\coordinate (hpppea) at (\hxxxe, \hyyya);
\coordinate (hpppeb) at (\hxxxe, \hyyyb);
\coordinate (hpppec) at (\hxxxe, \hyyyc);
\coordinate (hppped) at (\hxxxe, \hyyyd);
\coordinate (hpppee) at (\hxxxe, \hyyye);
\coordinate (hpppef) at (\hxxxe, \hyyyf);
\coordinate (hpppeg) at (\hxxxe, \hyyyg);
\coordinate (hpppeh) at (\hxxxe, \hyyyh);
\coordinate (hpppei) at (\hxxxe, \hyyyi);
\coordinate (hpppej) at (\hxxxe, \hyyyj);
\coordinate (hpppek) at (\hxxxe, \hyyyk);
\coordinate (hpppel) at (\hxxxe, \hyyyl);
\coordinate (hpppem) at (\hxxxe, \hyyym);
\coordinate (hpppen) at (\hxxxe, \hyyyn);
\coordinate (hpppeo) at (\hxxxe, \hyyyo);
\coordinate (hpppep) at (\hxxxe, \hyyyp);
\coordinate (hpppeq) at (\hxxxe, \hyyyq);
\coordinate (hppper) at (\hxxxe, \hyyyr);
\coordinate (hpppes) at (\hxxxe, \hyyys);
\coordinate (hpppet) at (\hxxxe, \hyyyt);
\coordinate (hpppeu) at (\hxxxe, \hyyyu);
\coordinate (hpppev) at (\hxxxe, \hyyyv);
\coordinate (hpppew) at (\hxxxe, \hyyyw);
\coordinate (hpppex) at (\hxxxe, \hyyyx);
\coordinate (hpppey) at (\hxxxe, \hyyyy);
\coordinate (hpppez) at (\hxxxe, \hyyyz);
\coordinate (hpppfa) at (\hxxxf, \hyyya);
\coordinate (hpppfb) at (\hxxxf, \hyyyb);
\coordinate (hpppfc) at (\hxxxf, \hyyyc);
\coordinate (hpppfd) at (\hxxxf, \hyyyd);
\coordinate (hpppfe) at (\hxxxf, \hyyye);
\coordinate (hpppff) at (\hxxxf, \hyyyf);
\coordinate (hpppfg) at (\hxxxf, \hyyyg);
\coordinate (hpppfh) at (\hxxxf, \hyyyh);
\coordinate (hpppfi) at (\hxxxf, \hyyyi);
\coordinate (hpppfj) at (\hxxxf, \hyyyj);
\coordinate (hpppfk) at (\hxxxf, \hyyyk);
\coordinate (hpppfl) at (\hxxxf, \hyyyl);
\coordinate (hpppfm) at (\hxxxf, \hyyym);
\coordinate (hpppfn) at (\hxxxf, \hyyyn);
\coordinate (hpppfo) at (\hxxxf, \hyyyo);
\coordinate (hpppfp) at (\hxxxf, \hyyyp);
\coordinate (hpppfq) at (\hxxxf, \hyyyq);
\coordinate (hpppfr) at (\hxxxf, \hyyyr);
\coordinate (hpppfs) at (\hxxxf, \hyyys);
\coordinate (hpppft) at (\hxxxf, \hyyyt);
\coordinate (hpppfu) at (\hxxxf, \hyyyu);
\coordinate (hpppfv) at (\hxxxf, \hyyyv);
\coordinate (hpppfw) at (\hxxxf, \hyyyw);
\coordinate (hpppfx) at (\hxxxf, \hyyyx);
\coordinate (hpppfy) at (\hxxxf, \hyyyy);
\coordinate (hpppfz) at (\hxxxf, \hyyyz);
\coordinate (hpppga) at (\hxxxg, \hyyya);
\coordinate (hpppgb) at (\hxxxg, \hyyyb);
\coordinate (hpppgc) at (\hxxxg, \hyyyc);
\coordinate (hpppgd) at (\hxxxg, \hyyyd);
\coordinate (hpppge) at (\hxxxg, \hyyye);
\coordinate (hpppgf) at (\hxxxg, \hyyyf);
\coordinate (hpppgg) at (\hxxxg, \hyyyg);
\coordinate (hpppgh) at (\hxxxg, \hyyyh);
\coordinate (hpppgi) at (\hxxxg, \hyyyi);
\coordinate (hpppgj) at (\hxxxg, \hyyyj);
\coordinate (hpppgk) at (\hxxxg, \hyyyk);
\coordinate (hpppgl) at (\hxxxg, \hyyyl);
\coordinate (hpppgm) at (\hxxxg, \hyyym);
\coordinate (hpppgn) at (\hxxxg, \hyyyn);
\coordinate (hpppgo) at (\hxxxg, \hyyyo);
\coordinate (hpppgp) at (\hxxxg, \hyyyp);
\coordinate (hpppgq) at (\hxxxg, \hyyyq);
\coordinate (hpppgr) at (\hxxxg, \hyyyr);
\coordinate (hpppgs) at (\hxxxg, \hyyys);
\coordinate (hpppgt) at (\hxxxg, \hyyyt);
\coordinate (hpppgu) at (\hxxxg, \hyyyu);
\coordinate (hpppgv) at (\hxxxg, \hyyyv);
\coordinate (hpppgw) at (\hxxxg, \hyyyw);
\coordinate (hpppgx) at (\hxxxg, \hyyyx);
\coordinate (hpppgy) at (\hxxxg, \hyyyy);
\coordinate (hpppgz) at (\hxxxg, \hyyyz);
\coordinate (hpppha) at (\hxxxh, \hyyya);
\coordinate (hppphb) at (\hxxxh, \hyyyb);
\coordinate (hppphc) at (\hxxxh, \hyyyc);
\coordinate (hppphd) at (\hxxxh, \hyyyd);
\coordinate (hppphe) at (\hxxxh, \hyyye);
\coordinate (hppphf) at (\hxxxh, \hyyyf);
\coordinate (hppphg) at (\hxxxh, \hyyyg);
\coordinate (hppphh) at (\hxxxh, \hyyyh);
\coordinate (hppphi) at (\hxxxh, \hyyyi);
\coordinate (hppphj) at (\hxxxh, \hyyyj);
\coordinate (hppphk) at (\hxxxh, \hyyyk);
\coordinate (hppphl) at (\hxxxh, \hyyyl);
\coordinate (hppphm) at (\hxxxh, \hyyym);
\coordinate (hppphn) at (\hxxxh, \hyyyn);
\coordinate (hpppho) at (\hxxxh, \hyyyo);
\coordinate (hppphp) at (\hxxxh, \hyyyp);
\coordinate (hppphq) at (\hxxxh, \hyyyq);
\coordinate (hppphr) at (\hxxxh, \hyyyr);
\coordinate (hppphs) at (\hxxxh, \hyyys);
\coordinate (hpppht) at (\hxxxh, \hyyyt);
\coordinate (hppphu) at (\hxxxh, \hyyyu);
\coordinate (hppphv) at (\hxxxh, \hyyyv);
\coordinate (hppphw) at (\hxxxh, \hyyyw);
\coordinate (hppphx) at (\hxxxh, \hyyyx);
\coordinate (hppphy) at (\hxxxh, \hyyyy);
\coordinate (hppphz) at (\hxxxh, \hyyyz);
\coordinate (hpppia) at (\hxxxi, \hyyya);
\coordinate (hpppib) at (\hxxxi, \hyyyb);
\coordinate (hpppic) at (\hxxxi, \hyyyc);
\coordinate (hpppid) at (\hxxxi, \hyyyd);
\coordinate (hpppie) at (\hxxxi, \hyyye);
\coordinate (hpppif) at (\hxxxi, \hyyyf);
\coordinate (hpppig) at (\hxxxi, \hyyyg);
\coordinate (hpppih) at (\hxxxi, \hyyyh);
\coordinate (hpppii) at (\hxxxi, \hyyyi);
\coordinate (hpppij) at (\hxxxi, \hyyyj);
\coordinate (hpppik) at (\hxxxi, \hyyyk);
\coordinate (hpppil) at (\hxxxi, \hyyyl);
\coordinate (hpppim) at (\hxxxi, \hyyym);
\coordinate (hpppin) at (\hxxxi, \hyyyn);
\coordinate (hpppio) at (\hxxxi, \hyyyo);
\coordinate (hpppip) at (\hxxxi, \hyyyp);
\coordinate (hpppiq) at (\hxxxi, \hyyyq);
\coordinate (hpppir) at (\hxxxi, \hyyyr);
\coordinate (hpppis) at (\hxxxi, \hyyys);
\coordinate (hpppit) at (\hxxxi, \hyyyt);
\coordinate (hpppiu) at (\hxxxi, \hyyyu);
\coordinate (hpppiv) at (\hxxxi, \hyyyv);
\coordinate (hpppiw) at (\hxxxi, \hyyyw);
\coordinate (hpppix) at (\hxxxi, \hyyyx);
\coordinate (hpppiy) at (\hxxxi, \hyyyy);
\coordinate (hpppiz) at (\hxxxi, \hyyyz);
\coordinate (hpppja) at (\hxxxj, \hyyya);
\coordinate (hpppjb) at (\hxxxj, \hyyyb);
\coordinate (hpppjc) at (\hxxxj, \hyyyc);
\coordinate (hpppjd) at (\hxxxj, \hyyyd);
\coordinate (hpppje) at (\hxxxj, \hyyye);
\coordinate (hpppjf) at (\hxxxj, \hyyyf);
\coordinate (hpppjg) at (\hxxxj, \hyyyg);
\coordinate (hpppjh) at (\hxxxj, \hyyyh);
\coordinate (hpppji) at (\hxxxj, \hyyyi);
\coordinate (hpppjj) at (\hxxxj, \hyyyj);
\coordinate (hpppjk) at (\hxxxj, \hyyyk);
\coordinate (hpppjl) at (\hxxxj, \hyyyl);
\coordinate (hpppjm) at (\hxxxj, \hyyym);
\coordinate (hpppjn) at (\hxxxj, \hyyyn);
\coordinate (hpppjo) at (\hxxxj, \hyyyo);
\coordinate (hpppjp) at (\hxxxj, \hyyyp);
\coordinate (hpppjq) at (\hxxxj, \hyyyq);
\coordinate (hpppjr) at (\hxxxj, \hyyyr);
\coordinate (hpppjs) at (\hxxxj, \hyyys);
\coordinate (hpppjt) at (\hxxxj, \hyyyt);
\coordinate (hpppju) at (\hxxxj, \hyyyu);
\coordinate (hpppjv) at (\hxxxj, \hyyyv);
\coordinate (hpppjw) at (\hxxxj, \hyyyw);
\coordinate (hpppjx) at (\hxxxj, \hyyyx);
\coordinate (hpppjy) at (\hxxxj, \hyyyy);
\coordinate (hpppjz) at (\hxxxj, \hyyyz);
\coordinate (hpppka) at (\hxxxk, \hyyya);
\coordinate (hpppkb) at (\hxxxk, \hyyyb);
\coordinate (hpppkc) at (\hxxxk, \hyyyc);
\coordinate (hpppkd) at (\hxxxk, \hyyyd);
\coordinate (hpppke) at (\hxxxk, \hyyye);
\coordinate (hpppkf) at (\hxxxk, \hyyyf);
\coordinate (hpppkg) at (\hxxxk, \hyyyg);
\coordinate (hpppkh) at (\hxxxk, \hyyyh);
\coordinate (hpppki) at (\hxxxk, \hyyyi);
\coordinate (hpppkj) at (\hxxxk, \hyyyj);
\coordinate (hpppkk) at (\hxxxk, \hyyyk);
\coordinate (hpppkl) at (\hxxxk, \hyyyl);
\coordinate (hpppkm) at (\hxxxk, \hyyym);
\coordinate (hpppkn) at (\hxxxk, \hyyyn);
\coordinate (hpppko) at (\hxxxk, \hyyyo);
\coordinate (hpppkp) at (\hxxxk, \hyyyp);
\coordinate (hpppkq) at (\hxxxk, \hyyyq);
\coordinate (hpppkr) at (\hxxxk, \hyyyr);
\coordinate (hpppks) at (\hxxxk, \hyyys);
\coordinate (hpppkt) at (\hxxxk, \hyyyt);
\coordinate (hpppku) at (\hxxxk, \hyyyu);
\coordinate (hpppkv) at (\hxxxk, \hyyyv);
\coordinate (hpppkw) at (\hxxxk, \hyyyw);
\coordinate (hpppkx) at (\hxxxk, \hyyyx);
\coordinate (hpppky) at (\hxxxk, \hyyyy);
\coordinate (hpppkz) at (\hxxxk, \hyyyz);
\coordinate (hpppla) at (\hxxxl, \hyyya);
\coordinate (hppplb) at (\hxxxl, \hyyyb);
\coordinate (hppplc) at (\hxxxl, \hyyyc);
\coordinate (hpppld) at (\hxxxl, \hyyyd);
\coordinate (hppple) at (\hxxxl, \hyyye);
\coordinate (hppplf) at (\hxxxl, \hyyyf);
\coordinate (hppplg) at (\hxxxl, \hyyyg);
\coordinate (hppplh) at (\hxxxl, \hyyyh);
\coordinate (hpppli) at (\hxxxl, \hyyyi);
\coordinate (hppplj) at (\hxxxl, \hyyyj);
\coordinate (hppplk) at (\hxxxl, \hyyyk);
\coordinate (hpppll) at (\hxxxl, \hyyyl);
\coordinate (hppplm) at (\hxxxl, \hyyym);
\coordinate (hpppln) at (\hxxxl, \hyyyn);
\coordinate (hppplo) at (\hxxxl, \hyyyo);
\coordinate (hppplp) at (\hxxxl, \hyyyp);
\coordinate (hppplq) at (\hxxxl, \hyyyq);
\coordinate (hppplr) at (\hxxxl, \hyyyr);
\coordinate (hpppls) at (\hxxxl, \hyyys);
\coordinate (hppplt) at (\hxxxl, \hyyyt);
\coordinate (hppplu) at (\hxxxl, \hyyyu);
\coordinate (hppplv) at (\hxxxl, \hyyyv);
\coordinate (hppplw) at (\hxxxl, \hyyyw);
\coordinate (hppplx) at (\hxxxl, \hyyyx);
\coordinate (hppply) at (\hxxxl, \hyyyy);
\coordinate (hppplz) at (\hxxxl, \hyyyz);
\coordinate (hpppma) at (\hxxxm, \hyyya);
\coordinate (hpppmb) at (\hxxxm, \hyyyb);
\coordinate (hpppmc) at (\hxxxm, \hyyyc);
\coordinate (hpppmd) at (\hxxxm, \hyyyd);
\coordinate (hpppme) at (\hxxxm, \hyyye);
\coordinate (hpppmf) at (\hxxxm, \hyyyf);
\coordinate (hpppmg) at (\hxxxm, \hyyyg);
\coordinate (hpppmh) at (\hxxxm, \hyyyh);
\coordinate (hpppmi) at (\hxxxm, \hyyyi);
\coordinate (hpppmj) at (\hxxxm, \hyyyj);
\coordinate (hpppmk) at (\hxxxm, \hyyyk);
\coordinate (hpppml) at (\hxxxm, \hyyyl);
\coordinate (hpppmm) at (\hxxxm, \hyyym);
\coordinate (hpppmn) at (\hxxxm, \hyyyn);
\coordinate (hpppmo) at (\hxxxm, \hyyyo);
\coordinate (hpppmp) at (\hxxxm, \hyyyp);
\coordinate (hpppmq) at (\hxxxm, \hyyyq);
\coordinate (hpppmr) at (\hxxxm, \hyyyr);
\coordinate (hpppms) at (\hxxxm, \hyyys);
\coordinate (hpppmt) at (\hxxxm, \hyyyt);
\coordinate (hpppmu) at (\hxxxm, \hyyyu);
\coordinate (hpppmv) at (\hxxxm, \hyyyv);
\coordinate (hpppmw) at (\hxxxm, \hyyyw);
\coordinate (hpppmx) at (\hxxxm, \hyyyx);
\coordinate (hpppmy) at (\hxxxm, \hyyyy);
\coordinate (hpppmz) at (\hxxxm, \hyyyz);
\coordinate (hpppna) at (\hxxxn, \hyyya);
\coordinate (hpppnb) at (\hxxxn, \hyyyb);
\coordinate (hpppnc) at (\hxxxn, \hyyyc);
\coordinate (hpppnd) at (\hxxxn, \hyyyd);
\coordinate (hpppne) at (\hxxxn, \hyyye);
\coordinate (hpppnf) at (\hxxxn, \hyyyf);
\coordinate (hpppng) at (\hxxxn, \hyyyg);
\coordinate (hpppnh) at (\hxxxn, \hyyyh);
\coordinate (hpppni) at (\hxxxn, \hyyyi);
\coordinate (hpppnj) at (\hxxxn, \hyyyj);
\coordinate (hpppnk) at (\hxxxn, \hyyyk);
\coordinate (hpppnl) at (\hxxxn, \hyyyl);
\coordinate (hpppnm) at (\hxxxn, \hyyym);
\coordinate (hpppnn) at (\hxxxn, \hyyyn);
\coordinate (hpppno) at (\hxxxn, \hyyyo);
\coordinate (hpppnp) at (\hxxxn, \hyyyp);
\coordinate (hpppnq) at (\hxxxn, \hyyyq);
\coordinate (hpppnr) at (\hxxxn, \hyyyr);
\coordinate (hpppns) at (\hxxxn, \hyyys);
\coordinate (hpppnt) at (\hxxxn, \hyyyt);
\coordinate (hpppnu) at (\hxxxn, \hyyyu);
\coordinate (hpppnv) at (\hxxxn, \hyyyv);
\coordinate (hpppnw) at (\hxxxn, \hyyyw);
\coordinate (hpppnx) at (\hxxxn, \hyyyx);
\coordinate (hpppny) at (\hxxxn, \hyyyy);
\coordinate (hpppnz) at (\hxxxn, \hyyyz);
\coordinate (hpppoa) at (\hxxxo, \hyyya);
\coordinate (hpppob) at (\hxxxo, \hyyyb);
\coordinate (hpppoc) at (\hxxxo, \hyyyc);
\coordinate (hpppod) at (\hxxxo, \hyyyd);
\coordinate (hpppoe) at (\hxxxo, \hyyye);
\coordinate (hpppof) at (\hxxxo, \hyyyf);
\coordinate (hpppog) at (\hxxxo, \hyyyg);
\coordinate (hpppoh) at (\hxxxo, \hyyyh);
\coordinate (hpppoi) at (\hxxxo, \hyyyi);
\coordinate (hpppoj) at (\hxxxo, \hyyyj);
\coordinate (hpppok) at (\hxxxo, \hyyyk);
\coordinate (hpppol) at (\hxxxo, \hyyyl);
\coordinate (hpppom) at (\hxxxo, \hyyym);
\coordinate (hpppon) at (\hxxxo, \hyyyn);
\coordinate (hpppoo) at (\hxxxo, \hyyyo);
\coordinate (hpppop) at (\hxxxo, \hyyyp);
\coordinate (hpppoq) at (\hxxxo, \hyyyq);
\coordinate (hpppor) at (\hxxxo, \hyyyr);
\coordinate (hpppos) at (\hxxxo, \hyyys);
\coordinate (hpppot) at (\hxxxo, \hyyyt);
\coordinate (hpppou) at (\hxxxo, \hyyyu);
\coordinate (hpppov) at (\hxxxo, \hyyyv);
\coordinate (hpppow) at (\hxxxo, \hyyyw);
\coordinate (hpppox) at (\hxxxo, \hyyyx);
\coordinate (hpppoy) at (\hxxxo, \hyyyy);
\coordinate (hpppoz) at (\hxxxo, \hyyyz);
\coordinate (hppppa) at (\hxxxp, \hyyya);
\coordinate (hppppb) at (\hxxxp, \hyyyb);
\coordinate (hppppc) at (\hxxxp, \hyyyc);
\coordinate (hppppd) at (\hxxxp, \hyyyd);
\coordinate (hppppe) at (\hxxxp, \hyyye);
\coordinate (hppppf) at (\hxxxp, \hyyyf);
\coordinate (hppppg) at (\hxxxp, \hyyyg);
\coordinate (hpppph) at (\hxxxp, \hyyyh);
\coordinate (hppppi) at (\hxxxp, \hyyyi);
\coordinate (hppppj) at (\hxxxp, \hyyyj);
\coordinate (hppppk) at (\hxxxp, \hyyyk);
\coordinate (hppppl) at (\hxxxp, \hyyyl);
\coordinate (hppppm) at (\hxxxp, \hyyym);
\coordinate (hppppn) at (\hxxxp, \hyyyn);
\coordinate (hppppo) at (\hxxxp, \hyyyo);
\coordinate (hppppp) at (\hxxxp, \hyyyp);
\coordinate (hppppq) at (\hxxxp, \hyyyq);
\coordinate (hppppr) at (\hxxxp, \hyyyr);
\coordinate (hpppps) at (\hxxxp, \hyyys);
\coordinate (hppppt) at (\hxxxp, \hyyyt);
\coordinate (hppppu) at (\hxxxp, \hyyyu);
\coordinate (hppppv) at (\hxxxp, \hyyyv);
\coordinate (hppppw) at (\hxxxp, \hyyyw);
\coordinate (hppppx) at (\hxxxp, \hyyyx);
\coordinate (hppppy) at (\hxxxp, \hyyyy);
\coordinate (hppppz) at (\hxxxp, \hyyyz);
\coordinate (hpppqa) at (\hxxxq, \hyyya);
\coordinate (hpppqb) at (\hxxxq, \hyyyb);
\coordinate (hpppqc) at (\hxxxq, \hyyyc);
\coordinate (hpppqd) at (\hxxxq, \hyyyd);
\coordinate (hpppqe) at (\hxxxq, \hyyye);
\coordinate (hpppqf) at (\hxxxq, \hyyyf);
\coordinate (hpppqg) at (\hxxxq, \hyyyg);
\coordinate (hpppqh) at (\hxxxq, \hyyyh);
\coordinate (hpppqi) at (\hxxxq, \hyyyi);
\coordinate (hpppqj) at (\hxxxq, \hyyyj);
\coordinate (hpppqk) at (\hxxxq, \hyyyk);
\coordinate (hpppql) at (\hxxxq, \hyyyl);
\coordinate (hpppqm) at (\hxxxq, \hyyym);
\coordinate (hpppqn) at (\hxxxq, \hyyyn);
\coordinate (hpppqo) at (\hxxxq, \hyyyo);
\coordinate (hpppqp) at (\hxxxq, \hyyyp);
\coordinate (hpppqq) at (\hxxxq, \hyyyq);
\coordinate (hpppqr) at (\hxxxq, \hyyyr);
\coordinate (hpppqs) at (\hxxxq, \hyyys);
\coordinate (hpppqt) at (\hxxxq, \hyyyt);
\coordinate (hpppqu) at (\hxxxq, \hyyyu);
\coordinate (hpppqv) at (\hxxxq, \hyyyv);
\coordinate (hpppqw) at (\hxxxq, \hyyyw);
\coordinate (hpppqx) at (\hxxxq, \hyyyx);
\coordinate (hpppqy) at (\hxxxq, \hyyyy);
\coordinate (hpppqz) at (\hxxxq, \hyyyz);
\coordinate (hpppra) at (\hxxxr, \hyyya);
\coordinate (hppprb) at (\hxxxr, \hyyyb);
\coordinate (hppprc) at (\hxxxr, \hyyyc);
\coordinate (hppprd) at (\hxxxr, \hyyyd);
\coordinate (hpppre) at (\hxxxr, \hyyye);
\coordinate (hppprf) at (\hxxxr, \hyyyf);
\coordinate (hppprg) at (\hxxxr, \hyyyg);
\coordinate (hppprh) at (\hxxxr, \hyyyh);
\coordinate (hpppri) at (\hxxxr, \hyyyi);
\coordinate (hppprj) at (\hxxxr, \hyyyj);
\coordinate (hppprk) at (\hxxxr, \hyyyk);
\coordinate (hppprl) at (\hxxxr, \hyyyl);
\coordinate (hppprm) at (\hxxxr, \hyyym);
\coordinate (hppprn) at (\hxxxr, \hyyyn);
\coordinate (hpppro) at (\hxxxr, \hyyyo);
\coordinate (hppprp) at (\hxxxr, \hyyyp);
\coordinate (hppprq) at (\hxxxr, \hyyyq);
\coordinate (hppprr) at (\hxxxr, \hyyyr);
\coordinate (hppprs) at (\hxxxr, \hyyys);
\coordinate (hppprt) at (\hxxxr, \hyyyt);
\coordinate (hpppru) at (\hxxxr, \hyyyu);
\coordinate (hppprv) at (\hxxxr, \hyyyv);
\coordinate (hppprw) at (\hxxxr, \hyyyw);
\coordinate (hppprx) at (\hxxxr, \hyyyx);
\coordinate (hpppry) at (\hxxxr, \hyyyy);
\coordinate (hppprz) at (\hxxxr, \hyyyz);
\coordinate (hpppsa) at (\hxxxs, \hyyya);
\coordinate (hpppsb) at (\hxxxs, \hyyyb);
\coordinate (hpppsc) at (\hxxxs, \hyyyc);
\coordinate (hpppsd) at (\hxxxs, \hyyyd);
\coordinate (hpppse) at (\hxxxs, \hyyye);
\coordinate (hpppsf) at (\hxxxs, \hyyyf);
\coordinate (hpppsg) at (\hxxxs, \hyyyg);
\coordinate (hpppsh) at (\hxxxs, \hyyyh);
\coordinate (hpppsi) at (\hxxxs, \hyyyi);
\coordinate (hpppsj) at (\hxxxs, \hyyyj);
\coordinate (hpppsk) at (\hxxxs, \hyyyk);
\coordinate (hpppsl) at (\hxxxs, \hyyyl);
\coordinate (hpppsm) at (\hxxxs, \hyyym);
\coordinate (hpppsn) at (\hxxxs, \hyyyn);
\coordinate (hpppso) at (\hxxxs, \hyyyo);
\coordinate (hpppsp) at (\hxxxs, \hyyyp);
\coordinate (hpppsq) at (\hxxxs, \hyyyq);
\coordinate (hpppsr) at (\hxxxs, \hyyyr);
\coordinate (hpppss) at (\hxxxs, \hyyys);
\coordinate (hpppst) at (\hxxxs, \hyyyt);
\coordinate (hpppsu) at (\hxxxs, \hyyyu);
\coordinate (hpppsv) at (\hxxxs, \hyyyv);
\coordinate (hpppsw) at (\hxxxs, \hyyyw);
\coordinate (hpppsx) at (\hxxxs, \hyyyx);
\coordinate (hpppsy) at (\hxxxs, \hyyyy);
\coordinate (hpppsz) at (\hxxxs, \hyyyz);
\coordinate (hpppta) at (\hxxxt, \hyyya);
\coordinate (hppptb) at (\hxxxt, \hyyyb);
\coordinate (hppptc) at (\hxxxt, \hyyyc);
\coordinate (hppptd) at (\hxxxt, \hyyyd);
\coordinate (hpppte) at (\hxxxt, \hyyye);
\coordinate (hppptf) at (\hxxxt, \hyyyf);
\coordinate (hppptg) at (\hxxxt, \hyyyg);
\coordinate (hpppth) at (\hxxxt, \hyyyh);
\coordinate (hpppti) at (\hxxxt, \hyyyi);
\coordinate (hppptj) at (\hxxxt, \hyyyj);
\coordinate (hppptk) at (\hxxxt, \hyyyk);
\coordinate (hppptl) at (\hxxxt, \hyyyl);
\coordinate (hppptm) at (\hxxxt, \hyyym);
\coordinate (hppptn) at (\hxxxt, \hyyyn);
\coordinate (hpppto) at (\hxxxt, \hyyyo);
\coordinate (hppptp) at (\hxxxt, \hyyyp);
\coordinate (hppptq) at (\hxxxt, \hyyyq);
\coordinate (hppptr) at (\hxxxt, \hyyyr);
\coordinate (hpppts) at (\hxxxt, \hyyys);
\coordinate (hppptt) at (\hxxxt, \hyyyt);
\coordinate (hppptu) at (\hxxxt, \hyyyu);
\coordinate (hppptv) at (\hxxxt, \hyyyv);
\coordinate (hppptw) at (\hxxxt, \hyyyw);
\coordinate (hppptx) at (\hxxxt, \hyyyx);
\coordinate (hpppty) at (\hxxxt, \hyyyy);
\coordinate (hppptz) at (\hxxxt, \hyyyz);
\coordinate (hpppua) at (\hxxxu, \hyyya);
\coordinate (hpppub) at (\hxxxu, \hyyyb);
\coordinate (hpppuc) at (\hxxxu, \hyyyc);
\coordinate (hpppud) at (\hxxxu, \hyyyd);
\coordinate (hpppue) at (\hxxxu, \hyyye);
\coordinate (hpppuf) at (\hxxxu, \hyyyf);
\coordinate (hpppug) at (\hxxxu, \hyyyg);
\coordinate (hpppuh) at (\hxxxu, \hyyyh);
\coordinate (hpppui) at (\hxxxu, \hyyyi);
\coordinate (hpppuj) at (\hxxxu, \hyyyj);
\coordinate (hpppuk) at (\hxxxu, \hyyyk);
\coordinate (hpppul) at (\hxxxu, \hyyyl);
\coordinate (hpppum) at (\hxxxu, \hyyym);
\coordinate (hpppun) at (\hxxxu, \hyyyn);
\coordinate (hpppuo) at (\hxxxu, \hyyyo);
\coordinate (hpppup) at (\hxxxu, \hyyyp);
\coordinate (hpppuq) at (\hxxxu, \hyyyq);
\coordinate (hpppur) at (\hxxxu, \hyyyr);
\coordinate (hpppus) at (\hxxxu, \hyyys);
\coordinate (hppput) at (\hxxxu, \hyyyt);
\coordinate (hpppuu) at (\hxxxu, \hyyyu);
\coordinate (hpppuv) at (\hxxxu, \hyyyv);
\coordinate (hpppuw) at (\hxxxu, \hyyyw);
\coordinate (hpppux) at (\hxxxu, \hyyyx);
\coordinate (hpppuy) at (\hxxxu, \hyyyy);
\coordinate (hpppuz) at (\hxxxu, \hyyyz);
\coordinate (hpppva) at (\hxxxv, \hyyya);
\coordinate (hpppvb) at (\hxxxv, \hyyyb);
\coordinate (hpppvc) at (\hxxxv, \hyyyc);
\coordinate (hpppvd) at (\hxxxv, \hyyyd);
\coordinate (hpppve) at (\hxxxv, \hyyye);
\coordinate (hpppvf) at (\hxxxv, \hyyyf);
\coordinate (hpppvg) at (\hxxxv, \hyyyg);
\coordinate (hpppvh) at (\hxxxv, \hyyyh);
\coordinate (hpppvi) at (\hxxxv, \hyyyi);
\coordinate (hpppvj) at (\hxxxv, \hyyyj);
\coordinate (hpppvk) at (\hxxxv, \hyyyk);
\coordinate (hpppvl) at (\hxxxv, \hyyyl);
\coordinate (hpppvm) at (\hxxxv, \hyyym);
\coordinate (hpppvn) at (\hxxxv, \hyyyn);
\coordinate (hpppvo) at (\hxxxv, \hyyyo);
\coordinate (hpppvp) at (\hxxxv, \hyyyp);
\coordinate (hpppvq) at (\hxxxv, \hyyyq);
\coordinate (hpppvr) at (\hxxxv, \hyyyr);
\coordinate (hpppvs) at (\hxxxv, \hyyys);
\coordinate (hpppvt) at (\hxxxv, \hyyyt);
\coordinate (hpppvu) at (\hxxxv, \hyyyu);
\coordinate (hpppvv) at (\hxxxv, \hyyyv);
\coordinate (hpppvw) at (\hxxxv, \hyyyw);
\coordinate (hpppvx) at (\hxxxv, \hyyyx);
\coordinate (hpppvy) at (\hxxxv, \hyyyy);
\coordinate (hpppvz) at (\hxxxv, \hyyyz);
\coordinate (hpppwa) at (\hxxxw, \hyyya);
\coordinate (hpppwb) at (\hxxxw, \hyyyb);
\coordinate (hpppwc) at (\hxxxw, \hyyyc);
\coordinate (hpppwd) at (\hxxxw, \hyyyd);
\coordinate (hpppwe) at (\hxxxw, \hyyye);
\coordinate (hpppwf) at (\hxxxw, \hyyyf);
\coordinate (hpppwg) at (\hxxxw, \hyyyg);
\coordinate (hpppwh) at (\hxxxw, \hyyyh);
\coordinate (hpppwi) at (\hxxxw, \hyyyi);
\coordinate (hpppwj) at (\hxxxw, \hyyyj);
\coordinate (hpppwk) at (\hxxxw, \hyyyk);
\coordinate (hpppwl) at (\hxxxw, \hyyyl);
\coordinate (hpppwm) at (\hxxxw, \hyyym);
\coordinate (hpppwn) at (\hxxxw, \hyyyn);
\coordinate (hpppwo) at (\hxxxw, \hyyyo);
\coordinate (hpppwp) at (\hxxxw, \hyyyp);
\coordinate (hpppwq) at (\hxxxw, \hyyyq);
\coordinate (hpppwr) at (\hxxxw, \hyyyr);
\coordinate (hpppws) at (\hxxxw, \hyyys);
\coordinate (hpppwt) at (\hxxxw, \hyyyt);
\coordinate (hpppwu) at (\hxxxw, \hyyyu);
\coordinate (hpppwv) at (\hxxxw, \hyyyv);
\coordinate (hpppww) at (\hxxxw, \hyyyw);
\coordinate (hpppwx) at (\hxxxw, \hyyyx);
\coordinate (hpppwy) at (\hxxxw, \hyyyy);
\coordinate (hpppwz) at (\hxxxw, \hyyyz);
\coordinate (hpppxa) at (\hxxxx, \hyyya);
\coordinate (hpppxb) at (\hxxxx, \hyyyb);
\coordinate (hpppxc) at (\hxxxx, \hyyyc);
\coordinate (hpppxd) at (\hxxxx, \hyyyd);
\coordinate (hpppxe) at (\hxxxx, \hyyye);
\coordinate (hpppxf) at (\hxxxx, \hyyyf);
\coordinate (hpppxg) at (\hxxxx, \hyyyg);
\coordinate (hpppxh) at (\hxxxx, \hyyyh);
\coordinate (hpppxi) at (\hxxxx, \hyyyi);
\coordinate (hpppxj) at (\hxxxx, \hyyyj);
\coordinate (hpppxk) at (\hxxxx, \hyyyk);
\coordinate (hpppxl) at (\hxxxx, \hyyyl);
\coordinate (hpppxm) at (\hxxxx, \hyyym);
\coordinate (hpppxn) at (\hxxxx, \hyyyn);
\coordinate (hpppxo) at (\hxxxx, \hyyyo);
\coordinate (hpppxp) at (\hxxxx, \hyyyp);
\coordinate (hpppxq) at (\hxxxx, \hyyyq);
\coordinate (hpppxr) at (\hxxxx, \hyyyr);
\coordinate (hpppxs) at (\hxxxx, \hyyys);
\coordinate (hpppxt) at (\hxxxx, \hyyyt);
\coordinate (hpppxu) at (\hxxxx, \hyyyu);
\coordinate (hpppxv) at (\hxxxx, \hyyyv);
\coordinate (hpppxw) at (\hxxxx, \hyyyw);
\coordinate (hpppxx) at (\hxxxx, \hyyyx);
\coordinate (hpppxy) at (\hxxxx, \hyyyy);
\coordinate (hpppxz) at (\hxxxx, \hyyyz);
\coordinate (hpppya) at (\hxxxy, \hyyya);
\coordinate (hpppyb) at (\hxxxy, \hyyyb);
\coordinate (hpppyc) at (\hxxxy, \hyyyc);
\coordinate (hpppyd) at (\hxxxy, \hyyyd);
\coordinate (hpppye) at (\hxxxy, \hyyye);
\coordinate (hpppyf) at (\hxxxy, \hyyyf);
\coordinate (hpppyg) at (\hxxxy, \hyyyg);
\coordinate (hpppyh) at (\hxxxy, \hyyyh);
\coordinate (hpppyi) at (\hxxxy, \hyyyi);
\coordinate (hpppyj) at (\hxxxy, \hyyyj);
\coordinate (hpppyk) at (\hxxxy, \hyyyk);
\coordinate (hpppyl) at (\hxxxy, \hyyyl);
\coordinate (hpppym) at (\hxxxy, \hyyym);
\coordinate (hpppyn) at (\hxxxy, \hyyyn);
\coordinate (hpppyo) at (\hxxxy, \hyyyo);
\coordinate (hpppyp) at (\hxxxy, \hyyyp);
\coordinate (hpppyq) at (\hxxxy, \hyyyq);
\coordinate (hpppyr) at (\hxxxy, \hyyyr);
\coordinate (hpppys) at (\hxxxy, \hyyys);
\coordinate (hpppyt) at (\hxxxy, \hyyyt);
\coordinate (hpppyu) at (\hxxxy, \hyyyu);
\coordinate (hpppyv) at (\hxxxy, \hyyyv);
\coordinate (hpppyw) at (\hxxxy, \hyyyw);
\coordinate (hpppyx) at (\hxxxy, \hyyyx);
\coordinate (hpppyy) at (\hxxxy, \hyyyy);
\coordinate (hpppyz) at (\hxxxy, \hyyyz);
\coordinate (hpppza) at (\hxxxz, \hyyya);
\coordinate (hpppzb) at (\hxxxz, \hyyyb);
\coordinate (hpppzc) at (\hxxxz, \hyyyc);
\coordinate (hpppzd) at (\hxxxz, \hyyyd);
\coordinate (hpppze) at (\hxxxz, \hyyye);
\coordinate (hpppzf) at (\hxxxz, \hyyyf);
\coordinate (hpppzg) at (\hxxxz, \hyyyg);
\coordinate (hpppzh) at (\hxxxz, \hyyyh);
\coordinate (hpppzi) at (\hxxxz, \hyyyi);
\coordinate (hpppzj) at (\hxxxz, \hyyyj);
\coordinate (hpppzk) at (\hxxxz, \hyyyk);
\coordinate (hpppzl) at (\hxxxz, \hyyyl);
\coordinate (hpppzm) at (\hxxxz, \hyyym);
\coordinate (hpppzn) at (\hxxxz, \hyyyn);
\coordinate (hpppzo) at (\hxxxz, \hyyyo);
\coordinate (hpppzp) at (\hxxxz, \hyyyp);
\coordinate (hpppzq) at (\hxxxz, \hyyyq);
\coordinate (hpppzr) at (\hxxxz, \hyyyr);
\coordinate (hpppzs) at (\hxxxz, \hyyys);
\coordinate (hpppzt) at (\hxxxz, \hyyyt);
\coordinate (hpppzu) at (\hxxxz, \hyyyu);
\coordinate (hpppzv) at (\hxxxz, \hyyyv);
\coordinate (hpppzw) at (\hxxxz, \hyyyw);
\coordinate (hpppzx) at (\hxxxz, \hyyyx);
\coordinate (hpppzy) at (\hxxxz, \hyyyy);
\coordinate (hpppzz) at (\hxxxz, \hyyyz);

%\gangprintcoordinateat{(0,0)}{The last coordinate values: }{($(hpppzz)$)}; 


% Draw related part of the coordinate system with dashed helplines (centered at (hpppii)) with letters as background, which would help to determine all coordinates. 
\coordinatebackground{h}{c}{d}{o};

% Step 2, draw key devices, their accessories, and take related coordinates of their pins, and may define more coordinates. 

% Draw the Opamp at the coordinate (hpppii) and name it as "swopamp".
\draw (hpppii) node [op amp, yscale=-1] (swopamp) {\ctikzflipy{Opamp}} ; 

% Its accessories and lables. 
\draw [-*](swopamp.down) -- ($(swopamp.down)+(0,1)$) node[right]{$V_+$}; 
\node at ($(swopamp.down)+(0.3,0.2)$) {7};  
\draw [-*](swopamp.up) -- ($(swopamp.up)+(0,-1)$) node[right]{$V_-$}; 
\node at ($(swopamp.up)+(0.3,-0.2)$) {4};

% Get the x- and y-components of the coordinates of the "+" and "-" pins. 
\getxyingivenunit{cm}{(swopamp.+)}{\swopampzx}{\swopampzy};
\getxyingivenunit{cm}{(swopamp.-)}{\swopampfx}{\swopampfy};

% Then define a few more coordinates, at least for keeping in mind.
\coordinate (plusshort) at ($(\hxxxg,\swopampzy)$);
%\fill  (plusshort) circle (2pt);  % May be commented later.
\coordinate (minusshort) at ($(\hxxxg,\swopampfy)$);
%\fill  (minusshort) circle (2pt); % May be commented later.
\coordinate (leftinter) at ($(\hxxxe,\swopampzy)$);
\fill  (leftinter) circle (2pt);

% Draw an "npn" at (hpppmi) and name it as "swQ".
\draw (hpppmi) node[npn](swQ){};

% Get the x- and y-components of the needed pins of it for later usage.
\getxyingivenunit{cm}{(swQ.C)}{\swQCx}{\swQCy};
\getxyingivenunit{cm}{(swQ.E)}{\swQEx}{\swQEy};

% Then define more coordinate(s).
\coordinate (Qcshort) at ($(\swQEx,\hyyyj)$);
%\fill  (Qcshort) circle (2pt); % May be commented later.
\coordinate (Qeshort) at ($(\swQEx,\hyyyf)$);
\fill  (Qeshort) circle (2pt) node [right] {$V_0$};

% Then the rectangle by the points (hpppef) -- (hpppej) -- (Qcshort) -- (Qeshort) forms a clear area for the key devices. 

% Connect the two devices.
\draw (swopamp.out) to [short, l=$I_B$, above] (swQ.B);

% Step 3, draw other little devices. For tidiness, better to give two units in length for each new device and align them up.

% For this specific circuit, let us attach the four bi-pole devices (maybe with their accessories) to each corner of the above mentioned rectangle area for the key devices, separately. 


\draw   (hppped) to [empty ZZener diode] (hpppef) -- (leftinter);
% The Latex system can not work properly when I put the following label into the above "[empty ZZener diode]" in the form of an additional "l= ..." option, then I have to employ the following "\node ..." command. Then the label should be aligned with the "ZZener" as much as possible even the coordinate system is modified later. Since the "\hyyyd" and "\hyyyf" micros are used to position the "ZZener", then better to use the center between them to locate the label, rather than using "\hyyye" directly. This idea is also applied in later "\node ..." commands. 
\node at ($(\hxxxe-1.3, \hyyyd*0.5+\hyyyf*0.5)$) {$V_Z = 5\textnormal{V}$};

\draw (hpppej) to [generic] (hpppel) -| (swQ.C);
\node at ($(\hxxxe-1.1, \hyyyj*0.5+\hyyyl*0.5)$) {$R_{1}=47k\Omega$};
\node [right] at (\swQCx,\hyyyj*0.5+\hyyyl*0.5) {$I_C \approx \beta I_B$};


\draw  (\swQEx, \hyyyd) to [generic] (\swQEx, \hyyyf) -- (swQ.E);
\node at ($(\swQEx+1.4, \hyyyd*0.5+\hyyyf*0.5)$) {$R_{E}=100k\Omega$};

% Draw the top area. 
%\draw  (\hxxxi-0.2,\hyyyn) --  (\hxxxi+0.2,\hyyyn) node [right] {$V_{cc}=15\textnormal{V}$} ;
%\draw [->] (hpppin) -- (hpppim) node [right] {$I$};  
%\draw  (hpppim) -- (hpppil);
%\fill  (hpppil) circle (2pt);

% Step 4, other shorts.
\draw  (swopamp.+)  to [short, l_=$I_+ \approx 0 $, above] (plusshort) -- (leftinter) -- (hpppej);

\draw  (swopamp.-)  to [short, l_=$I_- \approx 0 $, above] (minusshort) |- (Qeshort);

%Step 5, all the rest, especially labels. May also clean unnecessary staff, like the system background and dark points for showing newly defined coordinates previously. 
\draw [->] ($(\swQEx-0.4, \hyyyf - 0.4)$) -- node [left] {$I_E$} ($(\swQEx-0.4, \hyyyd + 0.4)$);

\draw [->] ($(\hxxxe+0.4, \hyyyl*0.5+\hyyyj*0.5 + 0.6)$) -- node [right] {$I_1$} ($(\hxxxe+0.4, \hyyyl*0.5+\hyyyj*0.5 - 0.6)$);

\end{circuitikz}


\newpage

{\Large Figure 6, the second circuit to be inserted.}



\vspace{2cm}
% page 185
\begin{circuitikz}[scale=1]
\pgfmathsetmacro{\totalsxxx}{26}
\pgfmathsetmacro{\totalsyyy}{26}
\pgfmathsetmacro{\sxxxspacing}{1}
\pgfmathsetmacro{\syyyspacing}{1}
\pgfmathsetmacro{\sxxxa}{-3}
\pgfmathsetmacro{\syyya}{-2}

\pgfmathsetmacro{\sxxxb}{\sxxxa + \sxxxspacing + 0.0 }
\pgfmathsetmacro{\sxxxc}{\sxxxb + \sxxxspacing + 0.0 }
\pgfmathsetmacro{\sxxxd}{\sxxxc + \sxxxspacing + 0.0 }
\pgfmathsetmacro{\sxxxe}{\sxxxd + \sxxxspacing + 0.0 }
\pgfmathsetmacro{\sxxxf}{\sxxxe + \sxxxspacing + 0.0 }
\pgfmathsetmacro{\sxxxg}{\sxxxf + \sxxxspacing + 0.0 }
\pgfmathsetmacro{\sxxxh}{\sxxxg + \sxxxspacing + 0.0 }
\pgfmathsetmacro{\sxxxi}{\sxxxh + \sxxxspacing + 0.0 }
\pgfmathsetmacro{\sxxxj}{\sxxxi + \sxxxspacing + 0.0 }
\pgfmathsetmacro{\sxxxk}{\sxxxj + \sxxxspacing + 0.0 }
\pgfmathsetmacro{\sxxxl}{\sxxxk + \sxxxspacing + 0.0 }
\pgfmathsetmacro{\sxxxm}{\sxxxl + \sxxxspacing + 0.0 }
\pgfmathsetmacro{\sxxxn}{\sxxxm + \sxxxspacing + 0.0 }
\pgfmathsetmacro{\sxxxo}{\sxxxn + \sxxxspacing + 0.0 }
\pgfmathsetmacro{\sxxxp}{\sxxxo + \sxxxspacing + 0.0 }
\pgfmathsetmacro{\sxxxq}{\sxxxp + \sxxxspacing + 0.0 }
\pgfmathsetmacro{\sxxxr}{\sxxxq + \sxxxspacing + 0.0 }
\pgfmathsetmacro{\sxxxs}{\sxxxr + \sxxxspacing + 0.0 }
\pgfmathsetmacro{\sxxxt}{\sxxxs + \sxxxspacing + 0.0 }
\pgfmathsetmacro{\sxxxu}{\sxxxt + \sxxxspacing + 0.0 }
\pgfmathsetmacro{\sxxxv}{\sxxxu + \sxxxspacing + 0.0 }
\pgfmathsetmacro{\sxxxw}{\sxxxv + \sxxxspacing + 0.0 }
\pgfmathsetmacro{\sxxxx}{\sxxxw + \sxxxspacing + 0.0 }
\pgfmathsetmacro{\sxxxy}{\sxxxx + \sxxxspacing + 0.0 }
\pgfmathsetmacro{\sxxxz}{\sxxxy + \sxxxspacing + 0.0 }

\pgfmathsetmacro{\syyyb}{\syyya + \syyyspacing + 0.0 }
\pgfmathsetmacro{\syyyc}{\syyyb + \syyyspacing + 0.0 }
\pgfmathsetmacro{\syyyd}{\syyyc + \syyyspacing + 0.0 }
\pgfmathsetmacro{\syyye}{\syyyd + \syyyspacing + 0.0 }
\pgfmathsetmacro{\syyyf}{\syyye + \syyyspacing + 0.0 }
\pgfmathsetmacro{\syyyg}{\syyyf + \syyyspacing + 0.0 }
\pgfmathsetmacro{\syyyh}{\syyyg + \syyyspacing + 0.0 }
\pgfmathsetmacro{\syyyi}{\syyyh + \syyyspacing + 0.0 }
\pgfmathsetmacro{\syyyj}{\syyyi + \syyyspacing + 0.0 }
\pgfmathsetmacro{\syyyk}{\syyyj + \syyyspacing + 0.0 }
\pgfmathsetmacro{\syyyl}{\syyyk + \syyyspacing + 0.0 }
\pgfmathsetmacro{\syyym}{\syyyl + \syyyspacing + 0.0 }
\pgfmathsetmacro{\syyyn}{\syyym + \syyyspacing + 0.0 }
\pgfmathsetmacro{\syyyo}{\syyyn + \syyyspacing + 0.0 }
\pgfmathsetmacro{\syyyp}{\syyyo + \syyyspacing + 0.0 }
\pgfmathsetmacro{\syyyq}{\syyyp + \syyyspacing + 0.0 }
\pgfmathsetmacro{\syyyr}{\syyyq + \syyyspacing + 0.0 }
\pgfmathsetmacro{\syyys}{\syyyr + \syyyspacing + 0.0 }
\pgfmathsetmacro{\syyyt}{\syyys + \syyyspacing + 0.0 }
\pgfmathsetmacro{\syyyu}{\syyyt + \syyyspacing + 0.0 }
\pgfmathsetmacro{\syyyv}{\syyyu + \syyyspacing + 0.0 }
\pgfmathsetmacro{\syyyw}{\syyyv + \syyyspacing + 0.0 }
\pgfmathsetmacro{\syyyx}{\syyyw + \syyyspacing + 0.0 }
\pgfmathsetmacro{\syyyy}{\syyyx + \syyyspacing + 0.0 }
\pgfmathsetmacro{\syyyz}{\syyyy + \syyyspacing + 0.0 }

\coordinate (spppaa) at (\sxxxa, \syyya);
\coordinate (spppab) at (\sxxxa, \syyyb);
\coordinate (spppac) at (\sxxxa, \syyyc);
\coordinate (spppad) at (\sxxxa, \syyyd);
\coordinate (spppae) at (\sxxxa, \syyye);
\coordinate (spppaf) at (\sxxxa, \syyyf);
\coordinate (spppag) at (\sxxxa, \syyyg);
\coordinate (spppah) at (\sxxxa, \syyyh);
\coordinate (spppai) at (\sxxxa, \syyyi);
\coordinate (spppaj) at (\sxxxa, \syyyj);
\coordinate (spppak) at (\sxxxa, \syyyk);
\coordinate (spppal) at (\sxxxa, \syyyl);
\coordinate (spppam) at (\sxxxa, \syyym);
\coordinate (spppan) at (\sxxxa, \syyyn);
\coordinate (spppao) at (\sxxxa, \syyyo);
\coordinate (spppap) at (\sxxxa, \syyyp);
\coordinate (spppaq) at (\sxxxa, \syyyq);
\coordinate (spppar) at (\sxxxa, \syyyr);
\coordinate (spppas) at (\sxxxa, \syyys);
\coordinate (spppat) at (\sxxxa, \syyyt);
\coordinate (spppau) at (\sxxxa, \syyyu);
\coordinate (spppav) at (\sxxxa, \syyyv);
\coordinate (spppaw) at (\sxxxa, \syyyw);
\coordinate (spppax) at (\sxxxa, \syyyx);
\coordinate (spppay) at (\sxxxa, \syyyy);
\coordinate (spppaz) at (\sxxxa, \syyyz);
\coordinate (spppba) at (\sxxxb, \syyya);
\coordinate (spppbb) at (\sxxxb, \syyyb);
\coordinate (spppbc) at (\sxxxb, \syyyc);
\coordinate (spppbd) at (\sxxxb, \syyyd);
\coordinate (spppbe) at (\sxxxb, \syyye);
\coordinate (spppbf) at (\sxxxb, \syyyf);
\coordinate (spppbg) at (\sxxxb, \syyyg);
\coordinate (spppbh) at (\sxxxb, \syyyh);
\coordinate (spppbi) at (\sxxxb, \syyyi);
\coordinate (spppbj) at (\sxxxb, \syyyj);
\coordinate (spppbk) at (\sxxxb, \syyyk);
\coordinate (spppbl) at (\sxxxb, \syyyl);
\coordinate (spppbm) at (\sxxxb, \syyym);
\coordinate (spppbn) at (\sxxxb, \syyyn);
\coordinate (spppbo) at (\sxxxb, \syyyo);
\coordinate (spppbp) at (\sxxxb, \syyyp);
\coordinate (spppbq) at (\sxxxb, \syyyq);
\coordinate (spppbr) at (\sxxxb, \syyyr);
\coordinate (spppbs) at (\sxxxb, \syyys);
\coordinate (spppbt) at (\sxxxb, \syyyt);
\coordinate (spppbu) at (\sxxxb, \syyyu);
\coordinate (spppbv) at (\sxxxb, \syyyv);
\coordinate (spppbw) at (\sxxxb, \syyyw);
\coordinate (spppbx) at (\sxxxb, \syyyx);
\coordinate (spppby) at (\sxxxb, \syyyy);
\coordinate (spppbz) at (\sxxxb, \syyyz);
\coordinate (spppca) at (\sxxxc, \syyya);
\coordinate (spppcb) at (\sxxxc, \syyyb);
\coordinate (spppcc) at (\sxxxc, \syyyc);
\coordinate (spppcd) at (\sxxxc, \syyyd);
\coordinate (spppce) at (\sxxxc, \syyye);
\coordinate (spppcf) at (\sxxxc, \syyyf);
\coordinate (spppcg) at (\sxxxc, \syyyg);
\coordinate (spppch) at (\sxxxc, \syyyh);
\coordinate (spppci) at (\sxxxc, \syyyi);
\coordinate (spppcj) at (\sxxxc, \syyyj);
\coordinate (spppck) at (\sxxxc, \syyyk);
\coordinate (spppcl) at (\sxxxc, \syyyl);
\coordinate (spppcm) at (\sxxxc, \syyym);
\coordinate (spppcn) at (\sxxxc, \syyyn);
\coordinate (spppco) at (\sxxxc, \syyyo);
\coordinate (spppcp) at (\sxxxc, \syyyp);
\coordinate (spppcq) at (\sxxxc, \syyyq);
\coordinate (spppcr) at (\sxxxc, \syyyr);
\coordinate (spppcs) at (\sxxxc, \syyys);
\coordinate (spppct) at (\sxxxc, \syyyt);
\coordinate (spppcu) at (\sxxxc, \syyyu);
\coordinate (spppcv) at (\sxxxc, \syyyv);
\coordinate (spppcw) at (\sxxxc, \syyyw);
\coordinate (spppcx) at (\sxxxc, \syyyx);
\coordinate (spppcy) at (\sxxxc, \syyyy);
\coordinate (spppcz) at (\sxxxc, \syyyz);
\coordinate (spppda) at (\sxxxd, \syyya);
\coordinate (spppdb) at (\sxxxd, \syyyb);
\coordinate (spppdc) at (\sxxxd, \syyyc);
\coordinate (spppdd) at (\sxxxd, \syyyd);
\coordinate (spppde) at (\sxxxd, \syyye);
\coordinate (spppdf) at (\sxxxd, \syyyf);
\coordinate (spppdg) at (\sxxxd, \syyyg);
\coordinate (spppdh) at (\sxxxd, \syyyh);
\coordinate (spppdi) at (\sxxxd, \syyyi);
\coordinate (spppdj) at (\sxxxd, \syyyj);
\coordinate (spppdk) at (\sxxxd, \syyyk);
\coordinate (spppdl) at (\sxxxd, \syyyl);
\coordinate (spppdm) at (\sxxxd, \syyym);
\coordinate (spppdn) at (\sxxxd, \syyyn);
\coordinate (spppdo) at (\sxxxd, \syyyo);
\coordinate (spppdp) at (\sxxxd, \syyyp);
\coordinate (spppdq) at (\sxxxd, \syyyq);
\coordinate (spppdr) at (\sxxxd, \syyyr);
\coordinate (spppds) at (\sxxxd, \syyys);
\coordinate (spppdt) at (\sxxxd, \syyyt);
\coordinate (spppdu) at (\sxxxd, \syyyu);
\coordinate (spppdv) at (\sxxxd, \syyyv);
\coordinate (spppdw) at (\sxxxd, \syyyw);
\coordinate (spppdx) at (\sxxxd, \syyyx);
\coordinate (spppdy) at (\sxxxd, \syyyy);
\coordinate (spppdz) at (\sxxxd, \syyyz);
\coordinate (spppea) at (\sxxxe, \syyya);
\coordinate (spppeb) at (\sxxxe, \syyyb);
\coordinate (spppec) at (\sxxxe, \syyyc);
\coordinate (sppped) at (\sxxxe, \syyyd);
\coordinate (spppee) at (\sxxxe, \syyye);
\coordinate (spppef) at (\sxxxe, \syyyf);
\coordinate (spppeg) at (\sxxxe, \syyyg);
\coordinate (spppeh) at (\sxxxe, \syyyh);
\coordinate (spppei) at (\sxxxe, \syyyi);
\coordinate (spppej) at (\sxxxe, \syyyj);
\coordinate (spppek) at (\sxxxe, \syyyk);
\coordinate (spppel) at (\sxxxe, \syyyl);
\coordinate (spppem) at (\sxxxe, \syyym);
\coordinate (spppen) at (\sxxxe, \syyyn);
\coordinate (spppeo) at (\sxxxe, \syyyo);
\coordinate (spppep) at (\sxxxe, \syyyp);
\coordinate (spppeq) at (\sxxxe, \syyyq);
\coordinate (sppper) at (\sxxxe, \syyyr);
\coordinate (spppes) at (\sxxxe, \syyys);
\coordinate (spppet) at (\sxxxe, \syyyt);
\coordinate (spppeu) at (\sxxxe, \syyyu);
\coordinate (spppev) at (\sxxxe, \syyyv);
\coordinate (spppew) at (\sxxxe, \syyyw);
\coordinate (spppex) at (\sxxxe, \syyyx);
\coordinate (spppey) at (\sxxxe, \syyyy);
\coordinate (spppez) at (\sxxxe, \syyyz);
\coordinate (spppfa) at (\sxxxf, \syyya);
\coordinate (spppfb) at (\sxxxf, \syyyb);
\coordinate (spppfc) at (\sxxxf, \syyyc);
\coordinate (spppfd) at (\sxxxf, \syyyd);
\coordinate (spppfe) at (\sxxxf, \syyye);
\coordinate (spppff) at (\sxxxf, \syyyf);
\coordinate (spppfg) at (\sxxxf, \syyyg);
\coordinate (spppfh) at (\sxxxf, \syyyh);
\coordinate (spppfi) at (\sxxxf, \syyyi);
\coordinate (spppfj) at (\sxxxf, \syyyj);
\coordinate (spppfk) at (\sxxxf, \syyyk);
\coordinate (spppfl) at (\sxxxf, \syyyl);
\coordinate (spppfm) at (\sxxxf, \syyym);
\coordinate (spppfn) at (\sxxxf, \syyyn);
\coordinate (spppfo) at (\sxxxf, \syyyo);
\coordinate (spppfp) at (\sxxxf, \syyyp);
\coordinate (spppfq) at (\sxxxf, \syyyq);
\coordinate (spppfr) at (\sxxxf, \syyyr);
\coordinate (spppfs) at (\sxxxf, \syyys);
\coordinate (spppft) at (\sxxxf, \syyyt);
\coordinate (spppfu) at (\sxxxf, \syyyu);
\coordinate (spppfv) at (\sxxxf, \syyyv);
\coordinate (spppfw) at (\sxxxf, \syyyw);
\coordinate (spppfx) at (\sxxxf, \syyyx);
\coordinate (spppfy) at (\sxxxf, \syyyy);
\coordinate (spppfz) at (\sxxxf, \syyyz);
\coordinate (spppga) at (\sxxxg, \syyya);
\coordinate (spppgb) at (\sxxxg, \syyyb);
\coordinate (spppgc) at (\sxxxg, \syyyc);
\coordinate (spppgd) at (\sxxxg, \syyyd);
\coordinate (spppge) at (\sxxxg, \syyye);
\coordinate (spppgf) at (\sxxxg, \syyyf);
\coordinate (spppgg) at (\sxxxg, \syyyg);
\coordinate (spppgh) at (\sxxxg, \syyyh);
\coordinate (spppgi) at (\sxxxg, \syyyi);
\coordinate (spppgj) at (\sxxxg, \syyyj);
\coordinate (spppgk) at (\sxxxg, \syyyk);
\coordinate (spppgl) at (\sxxxg, \syyyl);
\coordinate (spppgm) at (\sxxxg, \syyym);
\coordinate (spppgn) at (\sxxxg, \syyyn);
\coordinate (spppgo) at (\sxxxg, \syyyo);
\coordinate (spppgp) at (\sxxxg, \syyyp);
\coordinate (spppgq) at (\sxxxg, \syyyq);
\coordinate (spppgr) at (\sxxxg, \syyyr);
\coordinate (spppgs) at (\sxxxg, \syyys);
\coordinate (spppgt) at (\sxxxg, \syyyt);
\coordinate (spppgu) at (\sxxxg, \syyyu);
\coordinate (spppgv) at (\sxxxg, \syyyv);
\coordinate (spppgw) at (\sxxxg, \syyyw);
\coordinate (spppgx) at (\sxxxg, \syyyx);
\coordinate (spppgy) at (\sxxxg, \syyyy);
\coordinate (spppgz) at (\sxxxg, \syyyz);
\coordinate (spppha) at (\sxxxh, \syyya);
\coordinate (sppphb) at (\sxxxh, \syyyb);
\coordinate (sppphc) at (\sxxxh, \syyyc);
\coordinate (sppphd) at (\sxxxh, \syyyd);
\coordinate (sppphe) at (\sxxxh, \syyye);
\coordinate (sppphf) at (\sxxxh, \syyyf);
\coordinate (sppphg) at (\sxxxh, \syyyg);
\coordinate (sppphh) at (\sxxxh, \syyyh);
\coordinate (sppphi) at (\sxxxh, \syyyi);
\coordinate (sppphj) at (\sxxxh, \syyyj);
\coordinate (sppphk) at (\sxxxh, \syyyk);
\coordinate (sppphl) at (\sxxxh, \syyyl);
\coordinate (sppphm) at (\sxxxh, \syyym);
\coordinate (sppphn) at (\sxxxh, \syyyn);
\coordinate (spppho) at (\sxxxh, \syyyo);
\coordinate (sppphp) at (\sxxxh, \syyyp);
\coordinate (sppphq) at (\sxxxh, \syyyq);
\coordinate (sppphr) at (\sxxxh, \syyyr);
\coordinate (sppphs) at (\sxxxh, \syyys);
\coordinate (spppht) at (\sxxxh, \syyyt);
\coordinate (sppphu) at (\sxxxh, \syyyu);
\coordinate (sppphv) at (\sxxxh, \syyyv);
\coordinate (sppphw) at (\sxxxh, \syyyw);
\coordinate (sppphx) at (\sxxxh, \syyyx);
\coordinate (sppphy) at (\sxxxh, \syyyy);
\coordinate (sppphz) at (\sxxxh, \syyyz);
\coordinate (spppia) at (\sxxxi, \syyya);
\coordinate (spppib) at (\sxxxi, \syyyb);
\coordinate (spppic) at (\sxxxi, \syyyc);
\coordinate (spppid) at (\sxxxi, \syyyd);
\coordinate (spppie) at (\sxxxi, \syyye);
\coordinate (spppif) at (\sxxxi, \syyyf);
\coordinate (spppig) at (\sxxxi, \syyyg);
\coordinate (spppih) at (\sxxxi, \syyyh);
\coordinate (spppii) at (\sxxxi, \syyyi);
\coordinate (spppij) at (\sxxxi, \syyyj);
\coordinate (spppik) at (\sxxxi, \syyyk);
\coordinate (spppil) at (\sxxxi, \syyyl);
\coordinate (spppim) at (\sxxxi, \syyym);
\coordinate (spppin) at (\sxxxi, \syyyn);
\coordinate (spppio) at (\sxxxi, \syyyo);
\coordinate (spppip) at (\sxxxi, \syyyp);
\coordinate (spppiq) at (\sxxxi, \syyyq);
\coordinate (spppir) at (\sxxxi, \syyyr);
\coordinate (spppis) at (\sxxxi, \syyys);
\coordinate (spppit) at (\sxxxi, \syyyt);
\coordinate (spppiu) at (\sxxxi, \syyyu);
\coordinate (spppiv) at (\sxxxi, \syyyv);
\coordinate (spppiw) at (\sxxxi, \syyyw);
\coordinate (spppix) at (\sxxxi, \syyyx);
\coordinate (spppiy) at (\sxxxi, \syyyy);
\coordinate (spppiz) at (\sxxxi, \syyyz);
\coordinate (spppja) at (\sxxxj, \syyya);
\coordinate (spppjb) at (\sxxxj, \syyyb);
\coordinate (spppjc) at (\sxxxj, \syyyc);
\coordinate (spppjd) at (\sxxxj, \syyyd);
\coordinate (spppje) at (\sxxxj, \syyye);
\coordinate (spppjf) at (\sxxxj, \syyyf);
\coordinate (spppjg) at (\sxxxj, \syyyg);
\coordinate (spppjh) at (\sxxxj, \syyyh);
\coordinate (spppji) at (\sxxxj, \syyyi);
\coordinate (spppjj) at (\sxxxj, \syyyj);
\coordinate (spppjk) at (\sxxxj, \syyyk);
\coordinate (spppjl) at (\sxxxj, \syyyl);
\coordinate (spppjm) at (\sxxxj, \syyym);
\coordinate (spppjn) at (\sxxxj, \syyyn);
\coordinate (spppjo) at (\sxxxj, \syyyo);
\coordinate (spppjp) at (\sxxxj, \syyyp);
\coordinate (spppjq) at (\sxxxj, \syyyq);
\coordinate (spppjr) at (\sxxxj, \syyyr);
\coordinate (spppjs) at (\sxxxj, \syyys);
\coordinate (spppjt) at (\sxxxj, \syyyt);
\coordinate (spppju) at (\sxxxj, \syyyu);
\coordinate (spppjv) at (\sxxxj, \syyyv);
\coordinate (spppjw) at (\sxxxj, \syyyw);
\coordinate (spppjx) at (\sxxxj, \syyyx);
\coordinate (spppjy) at (\sxxxj, \syyyy);
\coordinate (spppjz) at (\sxxxj, \syyyz);
\coordinate (spppka) at (\sxxxk, \syyya);
\coordinate (spppkb) at (\sxxxk, \syyyb);
\coordinate (spppkc) at (\sxxxk, \syyyc);
\coordinate (spppkd) at (\sxxxk, \syyyd);
\coordinate (spppke) at (\sxxxk, \syyye);
\coordinate (spppkf) at (\sxxxk, \syyyf);
\coordinate (spppkg) at (\sxxxk, \syyyg);
\coordinate (spppkh) at (\sxxxk, \syyyh);
\coordinate (spppki) at (\sxxxk, \syyyi);
\coordinate (spppkj) at (\sxxxk, \syyyj);
\coordinate (spppkk) at (\sxxxk, \syyyk);
\coordinate (spppkl) at (\sxxxk, \syyyl);
\coordinate (spppkm) at (\sxxxk, \syyym);
\coordinate (spppkn) at (\sxxxk, \syyyn);
\coordinate (spppko) at (\sxxxk, \syyyo);
\coordinate (spppkp) at (\sxxxk, \syyyp);
\coordinate (spppkq) at (\sxxxk, \syyyq);
\coordinate (spppkr) at (\sxxxk, \syyyr);
\coordinate (spppks) at (\sxxxk, \syyys);
\coordinate (spppkt) at (\sxxxk, \syyyt);
\coordinate (spppku) at (\sxxxk, \syyyu);
\coordinate (spppkv) at (\sxxxk, \syyyv);
\coordinate (spppkw) at (\sxxxk, \syyyw);
\coordinate (spppkx) at (\sxxxk, \syyyx);
\coordinate (spppky) at (\sxxxk, \syyyy);
\coordinate (spppkz) at (\sxxxk, \syyyz);
\coordinate (spppla) at (\sxxxl, \syyya);
\coordinate (sppplb) at (\sxxxl, \syyyb);
\coordinate (sppplc) at (\sxxxl, \syyyc);
\coordinate (spppld) at (\sxxxl, \syyyd);
\coordinate (sppple) at (\sxxxl, \syyye);
\coordinate (sppplf) at (\sxxxl, \syyyf);
\coordinate (sppplg) at (\sxxxl, \syyyg);
\coordinate (sppplh) at (\sxxxl, \syyyh);
\coordinate (spppli) at (\sxxxl, \syyyi);
\coordinate (sppplj) at (\sxxxl, \syyyj);
\coordinate (sppplk) at (\sxxxl, \syyyk);
\coordinate (spppll) at (\sxxxl, \syyyl);
\coordinate (sppplm) at (\sxxxl, \syyym);
\coordinate (spppln) at (\sxxxl, \syyyn);
\coordinate (sppplo) at (\sxxxl, \syyyo);
\coordinate (sppplp) at (\sxxxl, \syyyp);
\coordinate (sppplq) at (\sxxxl, \syyyq);
\coordinate (sppplr) at (\sxxxl, \syyyr);
\coordinate (spppls) at (\sxxxl, \syyys);
\coordinate (sppplt) at (\sxxxl, \syyyt);
\coordinate (sppplu) at (\sxxxl, \syyyu);
\coordinate (sppplv) at (\sxxxl, \syyyv);
\coordinate (sppplw) at (\sxxxl, \syyyw);
\coordinate (sppplx) at (\sxxxl, \syyyx);
\coordinate (sppply) at (\sxxxl, \syyyy);
\coordinate (sppplz) at (\sxxxl, \syyyz);
\coordinate (spppma) at (\sxxxm, \syyya);
\coordinate (spppmb) at (\sxxxm, \syyyb);
\coordinate (spppmc) at (\sxxxm, \syyyc);
\coordinate (spppmd) at (\sxxxm, \syyyd);
\coordinate (spppme) at (\sxxxm, \syyye);
\coordinate (spppmf) at (\sxxxm, \syyyf);
\coordinate (spppmg) at (\sxxxm, \syyyg);
\coordinate (spppmh) at (\sxxxm, \syyyh);
\coordinate (spppmi) at (\sxxxm, \syyyi);
\coordinate (spppmj) at (\sxxxm, \syyyj);
\coordinate (spppmk) at (\sxxxm, \syyyk);
\coordinate (spppml) at (\sxxxm, \syyyl);
\coordinate (spppmm) at (\sxxxm, \syyym);
\coordinate (spppmn) at (\sxxxm, \syyyn);
\coordinate (spppmo) at (\sxxxm, \syyyo);
\coordinate (spppmp) at (\sxxxm, \syyyp);
\coordinate (spppmq) at (\sxxxm, \syyyq);
\coordinate (spppmr) at (\sxxxm, \syyyr);
\coordinate (spppms) at (\sxxxm, \syyys);
\coordinate (spppmt) at (\sxxxm, \syyyt);
\coordinate (spppmu) at (\sxxxm, \syyyu);
\coordinate (spppmv) at (\sxxxm, \syyyv);
\coordinate (spppmw) at (\sxxxm, \syyyw);
\coordinate (spppmx) at (\sxxxm, \syyyx);
\coordinate (spppmy) at (\sxxxm, \syyyy);
\coordinate (spppmz) at (\sxxxm, \syyyz);
\coordinate (spppna) at (\sxxxn, \syyya);
\coordinate (spppnb) at (\sxxxn, \syyyb);
\coordinate (spppnc) at (\sxxxn, \syyyc);
\coordinate (spppnd) at (\sxxxn, \syyyd);
\coordinate (spppne) at (\sxxxn, \syyye);
\coordinate (spppnf) at (\sxxxn, \syyyf);
\coordinate (spppng) at (\sxxxn, \syyyg);
\coordinate (spppnh) at (\sxxxn, \syyyh);
\coordinate (spppni) at (\sxxxn, \syyyi);
\coordinate (spppnj) at (\sxxxn, \syyyj);
\coordinate (spppnk) at (\sxxxn, \syyyk);
\coordinate (spppnl) at (\sxxxn, \syyyl);
\coordinate (spppnm) at (\sxxxn, \syyym);
\coordinate (spppnn) at (\sxxxn, \syyyn);
\coordinate (spppno) at (\sxxxn, \syyyo);
\coordinate (spppnp) at (\sxxxn, \syyyp);
\coordinate (spppnq) at (\sxxxn, \syyyq);
\coordinate (spppnr) at (\sxxxn, \syyyr);
\coordinate (spppns) at (\sxxxn, \syyys);
\coordinate (spppnt) at (\sxxxn, \syyyt);
\coordinate (spppnu) at (\sxxxn, \syyyu);
\coordinate (spppnv) at (\sxxxn, \syyyv);
\coordinate (spppnw) at (\sxxxn, \syyyw);
\coordinate (spppnx) at (\sxxxn, \syyyx);
\coordinate (spppny) at (\sxxxn, \syyyy);
\coordinate (spppnz) at (\sxxxn, \syyyz);
\coordinate (spppoa) at (\sxxxo, \syyya);
\coordinate (spppob) at (\sxxxo, \syyyb);
\coordinate (spppoc) at (\sxxxo, \syyyc);
\coordinate (spppod) at (\sxxxo, \syyyd);
\coordinate (spppoe) at (\sxxxo, \syyye);
\coordinate (spppof) at (\sxxxo, \syyyf);
\coordinate (spppog) at (\sxxxo, \syyyg);
\coordinate (spppoh) at (\sxxxo, \syyyh);
\coordinate (spppoi) at (\sxxxo, \syyyi);
\coordinate (spppoj) at (\sxxxo, \syyyj);
\coordinate (spppok) at (\sxxxo, \syyyk);
\coordinate (spppol) at (\sxxxo, \syyyl);
\coordinate (spppom) at (\sxxxo, \syyym);
\coordinate (spppon) at (\sxxxo, \syyyn);
\coordinate (spppoo) at (\sxxxo, \syyyo);
\coordinate (spppop) at (\sxxxo, \syyyp);
\coordinate (spppoq) at (\sxxxo, \syyyq);
\coordinate (spppor) at (\sxxxo, \syyyr);
\coordinate (spppos) at (\sxxxo, \syyys);
\coordinate (spppot) at (\sxxxo, \syyyt);
\coordinate (spppou) at (\sxxxo, \syyyu);
\coordinate (spppov) at (\sxxxo, \syyyv);
\coordinate (spppow) at (\sxxxo, \syyyw);
\coordinate (spppox) at (\sxxxo, \syyyx);
\coordinate (spppoy) at (\sxxxo, \syyyy);
\coordinate (spppoz) at (\sxxxo, \syyyz);
\coordinate (sppppa) at (\sxxxp, \syyya);
\coordinate (sppppb) at (\sxxxp, \syyyb);
\coordinate (sppppc) at (\sxxxp, \syyyc);
\coordinate (sppppd) at (\sxxxp, \syyyd);
\coordinate (sppppe) at (\sxxxp, \syyye);
\coordinate (sppppf) at (\sxxxp, \syyyf);
\coordinate (sppppg) at (\sxxxp, \syyyg);
\coordinate (spppph) at (\sxxxp, \syyyh);
\coordinate (sppppi) at (\sxxxp, \syyyi);
\coordinate (sppppj) at (\sxxxp, \syyyj);
\coordinate (sppppk) at (\sxxxp, \syyyk);
\coordinate (sppppl) at (\sxxxp, \syyyl);
\coordinate (sppppm) at (\sxxxp, \syyym);
\coordinate (sppppn) at (\sxxxp, \syyyn);
\coordinate (sppppo) at (\sxxxp, \syyyo);
\coordinate (sppppp) at (\sxxxp, \syyyp);
\coordinate (sppppq) at (\sxxxp, \syyyq);
\coordinate (sppppr) at (\sxxxp, \syyyr);
\coordinate (spppps) at (\sxxxp, \syyys);
\coordinate (sppppt) at (\sxxxp, \syyyt);
\coordinate (sppppu) at (\sxxxp, \syyyu);
\coordinate (sppppv) at (\sxxxp, \syyyv);
\coordinate (sppppw) at (\sxxxp, \syyyw);
\coordinate (sppppx) at (\sxxxp, \syyyx);
\coordinate (sppppy) at (\sxxxp, \syyyy);
\coordinate (sppppz) at (\sxxxp, \syyyz);
\coordinate (spppqa) at (\sxxxq, \syyya);
\coordinate (spppqb) at (\sxxxq, \syyyb);
\coordinate (spppqc) at (\sxxxq, \syyyc);
\coordinate (spppqd) at (\sxxxq, \syyyd);
\coordinate (spppqe) at (\sxxxq, \syyye);
\coordinate (spppqf) at (\sxxxq, \syyyf);
\coordinate (spppqg) at (\sxxxq, \syyyg);
\coordinate (spppqh) at (\sxxxq, \syyyh);
\coordinate (spppqi) at (\sxxxq, \syyyi);
\coordinate (spppqj) at (\sxxxq, \syyyj);
\coordinate (spppqk) at (\sxxxq, \syyyk);
\coordinate (spppql) at (\sxxxq, \syyyl);
\coordinate (spppqm) at (\sxxxq, \syyym);
\coordinate (spppqn) at (\sxxxq, \syyyn);
\coordinate (spppqo) at (\sxxxq, \syyyo);
\coordinate (spppqp) at (\sxxxq, \syyyp);
\coordinate (spppqq) at (\sxxxq, \syyyq);
\coordinate (spppqr) at (\sxxxq, \syyyr);
\coordinate (spppqs) at (\sxxxq, \syyys);
\coordinate (spppqt) at (\sxxxq, \syyyt);
\coordinate (spppqu) at (\sxxxq, \syyyu);
\coordinate (spppqv) at (\sxxxq, \syyyv);
\coordinate (spppqw) at (\sxxxq, \syyyw);
\coordinate (spppqx) at (\sxxxq, \syyyx);
\coordinate (spppqy) at (\sxxxq, \syyyy);
\coordinate (spppqz) at (\sxxxq, \syyyz);
\coordinate (spppra) at (\sxxxr, \syyya);
\coordinate (sppprb) at (\sxxxr, \syyyb);
\coordinate (sppprc) at (\sxxxr, \syyyc);
\coordinate (sppprd) at (\sxxxr, \syyyd);
\coordinate (spppre) at (\sxxxr, \syyye);
\coordinate (sppprf) at (\sxxxr, \syyyf);
\coordinate (sppprg) at (\sxxxr, \syyyg);
\coordinate (sppprh) at (\sxxxr, \syyyh);
\coordinate (spppri) at (\sxxxr, \syyyi);
\coordinate (sppprj) at (\sxxxr, \syyyj);
\coordinate (sppprk) at (\sxxxr, \syyyk);
\coordinate (sppprl) at (\sxxxr, \syyyl);
\coordinate (sppprm) at (\sxxxr, \syyym);
\coordinate (sppprn) at (\sxxxr, \syyyn);
\coordinate (spppro) at (\sxxxr, \syyyo);
\coordinate (sppprp) at (\sxxxr, \syyyp);
\coordinate (sppprq) at (\sxxxr, \syyyq);
\coordinate (sppprr) at (\sxxxr, \syyyr);
\coordinate (sppprs) at (\sxxxr, \syyys);
\coordinate (sppprt) at (\sxxxr, \syyyt);
\coordinate (spppru) at (\sxxxr, \syyyu);
\coordinate (sppprv) at (\sxxxr, \syyyv);
\coordinate (sppprw) at (\sxxxr, \syyyw);
\coordinate (sppprx) at (\sxxxr, \syyyx);
\coordinate (spppry) at (\sxxxr, \syyyy);
\coordinate (sppprz) at (\sxxxr, \syyyz);
\coordinate (spppsa) at (\sxxxs, \syyya);
\coordinate (spppsb) at (\sxxxs, \syyyb);
\coordinate (spppsc) at (\sxxxs, \syyyc);
\coordinate (spppsd) at (\sxxxs, \syyyd);
\coordinate (spppse) at (\sxxxs, \syyye);
\coordinate (spppsf) at (\sxxxs, \syyyf);
\coordinate (spppsg) at (\sxxxs, \syyyg);
\coordinate (spppsh) at (\sxxxs, \syyyh);
\coordinate (spppsi) at (\sxxxs, \syyyi);
\coordinate (spppsj) at (\sxxxs, \syyyj);
\coordinate (spppsk) at (\sxxxs, \syyyk);
\coordinate (spppsl) at (\sxxxs, \syyyl);
\coordinate (spppsm) at (\sxxxs, \syyym);
\coordinate (spppsn) at (\sxxxs, \syyyn);
\coordinate (spppso) at (\sxxxs, \syyyo);
\coordinate (spppsp) at (\sxxxs, \syyyp);
\coordinate (spppsq) at (\sxxxs, \syyyq);
\coordinate (spppsr) at (\sxxxs, \syyyr);
\coordinate (spppss) at (\sxxxs, \syyys);
\coordinate (spppst) at (\sxxxs, \syyyt);
\coordinate (spppsu) at (\sxxxs, \syyyu);
\coordinate (spppsv) at (\sxxxs, \syyyv);
\coordinate (spppsw) at (\sxxxs, \syyyw);
\coordinate (spppsx) at (\sxxxs, \syyyx);
\coordinate (spppsy) at (\sxxxs, \syyyy);
\coordinate (spppsz) at (\sxxxs, \syyyz);
\coordinate (spppta) at (\sxxxt, \syyya);
\coordinate (sppptb) at (\sxxxt, \syyyb);
\coordinate (sppptc) at (\sxxxt, \syyyc);
\coordinate (sppptd) at (\sxxxt, \syyyd);
\coordinate (spppte) at (\sxxxt, \syyye);
\coordinate (sppptf) at (\sxxxt, \syyyf);
\coordinate (sppptg) at (\sxxxt, \syyyg);
\coordinate (spppth) at (\sxxxt, \syyyh);
\coordinate (spppti) at (\sxxxt, \syyyi);
\coordinate (sppptj) at (\sxxxt, \syyyj);
\coordinate (sppptk) at (\sxxxt, \syyyk);
\coordinate (sppptl) at (\sxxxt, \syyyl);
\coordinate (sppptm) at (\sxxxt, \syyym);
\coordinate (sppptn) at (\sxxxt, \syyyn);
\coordinate (spppto) at (\sxxxt, \syyyo);
\coordinate (sppptp) at (\sxxxt, \syyyp);
\coordinate (sppptq) at (\sxxxt, \syyyq);
\coordinate (sppptr) at (\sxxxt, \syyyr);
\coordinate (spppts) at (\sxxxt, \syyys);
\coordinate (sppptt) at (\sxxxt, \syyyt);
\coordinate (sppptu) at (\sxxxt, \syyyu);
\coordinate (sppptv) at (\sxxxt, \syyyv);
\coordinate (sppptw) at (\sxxxt, \syyyw);
\coordinate (sppptx) at (\sxxxt, \syyyx);
\coordinate (spppty) at (\sxxxt, \syyyy);
\coordinate (sppptz) at (\sxxxt, \syyyz);
\coordinate (spppua) at (\sxxxu, \syyya);
\coordinate (spppub) at (\sxxxu, \syyyb);
\coordinate (spppuc) at (\sxxxu, \syyyc);
\coordinate (spppud) at (\sxxxu, \syyyd);
\coordinate (spppue) at (\sxxxu, \syyye);
\coordinate (spppuf) at (\sxxxu, \syyyf);
\coordinate (spppug) at (\sxxxu, \syyyg);
\coordinate (spppuh) at (\sxxxu, \syyyh);
\coordinate (spppui) at (\sxxxu, \syyyi);
\coordinate (spppuj) at (\sxxxu, \syyyj);
\coordinate (spppuk) at (\sxxxu, \syyyk);
\coordinate (spppul) at (\sxxxu, \syyyl);
\coordinate (spppum) at (\sxxxu, \syyym);
\coordinate (spppun) at (\sxxxu, \syyyn);
\coordinate (spppuo) at (\sxxxu, \syyyo);
\coordinate (spppup) at (\sxxxu, \syyyp);
\coordinate (spppuq) at (\sxxxu, \syyyq);
\coordinate (spppur) at (\sxxxu, \syyyr);
\coordinate (spppus) at (\sxxxu, \syyys);
\coordinate (sppput) at (\sxxxu, \syyyt);
\coordinate (spppuu) at (\sxxxu, \syyyu);
\coordinate (spppuv) at (\sxxxu, \syyyv);
\coordinate (spppuw) at (\sxxxu, \syyyw);
\coordinate (spppux) at (\sxxxu, \syyyx);
\coordinate (spppuy) at (\sxxxu, \syyyy);
\coordinate (spppuz) at (\sxxxu, \syyyz);
\coordinate (spppva) at (\sxxxv, \syyya);
\coordinate (spppvb) at (\sxxxv, \syyyb);
\coordinate (spppvc) at (\sxxxv, \syyyc);
\coordinate (spppvd) at (\sxxxv, \syyyd);
\coordinate (spppve) at (\sxxxv, \syyye);
\coordinate (spppvf) at (\sxxxv, \syyyf);
\coordinate (spppvg) at (\sxxxv, \syyyg);
\coordinate (spppvh) at (\sxxxv, \syyyh);
\coordinate (spppvi) at (\sxxxv, \syyyi);
\coordinate (spppvj) at (\sxxxv, \syyyj);
\coordinate (spppvk) at (\sxxxv, \syyyk);
\coordinate (spppvl) at (\sxxxv, \syyyl);
\coordinate (spppvm) at (\sxxxv, \syyym);
\coordinate (spppvn) at (\sxxxv, \syyyn);
\coordinate (spppvo) at (\sxxxv, \syyyo);
\coordinate (spppvp) at (\sxxxv, \syyyp);
\coordinate (spppvq) at (\sxxxv, \syyyq);
\coordinate (spppvr) at (\sxxxv, \syyyr);
\coordinate (spppvs) at (\sxxxv, \syyys);
\coordinate (spppvt) at (\sxxxv, \syyyt);
\coordinate (spppvu) at (\sxxxv, \syyyu);
\coordinate (spppvv) at (\sxxxv, \syyyv);
\coordinate (spppvw) at (\sxxxv, \syyyw);
\coordinate (spppvx) at (\sxxxv, \syyyx);
\coordinate (spppvy) at (\sxxxv, \syyyy);
\coordinate (spppvz) at (\sxxxv, \syyyz);
\coordinate (spppwa) at (\sxxxw, \syyya);
\coordinate (spppwb) at (\sxxxw, \syyyb);
\coordinate (spppwc) at (\sxxxw, \syyyc);
\coordinate (spppwd) at (\sxxxw, \syyyd);
\coordinate (spppwe) at (\sxxxw, \syyye);
\coordinate (spppwf) at (\sxxxw, \syyyf);
\coordinate (spppwg) at (\sxxxw, \syyyg);
\coordinate (spppwh) at (\sxxxw, \syyyh);
\coordinate (spppwi) at (\sxxxw, \syyyi);
\coordinate (spppwj) at (\sxxxw, \syyyj);
\coordinate (spppwk) at (\sxxxw, \syyyk);
\coordinate (spppwl) at (\sxxxw, \syyyl);
\coordinate (spppwm) at (\sxxxw, \syyym);
\coordinate (spppwn) at (\sxxxw, \syyyn);
\coordinate (spppwo) at (\sxxxw, \syyyo);
\coordinate (spppwp) at (\sxxxw, \syyyp);
\coordinate (spppwq) at (\sxxxw, \syyyq);
\coordinate (spppwr) at (\sxxxw, \syyyr);
\coordinate (spppws) at (\sxxxw, \syyys);
\coordinate (spppwt) at (\sxxxw, \syyyt);
\coordinate (spppwu) at (\sxxxw, \syyyu);
\coordinate (spppwv) at (\sxxxw, \syyyv);
\coordinate (spppww) at (\sxxxw, \syyyw);
\coordinate (spppwx) at (\sxxxw, \syyyx);
\coordinate (spppwy) at (\sxxxw, \syyyy);
\coordinate (spppwz) at (\sxxxw, \syyyz);
\coordinate (spppxa) at (\sxxxx, \syyya);
\coordinate (spppxb) at (\sxxxx, \syyyb);
\coordinate (spppxc) at (\sxxxx, \syyyc);
\coordinate (spppxd) at (\sxxxx, \syyyd);
\coordinate (spppxe) at (\sxxxx, \syyye);
\coordinate (spppxf) at (\sxxxx, \syyyf);
\coordinate (spppxg) at (\sxxxx, \syyyg);
\coordinate (spppxh) at (\sxxxx, \syyyh);
\coordinate (spppxi) at (\sxxxx, \syyyi);
\coordinate (spppxj) at (\sxxxx, \syyyj);
\coordinate (spppxk) at (\sxxxx, \syyyk);
\coordinate (spppxl) at (\sxxxx, \syyyl);
\coordinate (spppxm) at (\sxxxx, \syyym);
\coordinate (spppxn) at (\sxxxx, \syyyn);
\coordinate (spppxo) at (\sxxxx, \syyyo);
\coordinate (spppxp) at (\sxxxx, \syyyp);
\coordinate (spppxq) at (\sxxxx, \syyyq);
\coordinate (spppxr) at (\sxxxx, \syyyr);
\coordinate (spppxs) at (\sxxxx, \syyys);
\coordinate (spppxt) at (\sxxxx, \syyyt);
\coordinate (spppxu) at (\sxxxx, \syyyu);
\coordinate (spppxv) at (\sxxxx, \syyyv);
\coordinate (spppxw) at (\sxxxx, \syyyw);
\coordinate (spppxx) at (\sxxxx, \syyyx);
\coordinate (spppxy) at (\sxxxx, \syyyy);
\coordinate (spppxz) at (\sxxxx, \syyyz);
\coordinate (spppya) at (\sxxxy, \syyya);
\coordinate (spppyb) at (\sxxxy, \syyyb);
\coordinate (spppyc) at (\sxxxy, \syyyc);
\coordinate (spppyd) at (\sxxxy, \syyyd);
\coordinate (spppye) at (\sxxxy, \syyye);
\coordinate (spppyf) at (\sxxxy, \syyyf);
\coordinate (spppyg) at (\sxxxy, \syyyg);
\coordinate (spppyh) at (\sxxxy, \syyyh);
\coordinate (spppyi) at (\sxxxy, \syyyi);
\coordinate (spppyj) at (\sxxxy, \syyyj);
\coordinate (spppyk) at (\sxxxy, \syyyk);
\coordinate (spppyl) at (\sxxxy, \syyyl);
\coordinate (spppym) at (\sxxxy, \syyym);
\coordinate (spppyn) at (\sxxxy, \syyyn);
\coordinate (spppyo) at (\sxxxy, \syyyo);
\coordinate (spppyp) at (\sxxxy, \syyyp);
\coordinate (spppyq) at (\sxxxy, \syyyq);
\coordinate (spppyr) at (\sxxxy, \syyyr);
\coordinate (spppys) at (\sxxxy, \syyys);
\coordinate (spppyt) at (\sxxxy, \syyyt);
\coordinate (spppyu) at (\sxxxy, \syyyu);
\coordinate (spppyv) at (\sxxxy, \syyyv);
\coordinate (spppyw) at (\sxxxy, \syyyw);
\coordinate (spppyx) at (\sxxxy, \syyyx);
\coordinate (spppyy) at (\sxxxy, \syyyy);
\coordinate (spppyz) at (\sxxxy, \syyyz);
\coordinate (spppza) at (\sxxxz, \syyya);
\coordinate (spppzb) at (\sxxxz, \syyyb);
\coordinate (spppzc) at (\sxxxz, \syyyc);
\coordinate (spppzd) at (\sxxxz, \syyyd);
\coordinate (spppze) at (\sxxxz, \syyye);
\coordinate (spppzf) at (\sxxxz, \syyyf);
\coordinate (spppzg) at (\sxxxz, \syyyg);
\coordinate (spppzh) at (\sxxxz, \syyyh);
\coordinate (spppzi) at (\sxxxz, \syyyi);
\coordinate (spppzj) at (\sxxxz, \syyyj);
\coordinate (spppzk) at (\sxxxz, \syyyk);
\coordinate (spppzl) at (\sxxxz, \syyyl);
\coordinate (spppzm) at (\sxxxz, \syyym);
\coordinate (spppzn) at (\sxxxz, \syyyn);
\coordinate (spppzo) at (\sxxxz, \syyyo);
\coordinate (spppzp) at (\sxxxz, \syyyp);
\coordinate (spppzq) at (\sxxxz, \syyyq);
\coordinate (spppzr) at (\sxxxz, \syyyr);
\coordinate (spppzs) at (\sxxxz, \syyys);
\coordinate (spppzt) at (\sxxxz, \syyyt);
\coordinate (spppzu) at (\sxxxz, \syyyu);
\coordinate (spppzv) at (\sxxxz, \syyyv);
\coordinate (spppzw) at (\sxxxz, \syyyw);
\coordinate (spppzx) at (\sxxxz, \syyyx);
\coordinate (spppzy) at (\sxxxz, \syyyy);
\coordinate (spppzz) at (\sxxxz, \syyyz);

%\gangprintcoordinateat{(0,0)}{The last coordinate values: }{($(spppzz)$)}; 

\coordinatebackground{s}{f}{g}{q};
\draw (spppni) node [op amp] (opamp) {};
\getxyingivenunit{cm}{(opamp.+)}{\opampzx}{\opampzy};
\getxyingivenunit{cm}{(opamp.-)}{\opampfx}{\opampfy};

\draw (\sxxxg,\opampzy) node [left] {$U_{we}$} to [R, l=$R_d$, -*]  (\sxxxj,\opampzy) 
to [R, l=$R_d$, *-*] (opamp.+)
to [C, l_=$C_{d2}$, *-] (\opampzx, \syyyf);

\draw (opamp.out) |- (\sxxxl,\syyyk) to [C, l_=$C_{d1}$, *-] (\sxxxj,\syyyk) to [short](\sxxxj,\opampzy);

\draw (opamp.-) -|  (\sxxxl,\syyyk);

\draw (opamp.out) to [short, *-] (\sxxxq,\syyyi);



\end{circuitikz}












\newpage


\vspace{2cm}

{\Large Figure 7, the coordinate system of the  original circuit is easily switched to the one with keyword ``g", which is adjusted to allocate  spaces for the two circuits to be inserted, without changing the circuit logic. (Keeping the ``demobygangliu" system unchanged, so that the above circuits using it are not affected.)} 

\begin{circuitikz}[scale=1]


% Circuits can be drawn by the following five major steps, as shown in the following example. 

% Step 1, preparations. 

% "Install" the coordinate system with keyword ``g".
\pgfmathsetmacro{\totalgxxx}{26}
\pgfmathsetmacro{\totalgyyy}{26}
\pgfmathsetmacro{\gxxxspacing}{1}
\pgfmathsetmacro{\gyyyspacing}{1}
\pgfmathsetmacro{\gxxxa}{-8}
\pgfmathsetmacro{\gyyya}{-8}

\pgfmathsetmacro{\gxxxb}{\gxxxa + \gxxxspacing + 0.0 }
\pgfmathsetmacro{\gxxxc}{\gxxxb + \gxxxspacing + 0.0 }
\pgfmathsetmacro{\gxxxd}{\gxxxc + \gxxxspacing + 0.0 }
\pgfmathsetmacro{\gxxxe}{\gxxxd + \gxxxspacing + 0.0 }
\pgfmathsetmacro{\gxxxf}{\gxxxe + \gxxxspacing + 0.0 }
\pgfmathsetmacro{\gxxxg}{\gxxxf + \gxxxspacing + 0.0 }
\pgfmathsetmacro{\gxxxh}{\gxxxg + \gxxxspacing + 0.0 }
\pgfmathsetmacro{\gxxxi}{\gxxxh + \gxxxspacing + 0.0 }
\pgfmathsetmacro{\gxxxj}{\gxxxi + \gxxxspacing + 0.0 }
\pgfmathsetmacro{\gxxxk}{\gxxxj + \gxxxspacing + 0.0 }
\pgfmathsetmacro{\gxxxl}{\gxxxk + \gxxxspacing + 8.0 }
\pgfmathsetmacro{\gxxxm}{\gxxxl + \gxxxspacing + 0.0 }
\pgfmathsetmacro{\gxxxn}{\gxxxm + \gxxxspacing + 0.0 }
\pgfmathsetmacro{\gxxxo}{\gxxxn + \gxxxspacing + 0.0 }
\pgfmathsetmacro{\gxxxp}{\gxxxo + \gxxxspacing + 0.0 }
\pgfmathsetmacro{\gxxxq}{\gxxxp + \gxxxspacing + 0.0 }
\pgfmathsetmacro{\gxxxr}{\gxxxq + \gxxxspacing + 0.0 }
\pgfmathsetmacro{\gxxxs}{\gxxxr + \gxxxspacing + 0.0 }
\pgfmathsetmacro{\gxxxt}{\gxxxs + \gxxxspacing + 0.0 }
\pgfmathsetmacro{\gxxxu}{\gxxxt + \gxxxspacing + 0.0 }
\pgfmathsetmacro{\gxxxv}{\gxxxu + \gxxxspacing + 0.0 }
\pgfmathsetmacro{\gxxxw}{\gxxxv + \gxxxspacing + 0.0 }
\pgfmathsetmacro{\gxxxx}{\gxxxw + \gxxxspacing + 0.0 }
\pgfmathsetmacro{\gxxxy}{\gxxxx + \gxxxspacing + 0.0 }
\pgfmathsetmacro{\gxxxz}{\gxxxy + \gxxxspacing + 0.0 }

\pgfmathsetmacro{\gyyyb}{\gyyya + \gyyyspacing + 2.0 }
\pgfmathsetmacro{\gyyyc}{\gyyyb + \gyyyspacing + 0.0 }
\pgfmathsetmacro{\gyyyd}{\gyyyc + \gyyyspacing + 0.0 }
\pgfmathsetmacro{\gyyye}{\gyyyd + \gyyyspacing + 0.0 }
\pgfmathsetmacro{\gyyyf}{\gyyye + \gyyyspacing + 0.0 }
\pgfmathsetmacro{\gyyyg}{\gyyyf + \gyyyspacing + 0.0 }
\pgfmathsetmacro{\gyyyh}{\gyyyg + \gyyyspacing + 0.0 }
\pgfmathsetmacro{\gyyyi}{\gyyyh + \gyyyspacing + 0.0 }
\pgfmathsetmacro{\gyyyj}{\gyyyi + \gyyyspacing + 0.0 }
\pgfmathsetmacro{\gyyyk}{\gyyyj + \gyyyspacing + 0.0 }
\pgfmathsetmacro{\gyyyl}{\gyyyk + \gyyyspacing + 12.0 }
\pgfmathsetmacro{\gyyym}{\gyyyl + \gyyyspacing + 0.0 }
\pgfmathsetmacro{\gyyyn}{\gyyym + \gyyyspacing + 0.0 }
\pgfmathsetmacro{\gyyyo}{\gyyyn + \gyyyspacing + 0.0 }
\pgfmathsetmacro{\gyyyp}{\gyyyo + \gyyyspacing + 0.0 }
\pgfmathsetmacro{\gyyyq}{\gyyyp + \gyyyspacing + 0.0 }
\pgfmathsetmacro{\gyyyr}{\gyyyq + \gyyyspacing + 0.0 }
\pgfmathsetmacro{\gyyys}{\gyyyr + \gyyyspacing + 0.0 }
\pgfmathsetmacro{\gyyyt}{\gyyys + \gyyyspacing + 0.0 }
\pgfmathsetmacro{\gyyyu}{\gyyyt + \gyyyspacing + 0.0 }
\pgfmathsetmacro{\gyyyv}{\gyyyu + \gyyyspacing + 0.0 }
\pgfmathsetmacro{\gyyyw}{\gyyyv + \gyyyspacing + 0.0 }
\pgfmathsetmacro{\gyyyx}{\gyyyw + \gyyyspacing + 0.0 }
\pgfmathsetmacro{\gyyyy}{\gyyyx + \gyyyspacing + 0.0 }
\pgfmathsetmacro{\gyyyz}{\gyyyy + \gyyyspacing + 0.0 }

\coordinate (gpppaa) at (\gxxxa, \gyyya);
\coordinate (gpppab) at (\gxxxa, \gyyyb);
\coordinate (gpppac) at (\gxxxa, \gyyyc);
\coordinate (gpppad) at (\gxxxa, \gyyyd);
\coordinate (gpppae) at (\gxxxa, \gyyye);
\coordinate (gpppaf) at (\gxxxa, \gyyyf);
\coordinate (gpppag) at (\gxxxa, \gyyyg);
\coordinate (gpppah) at (\gxxxa, \gyyyh);
\coordinate (gpppai) at (\gxxxa, \gyyyi);
\coordinate (gpppaj) at (\gxxxa, \gyyyj);
\coordinate (gpppak) at (\gxxxa, \gyyyk);
\coordinate (gpppal) at (\gxxxa, \gyyyl);
\coordinate (gpppam) at (\gxxxa, \gyyym);
\coordinate (gpppan) at (\gxxxa, \gyyyn);
\coordinate (gpppao) at (\gxxxa, \gyyyo);
\coordinate (gpppap) at (\gxxxa, \gyyyp);
\coordinate (gpppaq) at (\gxxxa, \gyyyq);
\coordinate (gpppar) at (\gxxxa, \gyyyr);
\coordinate (gpppas) at (\gxxxa, \gyyys);
\coordinate (gpppat) at (\gxxxa, \gyyyt);
\coordinate (gpppau) at (\gxxxa, \gyyyu);
\coordinate (gpppav) at (\gxxxa, \gyyyv);
\coordinate (gpppaw) at (\gxxxa, \gyyyw);
\coordinate (gpppax) at (\gxxxa, \gyyyx);
\coordinate (gpppay) at (\gxxxa, \gyyyy);
\coordinate (gpppaz) at (\gxxxa, \gyyyz);
\coordinate (gpppba) at (\gxxxb, \gyyya);
\coordinate (gpppbb) at (\gxxxb, \gyyyb);
\coordinate (gpppbc) at (\gxxxb, \gyyyc);
\coordinate (gpppbd) at (\gxxxb, \gyyyd);
\coordinate (gpppbe) at (\gxxxb, \gyyye);
\coordinate (gpppbf) at (\gxxxb, \gyyyf);
\coordinate (gpppbg) at (\gxxxb, \gyyyg);
\coordinate (gpppbh) at (\gxxxb, \gyyyh);
\coordinate (gpppbi) at (\gxxxb, \gyyyi);
\coordinate (gpppbj) at (\gxxxb, \gyyyj);
\coordinate (gpppbk) at (\gxxxb, \gyyyk);
\coordinate (gpppbl) at (\gxxxb, \gyyyl);
\coordinate (gpppbm) at (\gxxxb, \gyyym);
\coordinate (gpppbn) at (\gxxxb, \gyyyn);
\coordinate (gpppbo) at (\gxxxb, \gyyyo);
\coordinate (gpppbp) at (\gxxxb, \gyyyp);
\coordinate (gpppbq) at (\gxxxb, \gyyyq);
\coordinate (gpppbr) at (\gxxxb, \gyyyr);
\coordinate (gpppbs) at (\gxxxb, \gyyys);
\coordinate (gpppbt) at (\gxxxb, \gyyyt);
\coordinate (gpppbu) at (\gxxxb, \gyyyu);
\coordinate (gpppbv) at (\gxxxb, \gyyyv);
\coordinate (gpppbw) at (\gxxxb, \gyyyw);
\coordinate (gpppbx) at (\gxxxb, \gyyyx);
\coordinate (gpppby) at (\gxxxb, \gyyyy);
\coordinate (gpppbz) at (\gxxxb, \gyyyz);
\coordinate (gpppca) at (\gxxxc, \gyyya);
\coordinate (gpppcb) at (\gxxxc, \gyyyb);
\coordinate (gpppcc) at (\gxxxc, \gyyyc);
\coordinate (gpppcd) at (\gxxxc, \gyyyd);
\coordinate (gpppce) at (\gxxxc, \gyyye);
\coordinate (gpppcf) at (\gxxxc, \gyyyf);
\coordinate (gpppcg) at (\gxxxc, \gyyyg);
\coordinate (gpppch) at (\gxxxc, \gyyyh);
\coordinate (gpppci) at (\gxxxc, \gyyyi);
\coordinate (gpppcj) at (\gxxxc, \gyyyj);
\coordinate (gpppck) at (\gxxxc, \gyyyk);
\coordinate (gpppcl) at (\gxxxc, \gyyyl);
\coordinate (gpppcm) at (\gxxxc, \gyyym);
\coordinate (gpppcn) at (\gxxxc, \gyyyn);
\coordinate (gpppco) at (\gxxxc, \gyyyo);
\coordinate (gpppcp) at (\gxxxc, \gyyyp);
\coordinate (gpppcq) at (\gxxxc, \gyyyq);
\coordinate (gpppcr) at (\gxxxc, \gyyyr);
\coordinate (gpppcs) at (\gxxxc, \gyyys);
\coordinate (gpppct) at (\gxxxc, \gyyyt);
\coordinate (gpppcu) at (\gxxxc, \gyyyu);
\coordinate (gpppcv) at (\gxxxc, \gyyyv);
\coordinate (gpppcw) at (\gxxxc, \gyyyw);
\coordinate (gpppcx) at (\gxxxc, \gyyyx);
\coordinate (gpppcy) at (\gxxxc, \gyyyy);
\coordinate (gpppcz) at (\gxxxc, \gyyyz);
\coordinate (gpppda) at (\gxxxd, \gyyya);
\coordinate (gpppdb) at (\gxxxd, \gyyyb);
\coordinate (gpppdc) at (\gxxxd, \gyyyc);
\coordinate (gpppdd) at (\gxxxd, \gyyyd);
\coordinate (gpppde) at (\gxxxd, \gyyye);
\coordinate (gpppdf) at (\gxxxd, \gyyyf);
\coordinate (gpppdg) at (\gxxxd, \gyyyg);
\coordinate (gpppdh) at (\gxxxd, \gyyyh);
\coordinate (gpppdi) at (\gxxxd, \gyyyi);
\coordinate (gpppdj) at (\gxxxd, \gyyyj);
\coordinate (gpppdk) at (\gxxxd, \gyyyk);
\coordinate (gpppdl) at (\gxxxd, \gyyyl);
\coordinate (gpppdm) at (\gxxxd, \gyyym);
\coordinate (gpppdn) at (\gxxxd, \gyyyn);
\coordinate (gpppdo) at (\gxxxd, \gyyyo);
\coordinate (gpppdp) at (\gxxxd, \gyyyp);
\coordinate (gpppdq) at (\gxxxd, \gyyyq);
\coordinate (gpppdr) at (\gxxxd, \gyyyr);
\coordinate (gpppds) at (\gxxxd, \gyyys);
\coordinate (gpppdt) at (\gxxxd, \gyyyt);
\coordinate (gpppdu) at (\gxxxd, \gyyyu);
\coordinate (gpppdv) at (\gxxxd, \gyyyv);
\coordinate (gpppdw) at (\gxxxd, \gyyyw);
\coordinate (gpppdx) at (\gxxxd, \gyyyx);
\coordinate (gpppdy) at (\gxxxd, \gyyyy);
\coordinate (gpppdz) at (\gxxxd, \gyyyz);
\coordinate (gpppea) at (\gxxxe, \gyyya);
\coordinate (gpppeb) at (\gxxxe, \gyyyb);
\coordinate (gpppec) at (\gxxxe, \gyyyc);
\coordinate (gppped) at (\gxxxe, \gyyyd);
\coordinate (gpppee) at (\gxxxe, \gyyye);
\coordinate (gpppef) at (\gxxxe, \gyyyf);
\coordinate (gpppeg) at (\gxxxe, \gyyyg);
\coordinate (gpppeh) at (\gxxxe, \gyyyh);
\coordinate (gpppei) at (\gxxxe, \gyyyi);
\coordinate (gpppej) at (\gxxxe, \gyyyj);
\coordinate (gpppek) at (\gxxxe, \gyyyk);
\coordinate (gpppel) at (\gxxxe, \gyyyl);
\coordinate (gpppem) at (\gxxxe, \gyyym);
\coordinate (gpppen) at (\gxxxe, \gyyyn);
\coordinate (gpppeo) at (\gxxxe, \gyyyo);
\coordinate (gpppep) at (\gxxxe, \gyyyp);
\coordinate (gpppeq) at (\gxxxe, \gyyyq);
\coordinate (gppper) at (\gxxxe, \gyyyr);
\coordinate (gpppes) at (\gxxxe, \gyyys);
\coordinate (gpppet) at (\gxxxe, \gyyyt);
\coordinate (gpppeu) at (\gxxxe, \gyyyu);
\coordinate (gpppev) at (\gxxxe, \gyyyv);
\coordinate (gpppew) at (\gxxxe, \gyyyw);
\coordinate (gpppex) at (\gxxxe, \gyyyx);
\coordinate (gpppey) at (\gxxxe, \gyyyy);
\coordinate (gpppez) at (\gxxxe, \gyyyz);
\coordinate (gpppfa) at (\gxxxf, \gyyya);
\coordinate (gpppfb) at (\gxxxf, \gyyyb);
\coordinate (gpppfc) at (\gxxxf, \gyyyc);
\coordinate (gpppfd) at (\gxxxf, \gyyyd);
\coordinate (gpppfe) at (\gxxxf, \gyyye);
\coordinate (gpppff) at (\gxxxf, \gyyyf);
\coordinate (gpppfg) at (\gxxxf, \gyyyg);
\coordinate (gpppfh) at (\gxxxf, \gyyyh);
\coordinate (gpppfi) at (\gxxxf, \gyyyi);
\coordinate (gpppfj) at (\gxxxf, \gyyyj);
\coordinate (gpppfk) at (\gxxxf, \gyyyk);
\coordinate (gpppfl) at (\gxxxf, \gyyyl);
\coordinate (gpppfm) at (\gxxxf, \gyyym);
\coordinate (gpppfn) at (\gxxxf, \gyyyn);
\coordinate (gpppfo) at (\gxxxf, \gyyyo);
\coordinate (gpppfp) at (\gxxxf, \gyyyp);
\coordinate (gpppfq) at (\gxxxf, \gyyyq);
\coordinate (gpppfr) at (\gxxxf, \gyyyr);
\coordinate (gpppfs) at (\gxxxf, \gyyys);
\coordinate (gpppft) at (\gxxxf, \gyyyt);
\coordinate (gpppfu) at (\gxxxf, \gyyyu);
\coordinate (gpppfv) at (\gxxxf, \gyyyv);
\coordinate (gpppfw) at (\gxxxf, \gyyyw);
\coordinate (gpppfx) at (\gxxxf, \gyyyx);
\coordinate (gpppfy) at (\gxxxf, \gyyyy);
\coordinate (gpppfz) at (\gxxxf, \gyyyz);
\coordinate (gpppga) at (\gxxxg, \gyyya);
\coordinate (gpppgb) at (\gxxxg, \gyyyb);
\coordinate (gpppgc) at (\gxxxg, \gyyyc);
\coordinate (gpppgd) at (\gxxxg, \gyyyd);
\coordinate (gpppge) at (\gxxxg, \gyyye);
\coordinate (gpppgf) at (\gxxxg, \gyyyf);
\coordinate (gpppgg) at (\gxxxg, \gyyyg);
\coordinate (gpppgh) at (\gxxxg, \gyyyh);
\coordinate (gpppgi) at (\gxxxg, \gyyyi);
\coordinate (gpppgj) at (\gxxxg, \gyyyj);
\coordinate (gpppgk) at (\gxxxg, \gyyyk);
\coordinate (gpppgl) at (\gxxxg, \gyyyl);
\coordinate (gpppgm) at (\gxxxg, \gyyym);
\coordinate (gpppgn) at (\gxxxg, \gyyyn);
\coordinate (gpppgo) at (\gxxxg, \gyyyo);
\coordinate (gpppgp) at (\gxxxg, \gyyyp);
\coordinate (gpppgq) at (\gxxxg, \gyyyq);
\coordinate (gpppgr) at (\gxxxg, \gyyyr);
\coordinate (gpppgs) at (\gxxxg, \gyyys);
\coordinate (gpppgt) at (\gxxxg, \gyyyt);
\coordinate (gpppgu) at (\gxxxg, \gyyyu);
\coordinate (gpppgv) at (\gxxxg, \gyyyv);
\coordinate (gpppgw) at (\gxxxg, \gyyyw);
\coordinate (gpppgx) at (\gxxxg, \gyyyx);
\coordinate (gpppgy) at (\gxxxg, \gyyyy);
\coordinate (gpppgz) at (\gxxxg, \gyyyz);
\coordinate (gpppha) at (\gxxxh, \gyyya);
\coordinate (gppphb) at (\gxxxh, \gyyyb);
\coordinate (gppphc) at (\gxxxh, \gyyyc);
\coordinate (gppphd) at (\gxxxh, \gyyyd);
\coordinate (gppphe) at (\gxxxh, \gyyye);
\coordinate (gppphf) at (\gxxxh, \gyyyf);
\coordinate (gppphg) at (\gxxxh, \gyyyg);
\coordinate (gppphh) at (\gxxxh, \gyyyh);
\coordinate (gppphi) at (\gxxxh, \gyyyi);
\coordinate (gppphj) at (\gxxxh, \gyyyj);
\coordinate (gppphk) at (\gxxxh, \gyyyk);
\coordinate (gppphl) at (\gxxxh, \gyyyl);
\coordinate (gppphm) at (\gxxxh, \gyyym);
\coordinate (gppphn) at (\gxxxh, \gyyyn);
\coordinate (gpppho) at (\gxxxh, \gyyyo);
\coordinate (gppphp) at (\gxxxh, \gyyyp);
\coordinate (gppphq) at (\gxxxh, \gyyyq);
\coordinate (gppphr) at (\gxxxh, \gyyyr);
\coordinate (gppphs) at (\gxxxh, \gyyys);
\coordinate (gpppht) at (\gxxxh, \gyyyt);
\coordinate (gppphu) at (\gxxxh, \gyyyu);
\coordinate (gppphv) at (\gxxxh, \gyyyv);
\coordinate (gppphw) at (\gxxxh, \gyyyw);
\coordinate (gppphx) at (\gxxxh, \gyyyx);
\coordinate (gppphy) at (\gxxxh, \gyyyy);
\coordinate (gppphz) at (\gxxxh, \gyyyz);
\coordinate (gpppia) at (\gxxxi, \gyyya);
\coordinate (gpppib) at (\gxxxi, \gyyyb);
\coordinate (gpppic) at (\gxxxi, \gyyyc);
\coordinate (gpppid) at (\gxxxi, \gyyyd);
\coordinate (gpppie) at (\gxxxi, \gyyye);
\coordinate (gpppif) at (\gxxxi, \gyyyf);
\coordinate (gpppig) at (\gxxxi, \gyyyg);
\coordinate (gpppih) at (\gxxxi, \gyyyh);
\coordinate (gpppii) at (\gxxxi, \gyyyi);
\coordinate (gpppij) at (\gxxxi, \gyyyj);
\coordinate (gpppik) at (\gxxxi, \gyyyk);
\coordinate (gpppil) at (\gxxxi, \gyyyl);
\coordinate (gpppim) at (\gxxxi, \gyyym);
\coordinate (gpppin) at (\gxxxi, \gyyyn);
\coordinate (gpppio) at (\gxxxi, \gyyyo);
\coordinate (gpppip) at (\gxxxi, \gyyyp);
\coordinate (gpppiq) at (\gxxxi, \gyyyq);
\coordinate (gpppir) at (\gxxxi, \gyyyr);
\coordinate (gpppis) at (\gxxxi, \gyyys);
\coordinate (gpppit) at (\gxxxi, \gyyyt);
\coordinate (gpppiu) at (\gxxxi, \gyyyu);
\coordinate (gpppiv) at (\gxxxi, \gyyyv);
\coordinate (gpppiw) at (\gxxxi, \gyyyw);
\coordinate (gpppix) at (\gxxxi, \gyyyx);
\coordinate (gpppiy) at (\gxxxi, \gyyyy);
\coordinate (gpppiz) at (\gxxxi, \gyyyz);
\coordinate (gpppja) at (\gxxxj, \gyyya);
\coordinate (gpppjb) at (\gxxxj, \gyyyb);
\coordinate (gpppjc) at (\gxxxj, \gyyyc);
\coordinate (gpppjd) at (\gxxxj, \gyyyd);
\coordinate (gpppje) at (\gxxxj, \gyyye);
\coordinate (gpppjf) at (\gxxxj, \gyyyf);
\coordinate (gpppjg) at (\gxxxj, \gyyyg);
\coordinate (gpppjh) at (\gxxxj, \gyyyh);
\coordinate (gpppji) at (\gxxxj, \gyyyi);
\coordinate (gpppjj) at (\gxxxj, \gyyyj);
\coordinate (gpppjk) at (\gxxxj, \gyyyk);
\coordinate (gpppjl) at (\gxxxj, \gyyyl);
\coordinate (gpppjm) at (\gxxxj, \gyyym);
\coordinate (gpppjn) at (\gxxxj, \gyyyn);
\coordinate (gpppjo) at (\gxxxj, \gyyyo);
\coordinate (gpppjp) at (\gxxxj, \gyyyp);
\coordinate (gpppjq) at (\gxxxj, \gyyyq);
\coordinate (gpppjr) at (\gxxxj, \gyyyr);
\coordinate (gpppjs) at (\gxxxj, \gyyys);
\coordinate (gpppjt) at (\gxxxj, \gyyyt);
\coordinate (gpppju) at (\gxxxj, \gyyyu);
\coordinate (gpppjv) at (\gxxxj, \gyyyv);
\coordinate (gpppjw) at (\gxxxj, \gyyyw);
\coordinate (gpppjx) at (\gxxxj, \gyyyx);
\coordinate (gpppjy) at (\gxxxj, \gyyyy);
\coordinate (gpppjz) at (\gxxxj, \gyyyz);
\coordinate (gpppka) at (\gxxxk, \gyyya);
\coordinate (gpppkb) at (\gxxxk, \gyyyb);
\coordinate (gpppkc) at (\gxxxk, \gyyyc);
\coordinate (gpppkd) at (\gxxxk, \gyyyd);
\coordinate (gpppke) at (\gxxxk, \gyyye);
\coordinate (gpppkf) at (\gxxxk, \gyyyf);
\coordinate (gpppkg) at (\gxxxk, \gyyyg);
\coordinate (gpppkh) at (\gxxxk, \gyyyh);
\coordinate (gpppki) at (\gxxxk, \gyyyi);
\coordinate (gpppkj) at (\gxxxk, \gyyyj);
\coordinate (gpppkk) at (\gxxxk, \gyyyk);
\coordinate (gpppkl) at (\gxxxk, \gyyyl);
\coordinate (gpppkm) at (\gxxxk, \gyyym);
\coordinate (gpppkn) at (\gxxxk, \gyyyn);
\coordinate (gpppko) at (\gxxxk, \gyyyo);
\coordinate (gpppkp) at (\gxxxk, \gyyyp);
\coordinate (gpppkq) at (\gxxxk, \gyyyq);
\coordinate (gpppkr) at (\gxxxk, \gyyyr);
\coordinate (gpppks) at (\gxxxk, \gyyys);
\coordinate (gpppkt) at (\gxxxk, \gyyyt);
\coordinate (gpppku) at (\gxxxk, \gyyyu);
\coordinate (gpppkv) at (\gxxxk, \gyyyv);
\coordinate (gpppkw) at (\gxxxk, \gyyyw);
\coordinate (gpppkx) at (\gxxxk, \gyyyx);
\coordinate (gpppky) at (\gxxxk, \gyyyy);
\coordinate (gpppkz) at (\gxxxk, \gyyyz);
\coordinate (gpppla) at (\gxxxl, \gyyya);
\coordinate (gppplb) at (\gxxxl, \gyyyb);
\coordinate (gppplc) at (\gxxxl, \gyyyc);
\coordinate (gpppld) at (\gxxxl, \gyyyd);
\coordinate (gppple) at (\gxxxl, \gyyye);
\coordinate (gppplf) at (\gxxxl, \gyyyf);
\coordinate (gppplg) at (\gxxxl, \gyyyg);
\coordinate (gppplh) at (\gxxxl, \gyyyh);
\coordinate (gpppli) at (\gxxxl, \gyyyi);
\coordinate (gppplj) at (\gxxxl, \gyyyj);
\coordinate (gppplk) at (\gxxxl, \gyyyk);
\coordinate (gpppll) at (\gxxxl, \gyyyl);
\coordinate (gppplm) at (\gxxxl, \gyyym);
\coordinate (gpppln) at (\gxxxl, \gyyyn);
\coordinate (gppplo) at (\gxxxl, \gyyyo);
\coordinate (gppplp) at (\gxxxl, \gyyyp);
\coordinate (gppplq) at (\gxxxl, \gyyyq);
\coordinate (gppplr) at (\gxxxl, \gyyyr);
\coordinate (gpppls) at (\gxxxl, \gyyys);
\coordinate (gppplt) at (\gxxxl, \gyyyt);
\coordinate (gppplu) at (\gxxxl, \gyyyu);
\coordinate (gppplv) at (\gxxxl, \gyyyv);
\coordinate (gppplw) at (\gxxxl, \gyyyw);
\coordinate (gppplx) at (\gxxxl, \gyyyx);
\coordinate (gppply) at (\gxxxl, \gyyyy);
\coordinate (gppplz) at (\gxxxl, \gyyyz);
\coordinate (gpppma) at (\gxxxm, \gyyya);
\coordinate (gpppmb) at (\gxxxm, \gyyyb);
\coordinate (gpppmc) at (\gxxxm, \gyyyc);
\coordinate (gpppmd) at (\gxxxm, \gyyyd);
\coordinate (gpppme) at (\gxxxm, \gyyye);
\coordinate (gpppmf) at (\gxxxm, \gyyyf);
\coordinate (gpppmg) at (\gxxxm, \gyyyg);
\coordinate (gpppmh) at (\gxxxm, \gyyyh);
\coordinate (gpppmi) at (\gxxxm, \gyyyi);
\coordinate (gpppmj) at (\gxxxm, \gyyyj);
\coordinate (gpppmk) at (\gxxxm, \gyyyk);
\coordinate (gpppml) at (\gxxxm, \gyyyl);
\coordinate (gpppmm) at (\gxxxm, \gyyym);
\coordinate (gpppmn) at (\gxxxm, \gyyyn);
\coordinate (gpppmo) at (\gxxxm, \gyyyo);
\coordinate (gpppmp) at (\gxxxm, \gyyyp);
\coordinate (gpppmq) at (\gxxxm, \gyyyq);
\coordinate (gpppmr) at (\gxxxm, \gyyyr);
\coordinate (gpppms) at (\gxxxm, \gyyys);
\coordinate (gpppmt) at (\gxxxm, \gyyyt);
\coordinate (gpppmu) at (\gxxxm, \gyyyu);
\coordinate (gpppmv) at (\gxxxm, \gyyyv);
\coordinate (gpppmw) at (\gxxxm, \gyyyw);
\coordinate (gpppmx) at (\gxxxm, \gyyyx);
\coordinate (gpppmy) at (\gxxxm, \gyyyy);
\coordinate (gpppmz) at (\gxxxm, \gyyyz);
\coordinate (gpppna) at (\gxxxn, \gyyya);
\coordinate (gpppnb) at (\gxxxn, \gyyyb);
\coordinate (gpppnc) at (\gxxxn, \gyyyc);
\coordinate (gpppnd) at (\gxxxn, \gyyyd);
\coordinate (gpppne) at (\gxxxn, \gyyye);
\coordinate (gpppnf) at (\gxxxn, \gyyyf);
\coordinate (gpppng) at (\gxxxn, \gyyyg);
\coordinate (gpppnh) at (\gxxxn, \gyyyh);
\coordinate (gpppni) at (\gxxxn, \gyyyi);
\coordinate (gpppnj) at (\gxxxn, \gyyyj);
\coordinate (gpppnk) at (\gxxxn, \gyyyk);
\coordinate (gpppnl) at (\gxxxn, \gyyyl);
\coordinate (gpppnm) at (\gxxxn, \gyyym);
\coordinate (gpppnn) at (\gxxxn, \gyyyn);
\coordinate (gpppno) at (\gxxxn, \gyyyo);
\coordinate (gpppnp) at (\gxxxn, \gyyyp);
\coordinate (gpppnq) at (\gxxxn, \gyyyq);
\coordinate (gpppnr) at (\gxxxn, \gyyyr);
\coordinate (gpppns) at (\gxxxn, \gyyys);
\coordinate (gpppnt) at (\gxxxn, \gyyyt);
\coordinate (gpppnu) at (\gxxxn, \gyyyu);
\coordinate (gpppnv) at (\gxxxn, \gyyyv);
\coordinate (gpppnw) at (\gxxxn, \gyyyw);
\coordinate (gpppnx) at (\gxxxn, \gyyyx);
\coordinate (gpppny) at (\gxxxn, \gyyyy);
\coordinate (gpppnz) at (\gxxxn, \gyyyz);
\coordinate (gpppoa) at (\gxxxo, \gyyya);
\coordinate (gpppob) at (\gxxxo, \gyyyb);
\coordinate (gpppoc) at (\gxxxo, \gyyyc);
\coordinate (gpppod) at (\gxxxo, \gyyyd);
\coordinate (gpppoe) at (\gxxxo, \gyyye);
\coordinate (gpppof) at (\gxxxo, \gyyyf);
\coordinate (gpppog) at (\gxxxo, \gyyyg);
\coordinate (gpppoh) at (\gxxxo, \gyyyh);
\coordinate (gpppoi) at (\gxxxo, \gyyyi);
\coordinate (gpppoj) at (\gxxxo, \gyyyj);
\coordinate (gpppok) at (\gxxxo, \gyyyk);
\coordinate (gpppol) at (\gxxxo, \gyyyl);
\coordinate (gpppom) at (\gxxxo, \gyyym);
\coordinate (gpppon) at (\gxxxo, \gyyyn);
\coordinate (gpppoo) at (\gxxxo, \gyyyo);
\coordinate (gpppop) at (\gxxxo, \gyyyp);
\coordinate (gpppoq) at (\gxxxo, \gyyyq);
\coordinate (gpppor) at (\gxxxo, \gyyyr);
\coordinate (gpppos) at (\gxxxo, \gyyys);
\coordinate (gpppot) at (\gxxxo, \gyyyt);
\coordinate (gpppou) at (\gxxxo, \gyyyu);
\coordinate (gpppov) at (\gxxxo, \gyyyv);
\coordinate (gpppow) at (\gxxxo, \gyyyw);
\coordinate (gpppox) at (\gxxxo, \gyyyx);
\coordinate (gpppoy) at (\gxxxo, \gyyyy);
\coordinate (gpppoz) at (\gxxxo, \gyyyz);
\coordinate (gppppa) at (\gxxxp, \gyyya);
\coordinate (gppppb) at (\gxxxp, \gyyyb);
\coordinate (gppppc) at (\gxxxp, \gyyyc);
\coordinate (gppppd) at (\gxxxp, \gyyyd);
\coordinate (gppppe) at (\gxxxp, \gyyye);
\coordinate (gppppf) at (\gxxxp, \gyyyf);
\coordinate (gppppg) at (\gxxxp, \gyyyg);
\coordinate (gpppph) at (\gxxxp, \gyyyh);
\coordinate (gppppi) at (\gxxxp, \gyyyi);
\coordinate (gppppj) at (\gxxxp, \gyyyj);
\coordinate (gppppk) at (\gxxxp, \gyyyk);
\coordinate (gppppl) at (\gxxxp, \gyyyl);
\coordinate (gppppm) at (\gxxxp, \gyyym);
\coordinate (gppppn) at (\gxxxp, \gyyyn);
\coordinate (gppppo) at (\gxxxp, \gyyyo);
\coordinate (gppppp) at (\gxxxp, \gyyyp);
\coordinate (gppppq) at (\gxxxp, \gyyyq);
\coordinate (gppppr) at (\gxxxp, \gyyyr);
\coordinate (gpppps) at (\gxxxp, \gyyys);
\coordinate (gppppt) at (\gxxxp, \gyyyt);
\coordinate (gppppu) at (\gxxxp, \gyyyu);
\coordinate (gppppv) at (\gxxxp, \gyyyv);
\coordinate (gppppw) at (\gxxxp, \gyyyw);
\coordinate (gppppx) at (\gxxxp, \gyyyx);
\coordinate (gppppy) at (\gxxxp, \gyyyy);
\coordinate (gppppz) at (\gxxxp, \gyyyz);
\coordinate (gpppqa) at (\gxxxq, \gyyya);
\coordinate (gpppqb) at (\gxxxq, \gyyyb);
\coordinate (gpppqc) at (\gxxxq, \gyyyc);
\coordinate (gpppqd) at (\gxxxq, \gyyyd);
\coordinate (gpppqe) at (\gxxxq, \gyyye);
\coordinate (gpppqf) at (\gxxxq, \gyyyf);
\coordinate (gpppqg) at (\gxxxq, \gyyyg);
\coordinate (gpppqh) at (\gxxxq, \gyyyh);
\coordinate (gpppqi) at (\gxxxq, \gyyyi);
\coordinate (gpppqj) at (\gxxxq, \gyyyj);
\coordinate (gpppqk) at (\gxxxq, \gyyyk);
\coordinate (gpppql) at (\gxxxq, \gyyyl);
\coordinate (gpppqm) at (\gxxxq, \gyyym);
\coordinate (gpppqn) at (\gxxxq, \gyyyn);
\coordinate (gpppqo) at (\gxxxq, \gyyyo);
\coordinate (gpppqp) at (\gxxxq, \gyyyp);
\coordinate (gpppqq) at (\gxxxq, \gyyyq);
\coordinate (gpppqr) at (\gxxxq, \gyyyr);
\coordinate (gpppqs) at (\gxxxq, \gyyys);
\coordinate (gpppqt) at (\gxxxq, \gyyyt);
\coordinate (gpppqu) at (\gxxxq, \gyyyu);
\coordinate (gpppqv) at (\gxxxq, \gyyyv);
\coordinate (gpppqw) at (\gxxxq, \gyyyw);
\coordinate (gpppqx) at (\gxxxq, \gyyyx);
\coordinate (gpppqy) at (\gxxxq, \gyyyy);
\coordinate (gpppqz) at (\gxxxq, \gyyyz);
\coordinate (gpppra) at (\gxxxr, \gyyya);
\coordinate (gppprb) at (\gxxxr, \gyyyb);
\coordinate (gppprc) at (\gxxxr, \gyyyc);
\coordinate (gppprd) at (\gxxxr, \gyyyd);
\coordinate (gpppre) at (\gxxxr, \gyyye);
\coordinate (gppprf) at (\gxxxr, \gyyyf);
\coordinate (gppprg) at (\gxxxr, \gyyyg);
\coordinate (gppprh) at (\gxxxr, \gyyyh);
\coordinate (gpppri) at (\gxxxr, \gyyyi);
\coordinate (gppprj) at (\gxxxr, \gyyyj);
\coordinate (gppprk) at (\gxxxr, \gyyyk);
\coordinate (gppprl) at (\gxxxr, \gyyyl);
\coordinate (gppprm) at (\gxxxr, \gyyym);
\coordinate (gppprn) at (\gxxxr, \gyyyn);
\coordinate (gpppro) at (\gxxxr, \gyyyo);
\coordinate (gppprp) at (\gxxxr, \gyyyp);
\coordinate (gppprq) at (\gxxxr, \gyyyq);
\coordinate (gppprr) at (\gxxxr, \gyyyr);
\coordinate (gppprs) at (\gxxxr, \gyyys);
\coordinate (gppprt) at (\gxxxr, \gyyyt);
\coordinate (gpppru) at (\gxxxr, \gyyyu);
\coordinate (gppprv) at (\gxxxr, \gyyyv);
\coordinate (gppprw) at (\gxxxr, \gyyyw);
\coordinate (gppprx) at (\gxxxr, \gyyyx);
\coordinate (gpppry) at (\gxxxr, \gyyyy);
\coordinate (gppprz) at (\gxxxr, \gyyyz);
\coordinate (gpppsa) at (\gxxxs, \gyyya);
\coordinate (gpppsb) at (\gxxxs, \gyyyb);
\coordinate (gpppsc) at (\gxxxs, \gyyyc);
\coordinate (gpppsd) at (\gxxxs, \gyyyd);
\coordinate (gpppse) at (\gxxxs, \gyyye);
\coordinate (gpppsf) at (\gxxxs, \gyyyf);
\coordinate (gpppsg) at (\gxxxs, \gyyyg);
\coordinate (gpppsh) at (\gxxxs, \gyyyh);
\coordinate (gpppsi) at (\gxxxs, \gyyyi);
\coordinate (gpppsj) at (\gxxxs, \gyyyj);
\coordinate (gpppsk) at (\gxxxs, \gyyyk);
\coordinate (gpppsl) at (\gxxxs, \gyyyl);
\coordinate (gpppsm) at (\gxxxs, \gyyym);
\coordinate (gpppsn) at (\gxxxs, \gyyyn);
\coordinate (gpppso) at (\gxxxs, \gyyyo);
\coordinate (gpppsp) at (\gxxxs, \gyyyp);
\coordinate (gpppsq) at (\gxxxs, \gyyyq);
\coordinate (gpppsr) at (\gxxxs, \gyyyr);
\coordinate (gpppss) at (\gxxxs, \gyyys);
\coordinate (gpppst) at (\gxxxs, \gyyyt);
\coordinate (gpppsu) at (\gxxxs, \gyyyu);
\coordinate (gpppsv) at (\gxxxs, \gyyyv);
\coordinate (gpppsw) at (\gxxxs, \gyyyw);
\coordinate (gpppsx) at (\gxxxs, \gyyyx);
\coordinate (gpppsy) at (\gxxxs, \gyyyy);
\coordinate (gpppsz) at (\gxxxs, \gyyyz);
\coordinate (gpppta) at (\gxxxt, \gyyya);
\coordinate (gppptb) at (\gxxxt, \gyyyb);
\coordinate (gppptc) at (\gxxxt, \gyyyc);
\coordinate (gppptd) at (\gxxxt, \gyyyd);
\coordinate (gpppte) at (\gxxxt, \gyyye);
\coordinate (gppptf) at (\gxxxt, \gyyyf);
\coordinate (gppptg) at (\gxxxt, \gyyyg);
\coordinate (gpppth) at (\gxxxt, \gyyyh);
\coordinate (gpppti) at (\gxxxt, \gyyyi);
\coordinate (gppptj) at (\gxxxt, \gyyyj);
\coordinate (gppptk) at (\gxxxt, \gyyyk);
\coordinate (gppptl) at (\gxxxt, \gyyyl);
\coordinate (gppptm) at (\gxxxt, \gyyym);
\coordinate (gppptn) at (\gxxxt, \gyyyn);
\coordinate (gpppto) at (\gxxxt, \gyyyo);
\coordinate (gppptp) at (\gxxxt, \gyyyp);
\coordinate (gppptq) at (\gxxxt, \gyyyq);
\coordinate (gppptr) at (\gxxxt, \gyyyr);
\coordinate (gpppts) at (\gxxxt, \gyyys);
\coordinate (gppptt) at (\gxxxt, \gyyyt);
\coordinate (gppptu) at (\gxxxt, \gyyyu);
\coordinate (gppptv) at (\gxxxt, \gyyyv);
\coordinate (gppptw) at (\gxxxt, \gyyyw);
\coordinate (gppptx) at (\gxxxt, \gyyyx);
\coordinate (gpppty) at (\gxxxt, \gyyyy);
\coordinate (gppptz) at (\gxxxt, \gyyyz);
\coordinate (gpppua) at (\gxxxu, \gyyya);
\coordinate (gpppub) at (\gxxxu, \gyyyb);
\coordinate (gpppuc) at (\gxxxu, \gyyyc);
\coordinate (gpppud) at (\gxxxu, \gyyyd);
\coordinate (gpppue) at (\gxxxu, \gyyye);
\coordinate (gpppuf) at (\gxxxu, \gyyyf);
\coordinate (gpppug) at (\gxxxu, \gyyyg);
\coordinate (gpppuh) at (\gxxxu, \gyyyh);
\coordinate (gpppui) at (\gxxxu, \gyyyi);
\coordinate (gpppuj) at (\gxxxu, \gyyyj);
\coordinate (gpppuk) at (\gxxxu, \gyyyk);
\coordinate (gpppul) at (\gxxxu, \gyyyl);
\coordinate (gpppum) at (\gxxxu, \gyyym);
\coordinate (gpppun) at (\gxxxu, \gyyyn);
\coordinate (gpppuo) at (\gxxxu, \gyyyo);
\coordinate (gpppup) at (\gxxxu, \gyyyp);
\coordinate (gpppuq) at (\gxxxu, \gyyyq);
\coordinate (gpppur) at (\gxxxu, \gyyyr);
\coordinate (gpppus) at (\gxxxu, \gyyys);
\coordinate (gppput) at (\gxxxu, \gyyyt);
\coordinate (gpppuu) at (\gxxxu, \gyyyu);
\coordinate (gpppuv) at (\gxxxu, \gyyyv);
\coordinate (gpppuw) at (\gxxxu, \gyyyw);
\coordinate (gpppux) at (\gxxxu, \gyyyx);
\coordinate (gpppuy) at (\gxxxu, \gyyyy);
\coordinate (gpppuz) at (\gxxxu, \gyyyz);
\coordinate (gpppva) at (\gxxxv, \gyyya);
\coordinate (gpppvb) at (\gxxxv, \gyyyb);
\coordinate (gpppvc) at (\gxxxv, \gyyyc);
\coordinate (gpppvd) at (\gxxxv, \gyyyd);
\coordinate (gpppve) at (\gxxxv, \gyyye);
\coordinate (gpppvf) at (\gxxxv, \gyyyf);
\coordinate (gpppvg) at (\gxxxv, \gyyyg);
\coordinate (gpppvh) at (\gxxxv, \gyyyh);
\coordinate (gpppvi) at (\gxxxv, \gyyyi);
\coordinate (gpppvj) at (\gxxxv, \gyyyj);
\coordinate (gpppvk) at (\gxxxv, \gyyyk);
\coordinate (gpppvl) at (\gxxxv, \gyyyl);
\coordinate (gpppvm) at (\gxxxv, \gyyym);
\coordinate (gpppvn) at (\gxxxv, \gyyyn);
\coordinate (gpppvo) at (\gxxxv, \gyyyo);
\coordinate (gpppvp) at (\gxxxv, \gyyyp);
\coordinate (gpppvq) at (\gxxxv, \gyyyq);
\coordinate (gpppvr) at (\gxxxv, \gyyyr);
\coordinate (gpppvs) at (\gxxxv, \gyyys);
\coordinate (gpppvt) at (\gxxxv, \gyyyt);
\coordinate (gpppvu) at (\gxxxv, \gyyyu);
\coordinate (gpppvv) at (\gxxxv, \gyyyv);
\coordinate (gpppvw) at (\gxxxv, \gyyyw);
\coordinate (gpppvx) at (\gxxxv, \gyyyx);
\coordinate (gpppvy) at (\gxxxv, \gyyyy);
\coordinate (gpppvz) at (\gxxxv, \gyyyz);
\coordinate (gpppwa) at (\gxxxw, \gyyya);
\coordinate (gpppwb) at (\gxxxw, \gyyyb);
\coordinate (gpppwc) at (\gxxxw, \gyyyc);
\coordinate (gpppwd) at (\gxxxw, \gyyyd);
\coordinate (gpppwe) at (\gxxxw, \gyyye);
\coordinate (gpppwf) at (\gxxxw, \gyyyf);
\coordinate (gpppwg) at (\gxxxw, \gyyyg);
\coordinate (gpppwh) at (\gxxxw, \gyyyh);
\coordinate (gpppwi) at (\gxxxw, \gyyyi);
\coordinate (gpppwj) at (\gxxxw, \gyyyj);
\coordinate (gpppwk) at (\gxxxw, \gyyyk);
\coordinate (gpppwl) at (\gxxxw, \gyyyl);
\coordinate (gpppwm) at (\gxxxw, \gyyym);
\coordinate (gpppwn) at (\gxxxw, \gyyyn);
\coordinate (gpppwo) at (\gxxxw, \gyyyo);
\coordinate (gpppwp) at (\gxxxw, \gyyyp);
\coordinate (gpppwq) at (\gxxxw, \gyyyq);
\coordinate (gpppwr) at (\gxxxw, \gyyyr);
\coordinate (gpppws) at (\gxxxw, \gyyys);
\coordinate (gpppwt) at (\gxxxw, \gyyyt);
\coordinate (gpppwu) at (\gxxxw, \gyyyu);
\coordinate (gpppwv) at (\gxxxw, \gyyyv);
\coordinate (gpppww) at (\gxxxw, \gyyyw);
\coordinate (gpppwx) at (\gxxxw, \gyyyx);
\coordinate (gpppwy) at (\gxxxw, \gyyyy);
\coordinate (gpppwz) at (\gxxxw, \gyyyz);
\coordinate (gpppxa) at (\gxxxx, \gyyya);
\coordinate (gpppxb) at (\gxxxx, \gyyyb);
\coordinate (gpppxc) at (\gxxxx, \gyyyc);
\coordinate (gpppxd) at (\gxxxx, \gyyyd);
\coordinate (gpppxe) at (\gxxxx, \gyyye);
\coordinate (gpppxf) at (\gxxxx, \gyyyf);
\coordinate (gpppxg) at (\gxxxx, \gyyyg);
\coordinate (gpppxh) at (\gxxxx, \gyyyh);
\coordinate (gpppxi) at (\gxxxx, \gyyyi);
\coordinate (gpppxj) at (\gxxxx, \gyyyj);
\coordinate (gpppxk) at (\gxxxx, \gyyyk);
\coordinate (gpppxl) at (\gxxxx, \gyyyl);
\coordinate (gpppxm) at (\gxxxx, \gyyym);
\coordinate (gpppxn) at (\gxxxx, \gyyyn);
\coordinate (gpppxo) at (\gxxxx, \gyyyo);
\coordinate (gpppxp) at (\gxxxx, \gyyyp);
\coordinate (gpppxq) at (\gxxxx, \gyyyq);
\coordinate (gpppxr) at (\gxxxx, \gyyyr);
\coordinate (gpppxs) at (\gxxxx, \gyyys);
\coordinate (gpppxt) at (\gxxxx, \gyyyt);
\coordinate (gpppxu) at (\gxxxx, \gyyyu);
\coordinate (gpppxv) at (\gxxxx, \gyyyv);
\coordinate (gpppxw) at (\gxxxx, \gyyyw);
\coordinate (gpppxx) at (\gxxxx, \gyyyx);
\coordinate (gpppxy) at (\gxxxx, \gyyyy);
\coordinate (gpppxz) at (\gxxxx, \gyyyz);
\coordinate (gpppya) at (\gxxxy, \gyyya);
\coordinate (gpppyb) at (\gxxxy, \gyyyb);
\coordinate (gpppyc) at (\gxxxy, \gyyyc);
\coordinate (gpppyd) at (\gxxxy, \gyyyd);
\coordinate (gpppye) at (\gxxxy, \gyyye);
\coordinate (gpppyf) at (\gxxxy, \gyyyf);
\coordinate (gpppyg) at (\gxxxy, \gyyyg);
\coordinate (gpppyh) at (\gxxxy, \gyyyh);
\coordinate (gpppyi) at (\gxxxy, \gyyyi);
\coordinate (gpppyj) at (\gxxxy, \gyyyj);
\coordinate (gpppyk) at (\gxxxy, \gyyyk);
\coordinate (gpppyl) at (\gxxxy, \gyyyl);
\coordinate (gpppym) at (\gxxxy, \gyyym);
\coordinate (gpppyn) at (\gxxxy, \gyyyn);
\coordinate (gpppyo) at (\gxxxy, \gyyyo);
\coordinate (gpppyp) at (\gxxxy, \gyyyp);
\coordinate (gpppyq) at (\gxxxy, \gyyyq);
\coordinate (gpppyr) at (\gxxxy, \gyyyr);
\coordinate (gpppys) at (\gxxxy, \gyyys);
\coordinate (gpppyt) at (\gxxxy, \gyyyt);
\coordinate (gpppyu) at (\gxxxy, \gyyyu);
\coordinate (gpppyv) at (\gxxxy, \gyyyv);
\coordinate (gpppyw) at (\gxxxy, \gyyyw);
\coordinate (gpppyx) at (\gxxxy, \gyyyx);
\coordinate (gpppyy) at (\gxxxy, \gyyyy);
\coordinate (gpppyz) at (\gxxxy, \gyyyz);
\coordinate (gpppza) at (\gxxxz, \gyyya);
\coordinate (gpppzb) at (\gxxxz, \gyyyb);
\coordinate (gpppzc) at (\gxxxz, \gyyyc);
\coordinate (gpppzd) at (\gxxxz, \gyyyd);
\coordinate (gpppze) at (\gxxxz, \gyyye);
\coordinate (gpppzf) at (\gxxxz, \gyyyf);
\coordinate (gpppzg) at (\gxxxz, \gyyyg);
\coordinate (gpppzh) at (\gxxxz, \gyyyh);
\coordinate (gpppzi) at (\gxxxz, \gyyyi);
\coordinate (gpppzj) at (\gxxxz, \gyyyj);
\coordinate (gpppzk) at (\gxxxz, \gyyyk);
\coordinate (gpppzl) at (\gxxxz, \gyyyl);
\coordinate (gpppzm) at (\gxxxz, \gyyym);
\coordinate (gpppzn) at (\gxxxz, \gyyyn);
\coordinate (gpppzo) at (\gxxxz, \gyyyo);
\coordinate (gpppzp) at (\gxxxz, \gyyyp);
\coordinate (gpppzq) at (\gxxxz, \gyyyq);
\coordinate (gpppzr) at (\gxxxz, \gyyyr);
\coordinate (gpppzs) at (\gxxxz, \gyyys);
\coordinate (gpppzt) at (\gxxxz, \gyyyt);
\coordinate (gpppzu) at (\gxxxz, \gyyyu);
\coordinate (gpppzv) at (\gxxxz, \gyyyv);
\coordinate (gpppzw) at (\gxxxz, \gyyyw);
\coordinate (gpppzx) at (\gxxxz, \gyyyx);
\coordinate (gpppzy) at (\gxxxz, \gyyyy);
\coordinate (gpppzz) at (\gxxxz, \gyyyz);

%\gangprintcoordinateat{(0,0)}{The last coordinate values: }{($(gpppzz)$)}; 


% Draw related part of the coordinate system with dashed helplines (centered at (gpppii)) with letters as background, which would help to determine all coordinates. 
\coordinatebackground{g}{c}{d}{o};

% Step 2, draw key devices, their accessories, and take related coordinates of their pins, and may define more coordinates. 

% Draw the Opamp at the coordinate (gpppii) and name it as "swopamp".
\draw (gpppii) node [op amp, yscale=-1] (swopamp) {\ctikzflipy{Opamp}} ; 

% Its accessories and lables. 
\draw [-*](swopamp.down) -- ($(swopamp.down)+(0,1)$) node[right]{$V_+$}; 
\node at ($(swopamp.down)+(0.3,0.2)$) {7};  
\draw [-*](swopamp.up) -- ($(swopamp.up)+(0,-1)$) node[right]{$V_-$}; 
\node at ($(swopamp.up)+(0.3,-0.2)$) {4};

% Get the x- and y-components of the coordinates of the "+" and "-" pins. 
\getxyingivenunit{cm}{(swopamp.+)}{\swopampzx}{\swopampzy};
\getxyingivenunit{cm}{(swopamp.-)}{\swopampfx}{\swopampfy};

% Then define a few more coordinates, at least for keeping in mind.
\coordinate (plusshort) at ($(\gxxxg,\swopampzy)$);
%\fill  (plusshort) circle (2pt);  % May be commented later.
\coordinate (minusshort) at ($(\gxxxg,\swopampfy)$);
%\fill  (minusshort) circle (2pt); % May be commented later.
\coordinate (leftinter) at ($(\gxxxe,\swopampzy)$);
\fill  (leftinter) circle (2pt);

% Draw an "npn" at (gpppmi) and name it as "swQ".
\draw (gpppmi) node[npn](swQ){};

% Get the x- and y-components of the needed pins of it for later usage.
\getxyingivenunit{cm}{(swQ.C)}{\swQCx}{\swQCy};
\getxyingivenunit{cm}{(swQ.E)}{\swQEx}{\swQEy};

% Then define more coordinate(s).
\coordinate (Qcshort) at ($(\swQEx,\gyyyj)$);
%\fill  (Qcshort) circle (2pt); % May be commented later.
\coordinate (Qeshort) at ($(\swQEx,\gyyyf)$);
\fill  (Qeshort) circle (2pt) node [right] {$V_0$};

% Then the rectangle by the points (gpppef) -- (gpppej) -- (Qcshort) -- (Qeshort) forms a clear area for the key devices. 

% Connect the two devices.
\draw (swopamp.out) to [short, l=$I_B$, above] (swQ.B);

% Step 3, draw other little devices. For tidiness, better to give two units in length for each new device and align them up.

% For this specific circuit, let us attach the four bi-pole devices (maybe with their accessories) to each corner of the above mentioned rectangle area for the key devices, separately. 

\draw  (gppped) node [ground] {} to [empty ZZener diode] (gpppef) -- (leftinter);
% The Latex system can not work properly when I put the following label into the above "[empty ZZener diode]" in the form of an additional "l= ..." option, then I have to employ the following "\node ..." command. Then the label should be aligned with the "ZZener" as much as possible even the coordinate system is modified later. Since the "\gyyyd" and "\gyyyf" micros are used to position the "ZZener", then better to use the center between them to locate the label, rather than using "\gyyye" directly. This idea is also applied in later "\node ..." commands. 
\node at ($(\gxxxe-1.3, \gyyyd*0.5+\gyyyf*0.5)$) {$V_Z = 5\textnormal{V}$};

\draw (gpppej) to [generic] (gpppel) -| (swQ.C);
\node at ($(\gxxxe-1.1, \gyyyj*0.5+\gyyyl*0.5)$) {$R_{1}=47k\Omega$};
\node [right] at (\swQCx,\gyyyj*0.5+\gyyyl*0.5) {$I_C \approx \beta I_B$};

\draw  (\swQEx, \gyyyd) node [ground] {} to [generic] (\swQEx, \gyyyf) -- (swQ.E);
\node at ($(\swQEx+1.4, \gyyyd*0.5+\gyyyf*0.5)$) {$R_{E}=100k\Omega$};

% Draw the top area. 
\draw  (\gxxxi-0.2,\gyyyn) --  (\gxxxi+0.2,\gyyyn) node [right] {$V_{cc}=15\textnormal{V}$} ;
\draw [->] (gpppin) -- (gpppim) node [right] {$I$};  
\draw  (gpppim) -- (gpppil);
\fill  (gpppil) circle (2pt);

% Step 4, other shorts.
\draw  (swopamp.+)  to [short, l_=$I_+ \approx 0 $, above] (plusshort) -- (leftinter) -- (gpppej);

\draw  (swopamp.-)  to [short, l_=$I_- \approx 0 $, above] (minusshort) |- (Qeshort);

%Step 5, all the rest, especially labels. May also clean unnecessary staff, like the system background and dark points for showing newly defined coordinates previously. 
\draw [->] ($(\swQEx-0.4, \gyyyf - 0.4)$) -- node [left] {$I_E$} ($(\swQEx-0.4, \gyyyd + 0.4)$);

\draw [->] ($(\gxxxe+0.4, \gyyyl*0.5+\gyyyj*0.5 + 0.6)$) -- node [right] {$I_1$} ($(\gxxxe+0.4, \gyyyl*0.5+\gyyyj*0.5 - 0.6)$);


\end{circuitikz}

























\newpage


{\Large Figure 8, then the two circuits are inserted and connected.}

\begin{circuitikz}[scale=1]


% Circuits can be drawn by the following five major steps, as shown in the following example. 

% Step 1, preparations. 

% "Install" the coordinate system with keyword ``g".
\pgfmathsetmacro{\totalgxxx}{26}
\pgfmathsetmacro{\totalgyyy}{26}
\pgfmathsetmacro{\gxxxspacing}{1}
\pgfmathsetmacro{\gyyyspacing}{1}
\pgfmathsetmacro{\gxxxa}{-8}
\pgfmathsetmacro{\gyyya}{-8}

\pgfmathsetmacro{\gxxxb}{\gxxxa + \gxxxspacing + 0.0 }
\pgfmathsetmacro{\gxxxc}{\gxxxb + \gxxxspacing + 0.0 }
\pgfmathsetmacro{\gxxxd}{\gxxxc + \gxxxspacing + 0.0 }
\pgfmathsetmacro{\gxxxe}{\gxxxd + \gxxxspacing + 0.0 }
\pgfmathsetmacro{\gxxxf}{\gxxxe + \gxxxspacing + 0.0 }
\pgfmathsetmacro{\gxxxg}{\gxxxf + \gxxxspacing + 0.0 }
\pgfmathsetmacro{\gxxxh}{\gxxxg + \gxxxspacing + 0.0 }
\pgfmathsetmacro{\gxxxi}{\gxxxh + \gxxxspacing + 0.0 }
\pgfmathsetmacro{\gxxxj}{\gxxxi + \gxxxspacing + 0.0 }
\pgfmathsetmacro{\gxxxk}{\gxxxj + \gxxxspacing + 0.0 }
\pgfmathsetmacro{\gxxxl}{\gxxxk + \gxxxspacing + 8.0 }
\pgfmathsetmacro{\gxxxm}{\gxxxl + \gxxxspacing + 0.0 }
\pgfmathsetmacro{\gxxxn}{\gxxxm + \gxxxspacing + 0.0 }
\pgfmathsetmacro{\gxxxo}{\gxxxn + \gxxxspacing + 0.0 }
\pgfmathsetmacro{\gxxxp}{\gxxxo + \gxxxspacing + 0.0 }
\pgfmathsetmacro{\gxxxq}{\gxxxp + \gxxxspacing + 0.0 }
\pgfmathsetmacro{\gxxxr}{\gxxxq + \gxxxspacing + 0.0 }
\pgfmathsetmacro{\gxxxs}{\gxxxr + \gxxxspacing + 0.0 }
\pgfmathsetmacro{\gxxxt}{\gxxxs + \gxxxspacing + 0.0 }
\pgfmathsetmacro{\gxxxu}{\gxxxt + \gxxxspacing + 0.0 }
\pgfmathsetmacro{\gxxxv}{\gxxxu + \gxxxspacing + 0.0 }
\pgfmathsetmacro{\gxxxw}{\gxxxv + \gxxxspacing + 0.0 }
\pgfmathsetmacro{\gxxxx}{\gxxxw + \gxxxspacing + 0.0 }
\pgfmathsetmacro{\gxxxy}{\gxxxx + \gxxxspacing + 0.0 }
\pgfmathsetmacro{\gxxxz}{\gxxxy + \gxxxspacing + 0.0 }

\pgfmathsetmacro{\gyyyb}{\gyyya + \gyyyspacing + 2.0 }
\pgfmathsetmacro{\gyyyc}{\gyyyb + \gyyyspacing + 0.0 }
\pgfmathsetmacro{\gyyyd}{\gyyyc + \gyyyspacing + 0.0 }
\pgfmathsetmacro{\gyyye}{\gyyyd + \gyyyspacing + 0.0 }
\pgfmathsetmacro{\gyyyf}{\gyyye + \gyyyspacing + 0.0 }
\pgfmathsetmacro{\gyyyg}{\gyyyf + \gyyyspacing + 0.0 }
\pgfmathsetmacro{\gyyyh}{\gyyyg + \gyyyspacing + 0.0 }
\pgfmathsetmacro{\gyyyi}{\gyyyh + \gyyyspacing + 0.0 }
\pgfmathsetmacro{\gyyyj}{\gyyyi + \gyyyspacing + 0.0 }
\pgfmathsetmacro{\gyyyk}{\gyyyj + \gyyyspacing + 0.0 }
\pgfmathsetmacro{\gyyyl}{\gyyyk + \gyyyspacing + 12.0 }
\pgfmathsetmacro{\gyyym}{\gyyyl + \gyyyspacing + 0.0 }
\pgfmathsetmacro{\gyyyn}{\gyyym + \gyyyspacing + 0.0 }
\pgfmathsetmacro{\gyyyo}{\gyyyn + \gyyyspacing + 0.0 }
\pgfmathsetmacro{\gyyyp}{\gyyyo + \gyyyspacing + 0.0 }
\pgfmathsetmacro{\gyyyq}{\gyyyp + \gyyyspacing + 0.0 }
\pgfmathsetmacro{\gyyyr}{\gyyyq + \gyyyspacing + 0.0 }
\pgfmathsetmacro{\gyyys}{\gyyyr + \gyyyspacing + 0.0 }
\pgfmathsetmacro{\gyyyt}{\gyyys + \gyyyspacing + 0.0 }
\pgfmathsetmacro{\gyyyu}{\gyyyt + \gyyyspacing + 0.0 }
\pgfmathsetmacro{\gyyyv}{\gyyyu + \gyyyspacing + 0.0 }
\pgfmathsetmacro{\gyyyw}{\gyyyv + \gyyyspacing + 0.0 }
\pgfmathsetmacro{\gyyyx}{\gyyyw + \gyyyspacing + 0.0 }
\pgfmathsetmacro{\gyyyy}{\gyyyx + \gyyyspacing + 0.0 }
\pgfmathsetmacro{\gyyyz}{\gyyyy + \gyyyspacing + 0.0 }

\coordinate (gpppaa) at (\gxxxa, \gyyya);
\coordinate (gpppab) at (\gxxxa, \gyyyb);
\coordinate (gpppac) at (\gxxxa, \gyyyc);
\coordinate (gpppad) at (\gxxxa, \gyyyd);
\coordinate (gpppae) at (\gxxxa, \gyyye);
\coordinate (gpppaf) at (\gxxxa, \gyyyf);
\coordinate (gpppag) at (\gxxxa, \gyyyg);
\coordinate (gpppah) at (\gxxxa, \gyyyh);
\coordinate (gpppai) at (\gxxxa, \gyyyi);
\coordinate (gpppaj) at (\gxxxa, \gyyyj);
\coordinate (gpppak) at (\gxxxa, \gyyyk);
\coordinate (gpppal) at (\gxxxa, \gyyyl);
\coordinate (gpppam) at (\gxxxa, \gyyym);
\coordinate (gpppan) at (\gxxxa, \gyyyn);
\coordinate (gpppao) at (\gxxxa, \gyyyo);
\coordinate (gpppap) at (\gxxxa, \gyyyp);
\coordinate (gpppaq) at (\gxxxa, \gyyyq);
\coordinate (gpppar) at (\gxxxa, \gyyyr);
\coordinate (gpppas) at (\gxxxa, \gyyys);
\coordinate (gpppat) at (\gxxxa, \gyyyt);
\coordinate (gpppau) at (\gxxxa, \gyyyu);
\coordinate (gpppav) at (\gxxxa, \gyyyv);
\coordinate (gpppaw) at (\gxxxa, \gyyyw);
\coordinate (gpppax) at (\gxxxa, \gyyyx);
\coordinate (gpppay) at (\gxxxa, \gyyyy);
\coordinate (gpppaz) at (\gxxxa, \gyyyz);
\coordinate (gpppba) at (\gxxxb, \gyyya);
\coordinate (gpppbb) at (\gxxxb, \gyyyb);
\coordinate (gpppbc) at (\gxxxb, \gyyyc);
\coordinate (gpppbd) at (\gxxxb, \gyyyd);
\coordinate (gpppbe) at (\gxxxb, \gyyye);
\coordinate (gpppbf) at (\gxxxb, \gyyyf);
\coordinate (gpppbg) at (\gxxxb, \gyyyg);
\coordinate (gpppbh) at (\gxxxb, \gyyyh);
\coordinate (gpppbi) at (\gxxxb, \gyyyi);
\coordinate (gpppbj) at (\gxxxb, \gyyyj);
\coordinate (gpppbk) at (\gxxxb, \gyyyk);
\coordinate (gpppbl) at (\gxxxb, \gyyyl);
\coordinate (gpppbm) at (\gxxxb, \gyyym);
\coordinate (gpppbn) at (\gxxxb, \gyyyn);
\coordinate (gpppbo) at (\gxxxb, \gyyyo);
\coordinate (gpppbp) at (\gxxxb, \gyyyp);
\coordinate (gpppbq) at (\gxxxb, \gyyyq);
\coordinate (gpppbr) at (\gxxxb, \gyyyr);
\coordinate (gpppbs) at (\gxxxb, \gyyys);
\coordinate (gpppbt) at (\gxxxb, \gyyyt);
\coordinate (gpppbu) at (\gxxxb, \gyyyu);
\coordinate (gpppbv) at (\gxxxb, \gyyyv);
\coordinate (gpppbw) at (\gxxxb, \gyyyw);
\coordinate (gpppbx) at (\gxxxb, \gyyyx);
\coordinate (gpppby) at (\gxxxb, \gyyyy);
\coordinate (gpppbz) at (\gxxxb, \gyyyz);
\coordinate (gpppca) at (\gxxxc, \gyyya);
\coordinate (gpppcb) at (\gxxxc, \gyyyb);
\coordinate (gpppcc) at (\gxxxc, \gyyyc);
\coordinate (gpppcd) at (\gxxxc, \gyyyd);
\coordinate (gpppce) at (\gxxxc, \gyyye);
\coordinate (gpppcf) at (\gxxxc, \gyyyf);
\coordinate (gpppcg) at (\gxxxc, \gyyyg);
\coordinate (gpppch) at (\gxxxc, \gyyyh);
\coordinate (gpppci) at (\gxxxc, \gyyyi);
\coordinate (gpppcj) at (\gxxxc, \gyyyj);
\coordinate (gpppck) at (\gxxxc, \gyyyk);
\coordinate (gpppcl) at (\gxxxc, \gyyyl);
\coordinate (gpppcm) at (\gxxxc, \gyyym);
\coordinate (gpppcn) at (\gxxxc, \gyyyn);
\coordinate (gpppco) at (\gxxxc, \gyyyo);
\coordinate (gpppcp) at (\gxxxc, \gyyyp);
\coordinate (gpppcq) at (\gxxxc, \gyyyq);
\coordinate (gpppcr) at (\gxxxc, \gyyyr);
\coordinate (gpppcs) at (\gxxxc, \gyyys);
\coordinate (gpppct) at (\gxxxc, \gyyyt);
\coordinate (gpppcu) at (\gxxxc, \gyyyu);
\coordinate (gpppcv) at (\gxxxc, \gyyyv);
\coordinate (gpppcw) at (\gxxxc, \gyyyw);
\coordinate (gpppcx) at (\gxxxc, \gyyyx);
\coordinate (gpppcy) at (\gxxxc, \gyyyy);
\coordinate (gpppcz) at (\gxxxc, \gyyyz);
\coordinate (gpppda) at (\gxxxd, \gyyya);
\coordinate (gpppdb) at (\gxxxd, \gyyyb);
\coordinate (gpppdc) at (\gxxxd, \gyyyc);
\coordinate (gpppdd) at (\gxxxd, \gyyyd);
\coordinate (gpppde) at (\gxxxd, \gyyye);
\coordinate (gpppdf) at (\gxxxd, \gyyyf);
\coordinate (gpppdg) at (\gxxxd, \gyyyg);
\coordinate (gpppdh) at (\gxxxd, \gyyyh);
\coordinate (gpppdi) at (\gxxxd, \gyyyi);
\coordinate (gpppdj) at (\gxxxd, \gyyyj);
\coordinate (gpppdk) at (\gxxxd, \gyyyk);
\coordinate (gpppdl) at (\gxxxd, \gyyyl);
\coordinate (gpppdm) at (\gxxxd, \gyyym);
\coordinate (gpppdn) at (\gxxxd, \gyyyn);
\coordinate (gpppdo) at (\gxxxd, \gyyyo);
\coordinate (gpppdp) at (\gxxxd, \gyyyp);
\coordinate (gpppdq) at (\gxxxd, \gyyyq);
\coordinate (gpppdr) at (\gxxxd, \gyyyr);
\coordinate (gpppds) at (\gxxxd, \gyyys);
\coordinate (gpppdt) at (\gxxxd, \gyyyt);
\coordinate (gpppdu) at (\gxxxd, \gyyyu);
\coordinate (gpppdv) at (\gxxxd, \gyyyv);
\coordinate (gpppdw) at (\gxxxd, \gyyyw);
\coordinate (gpppdx) at (\gxxxd, \gyyyx);
\coordinate (gpppdy) at (\gxxxd, \gyyyy);
\coordinate (gpppdz) at (\gxxxd, \gyyyz);
\coordinate (gpppea) at (\gxxxe, \gyyya);
\coordinate (gpppeb) at (\gxxxe, \gyyyb);
\coordinate (gpppec) at (\gxxxe, \gyyyc);
\coordinate (gppped) at (\gxxxe, \gyyyd);
\coordinate (gpppee) at (\gxxxe, \gyyye);
\coordinate (gpppef) at (\gxxxe, \gyyyf);
\coordinate (gpppeg) at (\gxxxe, \gyyyg);
\coordinate (gpppeh) at (\gxxxe, \gyyyh);
\coordinate (gpppei) at (\gxxxe, \gyyyi);
\coordinate (gpppej) at (\gxxxe, \gyyyj);
\coordinate (gpppek) at (\gxxxe, \gyyyk);
\coordinate (gpppel) at (\gxxxe, \gyyyl);
\coordinate (gpppem) at (\gxxxe, \gyyym);
\coordinate (gpppen) at (\gxxxe, \gyyyn);
\coordinate (gpppeo) at (\gxxxe, \gyyyo);
\coordinate (gpppep) at (\gxxxe, \gyyyp);
\coordinate (gpppeq) at (\gxxxe, \gyyyq);
\coordinate (gppper) at (\gxxxe, \gyyyr);
\coordinate (gpppes) at (\gxxxe, \gyyys);
\coordinate (gpppet) at (\gxxxe, \gyyyt);
\coordinate (gpppeu) at (\gxxxe, \gyyyu);
\coordinate (gpppev) at (\gxxxe, \gyyyv);
\coordinate (gpppew) at (\gxxxe, \gyyyw);
\coordinate (gpppex) at (\gxxxe, \gyyyx);
\coordinate (gpppey) at (\gxxxe, \gyyyy);
\coordinate (gpppez) at (\gxxxe, \gyyyz);
\coordinate (gpppfa) at (\gxxxf, \gyyya);
\coordinate (gpppfb) at (\gxxxf, \gyyyb);
\coordinate (gpppfc) at (\gxxxf, \gyyyc);
\coordinate (gpppfd) at (\gxxxf, \gyyyd);
\coordinate (gpppfe) at (\gxxxf, \gyyye);
\coordinate (gpppff) at (\gxxxf, \gyyyf);
\coordinate (gpppfg) at (\gxxxf, \gyyyg);
\coordinate (gpppfh) at (\gxxxf, \gyyyh);
\coordinate (gpppfi) at (\gxxxf, \gyyyi);
\coordinate (gpppfj) at (\gxxxf, \gyyyj);
\coordinate (gpppfk) at (\gxxxf, \gyyyk);
\coordinate (gpppfl) at (\gxxxf, \gyyyl);
\coordinate (gpppfm) at (\gxxxf, \gyyym);
\coordinate (gpppfn) at (\gxxxf, \gyyyn);
\coordinate (gpppfo) at (\gxxxf, \gyyyo);
\coordinate (gpppfp) at (\gxxxf, \gyyyp);
\coordinate (gpppfq) at (\gxxxf, \gyyyq);
\coordinate (gpppfr) at (\gxxxf, \gyyyr);
\coordinate (gpppfs) at (\gxxxf, \gyyys);
\coordinate (gpppft) at (\gxxxf, \gyyyt);
\coordinate (gpppfu) at (\gxxxf, \gyyyu);
\coordinate (gpppfv) at (\gxxxf, \gyyyv);
\coordinate (gpppfw) at (\gxxxf, \gyyyw);
\coordinate (gpppfx) at (\gxxxf, \gyyyx);
\coordinate (gpppfy) at (\gxxxf, \gyyyy);
\coordinate (gpppfz) at (\gxxxf, \gyyyz);
\coordinate (gpppga) at (\gxxxg, \gyyya);
\coordinate (gpppgb) at (\gxxxg, \gyyyb);
\coordinate (gpppgc) at (\gxxxg, \gyyyc);
\coordinate (gpppgd) at (\gxxxg, \gyyyd);
\coordinate (gpppge) at (\gxxxg, \gyyye);
\coordinate (gpppgf) at (\gxxxg, \gyyyf);
\coordinate (gpppgg) at (\gxxxg, \gyyyg);
\coordinate (gpppgh) at (\gxxxg, \gyyyh);
\coordinate (gpppgi) at (\gxxxg, \gyyyi);
\coordinate (gpppgj) at (\gxxxg, \gyyyj);
\coordinate (gpppgk) at (\gxxxg, \gyyyk);
\coordinate (gpppgl) at (\gxxxg, \gyyyl);
\coordinate (gpppgm) at (\gxxxg, \gyyym);
\coordinate (gpppgn) at (\gxxxg, \gyyyn);
\coordinate (gpppgo) at (\gxxxg, \gyyyo);
\coordinate (gpppgp) at (\gxxxg, \gyyyp);
\coordinate (gpppgq) at (\gxxxg, \gyyyq);
\coordinate (gpppgr) at (\gxxxg, \gyyyr);
\coordinate (gpppgs) at (\gxxxg, \gyyys);
\coordinate (gpppgt) at (\gxxxg, \gyyyt);
\coordinate (gpppgu) at (\gxxxg, \gyyyu);
\coordinate (gpppgv) at (\gxxxg, \gyyyv);
\coordinate (gpppgw) at (\gxxxg, \gyyyw);
\coordinate (gpppgx) at (\gxxxg, \gyyyx);
\coordinate (gpppgy) at (\gxxxg, \gyyyy);
\coordinate (gpppgz) at (\gxxxg, \gyyyz);
\coordinate (gpppha) at (\gxxxh, \gyyya);
\coordinate (gppphb) at (\gxxxh, \gyyyb);
\coordinate (gppphc) at (\gxxxh, \gyyyc);
\coordinate (gppphd) at (\gxxxh, \gyyyd);
\coordinate (gppphe) at (\gxxxh, \gyyye);
\coordinate (gppphf) at (\gxxxh, \gyyyf);
\coordinate (gppphg) at (\gxxxh, \gyyyg);
\coordinate (gppphh) at (\gxxxh, \gyyyh);
\coordinate (gppphi) at (\gxxxh, \gyyyi);
\coordinate (gppphj) at (\gxxxh, \gyyyj);
\coordinate (gppphk) at (\gxxxh, \gyyyk);
\coordinate (gppphl) at (\gxxxh, \gyyyl);
\coordinate (gppphm) at (\gxxxh, \gyyym);
\coordinate (gppphn) at (\gxxxh, \gyyyn);
\coordinate (gpppho) at (\gxxxh, \gyyyo);
\coordinate (gppphp) at (\gxxxh, \gyyyp);
\coordinate (gppphq) at (\gxxxh, \gyyyq);
\coordinate (gppphr) at (\gxxxh, \gyyyr);
\coordinate (gppphs) at (\gxxxh, \gyyys);
\coordinate (gpppht) at (\gxxxh, \gyyyt);
\coordinate (gppphu) at (\gxxxh, \gyyyu);
\coordinate (gppphv) at (\gxxxh, \gyyyv);
\coordinate (gppphw) at (\gxxxh, \gyyyw);
\coordinate (gppphx) at (\gxxxh, \gyyyx);
\coordinate (gppphy) at (\gxxxh, \gyyyy);
\coordinate (gppphz) at (\gxxxh, \gyyyz);
\coordinate (gpppia) at (\gxxxi, \gyyya);
\coordinate (gpppib) at (\gxxxi, \gyyyb);
\coordinate (gpppic) at (\gxxxi, \gyyyc);
\coordinate (gpppid) at (\gxxxi, \gyyyd);
\coordinate (gpppie) at (\gxxxi, \gyyye);
\coordinate (gpppif) at (\gxxxi, \gyyyf);
\coordinate (gpppig) at (\gxxxi, \gyyyg);
\coordinate (gpppih) at (\gxxxi, \gyyyh);
\coordinate (gpppii) at (\gxxxi, \gyyyi);
\coordinate (gpppij) at (\gxxxi, \gyyyj);
\coordinate (gpppik) at (\gxxxi, \gyyyk);
\coordinate (gpppil) at (\gxxxi, \gyyyl);
\coordinate (gpppim) at (\gxxxi, \gyyym);
\coordinate (gpppin) at (\gxxxi, \gyyyn);
\coordinate (gpppio) at (\gxxxi, \gyyyo);
\coordinate (gpppip) at (\gxxxi, \gyyyp);
\coordinate (gpppiq) at (\gxxxi, \gyyyq);
\coordinate (gpppir) at (\gxxxi, \gyyyr);
\coordinate (gpppis) at (\gxxxi, \gyyys);
\coordinate (gpppit) at (\gxxxi, \gyyyt);
\coordinate (gpppiu) at (\gxxxi, \gyyyu);
\coordinate (gpppiv) at (\gxxxi, \gyyyv);
\coordinate (gpppiw) at (\gxxxi, \gyyyw);
\coordinate (gpppix) at (\gxxxi, \gyyyx);
\coordinate (gpppiy) at (\gxxxi, \gyyyy);
\coordinate (gpppiz) at (\gxxxi, \gyyyz);
\coordinate (gpppja) at (\gxxxj, \gyyya);
\coordinate (gpppjb) at (\gxxxj, \gyyyb);
\coordinate (gpppjc) at (\gxxxj, \gyyyc);
\coordinate (gpppjd) at (\gxxxj, \gyyyd);
\coordinate (gpppje) at (\gxxxj, \gyyye);
\coordinate (gpppjf) at (\gxxxj, \gyyyf);
\coordinate (gpppjg) at (\gxxxj, \gyyyg);
\coordinate (gpppjh) at (\gxxxj, \gyyyh);
\coordinate (gpppji) at (\gxxxj, \gyyyi);
\coordinate (gpppjj) at (\gxxxj, \gyyyj);
\coordinate (gpppjk) at (\gxxxj, \gyyyk);
\coordinate (gpppjl) at (\gxxxj, \gyyyl);
\coordinate (gpppjm) at (\gxxxj, \gyyym);
\coordinate (gpppjn) at (\gxxxj, \gyyyn);
\coordinate (gpppjo) at (\gxxxj, \gyyyo);
\coordinate (gpppjp) at (\gxxxj, \gyyyp);
\coordinate (gpppjq) at (\gxxxj, \gyyyq);
\coordinate (gpppjr) at (\gxxxj, \gyyyr);
\coordinate (gpppjs) at (\gxxxj, \gyyys);
\coordinate (gpppjt) at (\gxxxj, \gyyyt);
\coordinate (gpppju) at (\gxxxj, \gyyyu);
\coordinate (gpppjv) at (\gxxxj, \gyyyv);
\coordinate (gpppjw) at (\gxxxj, \gyyyw);
\coordinate (gpppjx) at (\gxxxj, \gyyyx);
\coordinate (gpppjy) at (\gxxxj, \gyyyy);
\coordinate (gpppjz) at (\gxxxj, \gyyyz);
\coordinate (gpppka) at (\gxxxk, \gyyya);
\coordinate (gpppkb) at (\gxxxk, \gyyyb);
\coordinate (gpppkc) at (\gxxxk, \gyyyc);
\coordinate (gpppkd) at (\gxxxk, \gyyyd);
\coordinate (gpppke) at (\gxxxk, \gyyye);
\coordinate (gpppkf) at (\gxxxk, \gyyyf);
\coordinate (gpppkg) at (\gxxxk, \gyyyg);
\coordinate (gpppkh) at (\gxxxk, \gyyyh);
\coordinate (gpppki) at (\gxxxk, \gyyyi);
\coordinate (gpppkj) at (\gxxxk, \gyyyj);
\coordinate (gpppkk) at (\gxxxk, \gyyyk);
\coordinate (gpppkl) at (\gxxxk, \gyyyl);
\coordinate (gpppkm) at (\gxxxk, \gyyym);
\coordinate (gpppkn) at (\gxxxk, \gyyyn);
\coordinate (gpppko) at (\gxxxk, \gyyyo);
\coordinate (gpppkp) at (\gxxxk, \gyyyp);
\coordinate (gpppkq) at (\gxxxk, \gyyyq);
\coordinate (gpppkr) at (\gxxxk, \gyyyr);
\coordinate (gpppks) at (\gxxxk, \gyyys);
\coordinate (gpppkt) at (\gxxxk, \gyyyt);
\coordinate (gpppku) at (\gxxxk, \gyyyu);
\coordinate (gpppkv) at (\gxxxk, \gyyyv);
\coordinate (gpppkw) at (\gxxxk, \gyyyw);
\coordinate (gpppkx) at (\gxxxk, \gyyyx);
\coordinate (gpppky) at (\gxxxk, \gyyyy);
\coordinate (gpppkz) at (\gxxxk, \gyyyz);
\coordinate (gpppla) at (\gxxxl, \gyyya);
\coordinate (gppplb) at (\gxxxl, \gyyyb);
\coordinate (gppplc) at (\gxxxl, \gyyyc);
\coordinate (gpppld) at (\gxxxl, \gyyyd);
\coordinate (gppple) at (\gxxxl, \gyyye);
\coordinate (gppplf) at (\gxxxl, \gyyyf);
\coordinate (gppplg) at (\gxxxl, \gyyyg);
\coordinate (gppplh) at (\gxxxl, \gyyyh);
\coordinate (gpppli) at (\gxxxl, \gyyyi);
\coordinate (gppplj) at (\gxxxl, \gyyyj);
\coordinate (gppplk) at (\gxxxl, \gyyyk);
\coordinate (gpppll) at (\gxxxl, \gyyyl);
\coordinate (gppplm) at (\gxxxl, \gyyym);
\coordinate (gpppln) at (\gxxxl, \gyyyn);
\coordinate (gppplo) at (\gxxxl, \gyyyo);
\coordinate (gppplp) at (\gxxxl, \gyyyp);
\coordinate (gppplq) at (\gxxxl, \gyyyq);
\coordinate (gppplr) at (\gxxxl, \gyyyr);
\coordinate (gpppls) at (\gxxxl, \gyyys);
\coordinate (gppplt) at (\gxxxl, \gyyyt);
\coordinate (gppplu) at (\gxxxl, \gyyyu);
\coordinate (gppplv) at (\gxxxl, \gyyyv);
\coordinate (gppplw) at (\gxxxl, \gyyyw);
\coordinate (gppplx) at (\gxxxl, \gyyyx);
\coordinate (gppply) at (\gxxxl, \gyyyy);
\coordinate (gppplz) at (\gxxxl, \gyyyz);
\coordinate (gpppma) at (\gxxxm, \gyyya);
\coordinate (gpppmb) at (\gxxxm, \gyyyb);
\coordinate (gpppmc) at (\gxxxm, \gyyyc);
\coordinate (gpppmd) at (\gxxxm, \gyyyd);
\coordinate (gpppme) at (\gxxxm, \gyyye);
\coordinate (gpppmf) at (\gxxxm, \gyyyf);
\coordinate (gpppmg) at (\gxxxm, \gyyyg);
\coordinate (gpppmh) at (\gxxxm, \gyyyh);
\coordinate (gpppmi) at (\gxxxm, \gyyyi);
\coordinate (gpppmj) at (\gxxxm, \gyyyj);
\coordinate (gpppmk) at (\gxxxm, \gyyyk);
\coordinate (gpppml) at (\gxxxm, \gyyyl);
\coordinate (gpppmm) at (\gxxxm, \gyyym);
\coordinate (gpppmn) at (\gxxxm, \gyyyn);
\coordinate (gpppmo) at (\gxxxm, \gyyyo);
\coordinate (gpppmp) at (\gxxxm, \gyyyp);
\coordinate (gpppmq) at (\gxxxm, \gyyyq);
\coordinate (gpppmr) at (\gxxxm, \gyyyr);
\coordinate (gpppms) at (\gxxxm, \gyyys);
\coordinate (gpppmt) at (\gxxxm, \gyyyt);
\coordinate (gpppmu) at (\gxxxm, \gyyyu);
\coordinate (gpppmv) at (\gxxxm, \gyyyv);
\coordinate (gpppmw) at (\gxxxm, \gyyyw);
\coordinate (gpppmx) at (\gxxxm, \gyyyx);
\coordinate (gpppmy) at (\gxxxm, \gyyyy);
\coordinate (gpppmz) at (\gxxxm, \gyyyz);
\coordinate (gpppna) at (\gxxxn, \gyyya);
\coordinate (gpppnb) at (\gxxxn, \gyyyb);
\coordinate (gpppnc) at (\gxxxn, \gyyyc);
\coordinate (gpppnd) at (\gxxxn, \gyyyd);
\coordinate (gpppne) at (\gxxxn, \gyyye);
\coordinate (gpppnf) at (\gxxxn, \gyyyf);
\coordinate (gpppng) at (\gxxxn, \gyyyg);
\coordinate (gpppnh) at (\gxxxn, \gyyyh);
\coordinate (gpppni) at (\gxxxn, \gyyyi);
\coordinate (gpppnj) at (\gxxxn, \gyyyj);
\coordinate (gpppnk) at (\gxxxn, \gyyyk);
\coordinate (gpppnl) at (\gxxxn, \gyyyl);
\coordinate (gpppnm) at (\gxxxn, \gyyym);
\coordinate (gpppnn) at (\gxxxn, \gyyyn);
\coordinate (gpppno) at (\gxxxn, \gyyyo);
\coordinate (gpppnp) at (\gxxxn, \gyyyp);
\coordinate (gpppnq) at (\gxxxn, \gyyyq);
\coordinate (gpppnr) at (\gxxxn, \gyyyr);
\coordinate (gpppns) at (\gxxxn, \gyyys);
\coordinate (gpppnt) at (\gxxxn, \gyyyt);
\coordinate (gpppnu) at (\gxxxn, \gyyyu);
\coordinate (gpppnv) at (\gxxxn, \gyyyv);
\coordinate (gpppnw) at (\gxxxn, \gyyyw);
\coordinate (gpppnx) at (\gxxxn, \gyyyx);
\coordinate (gpppny) at (\gxxxn, \gyyyy);
\coordinate (gpppnz) at (\gxxxn, \gyyyz);
\coordinate (gpppoa) at (\gxxxo, \gyyya);
\coordinate (gpppob) at (\gxxxo, \gyyyb);
\coordinate (gpppoc) at (\gxxxo, \gyyyc);
\coordinate (gpppod) at (\gxxxo, \gyyyd);
\coordinate (gpppoe) at (\gxxxo, \gyyye);
\coordinate (gpppof) at (\gxxxo, \gyyyf);
\coordinate (gpppog) at (\gxxxo, \gyyyg);
\coordinate (gpppoh) at (\gxxxo, \gyyyh);
\coordinate (gpppoi) at (\gxxxo, \gyyyi);
\coordinate (gpppoj) at (\gxxxo, \gyyyj);
\coordinate (gpppok) at (\gxxxo, \gyyyk);
\coordinate (gpppol) at (\gxxxo, \gyyyl);
\coordinate (gpppom) at (\gxxxo, \gyyym);
\coordinate (gpppon) at (\gxxxo, \gyyyn);
\coordinate (gpppoo) at (\gxxxo, \gyyyo);
\coordinate (gpppop) at (\gxxxo, \gyyyp);
\coordinate (gpppoq) at (\gxxxo, \gyyyq);
\coordinate (gpppor) at (\gxxxo, \gyyyr);
\coordinate (gpppos) at (\gxxxo, \gyyys);
\coordinate (gpppot) at (\gxxxo, \gyyyt);
\coordinate (gpppou) at (\gxxxo, \gyyyu);
\coordinate (gpppov) at (\gxxxo, \gyyyv);
\coordinate (gpppow) at (\gxxxo, \gyyyw);
\coordinate (gpppox) at (\gxxxo, \gyyyx);
\coordinate (gpppoy) at (\gxxxo, \gyyyy);
\coordinate (gpppoz) at (\gxxxo, \gyyyz);
\coordinate (gppppa) at (\gxxxp, \gyyya);
\coordinate (gppppb) at (\gxxxp, \gyyyb);
\coordinate (gppppc) at (\gxxxp, \gyyyc);
\coordinate (gppppd) at (\gxxxp, \gyyyd);
\coordinate (gppppe) at (\gxxxp, \gyyye);
\coordinate (gppppf) at (\gxxxp, \gyyyf);
\coordinate (gppppg) at (\gxxxp, \gyyyg);
\coordinate (gpppph) at (\gxxxp, \gyyyh);
\coordinate (gppppi) at (\gxxxp, \gyyyi);
\coordinate (gppppj) at (\gxxxp, \gyyyj);
\coordinate (gppppk) at (\gxxxp, \gyyyk);
\coordinate (gppppl) at (\gxxxp, \gyyyl);
\coordinate (gppppm) at (\gxxxp, \gyyym);
\coordinate (gppppn) at (\gxxxp, \gyyyn);
\coordinate (gppppo) at (\gxxxp, \gyyyo);
\coordinate (gppppp) at (\gxxxp, \gyyyp);
\coordinate (gppppq) at (\gxxxp, \gyyyq);
\coordinate (gppppr) at (\gxxxp, \gyyyr);
\coordinate (gpppps) at (\gxxxp, \gyyys);
\coordinate (gppppt) at (\gxxxp, \gyyyt);
\coordinate (gppppu) at (\gxxxp, \gyyyu);
\coordinate (gppppv) at (\gxxxp, \gyyyv);
\coordinate (gppppw) at (\gxxxp, \gyyyw);
\coordinate (gppppx) at (\gxxxp, \gyyyx);
\coordinate (gppppy) at (\gxxxp, \gyyyy);
\coordinate (gppppz) at (\gxxxp, \gyyyz);
\coordinate (gpppqa) at (\gxxxq, \gyyya);
\coordinate (gpppqb) at (\gxxxq, \gyyyb);
\coordinate (gpppqc) at (\gxxxq, \gyyyc);
\coordinate (gpppqd) at (\gxxxq, \gyyyd);
\coordinate (gpppqe) at (\gxxxq, \gyyye);
\coordinate (gpppqf) at (\gxxxq, \gyyyf);
\coordinate (gpppqg) at (\gxxxq, \gyyyg);
\coordinate (gpppqh) at (\gxxxq, \gyyyh);
\coordinate (gpppqi) at (\gxxxq, \gyyyi);
\coordinate (gpppqj) at (\gxxxq, \gyyyj);
\coordinate (gpppqk) at (\gxxxq, \gyyyk);
\coordinate (gpppql) at (\gxxxq, \gyyyl);
\coordinate (gpppqm) at (\gxxxq, \gyyym);
\coordinate (gpppqn) at (\gxxxq, \gyyyn);
\coordinate (gpppqo) at (\gxxxq, \gyyyo);
\coordinate (gpppqp) at (\gxxxq, \gyyyp);
\coordinate (gpppqq) at (\gxxxq, \gyyyq);
\coordinate (gpppqr) at (\gxxxq, \gyyyr);
\coordinate (gpppqs) at (\gxxxq, \gyyys);
\coordinate (gpppqt) at (\gxxxq, \gyyyt);
\coordinate (gpppqu) at (\gxxxq, \gyyyu);
\coordinate (gpppqv) at (\gxxxq, \gyyyv);
\coordinate (gpppqw) at (\gxxxq, \gyyyw);
\coordinate (gpppqx) at (\gxxxq, \gyyyx);
\coordinate (gpppqy) at (\gxxxq, \gyyyy);
\coordinate (gpppqz) at (\gxxxq, \gyyyz);
\coordinate (gpppra) at (\gxxxr, \gyyya);
\coordinate (gppprb) at (\gxxxr, \gyyyb);
\coordinate (gppprc) at (\gxxxr, \gyyyc);
\coordinate (gppprd) at (\gxxxr, \gyyyd);
\coordinate (gpppre) at (\gxxxr, \gyyye);
\coordinate (gppprf) at (\gxxxr, \gyyyf);
\coordinate (gppprg) at (\gxxxr, \gyyyg);
\coordinate (gppprh) at (\gxxxr, \gyyyh);
\coordinate (gpppri) at (\gxxxr, \gyyyi);
\coordinate (gppprj) at (\gxxxr, \gyyyj);
\coordinate (gppprk) at (\gxxxr, \gyyyk);
\coordinate (gppprl) at (\gxxxr, \gyyyl);
\coordinate (gppprm) at (\gxxxr, \gyyym);
\coordinate (gppprn) at (\gxxxr, \gyyyn);
\coordinate (gpppro) at (\gxxxr, \gyyyo);
\coordinate (gppprp) at (\gxxxr, \gyyyp);
\coordinate (gppprq) at (\gxxxr, \gyyyq);
\coordinate (gppprr) at (\gxxxr, \gyyyr);
\coordinate (gppprs) at (\gxxxr, \gyyys);
\coordinate (gppprt) at (\gxxxr, \gyyyt);
\coordinate (gpppru) at (\gxxxr, \gyyyu);
\coordinate (gppprv) at (\gxxxr, \gyyyv);
\coordinate (gppprw) at (\gxxxr, \gyyyw);
\coordinate (gppprx) at (\gxxxr, \gyyyx);
\coordinate (gpppry) at (\gxxxr, \gyyyy);
\coordinate (gppprz) at (\gxxxr, \gyyyz);
\coordinate (gpppsa) at (\gxxxs, \gyyya);
\coordinate (gpppsb) at (\gxxxs, \gyyyb);
\coordinate (gpppsc) at (\gxxxs, \gyyyc);
\coordinate (gpppsd) at (\gxxxs, \gyyyd);
\coordinate (gpppse) at (\gxxxs, \gyyye);
\coordinate (gpppsf) at (\gxxxs, \gyyyf);
\coordinate (gpppsg) at (\gxxxs, \gyyyg);
\coordinate (gpppsh) at (\gxxxs, \gyyyh);
\coordinate (gpppsi) at (\gxxxs, \gyyyi);
\coordinate (gpppsj) at (\gxxxs, \gyyyj);
\coordinate (gpppsk) at (\gxxxs, \gyyyk);
\coordinate (gpppsl) at (\gxxxs, \gyyyl);
\coordinate (gpppsm) at (\gxxxs, \gyyym);
\coordinate (gpppsn) at (\gxxxs, \gyyyn);
\coordinate (gpppso) at (\gxxxs, \gyyyo);
\coordinate (gpppsp) at (\gxxxs, \gyyyp);
\coordinate (gpppsq) at (\gxxxs, \gyyyq);
\coordinate (gpppsr) at (\gxxxs, \gyyyr);
\coordinate (gpppss) at (\gxxxs, \gyyys);
\coordinate (gpppst) at (\gxxxs, \gyyyt);
\coordinate (gpppsu) at (\gxxxs, \gyyyu);
\coordinate (gpppsv) at (\gxxxs, \gyyyv);
\coordinate (gpppsw) at (\gxxxs, \gyyyw);
\coordinate (gpppsx) at (\gxxxs, \gyyyx);
\coordinate (gpppsy) at (\gxxxs, \gyyyy);
\coordinate (gpppsz) at (\gxxxs, \gyyyz);
\coordinate (gpppta) at (\gxxxt, \gyyya);
\coordinate (gppptb) at (\gxxxt, \gyyyb);
\coordinate (gppptc) at (\gxxxt, \gyyyc);
\coordinate (gppptd) at (\gxxxt, \gyyyd);
\coordinate (gpppte) at (\gxxxt, \gyyye);
\coordinate (gppptf) at (\gxxxt, \gyyyf);
\coordinate (gppptg) at (\gxxxt, \gyyyg);
\coordinate (gpppth) at (\gxxxt, \gyyyh);
\coordinate (gpppti) at (\gxxxt, \gyyyi);
\coordinate (gppptj) at (\gxxxt, \gyyyj);
\coordinate (gppptk) at (\gxxxt, \gyyyk);
\coordinate (gppptl) at (\gxxxt, \gyyyl);
\coordinate (gppptm) at (\gxxxt, \gyyym);
\coordinate (gppptn) at (\gxxxt, \gyyyn);
\coordinate (gpppto) at (\gxxxt, \gyyyo);
\coordinate (gppptp) at (\gxxxt, \gyyyp);
\coordinate (gppptq) at (\gxxxt, \gyyyq);
\coordinate (gppptr) at (\gxxxt, \gyyyr);
\coordinate (gpppts) at (\gxxxt, \gyyys);
\coordinate (gppptt) at (\gxxxt, \gyyyt);
\coordinate (gppptu) at (\gxxxt, \gyyyu);
\coordinate (gppptv) at (\gxxxt, \gyyyv);
\coordinate (gppptw) at (\gxxxt, \gyyyw);
\coordinate (gppptx) at (\gxxxt, \gyyyx);
\coordinate (gpppty) at (\gxxxt, \gyyyy);
\coordinate (gppptz) at (\gxxxt, \gyyyz);
\coordinate (gpppua) at (\gxxxu, \gyyya);
\coordinate (gpppub) at (\gxxxu, \gyyyb);
\coordinate (gpppuc) at (\gxxxu, \gyyyc);
\coordinate (gpppud) at (\gxxxu, \gyyyd);
\coordinate (gpppue) at (\gxxxu, \gyyye);
\coordinate (gpppuf) at (\gxxxu, \gyyyf);
\coordinate (gpppug) at (\gxxxu, \gyyyg);
\coordinate (gpppuh) at (\gxxxu, \gyyyh);
\coordinate (gpppui) at (\gxxxu, \gyyyi);
\coordinate (gpppuj) at (\gxxxu, \gyyyj);
\coordinate (gpppuk) at (\gxxxu, \gyyyk);
\coordinate (gpppul) at (\gxxxu, \gyyyl);
\coordinate (gpppum) at (\gxxxu, \gyyym);
\coordinate (gpppun) at (\gxxxu, \gyyyn);
\coordinate (gpppuo) at (\gxxxu, \gyyyo);
\coordinate (gpppup) at (\gxxxu, \gyyyp);
\coordinate (gpppuq) at (\gxxxu, \gyyyq);
\coordinate (gpppur) at (\gxxxu, \gyyyr);
\coordinate (gpppus) at (\gxxxu, \gyyys);
\coordinate (gppput) at (\gxxxu, \gyyyt);
\coordinate (gpppuu) at (\gxxxu, \gyyyu);
\coordinate (gpppuv) at (\gxxxu, \gyyyv);
\coordinate (gpppuw) at (\gxxxu, \gyyyw);
\coordinate (gpppux) at (\gxxxu, \gyyyx);
\coordinate (gpppuy) at (\gxxxu, \gyyyy);
\coordinate (gpppuz) at (\gxxxu, \gyyyz);
\coordinate (gpppva) at (\gxxxv, \gyyya);
\coordinate (gpppvb) at (\gxxxv, \gyyyb);
\coordinate (gpppvc) at (\gxxxv, \gyyyc);
\coordinate (gpppvd) at (\gxxxv, \gyyyd);
\coordinate (gpppve) at (\gxxxv, \gyyye);
\coordinate (gpppvf) at (\gxxxv, \gyyyf);
\coordinate (gpppvg) at (\gxxxv, \gyyyg);
\coordinate (gpppvh) at (\gxxxv, \gyyyh);
\coordinate (gpppvi) at (\gxxxv, \gyyyi);
\coordinate (gpppvj) at (\gxxxv, \gyyyj);
\coordinate (gpppvk) at (\gxxxv, \gyyyk);
\coordinate (gpppvl) at (\gxxxv, \gyyyl);
\coordinate (gpppvm) at (\gxxxv, \gyyym);
\coordinate (gpppvn) at (\gxxxv, \gyyyn);
\coordinate (gpppvo) at (\gxxxv, \gyyyo);
\coordinate (gpppvp) at (\gxxxv, \gyyyp);
\coordinate (gpppvq) at (\gxxxv, \gyyyq);
\coordinate (gpppvr) at (\gxxxv, \gyyyr);
\coordinate (gpppvs) at (\gxxxv, \gyyys);
\coordinate (gpppvt) at (\gxxxv, \gyyyt);
\coordinate (gpppvu) at (\gxxxv, \gyyyu);
\coordinate (gpppvv) at (\gxxxv, \gyyyv);
\coordinate (gpppvw) at (\gxxxv, \gyyyw);
\coordinate (gpppvx) at (\gxxxv, \gyyyx);
\coordinate (gpppvy) at (\gxxxv, \gyyyy);
\coordinate (gpppvz) at (\gxxxv, \gyyyz);
\coordinate (gpppwa) at (\gxxxw, \gyyya);
\coordinate (gpppwb) at (\gxxxw, \gyyyb);
\coordinate (gpppwc) at (\gxxxw, \gyyyc);
\coordinate (gpppwd) at (\gxxxw, \gyyyd);
\coordinate (gpppwe) at (\gxxxw, \gyyye);
\coordinate (gpppwf) at (\gxxxw, \gyyyf);
\coordinate (gpppwg) at (\gxxxw, \gyyyg);
\coordinate (gpppwh) at (\gxxxw, \gyyyh);
\coordinate (gpppwi) at (\gxxxw, \gyyyi);
\coordinate (gpppwj) at (\gxxxw, \gyyyj);
\coordinate (gpppwk) at (\gxxxw, \gyyyk);
\coordinate (gpppwl) at (\gxxxw, \gyyyl);
\coordinate (gpppwm) at (\gxxxw, \gyyym);
\coordinate (gpppwn) at (\gxxxw, \gyyyn);
\coordinate (gpppwo) at (\gxxxw, \gyyyo);
\coordinate (gpppwp) at (\gxxxw, \gyyyp);
\coordinate (gpppwq) at (\gxxxw, \gyyyq);
\coordinate (gpppwr) at (\gxxxw, \gyyyr);
\coordinate (gpppws) at (\gxxxw, \gyyys);
\coordinate (gpppwt) at (\gxxxw, \gyyyt);
\coordinate (gpppwu) at (\gxxxw, \gyyyu);
\coordinate (gpppwv) at (\gxxxw, \gyyyv);
\coordinate (gpppww) at (\gxxxw, \gyyyw);
\coordinate (gpppwx) at (\gxxxw, \gyyyx);
\coordinate (gpppwy) at (\gxxxw, \gyyyy);
\coordinate (gpppwz) at (\gxxxw, \gyyyz);
\coordinate (gpppxa) at (\gxxxx, \gyyya);
\coordinate (gpppxb) at (\gxxxx, \gyyyb);
\coordinate (gpppxc) at (\gxxxx, \gyyyc);
\coordinate (gpppxd) at (\gxxxx, \gyyyd);
\coordinate (gpppxe) at (\gxxxx, \gyyye);
\coordinate (gpppxf) at (\gxxxx, \gyyyf);
\coordinate (gpppxg) at (\gxxxx, \gyyyg);
\coordinate (gpppxh) at (\gxxxx, \gyyyh);
\coordinate (gpppxi) at (\gxxxx, \gyyyi);
\coordinate (gpppxj) at (\gxxxx, \gyyyj);
\coordinate (gpppxk) at (\gxxxx, \gyyyk);
\coordinate (gpppxl) at (\gxxxx, \gyyyl);
\coordinate (gpppxm) at (\gxxxx, \gyyym);
\coordinate (gpppxn) at (\gxxxx, \gyyyn);
\coordinate (gpppxo) at (\gxxxx, \gyyyo);
\coordinate (gpppxp) at (\gxxxx, \gyyyp);
\coordinate (gpppxq) at (\gxxxx, \gyyyq);
\coordinate (gpppxr) at (\gxxxx, \gyyyr);
\coordinate (gpppxs) at (\gxxxx, \gyyys);
\coordinate (gpppxt) at (\gxxxx, \gyyyt);
\coordinate (gpppxu) at (\gxxxx, \gyyyu);
\coordinate (gpppxv) at (\gxxxx, \gyyyv);
\coordinate (gpppxw) at (\gxxxx, \gyyyw);
\coordinate (gpppxx) at (\gxxxx, \gyyyx);
\coordinate (gpppxy) at (\gxxxx, \gyyyy);
\coordinate (gpppxz) at (\gxxxx, \gyyyz);
\coordinate (gpppya) at (\gxxxy, \gyyya);
\coordinate (gpppyb) at (\gxxxy, \gyyyb);
\coordinate (gpppyc) at (\gxxxy, \gyyyc);
\coordinate (gpppyd) at (\gxxxy, \gyyyd);
\coordinate (gpppye) at (\gxxxy, \gyyye);
\coordinate (gpppyf) at (\gxxxy, \gyyyf);
\coordinate (gpppyg) at (\gxxxy, \gyyyg);
\coordinate (gpppyh) at (\gxxxy, \gyyyh);
\coordinate (gpppyi) at (\gxxxy, \gyyyi);
\coordinate (gpppyj) at (\gxxxy, \gyyyj);
\coordinate (gpppyk) at (\gxxxy, \gyyyk);
\coordinate (gpppyl) at (\gxxxy, \gyyyl);
\coordinate (gpppym) at (\gxxxy, \gyyym);
\coordinate (gpppyn) at (\gxxxy, \gyyyn);
\coordinate (gpppyo) at (\gxxxy, \gyyyo);
\coordinate (gpppyp) at (\gxxxy, \gyyyp);
\coordinate (gpppyq) at (\gxxxy, \gyyyq);
\coordinate (gpppyr) at (\gxxxy, \gyyyr);
\coordinate (gpppys) at (\gxxxy, \gyyys);
\coordinate (gpppyt) at (\gxxxy, \gyyyt);
\coordinate (gpppyu) at (\gxxxy, \gyyyu);
\coordinate (gpppyv) at (\gxxxy, \gyyyv);
\coordinate (gpppyw) at (\gxxxy, \gyyyw);
\coordinate (gpppyx) at (\gxxxy, \gyyyx);
\coordinate (gpppyy) at (\gxxxy, \gyyyy);
\coordinate (gpppyz) at (\gxxxy, \gyyyz);
\coordinate (gpppza) at (\gxxxz, \gyyya);
\coordinate (gpppzb) at (\gxxxz, \gyyyb);
\coordinate (gpppzc) at (\gxxxz, \gyyyc);
\coordinate (gpppzd) at (\gxxxz, \gyyyd);
\coordinate (gpppze) at (\gxxxz, \gyyye);
\coordinate (gpppzf) at (\gxxxz, \gyyyf);
\coordinate (gpppzg) at (\gxxxz, \gyyyg);
\coordinate (gpppzh) at (\gxxxz, \gyyyh);
\coordinate (gpppzi) at (\gxxxz, \gyyyi);
\coordinate (gpppzj) at (\gxxxz, \gyyyj);
\coordinate (gpppzk) at (\gxxxz, \gyyyk);
\coordinate (gpppzl) at (\gxxxz, \gyyyl);
\coordinate (gpppzm) at (\gxxxz, \gyyym);
\coordinate (gpppzn) at (\gxxxz, \gyyyn);
\coordinate (gpppzo) at (\gxxxz, \gyyyo);
\coordinate (gpppzp) at (\gxxxz, \gyyyp);
\coordinate (gpppzq) at (\gxxxz, \gyyyq);
\coordinate (gpppzr) at (\gxxxz, \gyyyr);
\coordinate (gpppzs) at (\gxxxz, \gyyys);
\coordinate (gpppzt) at (\gxxxz, \gyyyt);
\coordinate (gpppzu) at (\gxxxz, \gyyyu);
\coordinate (gpppzv) at (\gxxxz, \gyyyv);
\coordinate (gpppzw) at (\gxxxz, \gyyyw);
\coordinate (gpppzx) at (\gxxxz, \gyyyx);
\coordinate (gpppzy) at (\gxxxz, \gyyyy);
\coordinate (gpppzz) at (\gxxxz, \gyyyz);

%\gangprintcoordinateat{(0,0)}{The last coordinate values: }{($(gpppzz)$)}; 


% Draw related part of the coordinate system with dashed helplines (centered at (gpppii)) with letters as background, which would help to determine all coordinates. 
\coordinatebackground{g}{c}{d}{o};

% Step 2, draw key devices, their accessories, and take related coordinates of their pins, and may define more coordinates. 

% Draw the Opamp at the coordinate (gpppii) and name it as "swopamp".
\draw (gpppii) node [op amp, yscale=-1] (swopamp) {\ctikzflipy{Opamp}} ; 

% Its accessories and lables. 
\draw [-*](swopamp.down) -- ($(swopamp.down)+(0,1)$) node[right]{$V_+$}; 
\node at ($(swopamp.down)+(0.3,0.2)$) {7};  
\draw [-*](swopamp.up) -- ($(swopamp.up)+(0,-1)$) node[right]{$V_-$}; 
\node at ($(swopamp.up)+(0.3,-0.2)$) {4};

% Get the x- and y-components of the coordinates of the "+" and "-" pins. 
\getxyingivenunit{cm}{(swopamp.+)}{\swopampzx}{\swopampzy};
\getxyingivenunit{cm}{(swopamp.-)}{\swopampfx}{\swopampfy};

% Then define a few more coordinates, at least for keeping in mind.
\coordinate (plusshort) at ($(\gxxxg,\swopampzy)$);
%\fill  (plusshort) circle (2pt);  % May be commented later.
\coordinate (minusshort) at ($(\gxxxg,\swopampfy)$);
%\fill  (minusshort) circle (2pt); % May be commented later.
\coordinate (leftinter) at ($(\gxxxe,\swopampzy)$);
\fill  (leftinter) circle (2pt);

% Draw an "npn" at (gpppmi) and name it as "swQ".
\draw (gpppmi) node[npn](swQ){};

% Get the x- and y-components of the needed pins of it for later usage.
\getxyingivenunit{cm}{(swQ.C)}{\swQCx}{\swQCy};
\getxyingivenunit{cm}{(swQ.E)}{\swQEx}{\swQEy};

% Then define more coordinate(s).
\coordinate (Qcshort) at ($(\swQEx,\gyyyj)$);
%\fill  (Qcshort) circle (2pt); % May be commented later.
\coordinate (Qeshort) at ($(\swQEx,\gyyyf)$);
\fill  (Qeshort) circle (2pt) node [right] {$V_0$};

% Then the rectangle by the points (gpppef) -- (gpppej) -- (Qcshort) -- (Qeshort) forms a clear area for the key devices. 

% Connect the two devices.
\draw (swopamp.out) to [short, l=$I_B$, above] (swQ.B);

% Step 3, draw other little devices. For tidiness, better to give two units in length for each new device and align them up.

% For this specific circuit, let us attach the four bi-pole devices (maybe with their accessories) to each corner of the above mentioned rectangle area for the key devices, separately. 

\draw  (gppped) node [ground] {} to [empty ZZener diode] (gpppef) -- (leftinter);
% The Latex system can not work properly when I put the following label into the above "[empty ZZener diode]" in the form of an additional "l= ..." option, then I have to employ the following "\node ..." command. Then the label should be aligned with the "ZZener" as much as possible even the coordinate system is modified later. Since the "\gyyyd" and "\gyyyf" micros are used to position the "ZZener", then better to use the center between them to locate the label, rather than using "\gyyye" directly. This idea is also applied in later "\node ..." commands. 
\node at ($(\gxxxe-1.3, \gyyyd*0.5+\gyyyf*0.5)$) {$V_Z = 5\textnormal{V}$};

\draw (gpppej) to [generic] (gpppel) -| (swQ.C);
\node at ($(\gxxxe-1.1, \gyyyj*0.5+\gyyyl*0.5)$) {$R_{1}=47k\Omega$};
\node [right] at (\swQCx,\gyyyj*0.5+\gyyyl*0.5) {$I_C \approx \beta I_B$};

\draw  (\swQEx, \gyyyd) node [ground] {} to [generic] (\swQEx, \gyyyf) -- (swQ.E);
\node at ($(\swQEx+1.4, \gyyyd*0.5+\gyyyf*0.5)$) {$R_{E}=100k\Omega$};

% Draw the top area. 
\draw  (\gxxxi-0.2,\gyyyn) --  (\gxxxi+0.2,\gyyyn) node [right] {$V_{cc}=15\textnormal{V}$} ;
\draw [->] (gpppin) -- (gpppim) node [right] {$I$};  
\draw  (gpppim) -- (gpppil);
\fill  (gpppil) circle (2pt);

% Step 4, other shorts.
\draw  (swopamp.+)  to [short, l_=$I_+ \approx 0 $, above] (plusshort) -- (leftinter) -- (gpppej);

\draw  (swopamp.-)  to [short, l_=$I_- \approx 0 $, above] (minusshort) |- (Qeshort);

%Step 5, all the rest, especially labels. May also clean unnecessary staff, like the system background and dark points for showing newly defined coordinates previously. 
\draw [->] ($(\swQEx-0.4, \gyyyf - 0.4)$) -- node [left] {$I_E$} ($(\swQEx-0.4, \gyyyd + 0.4)$);

\draw [->] ($(\gxxxe+0.4, \gyyyl*0.5+\gyyyj*0.5 + 0.6)$) -- node [right] {$I_1$} ($(\gxxxe+0.4, \gyyyl*0.5+\gyyyj*0.5 - 0.6)$);


\pgfmathsetmacro{\bigswQCx}{\swQCx}


















% The following is the first circuit, which uses "h" coordinate system, to be inserted into and connected to the above circuit.
% It is modified based on Figure 2.

% Circuits can be drawn by the following five major steps, as shown in the following example. 

% Step 1, preparations. 

% "Install" the coordinate system with keyword "h".
\pgfmathsetmacro{\totalhxxx}{26}
\pgfmathsetmacro{\totalhyyy}{26}
\pgfmathsetmacro{\hxxxspacing}{1}
\pgfmathsetmacro{\hyyyspacing}{1}
\pgfmathsetmacro{\hxxxa}{-2.5}
\pgfmathsetmacro{\hyyya}{06.5}

\pgfmathsetmacro{\hxxxb}{\hxxxa + \hxxxspacing + 0.0 }
\pgfmathsetmacro{\hxxxc}{\hxxxb + \hxxxspacing + 0.0 }
\pgfmathsetmacro{\hxxxd}{\hxxxc + \hxxxspacing + 0.0 }
\pgfmathsetmacro{\hxxxe}{\hxxxd + \hxxxspacing + 0.0 }
\pgfmathsetmacro{\hxxxf}{\hxxxe + \hxxxspacing + 0.0 }
\pgfmathsetmacro{\hxxxg}{\hxxxf + \hxxxspacing + 0.0 }
\pgfmathsetmacro{\hxxxh}{\hxxxg + \hxxxspacing + 0.0 }
\pgfmathsetmacro{\hxxxi}{\hxxxh + \hxxxspacing + 0.0 }
\pgfmathsetmacro{\hxxxj}{\hxxxi + \hxxxspacing + 0.0 }
\pgfmathsetmacro{\hxxxk}{\hxxxj + \hxxxspacing + 0.0 }
\pgfmathsetmacro{\hxxxl}{\hxxxk + \hxxxspacing + 0.0 }
\pgfmathsetmacro{\hxxxm}{\hxxxl + \hxxxspacing + 0.0 }
\pgfmathsetmacro{\hxxxn}{\hxxxm + \hxxxspacing + 0.0 }
\pgfmathsetmacro{\hxxxo}{\hxxxn + \hxxxspacing + 0.0 }
\pgfmathsetmacro{\hxxxp}{\hxxxo + \hxxxspacing + 0.0 }
\pgfmathsetmacro{\hxxxq}{\hxxxp + \hxxxspacing + 0.0 }
\pgfmathsetmacro{\hxxxr}{\hxxxq + \hxxxspacing + 0.0 }
\pgfmathsetmacro{\hxxxs}{\hxxxr + \hxxxspacing + 0.0 }
\pgfmathsetmacro{\hxxxt}{\hxxxs + \hxxxspacing + 0.0 }
\pgfmathsetmacro{\hxxxu}{\hxxxt + \hxxxspacing + 0.0 }
\pgfmathsetmacro{\hxxxv}{\hxxxu + \hxxxspacing + 0.0 }
\pgfmathsetmacro{\hxxxw}{\hxxxv + \hxxxspacing + 0.0 }
\pgfmathsetmacro{\hxxxx}{\hxxxw + \hxxxspacing + 0.0 }
\pgfmathsetmacro{\hxxxy}{\hxxxx + \hxxxspacing + 0.0 }
\pgfmathsetmacro{\hxxxz}{\hxxxy + \hxxxspacing + 0.0 }

\pgfmathsetmacro{\hyyyb}{\hyyya + \hyyyspacing + 0.0 }
\pgfmathsetmacro{\hyyyc}{\hyyyb + \hyyyspacing + 0.0 }
\pgfmathsetmacro{\hyyyd}{\hyyyc + \hyyyspacing + 0.0 }
\pgfmathsetmacro{\hyyye}{\hyyyd + \hyyyspacing + 0.0 }
\pgfmathsetmacro{\hyyyf}{\hyyye + \hyyyspacing + 0.0 }
\pgfmathsetmacro{\hyyyg}{\hyyyf + \hyyyspacing + 0.0 }
\pgfmathsetmacro{\hyyyh}{\hyyyg + \hyyyspacing + 0.0 }
\pgfmathsetmacro{\hyyyi}{\hyyyh + \hyyyspacing  -1.0 }
\pgfmathsetmacro{\hyyyj}{\hyyyi + \hyyyspacing + 0.0 }
\pgfmathsetmacro{\hyyyk}{\hyyyj + \hyyyspacing + 0.0 }
\pgfmathsetmacro{\hyyyl}{\hyyyk + \hyyyspacing + 0.0 }
\pgfmathsetmacro{\hyyym}{\hyyyl + \hyyyspacing + 0.0 }
\pgfmathsetmacro{\hyyyn}{\hyyym + \hyyyspacing + 0.0 }
\pgfmathsetmacro{\hyyyo}{\hyyyn + \hyyyspacing + 0.0 }
\pgfmathsetmacro{\hyyyp}{\hyyyo + \hyyyspacing + 0.0 }
\pgfmathsetmacro{\hyyyq}{\hyyyp + \hyyyspacing + 0.0 }
\pgfmathsetmacro{\hyyyr}{\hyyyq + \hyyyspacing + 0.0 }
\pgfmathsetmacro{\hyyys}{\hyyyr + \hyyyspacing + 0.0 }
\pgfmathsetmacro{\hyyyt}{\hyyys + \hyyyspacing + 0.0 }
\pgfmathsetmacro{\hyyyu}{\hyyyt + \hyyyspacing + 0.0 }
\pgfmathsetmacro{\hyyyv}{\hyyyu + \hyyyspacing + 0.0 }
\pgfmathsetmacro{\hyyyw}{\hyyyv + \hyyyspacing + 0.0 }
\pgfmathsetmacro{\hyyyx}{\hyyyw + \hyyyspacing + 0.0 }
\pgfmathsetmacro{\hyyyy}{\hyyyx + \hyyyspacing + 0.0 }
\pgfmathsetmacro{\hyyyz}{\hyyyy + \hyyyspacing + 0.0 }

\coordinate (hpppaa) at (\hxxxa, \hyyya);
\coordinate (hpppab) at (\hxxxa, \hyyyb);
\coordinate (hpppac) at (\hxxxa, \hyyyc);
\coordinate (hpppad) at (\hxxxa, \hyyyd);
\coordinate (hpppae) at (\hxxxa, \hyyye);
\coordinate (hpppaf) at (\hxxxa, \hyyyf);
\coordinate (hpppag) at (\hxxxa, \hyyyg);
\coordinate (hpppah) at (\hxxxa, \hyyyh);
\coordinate (hpppai) at (\hxxxa, \hyyyi);
\coordinate (hpppaj) at (\hxxxa, \hyyyj);
\coordinate (hpppak) at (\hxxxa, \hyyyk);
\coordinate (hpppal) at (\hxxxa, \hyyyl);
\coordinate (hpppam) at (\hxxxa, \hyyym);
\coordinate (hpppan) at (\hxxxa, \hyyyn);
\coordinate (hpppao) at (\hxxxa, \hyyyo);
\coordinate (hpppap) at (\hxxxa, \hyyyp);
\coordinate (hpppaq) at (\hxxxa, \hyyyq);
\coordinate (hpppar) at (\hxxxa, \hyyyr);
\coordinate (hpppas) at (\hxxxa, \hyyys);
\coordinate (hpppat) at (\hxxxa, \hyyyt);
\coordinate (hpppau) at (\hxxxa, \hyyyu);
\coordinate (hpppav) at (\hxxxa, \hyyyv);
\coordinate (hpppaw) at (\hxxxa, \hyyyw);
\coordinate (hpppax) at (\hxxxa, \hyyyx);
\coordinate (hpppay) at (\hxxxa, \hyyyy);
\coordinate (hpppaz) at (\hxxxa, \hyyyz);
\coordinate (hpppba) at (\hxxxb, \hyyya);
\coordinate (hpppbb) at (\hxxxb, \hyyyb);
\coordinate (hpppbc) at (\hxxxb, \hyyyc);
\coordinate (hpppbd) at (\hxxxb, \hyyyd);
\coordinate (hpppbe) at (\hxxxb, \hyyye);
\coordinate (hpppbf) at (\hxxxb, \hyyyf);
\coordinate (hpppbg) at (\hxxxb, \hyyyg);
\coordinate (hpppbh) at (\hxxxb, \hyyyh);
\coordinate (hpppbi) at (\hxxxb, \hyyyi);
\coordinate (hpppbj) at (\hxxxb, \hyyyj);
\coordinate (hpppbk) at (\hxxxb, \hyyyk);
\coordinate (hpppbl) at (\hxxxb, \hyyyl);
\coordinate (hpppbm) at (\hxxxb, \hyyym);
\coordinate (hpppbn) at (\hxxxb, \hyyyn);
\coordinate (hpppbo) at (\hxxxb, \hyyyo);
\coordinate (hpppbp) at (\hxxxb, \hyyyp);
\coordinate (hpppbq) at (\hxxxb, \hyyyq);
\coordinate (hpppbr) at (\hxxxb, \hyyyr);
\coordinate (hpppbs) at (\hxxxb, \hyyys);
\coordinate (hpppbt) at (\hxxxb, \hyyyt);
\coordinate (hpppbu) at (\hxxxb, \hyyyu);
\coordinate (hpppbv) at (\hxxxb, \hyyyv);
\coordinate (hpppbw) at (\hxxxb, \hyyyw);
\coordinate (hpppbx) at (\hxxxb, \hyyyx);
\coordinate (hpppby) at (\hxxxb, \hyyyy);
\coordinate (hpppbz) at (\hxxxb, \hyyyz);
\coordinate (hpppca) at (\hxxxc, \hyyya);
\coordinate (hpppcb) at (\hxxxc, \hyyyb);
\coordinate (hpppcc) at (\hxxxc, \hyyyc);
\coordinate (hpppcd) at (\hxxxc, \hyyyd);
\coordinate (hpppce) at (\hxxxc, \hyyye);
\coordinate (hpppcf) at (\hxxxc, \hyyyf);
\coordinate (hpppcg) at (\hxxxc, \hyyyg);
\coordinate (hpppch) at (\hxxxc, \hyyyh);
\coordinate (hpppci) at (\hxxxc, \hyyyi);
\coordinate (hpppcj) at (\hxxxc, \hyyyj);
\coordinate (hpppck) at (\hxxxc, \hyyyk);
\coordinate (hpppcl) at (\hxxxc, \hyyyl);
\coordinate (hpppcm) at (\hxxxc, \hyyym);
\coordinate (hpppcn) at (\hxxxc, \hyyyn);
\coordinate (hpppco) at (\hxxxc, \hyyyo);
\coordinate (hpppcp) at (\hxxxc, \hyyyp);
\coordinate (hpppcq) at (\hxxxc, \hyyyq);
\coordinate (hpppcr) at (\hxxxc, \hyyyr);
\coordinate (hpppcs) at (\hxxxc, \hyyys);
\coordinate (hpppct) at (\hxxxc, \hyyyt);
\coordinate (hpppcu) at (\hxxxc, \hyyyu);
\coordinate (hpppcv) at (\hxxxc, \hyyyv);
\coordinate (hpppcw) at (\hxxxc, \hyyyw);
\coordinate (hpppcx) at (\hxxxc, \hyyyx);
\coordinate (hpppcy) at (\hxxxc, \hyyyy);
\coordinate (hpppcz) at (\hxxxc, \hyyyz);
\coordinate (hpppda) at (\hxxxd, \hyyya);
\coordinate (hpppdb) at (\hxxxd, \hyyyb);
\coordinate (hpppdc) at (\hxxxd, \hyyyc);
\coordinate (hpppdd) at (\hxxxd, \hyyyd);
\coordinate (hpppde) at (\hxxxd, \hyyye);
\coordinate (hpppdf) at (\hxxxd, \hyyyf);
\coordinate (hpppdg) at (\hxxxd, \hyyyg);
\coordinate (hpppdh) at (\hxxxd, \hyyyh);
\coordinate (hpppdi) at (\hxxxd, \hyyyi);
\coordinate (hpppdj) at (\hxxxd, \hyyyj);
\coordinate (hpppdk) at (\hxxxd, \hyyyk);
\coordinate (hpppdl) at (\hxxxd, \hyyyl);
\coordinate (hpppdm) at (\hxxxd, \hyyym);
\coordinate (hpppdn) at (\hxxxd, \hyyyn);
\coordinate (hpppdo) at (\hxxxd, \hyyyo);
\coordinate (hpppdp) at (\hxxxd, \hyyyp);
\coordinate (hpppdq) at (\hxxxd, \hyyyq);
\coordinate (hpppdr) at (\hxxxd, \hyyyr);
\coordinate (hpppds) at (\hxxxd, \hyyys);
\coordinate (hpppdt) at (\hxxxd, \hyyyt);
\coordinate (hpppdu) at (\hxxxd, \hyyyu);
\coordinate (hpppdv) at (\hxxxd, \hyyyv);
\coordinate (hpppdw) at (\hxxxd, \hyyyw);
\coordinate (hpppdx) at (\hxxxd, \hyyyx);
\coordinate (hpppdy) at (\hxxxd, \hyyyy);
\coordinate (hpppdz) at (\hxxxd, \hyyyz);
\coordinate (hpppea) at (\hxxxe, \hyyya);
\coordinate (hpppeb) at (\hxxxe, \hyyyb);
\coordinate (hpppec) at (\hxxxe, \hyyyc);
\coordinate (hppped) at (\hxxxe, \hyyyd);
\coordinate (hpppee) at (\hxxxe, \hyyye);
\coordinate (hpppef) at (\hxxxe, \hyyyf);
\coordinate (hpppeg) at (\hxxxe, \hyyyg);
\coordinate (hpppeh) at (\hxxxe, \hyyyh);
\coordinate (hpppei) at (\hxxxe, \hyyyi);
\coordinate (hpppej) at (\hxxxe, \hyyyj);
\coordinate (hpppek) at (\hxxxe, \hyyyk);
\coordinate (hpppel) at (\hxxxe, \hyyyl);
\coordinate (hpppem) at (\hxxxe, \hyyym);
\coordinate (hpppen) at (\hxxxe, \hyyyn);
\coordinate (hpppeo) at (\hxxxe, \hyyyo);
\coordinate (hpppep) at (\hxxxe, \hyyyp);
\coordinate (hpppeq) at (\hxxxe, \hyyyq);
\coordinate (hppper) at (\hxxxe, \hyyyr);
\coordinate (hpppes) at (\hxxxe, \hyyys);
\coordinate (hpppet) at (\hxxxe, \hyyyt);
\coordinate (hpppeu) at (\hxxxe, \hyyyu);
\coordinate (hpppev) at (\hxxxe, \hyyyv);
\coordinate (hpppew) at (\hxxxe, \hyyyw);
\coordinate (hpppex) at (\hxxxe, \hyyyx);
\coordinate (hpppey) at (\hxxxe, \hyyyy);
\coordinate (hpppez) at (\hxxxe, \hyyyz);
\coordinate (hpppfa) at (\hxxxf, \hyyya);
\coordinate (hpppfb) at (\hxxxf, \hyyyb);
\coordinate (hpppfc) at (\hxxxf, \hyyyc);
\coordinate (hpppfd) at (\hxxxf, \hyyyd);
\coordinate (hpppfe) at (\hxxxf, \hyyye);
\coordinate (hpppff) at (\hxxxf, \hyyyf);
\coordinate (hpppfg) at (\hxxxf, \hyyyg);
\coordinate (hpppfh) at (\hxxxf, \hyyyh);
\coordinate (hpppfi) at (\hxxxf, \hyyyi);
\coordinate (hpppfj) at (\hxxxf, \hyyyj);
\coordinate (hpppfk) at (\hxxxf, \hyyyk);
\coordinate (hpppfl) at (\hxxxf, \hyyyl);
\coordinate (hpppfm) at (\hxxxf, \hyyym);
\coordinate (hpppfn) at (\hxxxf, \hyyyn);
\coordinate (hpppfo) at (\hxxxf, \hyyyo);
\coordinate (hpppfp) at (\hxxxf, \hyyyp);
\coordinate (hpppfq) at (\hxxxf, \hyyyq);
\coordinate (hpppfr) at (\hxxxf, \hyyyr);
\coordinate (hpppfs) at (\hxxxf, \hyyys);
\coordinate (hpppft) at (\hxxxf, \hyyyt);
\coordinate (hpppfu) at (\hxxxf, \hyyyu);
\coordinate (hpppfv) at (\hxxxf, \hyyyv);
\coordinate (hpppfw) at (\hxxxf, \hyyyw);
\coordinate (hpppfx) at (\hxxxf, \hyyyx);
\coordinate (hpppfy) at (\hxxxf, \hyyyy);
\coordinate (hpppfz) at (\hxxxf, \hyyyz);
\coordinate (hpppga) at (\hxxxg, \hyyya);
\coordinate (hpppgb) at (\hxxxg, \hyyyb);
\coordinate (hpppgc) at (\hxxxg, \hyyyc);
\coordinate (hpppgd) at (\hxxxg, \hyyyd);
\coordinate (hpppge) at (\hxxxg, \hyyye);
\coordinate (hpppgf) at (\hxxxg, \hyyyf);
\coordinate (hpppgg) at (\hxxxg, \hyyyg);
\coordinate (hpppgh) at (\hxxxg, \hyyyh);
\coordinate (hpppgi) at (\hxxxg, \hyyyi);
\coordinate (hpppgj) at (\hxxxg, \hyyyj);
\coordinate (hpppgk) at (\hxxxg, \hyyyk);
\coordinate (hpppgl) at (\hxxxg, \hyyyl);
\coordinate (hpppgm) at (\hxxxg, \hyyym);
\coordinate (hpppgn) at (\hxxxg, \hyyyn);
\coordinate (hpppgo) at (\hxxxg, \hyyyo);
\coordinate (hpppgp) at (\hxxxg, \hyyyp);
\coordinate (hpppgq) at (\hxxxg, \hyyyq);
\coordinate (hpppgr) at (\hxxxg, \hyyyr);
\coordinate (hpppgs) at (\hxxxg, \hyyys);
\coordinate (hpppgt) at (\hxxxg, \hyyyt);
\coordinate (hpppgu) at (\hxxxg, \hyyyu);
\coordinate (hpppgv) at (\hxxxg, \hyyyv);
\coordinate (hpppgw) at (\hxxxg, \hyyyw);
\coordinate (hpppgx) at (\hxxxg, \hyyyx);
\coordinate (hpppgy) at (\hxxxg, \hyyyy);
\coordinate (hpppgz) at (\hxxxg, \hyyyz);
\coordinate (hpppha) at (\hxxxh, \hyyya);
\coordinate (hppphb) at (\hxxxh, \hyyyb);
\coordinate (hppphc) at (\hxxxh, \hyyyc);
\coordinate (hppphd) at (\hxxxh, \hyyyd);
\coordinate (hppphe) at (\hxxxh, \hyyye);
\coordinate (hppphf) at (\hxxxh, \hyyyf);
\coordinate (hppphg) at (\hxxxh, \hyyyg);
\coordinate (hppphh) at (\hxxxh, \hyyyh);
\coordinate (hppphi) at (\hxxxh, \hyyyi);
\coordinate (hppphj) at (\hxxxh, \hyyyj);
\coordinate (hppphk) at (\hxxxh, \hyyyk);
\coordinate (hppphl) at (\hxxxh, \hyyyl);
\coordinate (hppphm) at (\hxxxh, \hyyym);
\coordinate (hppphn) at (\hxxxh, \hyyyn);
\coordinate (hpppho) at (\hxxxh, \hyyyo);
\coordinate (hppphp) at (\hxxxh, \hyyyp);
\coordinate (hppphq) at (\hxxxh, \hyyyq);
\coordinate (hppphr) at (\hxxxh, \hyyyr);
\coordinate (hppphs) at (\hxxxh, \hyyys);
\coordinate (hpppht) at (\hxxxh, \hyyyt);
\coordinate (hppphu) at (\hxxxh, \hyyyu);
\coordinate (hppphv) at (\hxxxh, \hyyyv);
\coordinate (hppphw) at (\hxxxh, \hyyyw);
\coordinate (hppphx) at (\hxxxh, \hyyyx);
\coordinate (hppphy) at (\hxxxh, \hyyyy);
\coordinate (hppphz) at (\hxxxh, \hyyyz);
\coordinate (hpppia) at (\hxxxi, \hyyya);
\coordinate (hpppib) at (\hxxxi, \hyyyb);
\coordinate (hpppic) at (\hxxxi, \hyyyc);
\coordinate (hpppid) at (\hxxxi, \hyyyd);
\coordinate (hpppie) at (\hxxxi, \hyyye);
\coordinate (hpppif) at (\hxxxi, \hyyyf);
\coordinate (hpppig) at (\hxxxi, \hyyyg);
\coordinate (hpppih) at (\hxxxi, \hyyyh);
\coordinate (hpppii) at (\hxxxi, \hyyyi);
\coordinate (hpppij) at (\hxxxi, \hyyyj);
\coordinate (hpppik) at (\hxxxi, \hyyyk);
\coordinate (hpppil) at (\hxxxi, \hyyyl);
\coordinate (hpppim) at (\hxxxi, \hyyym);
\coordinate (hpppin) at (\hxxxi, \hyyyn);
\coordinate (hpppio) at (\hxxxi, \hyyyo);
\coordinate (hpppip) at (\hxxxi, \hyyyp);
\coordinate (hpppiq) at (\hxxxi, \hyyyq);
\coordinate (hpppir) at (\hxxxi, \hyyyr);
\coordinate (hpppis) at (\hxxxi, \hyyys);
\coordinate (hpppit) at (\hxxxi, \hyyyt);
\coordinate (hpppiu) at (\hxxxi, \hyyyu);
\coordinate (hpppiv) at (\hxxxi, \hyyyv);
\coordinate (hpppiw) at (\hxxxi, \hyyyw);
\coordinate (hpppix) at (\hxxxi, \hyyyx);
\coordinate (hpppiy) at (\hxxxi, \hyyyy);
\coordinate (hpppiz) at (\hxxxi, \hyyyz);
\coordinate (hpppja) at (\hxxxj, \hyyya);
\coordinate (hpppjb) at (\hxxxj, \hyyyb);
\coordinate (hpppjc) at (\hxxxj, \hyyyc);
\coordinate (hpppjd) at (\hxxxj, \hyyyd);
\coordinate (hpppje) at (\hxxxj, \hyyye);
\coordinate (hpppjf) at (\hxxxj, \hyyyf);
\coordinate (hpppjg) at (\hxxxj, \hyyyg);
\coordinate (hpppjh) at (\hxxxj, \hyyyh);
\coordinate (hpppji) at (\hxxxj, \hyyyi);
\coordinate (hpppjj) at (\hxxxj, \hyyyj);
\coordinate (hpppjk) at (\hxxxj, \hyyyk);
\coordinate (hpppjl) at (\hxxxj, \hyyyl);
\coordinate (hpppjm) at (\hxxxj, \hyyym);
\coordinate (hpppjn) at (\hxxxj, \hyyyn);
\coordinate (hpppjo) at (\hxxxj, \hyyyo);
\coordinate (hpppjp) at (\hxxxj, \hyyyp);
\coordinate (hpppjq) at (\hxxxj, \hyyyq);
\coordinate (hpppjr) at (\hxxxj, \hyyyr);
\coordinate (hpppjs) at (\hxxxj, \hyyys);
\coordinate (hpppjt) at (\hxxxj, \hyyyt);
\coordinate (hpppju) at (\hxxxj, \hyyyu);
\coordinate (hpppjv) at (\hxxxj, \hyyyv);
\coordinate (hpppjw) at (\hxxxj, \hyyyw);
\coordinate (hpppjx) at (\hxxxj, \hyyyx);
\coordinate (hpppjy) at (\hxxxj, \hyyyy);
\coordinate (hpppjz) at (\hxxxj, \hyyyz);
\coordinate (hpppka) at (\hxxxk, \hyyya);
\coordinate (hpppkb) at (\hxxxk, \hyyyb);
\coordinate (hpppkc) at (\hxxxk, \hyyyc);
\coordinate (hpppkd) at (\hxxxk, \hyyyd);
\coordinate (hpppke) at (\hxxxk, \hyyye);
\coordinate (hpppkf) at (\hxxxk, \hyyyf);
\coordinate (hpppkg) at (\hxxxk, \hyyyg);
\coordinate (hpppkh) at (\hxxxk, \hyyyh);
\coordinate (hpppki) at (\hxxxk, \hyyyi);
\coordinate (hpppkj) at (\hxxxk, \hyyyj);
\coordinate (hpppkk) at (\hxxxk, \hyyyk);
\coordinate (hpppkl) at (\hxxxk, \hyyyl);
\coordinate (hpppkm) at (\hxxxk, \hyyym);
\coordinate (hpppkn) at (\hxxxk, \hyyyn);
\coordinate (hpppko) at (\hxxxk, \hyyyo);
\coordinate (hpppkp) at (\hxxxk, \hyyyp);
\coordinate (hpppkq) at (\hxxxk, \hyyyq);
\coordinate (hpppkr) at (\hxxxk, \hyyyr);
\coordinate (hpppks) at (\hxxxk, \hyyys);
\coordinate (hpppkt) at (\hxxxk, \hyyyt);
\coordinate (hpppku) at (\hxxxk, \hyyyu);
\coordinate (hpppkv) at (\hxxxk, \hyyyv);
\coordinate (hpppkw) at (\hxxxk, \hyyyw);
\coordinate (hpppkx) at (\hxxxk, \hyyyx);
\coordinate (hpppky) at (\hxxxk, \hyyyy);
\coordinate (hpppkz) at (\hxxxk, \hyyyz);
\coordinate (hpppla) at (\hxxxl, \hyyya);
\coordinate (hppplb) at (\hxxxl, \hyyyb);
\coordinate (hppplc) at (\hxxxl, \hyyyc);
\coordinate (hpppld) at (\hxxxl, \hyyyd);
\coordinate (hppple) at (\hxxxl, \hyyye);
\coordinate (hppplf) at (\hxxxl, \hyyyf);
\coordinate (hppplg) at (\hxxxl, \hyyyg);
\coordinate (hppplh) at (\hxxxl, \hyyyh);
\coordinate (hpppli) at (\hxxxl, \hyyyi);
\coordinate (hppplj) at (\hxxxl, \hyyyj);
\coordinate (hppplk) at (\hxxxl, \hyyyk);
\coordinate (hpppll) at (\hxxxl, \hyyyl);
\coordinate (hppplm) at (\hxxxl, \hyyym);
\coordinate (hpppln) at (\hxxxl, \hyyyn);
\coordinate (hppplo) at (\hxxxl, \hyyyo);
\coordinate (hppplp) at (\hxxxl, \hyyyp);
\coordinate (hppplq) at (\hxxxl, \hyyyq);
\coordinate (hppplr) at (\hxxxl, \hyyyr);
\coordinate (hpppls) at (\hxxxl, \hyyys);
\coordinate (hppplt) at (\hxxxl, \hyyyt);
\coordinate (hppplu) at (\hxxxl, \hyyyu);
\coordinate (hppplv) at (\hxxxl, \hyyyv);
\coordinate (hppplw) at (\hxxxl, \hyyyw);
\coordinate (hppplx) at (\hxxxl, \hyyyx);
\coordinate (hppply) at (\hxxxl, \hyyyy);
\coordinate (hppplz) at (\hxxxl, \hyyyz);
\coordinate (hpppma) at (\hxxxm, \hyyya);
\coordinate (hpppmb) at (\hxxxm, \hyyyb);
\coordinate (hpppmc) at (\hxxxm, \hyyyc);
\coordinate (hpppmd) at (\hxxxm, \hyyyd);
\coordinate (hpppme) at (\hxxxm, \hyyye);
\coordinate (hpppmf) at (\hxxxm, \hyyyf);
\coordinate (hpppmg) at (\hxxxm, \hyyyg);
\coordinate (hpppmh) at (\hxxxm, \hyyyh);
\coordinate (hpppmi) at (\hxxxm, \hyyyi);
\coordinate (hpppmj) at (\hxxxm, \hyyyj);
\coordinate (hpppmk) at (\hxxxm, \hyyyk);
\coordinate (hpppml) at (\hxxxm, \hyyyl);
\coordinate (hpppmm) at (\hxxxm, \hyyym);
\coordinate (hpppmn) at (\hxxxm, \hyyyn);
\coordinate (hpppmo) at (\hxxxm, \hyyyo);
\coordinate (hpppmp) at (\hxxxm, \hyyyp);
\coordinate (hpppmq) at (\hxxxm, \hyyyq);
\coordinate (hpppmr) at (\hxxxm, \hyyyr);
\coordinate (hpppms) at (\hxxxm, \hyyys);
\coordinate (hpppmt) at (\hxxxm, \hyyyt);
\coordinate (hpppmu) at (\hxxxm, \hyyyu);
\coordinate (hpppmv) at (\hxxxm, \hyyyv);
\coordinate (hpppmw) at (\hxxxm, \hyyyw);
\coordinate (hpppmx) at (\hxxxm, \hyyyx);
\coordinate (hpppmy) at (\hxxxm, \hyyyy);
\coordinate (hpppmz) at (\hxxxm, \hyyyz);
\coordinate (hpppna) at (\hxxxn, \hyyya);
\coordinate (hpppnb) at (\hxxxn, \hyyyb);
\coordinate (hpppnc) at (\hxxxn, \hyyyc);
\coordinate (hpppnd) at (\hxxxn, \hyyyd);
\coordinate (hpppne) at (\hxxxn, \hyyye);
\coordinate (hpppnf) at (\hxxxn, \hyyyf);
\coordinate (hpppng) at (\hxxxn, \hyyyg);
\coordinate (hpppnh) at (\hxxxn, \hyyyh);
\coordinate (hpppni) at (\hxxxn, \hyyyi);
\coordinate (hpppnj) at (\hxxxn, \hyyyj);
\coordinate (hpppnk) at (\hxxxn, \hyyyk);
\coordinate (hpppnl) at (\hxxxn, \hyyyl);
\coordinate (hpppnm) at (\hxxxn, \hyyym);
\coordinate (hpppnn) at (\hxxxn, \hyyyn);
\coordinate (hpppno) at (\hxxxn, \hyyyo);
\coordinate (hpppnp) at (\hxxxn, \hyyyp);
\coordinate (hpppnq) at (\hxxxn, \hyyyq);
\coordinate (hpppnr) at (\hxxxn, \hyyyr);
\coordinate (hpppns) at (\hxxxn, \hyyys);
\coordinate (hpppnt) at (\hxxxn, \hyyyt);
\coordinate (hpppnu) at (\hxxxn, \hyyyu);
\coordinate (hpppnv) at (\hxxxn, \hyyyv);
\coordinate (hpppnw) at (\hxxxn, \hyyyw);
\coordinate (hpppnx) at (\hxxxn, \hyyyx);
\coordinate (hpppny) at (\hxxxn, \hyyyy);
\coordinate (hpppnz) at (\hxxxn, \hyyyz);
\coordinate (hpppoa) at (\hxxxo, \hyyya);
\coordinate (hpppob) at (\hxxxo, \hyyyb);
\coordinate (hpppoc) at (\hxxxo, \hyyyc);
\coordinate (hpppod) at (\hxxxo, \hyyyd);
\coordinate (hpppoe) at (\hxxxo, \hyyye);
\coordinate (hpppof) at (\hxxxo, \hyyyf);
\coordinate (hpppog) at (\hxxxo, \hyyyg);
\coordinate (hpppoh) at (\hxxxo, \hyyyh);
\coordinate (hpppoi) at (\hxxxo, \hyyyi);
\coordinate (hpppoj) at (\hxxxo, \hyyyj);
\coordinate (hpppok) at (\hxxxo, \hyyyk);
\coordinate (hpppol) at (\hxxxo, \hyyyl);
\coordinate (hpppom) at (\hxxxo, \hyyym);
\coordinate (hpppon) at (\hxxxo, \hyyyn);
\coordinate (hpppoo) at (\hxxxo, \hyyyo);
\coordinate (hpppop) at (\hxxxo, \hyyyp);
\coordinate (hpppoq) at (\hxxxo, \hyyyq);
\coordinate (hpppor) at (\hxxxo, \hyyyr);
\coordinate (hpppos) at (\hxxxo, \hyyys);
\coordinate (hpppot) at (\hxxxo, \hyyyt);
\coordinate (hpppou) at (\hxxxo, \hyyyu);
\coordinate (hpppov) at (\hxxxo, \hyyyv);
\coordinate (hpppow) at (\hxxxo, \hyyyw);
\coordinate (hpppox) at (\hxxxo, \hyyyx);
\coordinate (hpppoy) at (\hxxxo, \hyyyy);
\coordinate (hpppoz) at (\hxxxo, \hyyyz);
\coordinate (hppppa) at (\hxxxp, \hyyya);
\coordinate (hppppb) at (\hxxxp, \hyyyb);
\coordinate (hppppc) at (\hxxxp, \hyyyc);
\coordinate (hppppd) at (\hxxxp, \hyyyd);
\coordinate (hppppe) at (\hxxxp, \hyyye);
\coordinate (hppppf) at (\hxxxp, \hyyyf);
\coordinate (hppppg) at (\hxxxp, \hyyyg);
\coordinate (hpppph) at (\hxxxp, \hyyyh);
\coordinate (hppppi) at (\hxxxp, \hyyyi);
\coordinate (hppppj) at (\hxxxp, \hyyyj);
\coordinate (hppppk) at (\hxxxp, \hyyyk);
\coordinate (hppppl) at (\hxxxp, \hyyyl);
\coordinate (hppppm) at (\hxxxp, \hyyym);
\coordinate (hppppn) at (\hxxxp, \hyyyn);
\coordinate (hppppo) at (\hxxxp, \hyyyo);
\coordinate (hppppp) at (\hxxxp, \hyyyp);
\coordinate (hppppq) at (\hxxxp, \hyyyq);
\coordinate (hppppr) at (\hxxxp, \hyyyr);
\coordinate (hpppps) at (\hxxxp, \hyyys);
\coordinate (hppppt) at (\hxxxp, \hyyyt);
\coordinate (hppppu) at (\hxxxp, \hyyyu);
\coordinate (hppppv) at (\hxxxp, \hyyyv);
\coordinate (hppppw) at (\hxxxp, \hyyyw);
\coordinate (hppppx) at (\hxxxp, \hyyyx);
\coordinate (hppppy) at (\hxxxp, \hyyyy);
\coordinate (hppppz) at (\hxxxp, \hyyyz);
\coordinate (hpppqa) at (\hxxxq, \hyyya);
\coordinate (hpppqb) at (\hxxxq, \hyyyb);
\coordinate (hpppqc) at (\hxxxq, \hyyyc);
\coordinate (hpppqd) at (\hxxxq, \hyyyd);
\coordinate (hpppqe) at (\hxxxq, \hyyye);
\coordinate (hpppqf) at (\hxxxq, \hyyyf);
\coordinate (hpppqg) at (\hxxxq, \hyyyg);
\coordinate (hpppqh) at (\hxxxq, \hyyyh);
\coordinate (hpppqi) at (\hxxxq, \hyyyi);
\coordinate (hpppqj) at (\hxxxq, \hyyyj);
\coordinate (hpppqk) at (\hxxxq, \hyyyk);
\coordinate (hpppql) at (\hxxxq, \hyyyl);
\coordinate (hpppqm) at (\hxxxq, \hyyym);
\coordinate (hpppqn) at (\hxxxq, \hyyyn);
\coordinate (hpppqo) at (\hxxxq, \hyyyo);
\coordinate (hpppqp) at (\hxxxq, \hyyyp);
\coordinate (hpppqq) at (\hxxxq, \hyyyq);
\coordinate (hpppqr) at (\hxxxq, \hyyyr);
\coordinate (hpppqs) at (\hxxxq, \hyyys);
\coordinate (hpppqt) at (\hxxxq, \hyyyt);
\coordinate (hpppqu) at (\hxxxq, \hyyyu);
\coordinate (hpppqv) at (\hxxxq, \hyyyv);
\coordinate (hpppqw) at (\hxxxq, \hyyyw);
\coordinate (hpppqx) at (\hxxxq, \hyyyx);
\coordinate (hpppqy) at (\hxxxq, \hyyyy);
\coordinate (hpppqz) at (\hxxxq, \hyyyz);
\coordinate (hpppra) at (\hxxxr, \hyyya);
\coordinate (hppprb) at (\hxxxr, \hyyyb);
\coordinate (hppprc) at (\hxxxr, \hyyyc);
\coordinate (hppprd) at (\hxxxr, \hyyyd);
\coordinate (hpppre) at (\hxxxr, \hyyye);
\coordinate (hppprf) at (\hxxxr, \hyyyf);
\coordinate (hppprg) at (\hxxxr, \hyyyg);
\coordinate (hppprh) at (\hxxxr, \hyyyh);
\coordinate (hpppri) at (\hxxxr, \hyyyi);
\coordinate (hppprj) at (\hxxxr, \hyyyj);
\coordinate (hppprk) at (\hxxxr, \hyyyk);
\coordinate (hppprl) at (\hxxxr, \hyyyl);
\coordinate (hppprm) at (\hxxxr, \hyyym);
\coordinate (hppprn) at (\hxxxr, \hyyyn);
\coordinate (hpppro) at (\hxxxr, \hyyyo);
\coordinate (hppprp) at (\hxxxr, \hyyyp);
\coordinate (hppprq) at (\hxxxr, \hyyyq);
\coordinate (hppprr) at (\hxxxr, \hyyyr);
\coordinate (hppprs) at (\hxxxr, \hyyys);
\coordinate (hppprt) at (\hxxxr, \hyyyt);
\coordinate (hpppru) at (\hxxxr, \hyyyu);
\coordinate (hppprv) at (\hxxxr, \hyyyv);
\coordinate (hppprw) at (\hxxxr, \hyyyw);
\coordinate (hppprx) at (\hxxxr, \hyyyx);
\coordinate (hpppry) at (\hxxxr, \hyyyy);
\coordinate (hppprz) at (\hxxxr, \hyyyz);
\coordinate (hpppsa) at (\hxxxs, \hyyya);
\coordinate (hpppsb) at (\hxxxs, \hyyyb);
\coordinate (hpppsc) at (\hxxxs, \hyyyc);
\coordinate (hpppsd) at (\hxxxs, \hyyyd);
\coordinate (hpppse) at (\hxxxs, \hyyye);
\coordinate (hpppsf) at (\hxxxs, \hyyyf);
\coordinate (hpppsg) at (\hxxxs, \hyyyg);
\coordinate (hpppsh) at (\hxxxs, \hyyyh);
\coordinate (hpppsi) at (\hxxxs, \hyyyi);
\coordinate (hpppsj) at (\hxxxs, \hyyyj);
\coordinate (hpppsk) at (\hxxxs, \hyyyk);
\coordinate (hpppsl) at (\hxxxs, \hyyyl);
\coordinate (hpppsm) at (\hxxxs, \hyyym);
\coordinate (hpppsn) at (\hxxxs, \hyyyn);
\coordinate (hpppso) at (\hxxxs, \hyyyo);
\coordinate (hpppsp) at (\hxxxs, \hyyyp);
\coordinate (hpppsq) at (\hxxxs, \hyyyq);
\coordinate (hpppsr) at (\hxxxs, \hyyyr);
\coordinate (hpppss) at (\hxxxs, \hyyys);
\coordinate (hpppst) at (\hxxxs, \hyyyt);
\coordinate (hpppsu) at (\hxxxs, \hyyyu);
\coordinate (hpppsv) at (\hxxxs, \hyyyv);
\coordinate (hpppsw) at (\hxxxs, \hyyyw);
\coordinate (hpppsx) at (\hxxxs, \hyyyx);
\coordinate (hpppsy) at (\hxxxs, \hyyyy);
\coordinate (hpppsz) at (\hxxxs, \hyyyz);
\coordinate (hpppta) at (\hxxxt, \hyyya);
\coordinate (hppptb) at (\hxxxt, \hyyyb);
\coordinate (hppptc) at (\hxxxt, \hyyyc);
\coordinate (hppptd) at (\hxxxt, \hyyyd);
\coordinate (hpppte) at (\hxxxt, \hyyye);
\coordinate (hppptf) at (\hxxxt, \hyyyf);
\coordinate (hppptg) at (\hxxxt, \hyyyg);
\coordinate (hpppth) at (\hxxxt, \hyyyh);
\coordinate (hpppti) at (\hxxxt, \hyyyi);
\coordinate (hppptj) at (\hxxxt, \hyyyj);
\coordinate (hppptk) at (\hxxxt, \hyyyk);
\coordinate (hppptl) at (\hxxxt, \hyyyl);
\coordinate (hppptm) at (\hxxxt, \hyyym);
\coordinate (hppptn) at (\hxxxt, \hyyyn);
\coordinate (hpppto) at (\hxxxt, \hyyyo);
\coordinate (hppptp) at (\hxxxt, \hyyyp);
\coordinate (hppptq) at (\hxxxt, \hyyyq);
\coordinate (hppptr) at (\hxxxt, \hyyyr);
\coordinate (hpppts) at (\hxxxt, \hyyys);
\coordinate (hppptt) at (\hxxxt, \hyyyt);
\coordinate (hppptu) at (\hxxxt, \hyyyu);
\coordinate (hppptv) at (\hxxxt, \hyyyv);
\coordinate (hppptw) at (\hxxxt, \hyyyw);
\coordinate (hppptx) at (\hxxxt, \hyyyx);
\coordinate (hpppty) at (\hxxxt, \hyyyy);
\coordinate (hppptz) at (\hxxxt, \hyyyz);
\coordinate (hpppua) at (\hxxxu, \hyyya);
\coordinate (hpppub) at (\hxxxu, \hyyyb);
\coordinate (hpppuc) at (\hxxxu, \hyyyc);
\coordinate (hpppud) at (\hxxxu, \hyyyd);
\coordinate (hpppue) at (\hxxxu, \hyyye);
\coordinate (hpppuf) at (\hxxxu, \hyyyf);
\coordinate (hpppug) at (\hxxxu, \hyyyg);
\coordinate (hpppuh) at (\hxxxu, \hyyyh);
\coordinate (hpppui) at (\hxxxu, \hyyyi);
\coordinate (hpppuj) at (\hxxxu, \hyyyj);
\coordinate (hpppuk) at (\hxxxu, \hyyyk);
\coordinate (hpppul) at (\hxxxu, \hyyyl);
\coordinate (hpppum) at (\hxxxu, \hyyym);
\coordinate (hpppun) at (\hxxxu, \hyyyn);
\coordinate (hpppuo) at (\hxxxu, \hyyyo);
\coordinate (hpppup) at (\hxxxu, \hyyyp);
\coordinate (hpppuq) at (\hxxxu, \hyyyq);
\coordinate (hpppur) at (\hxxxu, \hyyyr);
\coordinate (hpppus) at (\hxxxu, \hyyys);
\coordinate (hppput) at (\hxxxu, \hyyyt);
\coordinate (hpppuu) at (\hxxxu, \hyyyu);
\coordinate (hpppuv) at (\hxxxu, \hyyyv);
\coordinate (hpppuw) at (\hxxxu, \hyyyw);
\coordinate (hpppux) at (\hxxxu, \hyyyx);
\coordinate (hpppuy) at (\hxxxu, \hyyyy);
\coordinate (hpppuz) at (\hxxxu, \hyyyz);
\coordinate (hpppva) at (\hxxxv, \hyyya);
\coordinate (hpppvb) at (\hxxxv, \hyyyb);
\coordinate (hpppvc) at (\hxxxv, \hyyyc);
\coordinate (hpppvd) at (\hxxxv, \hyyyd);
\coordinate (hpppve) at (\hxxxv, \hyyye);
\coordinate (hpppvf) at (\hxxxv, \hyyyf);
\coordinate (hpppvg) at (\hxxxv, \hyyyg);
\coordinate (hpppvh) at (\hxxxv, \hyyyh);
\coordinate (hpppvi) at (\hxxxv, \hyyyi);
\coordinate (hpppvj) at (\hxxxv, \hyyyj);
\coordinate (hpppvk) at (\hxxxv, \hyyyk);
\coordinate (hpppvl) at (\hxxxv, \hyyyl);
\coordinate (hpppvm) at (\hxxxv, \hyyym);
\coordinate (hpppvn) at (\hxxxv, \hyyyn);
\coordinate (hpppvo) at (\hxxxv, \hyyyo);
\coordinate (hpppvp) at (\hxxxv, \hyyyp);
\coordinate (hpppvq) at (\hxxxv, \hyyyq);
\coordinate (hpppvr) at (\hxxxv, \hyyyr);
\coordinate (hpppvs) at (\hxxxv, \hyyys);
\coordinate (hpppvt) at (\hxxxv, \hyyyt);
\coordinate (hpppvu) at (\hxxxv, \hyyyu);
\coordinate (hpppvv) at (\hxxxv, \hyyyv);
\coordinate (hpppvw) at (\hxxxv, \hyyyw);
\coordinate (hpppvx) at (\hxxxv, \hyyyx);
\coordinate (hpppvy) at (\hxxxv, \hyyyy);
\coordinate (hpppvz) at (\hxxxv, \hyyyz);
\coordinate (hpppwa) at (\hxxxw, \hyyya);
\coordinate (hpppwb) at (\hxxxw, \hyyyb);
\coordinate (hpppwc) at (\hxxxw, \hyyyc);
\coordinate (hpppwd) at (\hxxxw, \hyyyd);
\coordinate (hpppwe) at (\hxxxw, \hyyye);
\coordinate (hpppwf) at (\hxxxw, \hyyyf);
\coordinate (hpppwg) at (\hxxxw, \hyyyg);
\coordinate (hpppwh) at (\hxxxw, \hyyyh);
\coordinate (hpppwi) at (\hxxxw, \hyyyi);
\coordinate (hpppwj) at (\hxxxw, \hyyyj);
\coordinate (hpppwk) at (\hxxxw, \hyyyk);
\coordinate (hpppwl) at (\hxxxw, \hyyyl);
\coordinate (hpppwm) at (\hxxxw, \hyyym);
\coordinate (hpppwn) at (\hxxxw, \hyyyn);
\coordinate (hpppwo) at (\hxxxw, \hyyyo);
\coordinate (hpppwp) at (\hxxxw, \hyyyp);
\coordinate (hpppwq) at (\hxxxw, \hyyyq);
\coordinate (hpppwr) at (\hxxxw, \hyyyr);
\coordinate (hpppws) at (\hxxxw, \hyyys);
\coordinate (hpppwt) at (\hxxxw, \hyyyt);
\coordinate (hpppwu) at (\hxxxw, \hyyyu);
\coordinate (hpppwv) at (\hxxxw, \hyyyv);
\coordinate (hpppww) at (\hxxxw, \hyyyw);
\coordinate (hpppwx) at (\hxxxw, \hyyyx);
\coordinate (hpppwy) at (\hxxxw, \hyyyy);
\coordinate (hpppwz) at (\hxxxw, \hyyyz);
\coordinate (hpppxa) at (\hxxxx, \hyyya);
\coordinate (hpppxb) at (\hxxxx, \hyyyb);
\coordinate (hpppxc) at (\hxxxx, \hyyyc);
\coordinate (hpppxd) at (\hxxxx, \hyyyd);
\coordinate (hpppxe) at (\hxxxx, \hyyye);
\coordinate (hpppxf) at (\hxxxx, \hyyyf);
\coordinate (hpppxg) at (\hxxxx, \hyyyg);
\coordinate (hpppxh) at (\hxxxx, \hyyyh);
\coordinate (hpppxi) at (\hxxxx, \hyyyi);
\coordinate (hpppxj) at (\hxxxx, \hyyyj);
\coordinate (hpppxk) at (\hxxxx, \hyyyk);
\coordinate (hpppxl) at (\hxxxx, \hyyyl);
\coordinate (hpppxm) at (\hxxxx, \hyyym);
\coordinate (hpppxn) at (\hxxxx, \hyyyn);
\coordinate (hpppxo) at (\hxxxx, \hyyyo);
\coordinate (hpppxp) at (\hxxxx, \hyyyp);
\coordinate (hpppxq) at (\hxxxx, \hyyyq);
\coordinate (hpppxr) at (\hxxxx, \hyyyr);
\coordinate (hpppxs) at (\hxxxx, \hyyys);
\coordinate (hpppxt) at (\hxxxx, \hyyyt);
\coordinate (hpppxu) at (\hxxxx, \hyyyu);
\coordinate (hpppxv) at (\hxxxx, \hyyyv);
\coordinate (hpppxw) at (\hxxxx, \hyyyw);
\coordinate (hpppxx) at (\hxxxx, \hyyyx);
\coordinate (hpppxy) at (\hxxxx, \hyyyy);
\coordinate (hpppxz) at (\hxxxx, \hyyyz);
\coordinate (hpppya) at (\hxxxy, \hyyya);
\coordinate (hpppyb) at (\hxxxy, \hyyyb);
\coordinate (hpppyc) at (\hxxxy, \hyyyc);
\coordinate (hpppyd) at (\hxxxy, \hyyyd);
\coordinate (hpppye) at (\hxxxy, \hyyye);
\coordinate (hpppyf) at (\hxxxy, \hyyyf);
\coordinate (hpppyg) at (\hxxxy, \hyyyg);
\coordinate (hpppyh) at (\hxxxy, \hyyyh);
\coordinate (hpppyi) at (\hxxxy, \hyyyi);
\coordinate (hpppyj) at (\hxxxy, \hyyyj);
\coordinate (hpppyk) at (\hxxxy, \hyyyk);
\coordinate (hpppyl) at (\hxxxy, \hyyyl);
\coordinate (hpppym) at (\hxxxy, \hyyym);
\coordinate (hpppyn) at (\hxxxy, \hyyyn);
\coordinate (hpppyo) at (\hxxxy, \hyyyo);
\coordinate (hpppyp) at (\hxxxy, \hyyyp);
\coordinate (hpppyq) at (\hxxxy, \hyyyq);
\coordinate (hpppyr) at (\hxxxy, \hyyyr);
\coordinate (hpppys) at (\hxxxy, \hyyys);
\coordinate (hpppyt) at (\hxxxy, \hyyyt);
\coordinate (hpppyu) at (\hxxxy, \hyyyu);
\coordinate (hpppyv) at (\hxxxy, \hyyyv);
\coordinate (hpppyw) at (\hxxxy, \hyyyw);
\coordinate (hpppyx) at (\hxxxy, \hyyyx);
\coordinate (hpppyy) at (\hxxxy, \hyyyy);
\coordinate (hpppyz) at (\hxxxy, \hyyyz);
\coordinate (hpppza) at (\hxxxz, \hyyya);
\coordinate (hpppzb) at (\hxxxz, \hyyyb);
\coordinate (hpppzc) at (\hxxxz, \hyyyc);
\coordinate (hpppzd) at (\hxxxz, \hyyyd);
\coordinate (hpppze) at (\hxxxz, \hyyye);
\coordinate (hpppzf) at (\hxxxz, \hyyyf);
\coordinate (hpppzg) at (\hxxxz, \hyyyg);
\coordinate (hpppzh) at (\hxxxz, \hyyyh);
\coordinate (hpppzi) at (\hxxxz, \hyyyi);
\coordinate (hpppzj) at (\hxxxz, \hyyyj);
\coordinate (hpppzk) at (\hxxxz, \hyyyk);
\coordinate (hpppzl) at (\hxxxz, \hyyyl);
\coordinate (hpppzm) at (\hxxxz, \hyyym);
\coordinate (hpppzn) at (\hxxxz, \hyyyn);
\coordinate (hpppzo) at (\hxxxz, \hyyyo);
\coordinate (hpppzp) at (\hxxxz, \hyyyp);
\coordinate (hpppzq) at (\hxxxz, \hyyyq);
\coordinate (hpppzr) at (\hxxxz, \hyyyr);
\coordinate (hpppzs) at (\hxxxz, \hyyys);
\coordinate (hpppzt) at (\hxxxz, \hyyyt);
\coordinate (hpppzu) at (\hxxxz, \hyyyu);
\coordinate (hpppzv) at (\hxxxz, \hyyyv);
\coordinate (hpppzw) at (\hxxxz, \hyyyw);
\coordinate (hpppzx) at (\hxxxz, \hyyyx);
\coordinate (hpppzy) at (\hxxxz, \hyyyy);
\coordinate (hpppzz) at (\hxxxz, \hyyyz);

%\gangprintcoordinateat{(0,0)}{The last coordinate values: }{($(hpppzz)$)}; 


% Draw related part of the coordinate system with dashed helplines (centered at (hpppii)) with letters as background, which would help to determine all coordinates. 
%\coordinatebackground{h}{c}{d}{o};

% Step 2, draw key devices, their accessories, and take related coordinates of their pins, and may define more coordinates. 

% Draw the Opamp at the coordinate (hpppii) and name it as "swopamp".
\draw (hpppii) node [op amp, yscale=-1] (swopamp) {\ctikzflipy{Opamp}} ; 

% Its accessories and lables. 
\draw [-*](swopamp.down) -- ($(swopamp.down)+(0,1)$) node[right]{$V_+$}; 
\node at ($(swopamp.down)+(0.3,0.2)$) {7};  
\draw [-*](swopamp.up) -- ($(swopamp.up)+(0,-1)$) node[right]{$V_-$}; 
\node at ($(swopamp.up)+(0.3,-0.2)$) {4};

% Get the x- and y-components of the coordinates of the "+" and "-" pins. 
\getxyingivenunit{cm}{(swopamp.+)}{\swopampzx}{\swopampzy};
\getxyingivenunit{cm}{(swopamp.-)}{\swopampfx}{\swopampfy};

% Then define a few more coordinates, at least for keeping in mind.
\coordinate (plusshort) at ($(\hxxxg,\swopampzy)$);
%\fill  (plusshort) circle (2pt);  % May be commented later.
\coordinate (minusshort) at ($(\hxxxg,\swopampfy)$);
%\fill  (minusshort) circle (2pt); % May be commented later.
\coordinate (leftinter) at ($(\hxxxe,\swopampzy)$);
\fill  (leftinter) circle (2pt);

% Draw an "npn" at (hpppmi) and name it as "swQ".
\draw (hpppmi) node[npn](swQ){};

% Get the x- and y-components of the needed pins of it for later usage.
\getxyingivenunit{cm}{(swQ.C)}{\swQCx}{\swQCy};
\getxyingivenunit{cm}{(swQ.E)}{\swQEx}{\swQEy};

% Then define more coordinate(s).
\coordinate (Qcshort) at ($(\swQEx,\hyyyj)$);
%\fill  (Qcshort) circle (2pt); % May be commented later.
\coordinate (Qeshort) at ($(\swQEx,\hyyyf)$);
\fill  (Qeshort) circle (2pt) node [right] {$V_0$};

% Then the rectangle by the points (hpppef) -- (hpppej) -- (Qcshort) -- (Qeshort) forms a clear area for the key devices. 

% Connect the two devices.
\draw (swopamp.out) to [short, l=$I_B$, above] (swQ.B);

% Step 3, draw other little devices. For tidiness, better to give two units in length for each new device and align them up.

% For this specific circuit, let us attach the four bi-pole devices (maybe with their accessories) to each corner of the above mentioned rectangle area for the key devices, separately. 

\fill  (\hxxxe,\gyyyi) circle (2pt);
\draw  (\hxxxe,\gyyyi) -- (hppped) to [empty ZZener diode] (hpppef) -- (leftinter);
% The Latex system can not work properly when I put the following label into the above "[empty ZZener diode]" in the form of an additional "l= ..." option, then I have to employ the following "\node ..." command. Then the label should be aligned with the "ZZener" as much as possible even the coordinate system is modified later. Since the "\hyyyd" and "\hyyyf" micros are used to position the "ZZener", then better to use the center between them to locate the label, rather than using "\hyyye" directly. This idea is also applied in later "\node ..." commands. 
\node at ($(\hxxxe-1.3, \hyyyd*0.5+\hyyyf*0.5)$) {$V_Z = 5\textnormal{V}$};

\draw (hpppej) to [generic] (hpppel) -| (swQ.C);
\node at ($(\hxxxe-1.1, \hyyyj*0.5+\hyyyl*0.5)$) {$R_{1}=47k\Omega$};
\node [right] at (\swQCx,\hyyyj*0.5+\hyyyl*0.5) {$I_C \approx \beta I_B$};

\fill  (\bigswQCx, \hyyyd) circle (2pt);
\draw  (\bigswQCx, \hyyyd) -- (\swQEx, \hyyyd) to [generic] (\swQEx, \hyyyf) -- (swQ.E);
\node at ($(\swQEx+1.4, \hyyyd*0.5+\hyyyf*0.5)$) {$R_{E}=100k\Omega$};

% Draw the top area. 
%\draw  (\hxxxi-0.2,\hyyyn) --  (\hxxxi+0.2,\hyyyn) node [right] {$V_{cc}=15\textnormal{V}$} ;
%\draw [->] (hpppin) -- (hpppim) node [right] {$I$};  
%\draw  (hpppim) -- (hpppil);
%\fill  (hpppil) circle (2pt);

% Step 4, other shorts.
\draw  (swopamp.+)  to [short, l_=$I_+ \approx 0 $, above] (plusshort) -- (leftinter) -- (hpppej);

\draw  (swopamp.-)  to [short, l_=$I_- \approx 0 $, above] (minusshort) |- (Qeshort);

%Step 5, all the rest, especially labels. May also clean unnecessary staff, like the system background and dark points for showing newly defined coordinates previously. 
\draw [->] ($(\swQEx-0.4, \hyyyf - 0.4)$) -- node [left] {$I_E$} ($(\swQEx-0.4, \hyyyd + 0.4)$);

\draw [->] ($(\hxxxe+0.4, \hyyyl*0.5+\hyyyj*0.5 + 0.6)$) -- node [right] {$I_1$} ($(\hxxxe+0.4, \hyyyl*0.5+\hyyyj*0.5 - 0.6)$);

\pgfmathsetmacro{\secondopampzy}{\swopampzy}





















% The following is the second circuit, which uses "s" coordinate system, to be inserted into and connected the above circuit as well. It is modified based on Figure 6. 

\pgfmathsetmacro{\totalsxxx}{26}
\pgfmathsetmacro{\totalsyyy}{26}
\pgfmathsetmacro{\sxxxspacing}{1}
\pgfmathsetmacro{\syyyspacing}{1}
\pgfmathsetmacro{\sxxxa}{-3}
\pgfmathsetmacro{\syyya}{-2}

\pgfmathsetmacro{\sxxxb}{\sxxxa + \sxxxspacing + 0.0 }
\pgfmathsetmacro{\sxxxc}{\sxxxb + \sxxxspacing + 0.0 }
\pgfmathsetmacro{\sxxxd}{\sxxxc + \sxxxspacing + 0.0 }
\pgfmathsetmacro{\sxxxe}{\sxxxd + \sxxxspacing + 0.0 }
\pgfmathsetmacro{\sxxxf}{\sxxxe + \sxxxspacing + 0.0 }
\pgfmathsetmacro{\sxxxg}{\sxxxf + \sxxxspacing + 0.0 }
\pgfmathsetmacro{\sxxxh}{\sxxxg + \sxxxspacing + 0.0 }
\pgfmathsetmacro{\sxxxi}{\sxxxh + \sxxxspacing + 0.0 }
\pgfmathsetmacro{\sxxxj}{\sxxxi + \sxxxspacing + 0.0 }
\pgfmathsetmacro{\sxxxk}{\sxxxj + \sxxxspacing + 0.0 }
\pgfmathsetmacro{\sxxxl}{\sxxxk + \sxxxspacing + 0.0 }
\pgfmathsetmacro{\sxxxm}{\sxxxl + \sxxxspacing + 0.0 }
\pgfmathsetmacro{\sxxxn}{\sxxxm + \sxxxspacing + 0.0 }
\pgfmathsetmacro{\sxxxo}{\sxxxn + \sxxxspacing + 0.0 }
\pgfmathsetmacro{\sxxxp}{\sxxxo + \sxxxspacing + 0.0 }
\pgfmathsetmacro{\sxxxq}{\sxxxp + \sxxxspacing + 0.0 }
\pgfmathsetmacro{\sxxxr}{\sxxxq + \sxxxspacing + 0.0 }
\pgfmathsetmacro{\sxxxs}{\sxxxr + \sxxxspacing + 0.0 }
\pgfmathsetmacro{\sxxxt}{\sxxxs + \sxxxspacing + 0.0 }
\pgfmathsetmacro{\sxxxu}{\sxxxt + \sxxxspacing + 0.0 }
\pgfmathsetmacro{\sxxxv}{\sxxxu + \sxxxspacing + 0.0 }
\pgfmathsetmacro{\sxxxw}{\sxxxv + \sxxxspacing + 0.0 }
\pgfmathsetmacro{\sxxxx}{\sxxxw + \sxxxspacing + 0.0 }
\pgfmathsetmacro{\sxxxy}{\sxxxx + \sxxxspacing + 0.0 }
\pgfmathsetmacro{\sxxxz}{\sxxxy + \sxxxspacing + 0.0 }

\pgfmathsetmacro{\syyyb}{\syyya + \syyyspacing + 0.0 }
\pgfmathsetmacro{\syyyc}{\syyyb + \syyyspacing + 0.0 }
\pgfmathsetmacro{\syyyd}{\syyyc + \syyyspacing + 0.0 }
\pgfmathsetmacro{\syyye}{\syyyd + \syyyspacing + 0.0 }
\pgfmathsetmacro{\syyyf}{\syyye + \syyyspacing + 0.0 }
\pgfmathsetmacro{\syyyg}{\syyyf + \syyyspacing + 0.0 }
\pgfmathsetmacro{\syyyh}{\syyyg + \syyyspacing + 0.0 }
\pgfmathsetmacro{\syyyi}{\syyyh + \syyyspacing + 0.0 }
\pgfmathsetmacro{\syyyj}{\syyyi + \syyyspacing + 0.0 }
\pgfmathsetmacro{\syyyk}{\syyyj + \syyyspacing + 0.0 }
\pgfmathsetmacro{\syyyl}{\syyyk + \syyyspacing + 0.0 }
\pgfmathsetmacro{\syyym}{\syyyl + \syyyspacing + 0.0 }
\pgfmathsetmacro{\syyyn}{\syyym + \syyyspacing + 0.0 }
\pgfmathsetmacro{\syyyo}{\syyyn + \syyyspacing + 0.0 }
\pgfmathsetmacro{\syyyp}{\syyyo + \syyyspacing + 0.0 }
\pgfmathsetmacro{\syyyq}{\syyyp + \syyyspacing + 0.0 }
\pgfmathsetmacro{\syyyr}{\syyyq + \syyyspacing + 0.0 }
\pgfmathsetmacro{\syyys}{\syyyr + \syyyspacing + 0.0 }
\pgfmathsetmacro{\syyyt}{\syyys + \syyyspacing + 0.0 }
\pgfmathsetmacro{\syyyu}{\syyyt + \syyyspacing + 0.0 }
\pgfmathsetmacro{\syyyv}{\syyyu + \syyyspacing + 0.0 }
\pgfmathsetmacro{\syyyw}{\syyyv + \syyyspacing + 0.0 }
\pgfmathsetmacro{\syyyx}{\syyyw + \syyyspacing + 0.0 }
\pgfmathsetmacro{\syyyy}{\syyyx + \syyyspacing + 0.0 }
\pgfmathsetmacro{\syyyz}{\syyyy + \syyyspacing + 0.0 }

\coordinate (spppaa) at (\sxxxa, \syyya);
\coordinate (spppab) at (\sxxxa, \syyyb);
\coordinate (spppac) at (\sxxxa, \syyyc);
\coordinate (spppad) at (\sxxxa, \syyyd);
\coordinate (spppae) at (\sxxxa, \syyye);
\coordinate (spppaf) at (\sxxxa, \syyyf);
\coordinate (spppag) at (\sxxxa, \syyyg);
\coordinate (spppah) at (\sxxxa, \syyyh);
\coordinate (spppai) at (\sxxxa, \syyyi);
\coordinate (spppaj) at (\sxxxa, \syyyj);
\coordinate (spppak) at (\sxxxa, \syyyk);
\coordinate (spppal) at (\sxxxa, \syyyl);
\coordinate (spppam) at (\sxxxa, \syyym);
\coordinate (spppan) at (\sxxxa, \syyyn);
\coordinate (spppao) at (\sxxxa, \syyyo);
\coordinate (spppap) at (\sxxxa, \syyyp);
\coordinate (spppaq) at (\sxxxa, \syyyq);
\coordinate (spppar) at (\sxxxa, \syyyr);
\coordinate (spppas) at (\sxxxa, \syyys);
\coordinate (spppat) at (\sxxxa, \syyyt);
\coordinate (spppau) at (\sxxxa, \syyyu);
\coordinate (spppav) at (\sxxxa, \syyyv);
\coordinate (spppaw) at (\sxxxa, \syyyw);
\coordinate (spppax) at (\sxxxa, \syyyx);
\coordinate (spppay) at (\sxxxa, \syyyy);
\coordinate (spppaz) at (\sxxxa, \syyyz);
\coordinate (spppba) at (\sxxxb, \syyya);
\coordinate (spppbb) at (\sxxxb, \syyyb);
\coordinate (spppbc) at (\sxxxb, \syyyc);
\coordinate (spppbd) at (\sxxxb, \syyyd);
\coordinate (spppbe) at (\sxxxb, \syyye);
\coordinate (spppbf) at (\sxxxb, \syyyf);
\coordinate (spppbg) at (\sxxxb, \syyyg);
\coordinate (spppbh) at (\sxxxb, \syyyh);
\coordinate (spppbi) at (\sxxxb, \syyyi);
\coordinate (spppbj) at (\sxxxb, \syyyj);
\coordinate (spppbk) at (\sxxxb, \syyyk);
\coordinate (spppbl) at (\sxxxb, \syyyl);
\coordinate (spppbm) at (\sxxxb, \syyym);
\coordinate (spppbn) at (\sxxxb, \syyyn);
\coordinate (spppbo) at (\sxxxb, \syyyo);
\coordinate (spppbp) at (\sxxxb, \syyyp);
\coordinate (spppbq) at (\sxxxb, \syyyq);
\coordinate (spppbr) at (\sxxxb, \syyyr);
\coordinate (spppbs) at (\sxxxb, \syyys);
\coordinate (spppbt) at (\sxxxb, \syyyt);
\coordinate (spppbu) at (\sxxxb, \syyyu);
\coordinate (spppbv) at (\sxxxb, \syyyv);
\coordinate (spppbw) at (\sxxxb, \syyyw);
\coordinate (spppbx) at (\sxxxb, \syyyx);
\coordinate (spppby) at (\sxxxb, \syyyy);
\coordinate (spppbz) at (\sxxxb, \syyyz);
\coordinate (spppca) at (\sxxxc, \syyya);
\coordinate (spppcb) at (\sxxxc, \syyyb);
\coordinate (spppcc) at (\sxxxc, \syyyc);
\coordinate (spppcd) at (\sxxxc, \syyyd);
\coordinate (spppce) at (\sxxxc, \syyye);
\coordinate (spppcf) at (\sxxxc, \syyyf);
\coordinate (spppcg) at (\sxxxc, \syyyg);
\coordinate (spppch) at (\sxxxc, \syyyh);
\coordinate (spppci) at (\sxxxc, \syyyi);
\coordinate (spppcj) at (\sxxxc, \syyyj);
\coordinate (spppck) at (\sxxxc, \syyyk);
\coordinate (spppcl) at (\sxxxc, \syyyl);
\coordinate (spppcm) at (\sxxxc, \syyym);
\coordinate (spppcn) at (\sxxxc, \syyyn);
\coordinate (spppco) at (\sxxxc, \syyyo);
\coordinate (spppcp) at (\sxxxc, \syyyp);
\coordinate (spppcq) at (\sxxxc, \syyyq);
\coordinate (spppcr) at (\sxxxc, \syyyr);
\coordinate (spppcs) at (\sxxxc, \syyys);
\coordinate (spppct) at (\sxxxc, \syyyt);
\coordinate (spppcu) at (\sxxxc, \syyyu);
\coordinate (spppcv) at (\sxxxc, \syyyv);
\coordinate (spppcw) at (\sxxxc, \syyyw);
\coordinate (spppcx) at (\sxxxc, \syyyx);
\coordinate (spppcy) at (\sxxxc, \syyyy);
\coordinate (spppcz) at (\sxxxc, \syyyz);
\coordinate (spppda) at (\sxxxd, \syyya);
\coordinate (spppdb) at (\sxxxd, \syyyb);
\coordinate (spppdc) at (\sxxxd, \syyyc);
\coordinate (spppdd) at (\sxxxd, \syyyd);
\coordinate (spppde) at (\sxxxd, \syyye);
\coordinate (spppdf) at (\sxxxd, \syyyf);
\coordinate (spppdg) at (\sxxxd, \syyyg);
\coordinate (spppdh) at (\sxxxd, \syyyh);
\coordinate (spppdi) at (\sxxxd, \syyyi);
\coordinate (spppdj) at (\sxxxd, \syyyj);
\coordinate (spppdk) at (\sxxxd, \syyyk);
\coordinate (spppdl) at (\sxxxd, \syyyl);
\coordinate (spppdm) at (\sxxxd, \syyym);
\coordinate (spppdn) at (\sxxxd, \syyyn);
\coordinate (spppdo) at (\sxxxd, \syyyo);
\coordinate (spppdp) at (\sxxxd, \syyyp);
\coordinate (spppdq) at (\sxxxd, \syyyq);
\coordinate (spppdr) at (\sxxxd, \syyyr);
\coordinate (spppds) at (\sxxxd, \syyys);
\coordinate (spppdt) at (\sxxxd, \syyyt);
\coordinate (spppdu) at (\sxxxd, \syyyu);
\coordinate (spppdv) at (\sxxxd, \syyyv);
\coordinate (spppdw) at (\sxxxd, \syyyw);
\coordinate (spppdx) at (\sxxxd, \syyyx);
\coordinate (spppdy) at (\sxxxd, \syyyy);
\coordinate (spppdz) at (\sxxxd, \syyyz);
\coordinate (spppea) at (\sxxxe, \syyya);
\coordinate (spppeb) at (\sxxxe, \syyyb);
\coordinate (spppec) at (\sxxxe, \syyyc);
\coordinate (sppped) at (\sxxxe, \syyyd);
\coordinate (spppee) at (\sxxxe, \syyye);
\coordinate (spppef) at (\sxxxe, \syyyf);
\coordinate (spppeg) at (\sxxxe, \syyyg);
\coordinate (spppeh) at (\sxxxe, \syyyh);
\coordinate (spppei) at (\sxxxe, \syyyi);
\coordinate (spppej) at (\sxxxe, \syyyj);
\coordinate (spppek) at (\sxxxe, \syyyk);
\coordinate (spppel) at (\sxxxe, \syyyl);
\coordinate (spppem) at (\sxxxe, \syyym);
\coordinate (spppen) at (\sxxxe, \syyyn);
\coordinate (spppeo) at (\sxxxe, \syyyo);
\coordinate (spppep) at (\sxxxe, \syyyp);
\coordinate (spppeq) at (\sxxxe, \syyyq);
\coordinate (sppper) at (\sxxxe, \syyyr);
\coordinate (spppes) at (\sxxxe, \syyys);
\coordinate (spppet) at (\sxxxe, \syyyt);
\coordinate (spppeu) at (\sxxxe, \syyyu);
\coordinate (spppev) at (\sxxxe, \syyyv);
\coordinate (spppew) at (\sxxxe, \syyyw);
\coordinate (spppex) at (\sxxxe, \syyyx);
\coordinate (spppey) at (\sxxxe, \syyyy);
\coordinate (spppez) at (\sxxxe, \syyyz);
\coordinate (spppfa) at (\sxxxf, \syyya);
\coordinate (spppfb) at (\sxxxf, \syyyb);
\coordinate (spppfc) at (\sxxxf, \syyyc);
\coordinate (spppfd) at (\sxxxf, \syyyd);
\coordinate (spppfe) at (\sxxxf, \syyye);
\coordinate (spppff) at (\sxxxf, \syyyf);
\coordinate (spppfg) at (\sxxxf, \syyyg);
\coordinate (spppfh) at (\sxxxf, \syyyh);
\coordinate (spppfi) at (\sxxxf, \syyyi);
\coordinate (spppfj) at (\sxxxf, \syyyj);
\coordinate (spppfk) at (\sxxxf, \syyyk);
\coordinate (spppfl) at (\sxxxf, \syyyl);
\coordinate (spppfm) at (\sxxxf, \syyym);
\coordinate (spppfn) at (\sxxxf, \syyyn);
\coordinate (spppfo) at (\sxxxf, \syyyo);
\coordinate (spppfp) at (\sxxxf, \syyyp);
\coordinate (spppfq) at (\sxxxf, \syyyq);
\coordinate (spppfr) at (\sxxxf, \syyyr);
\coordinate (spppfs) at (\sxxxf, \syyys);
\coordinate (spppft) at (\sxxxf, \syyyt);
\coordinate (spppfu) at (\sxxxf, \syyyu);
\coordinate (spppfv) at (\sxxxf, \syyyv);
\coordinate (spppfw) at (\sxxxf, \syyyw);
\coordinate (spppfx) at (\sxxxf, \syyyx);
\coordinate (spppfy) at (\sxxxf, \syyyy);
\coordinate (spppfz) at (\sxxxf, \syyyz);
\coordinate (spppga) at (\sxxxg, \syyya);
\coordinate (spppgb) at (\sxxxg, \syyyb);
\coordinate (spppgc) at (\sxxxg, \syyyc);
\coordinate (spppgd) at (\sxxxg, \syyyd);
\coordinate (spppge) at (\sxxxg, \syyye);
\coordinate (spppgf) at (\sxxxg, \syyyf);
\coordinate (spppgg) at (\sxxxg, \syyyg);
\coordinate (spppgh) at (\sxxxg, \syyyh);
\coordinate (spppgi) at (\sxxxg, \syyyi);
\coordinate (spppgj) at (\sxxxg, \syyyj);
\coordinate (spppgk) at (\sxxxg, \syyyk);
\coordinate (spppgl) at (\sxxxg, \syyyl);
\coordinate (spppgm) at (\sxxxg, \syyym);
\coordinate (spppgn) at (\sxxxg, \syyyn);
\coordinate (spppgo) at (\sxxxg, \syyyo);
\coordinate (spppgp) at (\sxxxg, \syyyp);
\coordinate (spppgq) at (\sxxxg, \syyyq);
\coordinate (spppgr) at (\sxxxg, \syyyr);
\coordinate (spppgs) at (\sxxxg, \syyys);
\coordinate (spppgt) at (\sxxxg, \syyyt);
\coordinate (spppgu) at (\sxxxg, \syyyu);
\coordinate (spppgv) at (\sxxxg, \syyyv);
\coordinate (spppgw) at (\sxxxg, \syyyw);
\coordinate (spppgx) at (\sxxxg, \syyyx);
\coordinate (spppgy) at (\sxxxg, \syyyy);
\coordinate (spppgz) at (\sxxxg, \syyyz);
\coordinate (spppha) at (\sxxxh, \syyya);
\coordinate (sppphb) at (\sxxxh, \syyyb);
\coordinate (sppphc) at (\sxxxh, \syyyc);
\coordinate (sppphd) at (\sxxxh, \syyyd);
\coordinate (sppphe) at (\sxxxh, \syyye);
\coordinate (sppphf) at (\sxxxh, \syyyf);
\coordinate (sppphg) at (\sxxxh, \syyyg);
\coordinate (sppphh) at (\sxxxh, \syyyh);
\coordinate (sppphi) at (\sxxxh, \syyyi);
\coordinate (sppphj) at (\sxxxh, \syyyj);
\coordinate (sppphk) at (\sxxxh, \syyyk);
\coordinate (sppphl) at (\sxxxh, \syyyl);
\coordinate (sppphm) at (\sxxxh, \syyym);
\coordinate (sppphn) at (\sxxxh, \syyyn);
\coordinate (spppho) at (\sxxxh, \syyyo);
\coordinate (sppphp) at (\sxxxh, \syyyp);
\coordinate (sppphq) at (\sxxxh, \syyyq);
\coordinate (sppphr) at (\sxxxh, \syyyr);
\coordinate (sppphs) at (\sxxxh, \syyys);
\coordinate (spppht) at (\sxxxh, \syyyt);
\coordinate (sppphu) at (\sxxxh, \syyyu);
\coordinate (sppphv) at (\sxxxh, \syyyv);
\coordinate (sppphw) at (\sxxxh, \syyyw);
\coordinate (sppphx) at (\sxxxh, \syyyx);
\coordinate (sppphy) at (\sxxxh, \syyyy);
\coordinate (sppphz) at (\sxxxh, \syyyz);
\coordinate (spppia) at (\sxxxi, \syyya);
\coordinate (spppib) at (\sxxxi, \syyyb);
\coordinate (spppic) at (\sxxxi, \syyyc);
\coordinate (spppid) at (\sxxxi, \syyyd);
\coordinate (spppie) at (\sxxxi, \syyye);
\coordinate (spppif) at (\sxxxi, \syyyf);
\coordinate (spppig) at (\sxxxi, \syyyg);
\coordinate (spppih) at (\sxxxi, \syyyh);
\coordinate (spppii) at (\sxxxi, \syyyi);
\coordinate (spppij) at (\sxxxi, \syyyj);
\coordinate (spppik) at (\sxxxi, \syyyk);
\coordinate (spppil) at (\sxxxi, \syyyl);
\coordinate (spppim) at (\sxxxi, \syyym);
\coordinate (spppin) at (\sxxxi, \syyyn);
\coordinate (spppio) at (\sxxxi, \syyyo);
\coordinate (spppip) at (\sxxxi, \syyyp);
\coordinate (spppiq) at (\sxxxi, \syyyq);
\coordinate (spppir) at (\sxxxi, \syyyr);
\coordinate (spppis) at (\sxxxi, \syyys);
\coordinate (spppit) at (\sxxxi, \syyyt);
\coordinate (spppiu) at (\sxxxi, \syyyu);
\coordinate (spppiv) at (\sxxxi, \syyyv);
\coordinate (spppiw) at (\sxxxi, \syyyw);
\coordinate (spppix) at (\sxxxi, \syyyx);
\coordinate (spppiy) at (\sxxxi, \syyyy);
\coordinate (spppiz) at (\sxxxi, \syyyz);
\coordinate (spppja) at (\sxxxj, \syyya);
\coordinate (spppjb) at (\sxxxj, \syyyb);
\coordinate (spppjc) at (\sxxxj, \syyyc);
\coordinate (spppjd) at (\sxxxj, \syyyd);
\coordinate (spppje) at (\sxxxj, \syyye);
\coordinate (spppjf) at (\sxxxj, \syyyf);
\coordinate (spppjg) at (\sxxxj, \syyyg);
\coordinate (spppjh) at (\sxxxj, \syyyh);
\coordinate (spppji) at (\sxxxj, \syyyi);
\coordinate (spppjj) at (\sxxxj, \syyyj);
\coordinate (spppjk) at (\sxxxj, \syyyk);
\coordinate (spppjl) at (\sxxxj, \syyyl);
\coordinate (spppjm) at (\sxxxj, \syyym);
\coordinate (spppjn) at (\sxxxj, \syyyn);
\coordinate (spppjo) at (\sxxxj, \syyyo);
\coordinate (spppjp) at (\sxxxj, \syyyp);
\coordinate (spppjq) at (\sxxxj, \syyyq);
\coordinate (spppjr) at (\sxxxj, \syyyr);
\coordinate (spppjs) at (\sxxxj, \syyys);
\coordinate (spppjt) at (\sxxxj, \syyyt);
\coordinate (spppju) at (\sxxxj, \syyyu);
\coordinate (spppjv) at (\sxxxj, \syyyv);
\coordinate (spppjw) at (\sxxxj, \syyyw);
\coordinate (spppjx) at (\sxxxj, \syyyx);
\coordinate (spppjy) at (\sxxxj, \syyyy);
\coordinate (spppjz) at (\sxxxj, \syyyz);
\coordinate (spppka) at (\sxxxk, \syyya);
\coordinate (spppkb) at (\sxxxk, \syyyb);
\coordinate (spppkc) at (\sxxxk, \syyyc);
\coordinate (spppkd) at (\sxxxk, \syyyd);
\coordinate (spppke) at (\sxxxk, \syyye);
\coordinate (spppkf) at (\sxxxk, \syyyf);
\coordinate (spppkg) at (\sxxxk, \syyyg);
\coordinate (spppkh) at (\sxxxk, \syyyh);
\coordinate (spppki) at (\sxxxk, \syyyi);
\coordinate (spppkj) at (\sxxxk, \syyyj);
\coordinate (spppkk) at (\sxxxk, \syyyk);
\coordinate (spppkl) at (\sxxxk, \syyyl);
\coordinate (spppkm) at (\sxxxk, \syyym);
\coordinate (spppkn) at (\sxxxk, \syyyn);
\coordinate (spppko) at (\sxxxk, \syyyo);
\coordinate (spppkp) at (\sxxxk, \syyyp);
\coordinate (spppkq) at (\sxxxk, \syyyq);
\coordinate (spppkr) at (\sxxxk, \syyyr);
\coordinate (spppks) at (\sxxxk, \syyys);
\coordinate (spppkt) at (\sxxxk, \syyyt);
\coordinate (spppku) at (\sxxxk, \syyyu);
\coordinate (spppkv) at (\sxxxk, \syyyv);
\coordinate (spppkw) at (\sxxxk, \syyyw);
\coordinate (spppkx) at (\sxxxk, \syyyx);
\coordinate (spppky) at (\sxxxk, \syyyy);
\coordinate (spppkz) at (\sxxxk, \syyyz);
\coordinate (spppla) at (\sxxxl, \syyya);
\coordinate (sppplb) at (\sxxxl, \syyyb);
\coordinate (sppplc) at (\sxxxl, \syyyc);
\coordinate (spppld) at (\sxxxl, \syyyd);
\coordinate (sppple) at (\sxxxl, \syyye);
\coordinate (sppplf) at (\sxxxl, \syyyf);
\coordinate (sppplg) at (\sxxxl, \syyyg);
\coordinate (sppplh) at (\sxxxl, \syyyh);
\coordinate (spppli) at (\sxxxl, \syyyi);
\coordinate (sppplj) at (\sxxxl, \syyyj);
\coordinate (sppplk) at (\sxxxl, \syyyk);
\coordinate (spppll) at (\sxxxl, \syyyl);
\coordinate (sppplm) at (\sxxxl, \syyym);
\coordinate (spppln) at (\sxxxl, \syyyn);
\coordinate (sppplo) at (\sxxxl, \syyyo);
\coordinate (sppplp) at (\sxxxl, \syyyp);
\coordinate (sppplq) at (\sxxxl, \syyyq);
\coordinate (sppplr) at (\sxxxl, \syyyr);
\coordinate (spppls) at (\sxxxl, \syyys);
\coordinate (sppplt) at (\sxxxl, \syyyt);
\coordinate (sppplu) at (\sxxxl, \syyyu);
\coordinate (sppplv) at (\sxxxl, \syyyv);
\coordinate (sppplw) at (\sxxxl, \syyyw);
\coordinate (sppplx) at (\sxxxl, \syyyx);
\coordinate (sppply) at (\sxxxl, \syyyy);
\coordinate (sppplz) at (\sxxxl, \syyyz);
\coordinate (spppma) at (\sxxxm, \syyya);
\coordinate (spppmb) at (\sxxxm, \syyyb);
\coordinate (spppmc) at (\sxxxm, \syyyc);
\coordinate (spppmd) at (\sxxxm, \syyyd);
\coordinate (spppme) at (\sxxxm, \syyye);
\coordinate (spppmf) at (\sxxxm, \syyyf);
\coordinate (spppmg) at (\sxxxm, \syyyg);
\coordinate (spppmh) at (\sxxxm, \syyyh);
\coordinate (spppmi) at (\sxxxm, \syyyi);
\coordinate (spppmj) at (\sxxxm, \syyyj);
\coordinate (spppmk) at (\sxxxm, \syyyk);
\coordinate (spppml) at (\sxxxm, \syyyl);
\coordinate (spppmm) at (\sxxxm, \syyym);
\coordinate (spppmn) at (\sxxxm, \syyyn);
\coordinate (spppmo) at (\sxxxm, \syyyo);
\coordinate (spppmp) at (\sxxxm, \syyyp);
\coordinate (spppmq) at (\sxxxm, \syyyq);
\coordinate (spppmr) at (\sxxxm, \syyyr);
\coordinate (spppms) at (\sxxxm, \syyys);
\coordinate (spppmt) at (\sxxxm, \syyyt);
\coordinate (spppmu) at (\sxxxm, \syyyu);
\coordinate (spppmv) at (\sxxxm, \syyyv);
\coordinate (spppmw) at (\sxxxm, \syyyw);
\coordinate (spppmx) at (\sxxxm, \syyyx);
\coordinate (spppmy) at (\sxxxm, \syyyy);
\coordinate (spppmz) at (\sxxxm, \syyyz);
\coordinate (spppna) at (\sxxxn, \syyya);
\coordinate (spppnb) at (\sxxxn, \syyyb);
\coordinate (spppnc) at (\sxxxn, \syyyc);
\coordinate (spppnd) at (\sxxxn, \syyyd);
\coordinate (spppne) at (\sxxxn, \syyye);
\coordinate (spppnf) at (\sxxxn, \syyyf);
\coordinate (spppng) at (\sxxxn, \syyyg);
\coordinate (spppnh) at (\sxxxn, \syyyh);
\coordinate (spppni) at (\sxxxn, \syyyi);
\coordinate (spppnj) at (\sxxxn, \syyyj);
\coordinate (spppnk) at (\sxxxn, \syyyk);
\coordinate (spppnl) at (\sxxxn, \syyyl);
\coordinate (spppnm) at (\sxxxn, \syyym);
\coordinate (spppnn) at (\sxxxn, \syyyn);
\coordinate (spppno) at (\sxxxn, \syyyo);
\coordinate (spppnp) at (\sxxxn, \syyyp);
\coordinate (spppnq) at (\sxxxn, \syyyq);
\coordinate (spppnr) at (\sxxxn, \syyyr);
\coordinate (spppns) at (\sxxxn, \syyys);
\coordinate (spppnt) at (\sxxxn, \syyyt);
\coordinate (spppnu) at (\sxxxn, \syyyu);
\coordinate (spppnv) at (\sxxxn, \syyyv);
\coordinate (spppnw) at (\sxxxn, \syyyw);
\coordinate (spppnx) at (\sxxxn, \syyyx);
\coordinate (spppny) at (\sxxxn, \syyyy);
\coordinate (spppnz) at (\sxxxn, \syyyz);
\coordinate (spppoa) at (\sxxxo, \syyya);
\coordinate (spppob) at (\sxxxo, \syyyb);
\coordinate (spppoc) at (\sxxxo, \syyyc);
\coordinate (spppod) at (\sxxxo, \syyyd);
\coordinate (spppoe) at (\sxxxo, \syyye);
\coordinate (spppof) at (\sxxxo, \syyyf);
\coordinate (spppog) at (\sxxxo, \syyyg);
\coordinate (spppoh) at (\sxxxo, \syyyh);
\coordinate (spppoi) at (\sxxxo, \syyyi);
\coordinate (spppoj) at (\sxxxo, \syyyj);
\coordinate (spppok) at (\sxxxo, \syyyk);
\coordinate (spppol) at (\sxxxo, \syyyl);
\coordinate (spppom) at (\sxxxo, \syyym);
\coordinate (spppon) at (\sxxxo, \syyyn);
\coordinate (spppoo) at (\sxxxo, \syyyo);
\coordinate (spppop) at (\sxxxo, \syyyp);
\coordinate (spppoq) at (\sxxxo, \syyyq);
\coordinate (spppor) at (\sxxxo, \syyyr);
\coordinate (spppos) at (\sxxxo, \syyys);
\coordinate (spppot) at (\sxxxo, \syyyt);
\coordinate (spppou) at (\sxxxo, \syyyu);
\coordinate (spppov) at (\sxxxo, \syyyv);
\coordinate (spppow) at (\sxxxo, \syyyw);
\coordinate (spppox) at (\sxxxo, \syyyx);
\coordinate (spppoy) at (\sxxxo, \syyyy);
\coordinate (spppoz) at (\sxxxo, \syyyz);
\coordinate (sppppa) at (\sxxxp, \syyya);
\coordinate (sppppb) at (\sxxxp, \syyyb);
\coordinate (sppppc) at (\sxxxp, \syyyc);
\coordinate (sppppd) at (\sxxxp, \syyyd);
\coordinate (sppppe) at (\sxxxp, \syyye);
\coordinate (sppppf) at (\sxxxp, \syyyf);
\coordinate (sppppg) at (\sxxxp, \syyyg);
\coordinate (spppph) at (\sxxxp, \syyyh);
\coordinate (sppppi) at (\sxxxp, \syyyi);
\coordinate (sppppj) at (\sxxxp, \syyyj);
\coordinate (sppppk) at (\sxxxp, \syyyk);
\coordinate (sppppl) at (\sxxxp, \syyyl);
\coordinate (sppppm) at (\sxxxp, \syyym);
\coordinate (sppppn) at (\sxxxp, \syyyn);
\coordinate (sppppo) at (\sxxxp, \syyyo);
\coordinate (sppppp) at (\sxxxp, \syyyp);
\coordinate (sppppq) at (\sxxxp, \syyyq);
\coordinate (sppppr) at (\sxxxp, \syyyr);
\coordinate (spppps) at (\sxxxp, \syyys);
\coordinate (sppppt) at (\sxxxp, \syyyt);
\coordinate (sppppu) at (\sxxxp, \syyyu);
\coordinate (sppppv) at (\sxxxp, \syyyv);
\coordinate (sppppw) at (\sxxxp, \syyyw);
\coordinate (sppppx) at (\sxxxp, \syyyx);
\coordinate (sppppy) at (\sxxxp, \syyyy);
\coordinate (sppppz) at (\sxxxp, \syyyz);
\coordinate (spppqa) at (\sxxxq, \syyya);
\coordinate (spppqb) at (\sxxxq, \syyyb);
\coordinate (spppqc) at (\sxxxq, \syyyc);
\coordinate (spppqd) at (\sxxxq, \syyyd);
\coordinate (spppqe) at (\sxxxq, \syyye);
\coordinate (spppqf) at (\sxxxq, \syyyf);
\coordinate (spppqg) at (\sxxxq, \syyyg);
\coordinate (spppqh) at (\sxxxq, \syyyh);
\coordinate (spppqi) at (\sxxxq, \syyyi);
\coordinate (spppqj) at (\sxxxq, \syyyj);
\coordinate (spppqk) at (\sxxxq, \syyyk);
\coordinate (spppql) at (\sxxxq, \syyyl);
\coordinate (spppqm) at (\sxxxq, \syyym);
\coordinate (spppqn) at (\sxxxq, \syyyn);
\coordinate (spppqo) at (\sxxxq, \syyyo);
\coordinate (spppqp) at (\sxxxq, \syyyp);
\coordinate (spppqq) at (\sxxxq, \syyyq);
\coordinate (spppqr) at (\sxxxq, \syyyr);
\coordinate (spppqs) at (\sxxxq, \syyys);
\coordinate (spppqt) at (\sxxxq, \syyyt);
\coordinate (spppqu) at (\sxxxq, \syyyu);
\coordinate (spppqv) at (\sxxxq, \syyyv);
\coordinate (spppqw) at (\sxxxq, \syyyw);
\coordinate (spppqx) at (\sxxxq, \syyyx);
\coordinate (spppqy) at (\sxxxq, \syyyy);
\coordinate (spppqz) at (\sxxxq, \syyyz);
\coordinate (spppra) at (\sxxxr, \syyya);
\coordinate (sppprb) at (\sxxxr, \syyyb);
\coordinate (sppprc) at (\sxxxr, \syyyc);
\coordinate (sppprd) at (\sxxxr, \syyyd);
\coordinate (spppre) at (\sxxxr, \syyye);
\coordinate (sppprf) at (\sxxxr, \syyyf);
\coordinate (sppprg) at (\sxxxr, \syyyg);
\coordinate (sppprh) at (\sxxxr, \syyyh);
\coordinate (spppri) at (\sxxxr, \syyyi);
\coordinate (sppprj) at (\sxxxr, \syyyj);
\coordinate (sppprk) at (\sxxxr, \syyyk);
\coordinate (sppprl) at (\sxxxr, \syyyl);
\coordinate (sppprm) at (\sxxxr, \syyym);
\coordinate (sppprn) at (\sxxxr, \syyyn);
\coordinate (spppro) at (\sxxxr, \syyyo);
\coordinate (sppprp) at (\sxxxr, \syyyp);
\coordinate (sppprq) at (\sxxxr, \syyyq);
\coordinate (sppprr) at (\sxxxr, \syyyr);
\coordinate (sppprs) at (\sxxxr, \syyys);
\coordinate (sppprt) at (\sxxxr, \syyyt);
\coordinate (spppru) at (\sxxxr, \syyyu);
\coordinate (sppprv) at (\sxxxr, \syyyv);
\coordinate (sppprw) at (\sxxxr, \syyyw);
\coordinate (sppprx) at (\sxxxr, \syyyx);
\coordinate (spppry) at (\sxxxr, \syyyy);
\coordinate (sppprz) at (\sxxxr, \syyyz);
\coordinate (spppsa) at (\sxxxs, \syyya);
\coordinate (spppsb) at (\sxxxs, \syyyb);
\coordinate (spppsc) at (\sxxxs, \syyyc);
\coordinate (spppsd) at (\sxxxs, \syyyd);
\coordinate (spppse) at (\sxxxs, \syyye);
\coordinate (spppsf) at (\sxxxs, \syyyf);
\coordinate (spppsg) at (\sxxxs, \syyyg);
\coordinate (spppsh) at (\sxxxs, \syyyh);
\coordinate (spppsi) at (\sxxxs, \syyyi);
\coordinate (spppsj) at (\sxxxs, \syyyj);
\coordinate (spppsk) at (\sxxxs, \syyyk);
\coordinate (spppsl) at (\sxxxs, \syyyl);
\coordinate (spppsm) at (\sxxxs, \syyym);
\coordinate (spppsn) at (\sxxxs, \syyyn);
\coordinate (spppso) at (\sxxxs, \syyyo);
\coordinate (spppsp) at (\sxxxs, \syyyp);
\coordinate (spppsq) at (\sxxxs, \syyyq);
\coordinate (spppsr) at (\sxxxs, \syyyr);
\coordinate (spppss) at (\sxxxs, \syyys);
\coordinate (spppst) at (\sxxxs, \syyyt);
\coordinate (spppsu) at (\sxxxs, \syyyu);
\coordinate (spppsv) at (\sxxxs, \syyyv);
\coordinate (spppsw) at (\sxxxs, \syyyw);
\coordinate (spppsx) at (\sxxxs, \syyyx);
\coordinate (spppsy) at (\sxxxs, \syyyy);
\coordinate (spppsz) at (\sxxxs, \syyyz);
\coordinate (spppta) at (\sxxxt, \syyya);
\coordinate (sppptb) at (\sxxxt, \syyyb);
\coordinate (sppptc) at (\sxxxt, \syyyc);
\coordinate (sppptd) at (\sxxxt, \syyyd);
\coordinate (spppte) at (\sxxxt, \syyye);
\coordinate (sppptf) at (\sxxxt, \syyyf);
\coordinate (sppptg) at (\sxxxt, \syyyg);
\coordinate (spppth) at (\sxxxt, \syyyh);
\coordinate (spppti) at (\sxxxt, \syyyi);
\coordinate (sppptj) at (\sxxxt, \syyyj);
\coordinate (sppptk) at (\sxxxt, \syyyk);
\coordinate (sppptl) at (\sxxxt, \syyyl);
\coordinate (sppptm) at (\sxxxt, \syyym);
\coordinate (sppptn) at (\sxxxt, \syyyn);
\coordinate (spppto) at (\sxxxt, \syyyo);
\coordinate (sppptp) at (\sxxxt, \syyyp);
\coordinate (sppptq) at (\sxxxt, \syyyq);
\coordinate (sppptr) at (\sxxxt, \syyyr);
\coordinate (spppts) at (\sxxxt, \syyys);
\coordinate (sppptt) at (\sxxxt, \syyyt);
\coordinate (sppptu) at (\sxxxt, \syyyu);
\coordinate (sppptv) at (\sxxxt, \syyyv);
\coordinate (sppptw) at (\sxxxt, \syyyw);
\coordinate (sppptx) at (\sxxxt, \syyyx);
\coordinate (spppty) at (\sxxxt, \syyyy);
\coordinate (sppptz) at (\sxxxt, \syyyz);
\coordinate (spppua) at (\sxxxu, \syyya);
\coordinate (spppub) at (\sxxxu, \syyyb);
\coordinate (spppuc) at (\sxxxu, \syyyc);
\coordinate (spppud) at (\sxxxu, \syyyd);
\coordinate (spppue) at (\sxxxu, \syyye);
\coordinate (spppuf) at (\sxxxu, \syyyf);
\coordinate (spppug) at (\sxxxu, \syyyg);
\coordinate (spppuh) at (\sxxxu, \syyyh);
\coordinate (spppui) at (\sxxxu, \syyyi);
\coordinate (spppuj) at (\sxxxu, \syyyj);
\coordinate (spppuk) at (\sxxxu, \syyyk);
\coordinate (spppul) at (\sxxxu, \syyyl);
\coordinate (spppum) at (\sxxxu, \syyym);
\coordinate (spppun) at (\sxxxu, \syyyn);
\coordinate (spppuo) at (\sxxxu, \syyyo);
\coordinate (spppup) at (\sxxxu, \syyyp);
\coordinate (spppuq) at (\sxxxu, \syyyq);
\coordinate (spppur) at (\sxxxu, \syyyr);
\coordinate (spppus) at (\sxxxu, \syyys);
\coordinate (sppput) at (\sxxxu, \syyyt);
\coordinate (spppuu) at (\sxxxu, \syyyu);
\coordinate (spppuv) at (\sxxxu, \syyyv);
\coordinate (spppuw) at (\sxxxu, \syyyw);
\coordinate (spppux) at (\sxxxu, \syyyx);
\coordinate (spppuy) at (\sxxxu, \syyyy);
\coordinate (spppuz) at (\sxxxu, \syyyz);
\coordinate (spppva) at (\sxxxv, \syyya);
\coordinate (spppvb) at (\sxxxv, \syyyb);
\coordinate (spppvc) at (\sxxxv, \syyyc);
\coordinate (spppvd) at (\sxxxv, \syyyd);
\coordinate (spppve) at (\sxxxv, \syyye);
\coordinate (spppvf) at (\sxxxv, \syyyf);
\coordinate (spppvg) at (\sxxxv, \syyyg);
\coordinate (spppvh) at (\sxxxv, \syyyh);
\coordinate (spppvi) at (\sxxxv, \syyyi);
\coordinate (spppvj) at (\sxxxv, \syyyj);
\coordinate (spppvk) at (\sxxxv, \syyyk);
\coordinate (spppvl) at (\sxxxv, \syyyl);
\coordinate (spppvm) at (\sxxxv, \syyym);
\coordinate (spppvn) at (\sxxxv, \syyyn);
\coordinate (spppvo) at (\sxxxv, \syyyo);
\coordinate (spppvp) at (\sxxxv, \syyyp);
\coordinate (spppvq) at (\sxxxv, \syyyq);
\coordinate (spppvr) at (\sxxxv, \syyyr);
\coordinate (spppvs) at (\sxxxv, \syyys);
\coordinate (spppvt) at (\sxxxv, \syyyt);
\coordinate (spppvu) at (\sxxxv, \syyyu);
\coordinate (spppvv) at (\sxxxv, \syyyv);
\coordinate (spppvw) at (\sxxxv, \syyyw);
\coordinate (spppvx) at (\sxxxv, \syyyx);
\coordinate (spppvy) at (\sxxxv, \syyyy);
\coordinate (spppvz) at (\sxxxv, \syyyz);
\coordinate (spppwa) at (\sxxxw, \syyya);
\coordinate (spppwb) at (\sxxxw, \syyyb);
\coordinate (spppwc) at (\sxxxw, \syyyc);
\coordinate (spppwd) at (\sxxxw, \syyyd);
\coordinate (spppwe) at (\sxxxw, \syyye);
\coordinate (spppwf) at (\sxxxw, \syyyf);
\coordinate (spppwg) at (\sxxxw, \syyyg);
\coordinate (spppwh) at (\sxxxw, \syyyh);
\coordinate (spppwi) at (\sxxxw, \syyyi);
\coordinate (spppwj) at (\sxxxw, \syyyj);
\coordinate (spppwk) at (\sxxxw, \syyyk);
\coordinate (spppwl) at (\sxxxw, \syyyl);
\coordinate (spppwm) at (\sxxxw, \syyym);
\coordinate (spppwn) at (\sxxxw, \syyyn);
\coordinate (spppwo) at (\sxxxw, \syyyo);
\coordinate (spppwp) at (\sxxxw, \syyyp);
\coordinate (spppwq) at (\sxxxw, \syyyq);
\coordinate (spppwr) at (\sxxxw, \syyyr);
\coordinate (spppws) at (\sxxxw, \syyys);
\coordinate (spppwt) at (\sxxxw, \syyyt);
\coordinate (spppwu) at (\sxxxw, \syyyu);
\coordinate (spppwv) at (\sxxxw, \syyyv);
\coordinate (spppww) at (\sxxxw, \syyyw);
\coordinate (spppwx) at (\sxxxw, \syyyx);
\coordinate (spppwy) at (\sxxxw, \syyyy);
\coordinate (spppwz) at (\sxxxw, \syyyz);
\coordinate (spppxa) at (\sxxxx, \syyya);
\coordinate (spppxb) at (\sxxxx, \syyyb);
\coordinate (spppxc) at (\sxxxx, \syyyc);
\coordinate (spppxd) at (\sxxxx, \syyyd);
\coordinate (spppxe) at (\sxxxx, \syyye);
\coordinate (spppxf) at (\sxxxx, \syyyf);
\coordinate (spppxg) at (\sxxxx, \syyyg);
\coordinate (spppxh) at (\sxxxx, \syyyh);
\coordinate (spppxi) at (\sxxxx, \syyyi);
\coordinate (spppxj) at (\sxxxx, \syyyj);
\coordinate (spppxk) at (\sxxxx, \syyyk);
\coordinate (spppxl) at (\sxxxx, \syyyl);
\coordinate (spppxm) at (\sxxxx, \syyym);
\coordinate (spppxn) at (\sxxxx, \syyyn);
\coordinate (spppxo) at (\sxxxx, \syyyo);
\coordinate (spppxp) at (\sxxxx, \syyyp);
\coordinate (spppxq) at (\sxxxx, \syyyq);
\coordinate (spppxr) at (\sxxxx, \syyyr);
\coordinate (spppxs) at (\sxxxx, \syyys);
\coordinate (spppxt) at (\sxxxx, \syyyt);
\coordinate (spppxu) at (\sxxxx, \syyyu);
\coordinate (spppxv) at (\sxxxx, \syyyv);
\coordinate (spppxw) at (\sxxxx, \syyyw);
\coordinate (spppxx) at (\sxxxx, \syyyx);
\coordinate (spppxy) at (\sxxxx, \syyyy);
\coordinate (spppxz) at (\sxxxx, \syyyz);
\coordinate (spppya) at (\sxxxy, \syyya);
\coordinate (spppyb) at (\sxxxy, \syyyb);
\coordinate (spppyc) at (\sxxxy, \syyyc);
\coordinate (spppyd) at (\sxxxy, \syyyd);
\coordinate (spppye) at (\sxxxy, \syyye);
\coordinate (spppyf) at (\sxxxy, \syyyf);
\coordinate (spppyg) at (\sxxxy, \syyyg);
\coordinate (spppyh) at (\sxxxy, \syyyh);
\coordinate (spppyi) at (\sxxxy, \syyyi);
\coordinate (spppyj) at (\sxxxy, \syyyj);
\coordinate (spppyk) at (\sxxxy, \syyyk);
\coordinate (spppyl) at (\sxxxy, \syyyl);
\coordinate (spppym) at (\sxxxy, \syyym);
\coordinate (spppyn) at (\sxxxy, \syyyn);
\coordinate (spppyo) at (\sxxxy, \syyyo);
\coordinate (spppyp) at (\sxxxy, \syyyp);
\coordinate (spppyq) at (\sxxxy, \syyyq);
\coordinate (spppyr) at (\sxxxy, \syyyr);
\coordinate (spppys) at (\sxxxy, \syyys);
\coordinate (spppyt) at (\sxxxy, \syyyt);
\coordinate (spppyu) at (\sxxxy, \syyyu);
\coordinate (spppyv) at (\sxxxy, \syyyv);
\coordinate (spppyw) at (\sxxxy, \syyyw);
\coordinate (spppyx) at (\sxxxy, \syyyx);
\coordinate (spppyy) at (\sxxxy, \syyyy);
\coordinate (spppyz) at (\sxxxy, \syyyz);
\coordinate (spppza) at (\sxxxz, \syyya);
\coordinate (spppzb) at (\sxxxz, \syyyb);
\coordinate (spppzc) at (\sxxxz, \syyyc);
\coordinate (spppzd) at (\sxxxz, \syyyd);
\coordinate (spppze) at (\sxxxz, \syyye);
\coordinate (spppzf) at (\sxxxz, \syyyf);
\coordinate (spppzg) at (\sxxxz, \syyyg);
\coordinate (spppzh) at (\sxxxz, \syyyh);
\coordinate (spppzi) at (\sxxxz, \syyyi);
\coordinate (spppzj) at (\sxxxz, \syyyj);
\coordinate (spppzk) at (\sxxxz, \syyyk);
\coordinate (spppzl) at (\sxxxz, \syyyl);
\coordinate (spppzm) at (\sxxxz, \syyym);
\coordinate (spppzn) at (\sxxxz, \syyyn);
\coordinate (spppzo) at (\sxxxz, \syyyo);
\coordinate (spppzp) at (\sxxxz, \syyyp);
\coordinate (spppzq) at (\sxxxz, \syyyq);
\coordinate (spppzr) at (\sxxxz, \syyyr);
\coordinate (spppzs) at (\sxxxz, \syyys);
\coordinate (spppzt) at (\sxxxz, \syyyt);
\coordinate (spppzu) at (\sxxxz, \syyyu);
\coordinate (spppzv) at (\sxxxz, \syyyv);
\coordinate (spppzw) at (\sxxxz, \syyyw);
\coordinate (spppzx) at (\sxxxz, \syyyx);
\coordinate (spppzy) at (\sxxxz, \syyyy);
\coordinate (spppzz) at (\sxxxz, \syyyz);

%\gangprintcoordinateat{(0,0)}{The last coordinate values: }{($(spppzz)$)}; 

%\coordinatebackground{s}{f}{g}{q};
\draw (spppni) node [op amp] (opamp) {};
\getxyingivenunit{cm}{(opamp.+)}{\opampzx}{\opampzy};
\getxyingivenunit{cm}{(opamp.-)}{\opampfx}{\opampfy};


\fill  (\sxxxg, \secondopampzy) circle (2pt);
\draw (\sxxxg, \secondopampzy) -- (\sxxxg,\opampzy) node [left] {$U_{we}$} to [R, l=$R_d$, -*]  (\sxxxj,\opampzy) 
to [R, l=$R_d$, *-*] (opamp.+)
to [C, l_=$C_{d2}$, *-] (\opampzx, \syyyf) -- (\opampzx, \gyyyi);
\fill (\opampzx, \gyyyi) circle (2pt);


\draw (opamp.out) |- (\sxxxl,\syyyk) to [C, l_=$C_{d1}$, *-] (\sxxxj,\syyyk) to [short](\sxxxj,\opampzy);

\draw (opamp.-) -|  (\sxxxl,\syyyk);

\getxyingivenunit{cm}{(opamp.out)}{\opampoutx}{\opampouty};
\draw (opamp.out) to [short, *-*] (\bigswQCx,\opampouty);

\end{circuitikz}

























\newpage

{\Large Figure 9, deleting the coordinate background by commenting out the only one command \\
``\textbackslash coordinatebackground ..." line.}


\begin{circuitikz}[scale=1]


% Circuits can be drawn by the following five major steps, as shown in the following example. 

% Step 1, preparations. 

% "Install" the coordinate system with keyword ``g".
\pgfmathsetmacro{\totalgxxx}{26}
\pgfmathsetmacro{\totalgyyy}{26}
\pgfmathsetmacro{\gxxxspacing}{1}
\pgfmathsetmacro{\gyyyspacing}{1}
\pgfmathsetmacro{\gxxxa}{-8}
\pgfmathsetmacro{\gyyya}{-8}

\pgfmathsetmacro{\gxxxb}{\gxxxa + \gxxxspacing + 0.0 }
\pgfmathsetmacro{\gxxxc}{\gxxxb + \gxxxspacing + 0.0 }
\pgfmathsetmacro{\gxxxd}{\gxxxc + \gxxxspacing + 0.0 }
\pgfmathsetmacro{\gxxxe}{\gxxxd + \gxxxspacing + 0.0 }
\pgfmathsetmacro{\gxxxf}{\gxxxe + \gxxxspacing + 0.0 }
\pgfmathsetmacro{\gxxxg}{\gxxxf + \gxxxspacing + 0.0 }
\pgfmathsetmacro{\gxxxh}{\gxxxg + \gxxxspacing + 0.0 }
\pgfmathsetmacro{\gxxxi}{\gxxxh + \gxxxspacing + 0.0 }
\pgfmathsetmacro{\gxxxj}{\gxxxi + \gxxxspacing + 0.0 }
\pgfmathsetmacro{\gxxxk}{\gxxxj + \gxxxspacing + 0.0 }
\pgfmathsetmacro{\gxxxl}{\gxxxk + \gxxxspacing + 8.0 }
\pgfmathsetmacro{\gxxxm}{\gxxxl + \gxxxspacing + 0.0 }
\pgfmathsetmacro{\gxxxn}{\gxxxm + \gxxxspacing + 0.0 }
\pgfmathsetmacro{\gxxxo}{\gxxxn + \gxxxspacing + 0.0 }
\pgfmathsetmacro{\gxxxp}{\gxxxo + \gxxxspacing + 0.0 }
\pgfmathsetmacro{\gxxxq}{\gxxxp + \gxxxspacing + 0.0 }
\pgfmathsetmacro{\gxxxr}{\gxxxq + \gxxxspacing + 0.0 }
\pgfmathsetmacro{\gxxxs}{\gxxxr + \gxxxspacing + 0.0 }
\pgfmathsetmacro{\gxxxt}{\gxxxs + \gxxxspacing + 0.0 }
\pgfmathsetmacro{\gxxxu}{\gxxxt + \gxxxspacing + 0.0 }
\pgfmathsetmacro{\gxxxv}{\gxxxu + \gxxxspacing + 0.0 }
\pgfmathsetmacro{\gxxxw}{\gxxxv + \gxxxspacing + 0.0 }
\pgfmathsetmacro{\gxxxx}{\gxxxw + \gxxxspacing + 0.0 }
\pgfmathsetmacro{\gxxxy}{\gxxxx + \gxxxspacing + 0.0 }
\pgfmathsetmacro{\gxxxz}{\gxxxy + \gxxxspacing + 0.0 }

\pgfmathsetmacro{\gyyyb}{\gyyya + \gyyyspacing + 2.0 }
\pgfmathsetmacro{\gyyyc}{\gyyyb + \gyyyspacing + 0.0 }
\pgfmathsetmacro{\gyyyd}{\gyyyc + \gyyyspacing + 0.0 }
\pgfmathsetmacro{\gyyye}{\gyyyd + \gyyyspacing + 0.0 }
\pgfmathsetmacro{\gyyyf}{\gyyye + \gyyyspacing + 0.0 }
\pgfmathsetmacro{\gyyyg}{\gyyyf + \gyyyspacing + 0.0 }
\pgfmathsetmacro{\gyyyh}{\gyyyg + \gyyyspacing + 0.0 }
\pgfmathsetmacro{\gyyyi}{\gyyyh + \gyyyspacing + 0.0 }
\pgfmathsetmacro{\gyyyj}{\gyyyi + \gyyyspacing + 0.0 }
\pgfmathsetmacro{\gyyyk}{\gyyyj + \gyyyspacing + 0.0 }
\pgfmathsetmacro{\gyyyl}{\gyyyk + \gyyyspacing + 12.0 }
\pgfmathsetmacro{\gyyym}{\gyyyl + \gyyyspacing + 0.0 }
\pgfmathsetmacro{\gyyyn}{\gyyym + \gyyyspacing + 0.0 }
\pgfmathsetmacro{\gyyyo}{\gyyyn + \gyyyspacing + 0.0 }
\pgfmathsetmacro{\gyyyp}{\gyyyo + \gyyyspacing + 0.0 }
\pgfmathsetmacro{\gyyyq}{\gyyyp + \gyyyspacing + 0.0 }
\pgfmathsetmacro{\gyyyr}{\gyyyq + \gyyyspacing + 0.0 }
\pgfmathsetmacro{\gyyys}{\gyyyr + \gyyyspacing + 0.0 }
\pgfmathsetmacro{\gyyyt}{\gyyys + \gyyyspacing + 0.0 }
\pgfmathsetmacro{\gyyyu}{\gyyyt + \gyyyspacing + 0.0 }
\pgfmathsetmacro{\gyyyv}{\gyyyu + \gyyyspacing + 0.0 }
\pgfmathsetmacro{\gyyyw}{\gyyyv + \gyyyspacing + 0.0 }
\pgfmathsetmacro{\gyyyx}{\gyyyw + \gyyyspacing + 0.0 }
\pgfmathsetmacro{\gyyyy}{\gyyyx + \gyyyspacing + 0.0 }
\pgfmathsetmacro{\gyyyz}{\gyyyy + \gyyyspacing + 0.0 }

\coordinate (gpppaa) at (\gxxxa, \gyyya);
\coordinate (gpppab) at (\gxxxa, \gyyyb);
\coordinate (gpppac) at (\gxxxa, \gyyyc);
\coordinate (gpppad) at (\gxxxa, \gyyyd);
\coordinate (gpppae) at (\gxxxa, \gyyye);
\coordinate (gpppaf) at (\gxxxa, \gyyyf);
\coordinate (gpppag) at (\gxxxa, \gyyyg);
\coordinate (gpppah) at (\gxxxa, \gyyyh);
\coordinate (gpppai) at (\gxxxa, \gyyyi);
\coordinate (gpppaj) at (\gxxxa, \gyyyj);
\coordinate (gpppak) at (\gxxxa, \gyyyk);
\coordinate (gpppal) at (\gxxxa, \gyyyl);
\coordinate (gpppam) at (\gxxxa, \gyyym);
\coordinate (gpppan) at (\gxxxa, \gyyyn);
\coordinate (gpppao) at (\gxxxa, \gyyyo);
\coordinate (gpppap) at (\gxxxa, \gyyyp);
\coordinate (gpppaq) at (\gxxxa, \gyyyq);
\coordinate (gpppar) at (\gxxxa, \gyyyr);
\coordinate (gpppas) at (\gxxxa, \gyyys);
\coordinate (gpppat) at (\gxxxa, \gyyyt);
\coordinate (gpppau) at (\gxxxa, \gyyyu);
\coordinate (gpppav) at (\gxxxa, \gyyyv);
\coordinate (gpppaw) at (\gxxxa, \gyyyw);
\coordinate (gpppax) at (\gxxxa, \gyyyx);
\coordinate (gpppay) at (\gxxxa, \gyyyy);
\coordinate (gpppaz) at (\gxxxa, \gyyyz);
\coordinate (gpppba) at (\gxxxb, \gyyya);
\coordinate (gpppbb) at (\gxxxb, \gyyyb);
\coordinate (gpppbc) at (\gxxxb, \gyyyc);
\coordinate (gpppbd) at (\gxxxb, \gyyyd);
\coordinate (gpppbe) at (\gxxxb, \gyyye);
\coordinate (gpppbf) at (\gxxxb, \gyyyf);
\coordinate (gpppbg) at (\gxxxb, \gyyyg);
\coordinate (gpppbh) at (\gxxxb, \gyyyh);
\coordinate (gpppbi) at (\gxxxb, \gyyyi);
\coordinate (gpppbj) at (\gxxxb, \gyyyj);
\coordinate (gpppbk) at (\gxxxb, \gyyyk);
\coordinate (gpppbl) at (\gxxxb, \gyyyl);
\coordinate (gpppbm) at (\gxxxb, \gyyym);
\coordinate (gpppbn) at (\gxxxb, \gyyyn);
\coordinate (gpppbo) at (\gxxxb, \gyyyo);
\coordinate (gpppbp) at (\gxxxb, \gyyyp);
\coordinate (gpppbq) at (\gxxxb, \gyyyq);
\coordinate (gpppbr) at (\gxxxb, \gyyyr);
\coordinate (gpppbs) at (\gxxxb, \gyyys);
\coordinate (gpppbt) at (\gxxxb, \gyyyt);
\coordinate (gpppbu) at (\gxxxb, \gyyyu);
\coordinate (gpppbv) at (\gxxxb, \gyyyv);
\coordinate (gpppbw) at (\gxxxb, \gyyyw);
\coordinate (gpppbx) at (\gxxxb, \gyyyx);
\coordinate (gpppby) at (\gxxxb, \gyyyy);
\coordinate (gpppbz) at (\gxxxb, \gyyyz);
\coordinate (gpppca) at (\gxxxc, \gyyya);
\coordinate (gpppcb) at (\gxxxc, \gyyyb);
\coordinate (gpppcc) at (\gxxxc, \gyyyc);
\coordinate (gpppcd) at (\gxxxc, \gyyyd);
\coordinate (gpppce) at (\gxxxc, \gyyye);
\coordinate (gpppcf) at (\gxxxc, \gyyyf);
\coordinate (gpppcg) at (\gxxxc, \gyyyg);
\coordinate (gpppch) at (\gxxxc, \gyyyh);
\coordinate (gpppci) at (\gxxxc, \gyyyi);
\coordinate (gpppcj) at (\gxxxc, \gyyyj);
\coordinate (gpppck) at (\gxxxc, \gyyyk);
\coordinate (gpppcl) at (\gxxxc, \gyyyl);
\coordinate (gpppcm) at (\gxxxc, \gyyym);
\coordinate (gpppcn) at (\gxxxc, \gyyyn);
\coordinate (gpppco) at (\gxxxc, \gyyyo);
\coordinate (gpppcp) at (\gxxxc, \gyyyp);
\coordinate (gpppcq) at (\gxxxc, \gyyyq);
\coordinate (gpppcr) at (\gxxxc, \gyyyr);
\coordinate (gpppcs) at (\gxxxc, \gyyys);
\coordinate (gpppct) at (\gxxxc, \gyyyt);
\coordinate (gpppcu) at (\gxxxc, \gyyyu);
\coordinate (gpppcv) at (\gxxxc, \gyyyv);
\coordinate (gpppcw) at (\gxxxc, \gyyyw);
\coordinate (gpppcx) at (\gxxxc, \gyyyx);
\coordinate (gpppcy) at (\gxxxc, \gyyyy);
\coordinate (gpppcz) at (\gxxxc, \gyyyz);
\coordinate (gpppda) at (\gxxxd, \gyyya);
\coordinate (gpppdb) at (\gxxxd, \gyyyb);
\coordinate (gpppdc) at (\gxxxd, \gyyyc);
\coordinate (gpppdd) at (\gxxxd, \gyyyd);
\coordinate (gpppde) at (\gxxxd, \gyyye);
\coordinate (gpppdf) at (\gxxxd, \gyyyf);
\coordinate (gpppdg) at (\gxxxd, \gyyyg);
\coordinate (gpppdh) at (\gxxxd, \gyyyh);
\coordinate (gpppdi) at (\gxxxd, \gyyyi);
\coordinate (gpppdj) at (\gxxxd, \gyyyj);
\coordinate (gpppdk) at (\gxxxd, \gyyyk);
\coordinate (gpppdl) at (\gxxxd, \gyyyl);
\coordinate (gpppdm) at (\gxxxd, \gyyym);
\coordinate (gpppdn) at (\gxxxd, \gyyyn);
\coordinate (gpppdo) at (\gxxxd, \gyyyo);
\coordinate (gpppdp) at (\gxxxd, \gyyyp);
\coordinate (gpppdq) at (\gxxxd, \gyyyq);
\coordinate (gpppdr) at (\gxxxd, \gyyyr);
\coordinate (gpppds) at (\gxxxd, \gyyys);
\coordinate (gpppdt) at (\gxxxd, \gyyyt);
\coordinate (gpppdu) at (\gxxxd, \gyyyu);
\coordinate (gpppdv) at (\gxxxd, \gyyyv);
\coordinate (gpppdw) at (\gxxxd, \gyyyw);
\coordinate (gpppdx) at (\gxxxd, \gyyyx);
\coordinate (gpppdy) at (\gxxxd, \gyyyy);
\coordinate (gpppdz) at (\gxxxd, \gyyyz);
\coordinate (gpppea) at (\gxxxe, \gyyya);
\coordinate (gpppeb) at (\gxxxe, \gyyyb);
\coordinate (gpppec) at (\gxxxe, \gyyyc);
\coordinate (gppped) at (\gxxxe, \gyyyd);
\coordinate (gpppee) at (\gxxxe, \gyyye);
\coordinate (gpppef) at (\gxxxe, \gyyyf);
\coordinate (gpppeg) at (\gxxxe, \gyyyg);
\coordinate (gpppeh) at (\gxxxe, \gyyyh);
\coordinate (gpppei) at (\gxxxe, \gyyyi);
\coordinate (gpppej) at (\gxxxe, \gyyyj);
\coordinate (gpppek) at (\gxxxe, \gyyyk);
\coordinate (gpppel) at (\gxxxe, \gyyyl);
\coordinate (gpppem) at (\gxxxe, \gyyym);
\coordinate (gpppen) at (\gxxxe, \gyyyn);
\coordinate (gpppeo) at (\gxxxe, \gyyyo);
\coordinate (gpppep) at (\gxxxe, \gyyyp);
\coordinate (gpppeq) at (\gxxxe, \gyyyq);
\coordinate (gppper) at (\gxxxe, \gyyyr);
\coordinate (gpppes) at (\gxxxe, \gyyys);
\coordinate (gpppet) at (\gxxxe, \gyyyt);
\coordinate (gpppeu) at (\gxxxe, \gyyyu);
\coordinate (gpppev) at (\gxxxe, \gyyyv);
\coordinate (gpppew) at (\gxxxe, \gyyyw);
\coordinate (gpppex) at (\gxxxe, \gyyyx);
\coordinate (gpppey) at (\gxxxe, \gyyyy);
\coordinate (gpppez) at (\gxxxe, \gyyyz);
\coordinate (gpppfa) at (\gxxxf, \gyyya);
\coordinate (gpppfb) at (\gxxxf, \gyyyb);
\coordinate (gpppfc) at (\gxxxf, \gyyyc);
\coordinate (gpppfd) at (\gxxxf, \gyyyd);
\coordinate (gpppfe) at (\gxxxf, \gyyye);
\coordinate (gpppff) at (\gxxxf, \gyyyf);
\coordinate (gpppfg) at (\gxxxf, \gyyyg);
\coordinate (gpppfh) at (\gxxxf, \gyyyh);
\coordinate (gpppfi) at (\gxxxf, \gyyyi);
\coordinate (gpppfj) at (\gxxxf, \gyyyj);
\coordinate (gpppfk) at (\gxxxf, \gyyyk);
\coordinate (gpppfl) at (\gxxxf, \gyyyl);
\coordinate (gpppfm) at (\gxxxf, \gyyym);
\coordinate (gpppfn) at (\gxxxf, \gyyyn);
\coordinate (gpppfo) at (\gxxxf, \gyyyo);
\coordinate (gpppfp) at (\gxxxf, \gyyyp);
\coordinate (gpppfq) at (\gxxxf, \gyyyq);
\coordinate (gpppfr) at (\gxxxf, \gyyyr);
\coordinate (gpppfs) at (\gxxxf, \gyyys);
\coordinate (gpppft) at (\gxxxf, \gyyyt);
\coordinate (gpppfu) at (\gxxxf, \gyyyu);
\coordinate (gpppfv) at (\gxxxf, \gyyyv);
\coordinate (gpppfw) at (\gxxxf, \gyyyw);
\coordinate (gpppfx) at (\gxxxf, \gyyyx);
\coordinate (gpppfy) at (\gxxxf, \gyyyy);
\coordinate (gpppfz) at (\gxxxf, \gyyyz);
\coordinate (gpppga) at (\gxxxg, \gyyya);
\coordinate (gpppgb) at (\gxxxg, \gyyyb);
\coordinate (gpppgc) at (\gxxxg, \gyyyc);
\coordinate (gpppgd) at (\gxxxg, \gyyyd);
\coordinate (gpppge) at (\gxxxg, \gyyye);
\coordinate (gpppgf) at (\gxxxg, \gyyyf);
\coordinate (gpppgg) at (\gxxxg, \gyyyg);
\coordinate (gpppgh) at (\gxxxg, \gyyyh);
\coordinate (gpppgi) at (\gxxxg, \gyyyi);
\coordinate (gpppgj) at (\gxxxg, \gyyyj);
\coordinate (gpppgk) at (\gxxxg, \gyyyk);
\coordinate (gpppgl) at (\gxxxg, \gyyyl);
\coordinate (gpppgm) at (\gxxxg, \gyyym);
\coordinate (gpppgn) at (\gxxxg, \gyyyn);
\coordinate (gpppgo) at (\gxxxg, \gyyyo);
\coordinate (gpppgp) at (\gxxxg, \gyyyp);
\coordinate (gpppgq) at (\gxxxg, \gyyyq);
\coordinate (gpppgr) at (\gxxxg, \gyyyr);
\coordinate (gpppgs) at (\gxxxg, \gyyys);
\coordinate (gpppgt) at (\gxxxg, \gyyyt);
\coordinate (gpppgu) at (\gxxxg, \gyyyu);
\coordinate (gpppgv) at (\gxxxg, \gyyyv);
\coordinate (gpppgw) at (\gxxxg, \gyyyw);
\coordinate (gpppgx) at (\gxxxg, \gyyyx);
\coordinate (gpppgy) at (\gxxxg, \gyyyy);
\coordinate (gpppgz) at (\gxxxg, \gyyyz);
\coordinate (gpppha) at (\gxxxh, \gyyya);
\coordinate (gppphb) at (\gxxxh, \gyyyb);
\coordinate (gppphc) at (\gxxxh, \gyyyc);
\coordinate (gppphd) at (\gxxxh, \gyyyd);
\coordinate (gppphe) at (\gxxxh, \gyyye);
\coordinate (gppphf) at (\gxxxh, \gyyyf);
\coordinate (gppphg) at (\gxxxh, \gyyyg);
\coordinate (gppphh) at (\gxxxh, \gyyyh);
\coordinate (gppphi) at (\gxxxh, \gyyyi);
\coordinate (gppphj) at (\gxxxh, \gyyyj);
\coordinate (gppphk) at (\gxxxh, \gyyyk);
\coordinate (gppphl) at (\gxxxh, \gyyyl);
\coordinate (gppphm) at (\gxxxh, \gyyym);
\coordinate (gppphn) at (\gxxxh, \gyyyn);
\coordinate (gpppho) at (\gxxxh, \gyyyo);
\coordinate (gppphp) at (\gxxxh, \gyyyp);
\coordinate (gppphq) at (\gxxxh, \gyyyq);
\coordinate (gppphr) at (\gxxxh, \gyyyr);
\coordinate (gppphs) at (\gxxxh, \gyyys);
\coordinate (gpppht) at (\gxxxh, \gyyyt);
\coordinate (gppphu) at (\gxxxh, \gyyyu);
\coordinate (gppphv) at (\gxxxh, \gyyyv);
\coordinate (gppphw) at (\gxxxh, \gyyyw);
\coordinate (gppphx) at (\gxxxh, \gyyyx);
\coordinate (gppphy) at (\gxxxh, \gyyyy);
\coordinate (gppphz) at (\gxxxh, \gyyyz);
\coordinate (gpppia) at (\gxxxi, \gyyya);
\coordinate (gpppib) at (\gxxxi, \gyyyb);
\coordinate (gpppic) at (\gxxxi, \gyyyc);
\coordinate (gpppid) at (\gxxxi, \gyyyd);
\coordinate (gpppie) at (\gxxxi, \gyyye);
\coordinate (gpppif) at (\gxxxi, \gyyyf);
\coordinate (gpppig) at (\gxxxi, \gyyyg);
\coordinate (gpppih) at (\gxxxi, \gyyyh);
\coordinate (gpppii) at (\gxxxi, \gyyyi);
\coordinate (gpppij) at (\gxxxi, \gyyyj);
\coordinate (gpppik) at (\gxxxi, \gyyyk);
\coordinate (gpppil) at (\gxxxi, \gyyyl);
\coordinate (gpppim) at (\gxxxi, \gyyym);
\coordinate (gpppin) at (\gxxxi, \gyyyn);
\coordinate (gpppio) at (\gxxxi, \gyyyo);
\coordinate (gpppip) at (\gxxxi, \gyyyp);
\coordinate (gpppiq) at (\gxxxi, \gyyyq);
\coordinate (gpppir) at (\gxxxi, \gyyyr);
\coordinate (gpppis) at (\gxxxi, \gyyys);
\coordinate (gpppit) at (\gxxxi, \gyyyt);
\coordinate (gpppiu) at (\gxxxi, \gyyyu);
\coordinate (gpppiv) at (\gxxxi, \gyyyv);
\coordinate (gpppiw) at (\gxxxi, \gyyyw);
\coordinate (gpppix) at (\gxxxi, \gyyyx);
\coordinate (gpppiy) at (\gxxxi, \gyyyy);
\coordinate (gpppiz) at (\gxxxi, \gyyyz);
\coordinate (gpppja) at (\gxxxj, \gyyya);
\coordinate (gpppjb) at (\gxxxj, \gyyyb);
\coordinate (gpppjc) at (\gxxxj, \gyyyc);
\coordinate (gpppjd) at (\gxxxj, \gyyyd);
\coordinate (gpppje) at (\gxxxj, \gyyye);
\coordinate (gpppjf) at (\gxxxj, \gyyyf);
\coordinate (gpppjg) at (\gxxxj, \gyyyg);
\coordinate (gpppjh) at (\gxxxj, \gyyyh);
\coordinate (gpppji) at (\gxxxj, \gyyyi);
\coordinate (gpppjj) at (\gxxxj, \gyyyj);
\coordinate (gpppjk) at (\gxxxj, \gyyyk);
\coordinate (gpppjl) at (\gxxxj, \gyyyl);
\coordinate (gpppjm) at (\gxxxj, \gyyym);
\coordinate (gpppjn) at (\gxxxj, \gyyyn);
\coordinate (gpppjo) at (\gxxxj, \gyyyo);
\coordinate (gpppjp) at (\gxxxj, \gyyyp);
\coordinate (gpppjq) at (\gxxxj, \gyyyq);
\coordinate (gpppjr) at (\gxxxj, \gyyyr);
\coordinate (gpppjs) at (\gxxxj, \gyyys);
\coordinate (gpppjt) at (\gxxxj, \gyyyt);
\coordinate (gpppju) at (\gxxxj, \gyyyu);
\coordinate (gpppjv) at (\gxxxj, \gyyyv);
\coordinate (gpppjw) at (\gxxxj, \gyyyw);
\coordinate (gpppjx) at (\gxxxj, \gyyyx);
\coordinate (gpppjy) at (\gxxxj, \gyyyy);
\coordinate (gpppjz) at (\gxxxj, \gyyyz);
\coordinate (gpppka) at (\gxxxk, \gyyya);
\coordinate (gpppkb) at (\gxxxk, \gyyyb);
\coordinate (gpppkc) at (\gxxxk, \gyyyc);
\coordinate (gpppkd) at (\gxxxk, \gyyyd);
\coordinate (gpppke) at (\gxxxk, \gyyye);
\coordinate (gpppkf) at (\gxxxk, \gyyyf);
\coordinate (gpppkg) at (\gxxxk, \gyyyg);
\coordinate (gpppkh) at (\gxxxk, \gyyyh);
\coordinate (gpppki) at (\gxxxk, \gyyyi);
\coordinate (gpppkj) at (\gxxxk, \gyyyj);
\coordinate (gpppkk) at (\gxxxk, \gyyyk);
\coordinate (gpppkl) at (\gxxxk, \gyyyl);
\coordinate (gpppkm) at (\gxxxk, \gyyym);
\coordinate (gpppkn) at (\gxxxk, \gyyyn);
\coordinate (gpppko) at (\gxxxk, \gyyyo);
\coordinate (gpppkp) at (\gxxxk, \gyyyp);
\coordinate (gpppkq) at (\gxxxk, \gyyyq);
\coordinate (gpppkr) at (\gxxxk, \gyyyr);
\coordinate (gpppks) at (\gxxxk, \gyyys);
\coordinate (gpppkt) at (\gxxxk, \gyyyt);
\coordinate (gpppku) at (\gxxxk, \gyyyu);
\coordinate (gpppkv) at (\gxxxk, \gyyyv);
\coordinate (gpppkw) at (\gxxxk, \gyyyw);
\coordinate (gpppkx) at (\gxxxk, \gyyyx);
\coordinate (gpppky) at (\gxxxk, \gyyyy);
\coordinate (gpppkz) at (\gxxxk, \gyyyz);
\coordinate (gpppla) at (\gxxxl, \gyyya);
\coordinate (gppplb) at (\gxxxl, \gyyyb);
\coordinate (gppplc) at (\gxxxl, \gyyyc);
\coordinate (gpppld) at (\gxxxl, \gyyyd);
\coordinate (gppple) at (\gxxxl, \gyyye);
\coordinate (gppplf) at (\gxxxl, \gyyyf);
\coordinate (gppplg) at (\gxxxl, \gyyyg);
\coordinate (gppplh) at (\gxxxl, \gyyyh);
\coordinate (gpppli) at (\gxxxl, \gyyyi);
\coordinate (gppplj) at (\gxxxl, \gyyyj);
\coordinate (gppplk) at (\gxxxl, \gyyyk);
\coordinate (gpppll) at (\gxxxl, \gyyyl);
\coordinate (gppplm) at (\gxxxl, \gyyym);
\coordinate (gpppln) at (\gxxxl, \gyyyn);
\coordinate (gppplo) at (\gxxxl, \gyyyo);
\coordinate (gppplp) at (\gxxxl, \gyyyp);
\coordinate (gppplq) at (\gxxxl, \gyyyq);
\coordinate (gppplr) at (\gxxxl, \gyyyr);
\coordinate (gpppls) at (\gxxxl, \gyyys);
\coordinate (gppplt) at (\gxxxl, \gyyyt);
\coordinate (gppplu) at (\gxxxl, \gyyyu);
\coordinate (gppplv) at (\gxxxl, \gyyyv);
\coordinate (gppplw) at (\gxxxl, \gyyyw);
\coordinate (gppplx) at (\gxxxl, \gyyyx);
\coordinate (gppply) at (\gxxxl, \gyyyy);
\coordinate (gppplz) at (\gxxxl, \gyyyz);
\coordinate (gpppma) at (\gxxxm, \gyyya);
\coordinate (gpppmb) at (\gxxxm, \gyyyb);
\coordinate (gpppmc) at (\gxxxm, \gyyyc);
\coordinate (gpppmd) at (\gxxxm, \gyyyd);
\coordinate (gpppme) at (\gxxxm, \gyyye);
\coordinate (gpppmf) at (\gxxxm, \gyyyf);
\coordinate (gpppmg) at (\gxxxm, \gyyyg);
\coordinate (gpppmh) at (\gxxxm, \gyyyh);
\coordinate (gpppmi) at (\gxxxm, \gyyyi);
\coordinate (gpppmj) at (\gxxxm, \gyyyj);
\coordinate (gpppmk) at (\gxxxm, \gyyyk);
\coordinate (gpppml) at (\gxxxm, \gyyyl);
\coordinate (gpppmm) at (\gxxxm, \gyyym);
\coordinate (gpppmn) at (\gxxxm, \gyyyn);
\coordinate (gpppmo) at (\gxxxm, \gyyyo);
\coordinate (gpppmp) at (\gxxxm, \gyyyp);
\coordinate (gpppmq) at (\gxxxm, \gyyyq);
\coordinate (gpppmr) at (\gxxxm, \gyyyr);
\coordinate (gpppms) at (\gxxxm, \gyyys);
\coordinate (gpppmt) at (\gxxxm, \gyyyt);
\coordinate (gpppmu) at (\gxxxm, \gyyyu);
\coordinate (gpppmv) at (\gxxxm, \gyyyv);
\coordinate (gpppmw) at (\gxxxm, \gyyyw);
\coordinate (gpppmx) at (\gxxxm, \gyyyx);
\coordinate (gpppmy) at (\gxxxm, \gyyyy);
\coordinate (gpppmz) at (\gxxxm, \gyyyz);
\coordinate (gpppna) at (\gxxxn, \gyyya);
\coordinate (gpppnb) at (\gxxxn, \gyyyb);
\coordinate (gpppnc) at (\gxxxn, \gyyyc);
\coordinate (gpppnd) at (\gxxxn, \gyyyd);
\coordinate (gpppne) at (\gxxxn, \gyyye);
\coordinate (gpppnf) at (\gxxxn, \gyyyf);
\coordinate (gpppng) at (\gxxxn, \gyyyg);
\coordinate (gpppnh) at (\gxxxn, \gyyyh);
\coordinate (gpppni) at (\gxxxn, \gyyyi);
\coordinate (gpppnj) at (\gxxxn, \gyyyj);
\coordinate (gpppnk) at (\gxxxn, \gyyyk);
\coordinate (gpppnl) at (\gxxxn, \gyyyl);
\coordinate (gpppnm) at (\gxxxn, \gyyym);
\coordinate (gpppnn) at (\gxxxn, \gyyyn);
\coordinate (gpppno) at (\gxxxn, \gyyyo);
\coordinate (gpppnp) at (\gxxxn, \gyyyp);
\coordinate (gpppnq) at (\gxxxn, \gyyyq);
\coordinate (gpppnr) at (\gxxxn, \gyyyr);
\coordinate (gpppns) at (\gxxxn, \gyyys);
\coordinate (gpppnt) at (\gxxxn, \gyyyt);
\coordinate (gpppnu) at (\gxxxn, \gyyyu);
\coordinate (gpppnv) at (\gxxxn, \gyyyv);
\coordinate (gpppnw) at (\gxxxn, \gyyyw);
\coordinate (gpppnx) at (\gxxxn, \gyyyx);
\coordinate (gpppny) at (\gxxxn, \gyyyy);
\coordinate (gpppnz) at (\gxxxn, \gyyyz);
\coordinate (gpppoa) at (\gxxxo, \gyyya);
\coordinate (gpppob) at (\gxxxo, \gyyyb);
\coordinate (gpppoc) at (\gxxxo, \gyyyc);
\coordinate (gpppod) at (\gxxxo, \gyyyd);
\coordinate (gpppoe) at (\gxxxo, \gyyye);
\coordinate (gpppof) at (\gxxxo, \gyyyf);
\coordinate (gpppog) at (\gxxxo, \gyyyg);
\coordinate (gpppoh) at (\gxxxo, \gyyyh);
\coordinate (gpppoi) at (\gxxxo, \gyyyi);
\coordinate (gpppoj) at (\gxxxo, \gyyyj);
\coordinate (gpppok) at (\gxxxo, \gyyyk);
\coordinate (gpppol) at (\gxxxo, \gyyyl);
\coordinate (gpppom) at (\gxxxo, \gyyym);
\coordinate (gpppon) at (\gxxxo, \gyyyn);
\coordinate (gpppoo) at (\gxxxo, \gyyyo);
\coordinate (gpppop) at (\gxxxo, \gyyyp);
\coordinate (gpppoq) at (\gxxxo, \gyyyq);
\coordinate (gpppor) at (\gxxxo, \gyyyr);
\coordinate (gpppos) at (\gxxxo, \gyyys);
\coordinate (gpppot) at (\gxxxo, \gyyyt);
\coordinate (gpppou) at (\gxxxo, \gyyyu);
\coordinate (gpppov) at (\gxxxo, \gyyyv);
\coordinate (gpppow) at (\gxxxo, \gyyyw);
\coordinate (gpppox) at (\gxxxo, \gyyyx);
\coordinate (gpppoy) at (\gxxxo, \gyyyy);
\coordinate (gpppoz) at (\gxxxo, \gyyyz);
\coordinate (gppppa) at (\gxxxp, \gyyya);
\coordinate (gppppb) at (\gxxxp, \gyyyb);
\coordinate (gppppc) at (\gxxxp, \gyyyc);
\coordinate (gppppd) at (\gxxxp, \gyyyd);
\coordinate (gppppe) at (\gxxxp, \gyyye);
\coordinate (gppppf) at (\gxxxp, \gyyyf);
\coordinate (gppppg) at (\gxxxp, \gyyyg);
\coordinate (gpppph) at (\gxxxp, \gyyyh);
\coordinate (gppppi) at (\gxxxp, \gyyyi);
\coordinate (gppppj) at (\gxxxp, \gyyyj);
\coordinate (gppppk) at (\gxxxp, \gyyyk);
\coordinate (gppppl) at (\gxxxp, \gyyyl);
\coordinate (gppppm) at (\gxxxp, \gyyym);
\coordinate (gppppn) at (\gxxxp, \gyyyn);
\coordinate (gppppo) at (\gxxxp, \gyyyo);
\coordinate (gppppp) at (\gxxxp, \gyyyp);
\coordinate (gppppq) at (\gxxxp, \gyyyq);
\coordinate (gppppr) at (\gxxxp, \gyyyr);
\coordinate (gpppps) at (\gxxxp, \gyyys);
\coordinate (gppppt) at (\gxxxp, \gyyyt);
\coordinate (gppppu) at (\gxxxp, \gyyyu);
\coordinate (gppppv) at (\gxxxp, \gyyyv);
\coordinate (gppppw) at (\gxxxp, \gyyyw);
\coordinate (gppppx) at (\gxxxp, \gyyyx);
\coordinate (gppppy) at (\gxxxp, \gyyyy);
\coordinate (gppppz) at (\gxxxp, \gyyyz);
\coordinate (gpppqa) at (\gxxxq, \gyyya);
\coordinate (gpppqb) at (\gxxxq, \gyyyb);
\coordinate (gpppqc) at (\gxxxq, \gyyyc);
\coordinate (gpppqd) at (\gxxxq, \gyyyd);
\coordinate (gpppqe) at (\gxxxq, \gyyye);
\coordinate (gpppqf) at (\gxxxq, \gyyyf);
\coordinate (gpppqg) at (\gxxxq, \gyyyg);
\coordinate (gpppqh) at (\gxxxq, \gyyyh);
\coordinate (gpppqi) at (\gxxxq, \gyyyi);
\coordinate (gpppqj) at (\gxxxq, \gyyyj);
\coordinate (gpppqk) at (\gxxxq, \gyyyk);
\coordinate (gpppql) at (\gxxxq, \gyyyl);
\coordinate (gpppqm) at (\gxxxq, \gyyym);
\coordinate (gpppqn) at (\gxxxq, \gyyyn);
\coordinate (gpppqo) at (\gxxxq, \gyyyo);
\coordinate (gpppqp) at (\gxxxq, \gyyyp);
\coordinate (gpppqq) at (\gxxxq, \gyyyq);
\coordinate (gpppqr) at (\gxxxq, \gyyyr);
\coordinate (gpppqs) at (\gxxxq, \gyyys);
\coordinate (gpppqt) at (\gxxxq, \gyyyt);
\coordinate (gpppqu) at (\gxxxq, \gyyyu);
\coordinate (gpppqv) at (\gxxxq, \gyyyv);
\coordinate (gpppqw) at (\gxxxq, \gyyyw);
\coordinate (gpppqx) at (\gxxxq, \gyyyx);
\coordinate (gpppqy) at (\gxxxq, \gyyyy);
\coordinate (gpppqz) at (\gxxxq, \gyyyz);
\coordinate (gpppra) at (\gxxxr, \gyyya);
\coordinate (gppprb) at (\gxxxr, \gyyyb);
\coordinate (gppprc) at (\gxxxr, \gyyyc);
\coordinate (gppprd) at (\gxxxr, \gyyyd);
\coordinate (gpppre) at (\gxxxr, \gyyye);
\coordinate (gppprf) at (\gxxxr, \gyyyf);
\coordinate (gppprg) at (\gxxxr, \gyyyg);
\coordinate (gppprh) at (\gxxxr, \gyyyh);
\coordinate (gpppri) at (\gxxxr, \gyyyi);
\coordinate (gppprj) at (\gxxxr, \gyyyj);
\coordinate (gppprk) at (\gxxxr, \gyyyk);
\coordinate (gppprl) at (\gxxxr, \gyyyl);
\coordinate (gppprm) at (\gxxxr, \gyyym);
\coordinate (gppprn) at (\gxxxr, \gyyyn);
\coordinate (gpppro) at (\gxxxr, \gyyyo);
\coordinate (gppprp) at (\gxxxr, \gyyyp);
\coordinate (gppprq) at (\gxxxr, \gyyyq);
\coordinate (gppprr) at (\gxxxr, \gyyyr);
\coordinate (gppprs) at (\gxxxr, \gyyys);
\coordinate (gppprt) at (\gxxxr, \gyyyt);
\coordinate (gpppru) at (\gxxxr, \gyyyu);
\coordinate (gppprv) at (\gxxxr, \gyyyv);
\coordinate (gppprw) at (\gxxxr, \gyyyw);
\coordinate (gppprx) at (\gxxxr, \gyyyx);
\coordinate (gpppry) at (\gxxxr, \gyyyy);
\coordinate (gppprz) at (\gxxxr, \gyyyz);
\coordinate (gpppsa) at (\gxxxs, \gyyya);
\coordinate (gpppsb) at (\gxxxs, \gyyyb);
\coordinate (gpppsc) at (\gxxxs, \gyyyc);
\coordinate (gpppsd) at (\gxxxs, \gyyyd);
\coordinate (gpppse) at (\gxxxs, \gyyye);
\coordinate (gpppsf) at (\gxxxs, \gyyyf);
\coordinate (gpppsg) at (\gxxxs, \gyyyg);
\coordinate (gpppsh) at (\gxxxs, \gyyyh);
\coordinate (gpppsi) at (\gxxxs, \gyyyi);
\coordinate (gpppsj) at (\gxxxs, \gyyyj);
\coordinate (gpppsk) at (\gxxxs, \gyyyk);
\coordinate (gpppsl) at (\gxxxs, \gyyyl);
\coordinate (gpppsm) at (\gxxxs, \gyyym);
\coordinate (gpppsn) at (\gxxxs, \gyyyn);
\coordinate (gpppso) at (\gxxxs, \gyyyo);
\coordinate (gpppsp) at (\gxxxs, \gyyyp);
\coordinate (gpppsq) at (\gxxxs, \gyyyq);
\coordinate (gpppsr) at (\gxxxs, \gyyyr);
\coordinate (gpppss) at (\gxxxs, \gyyys);
\coordinate (gpppst) at (\gxxxs, \gyyyt);
\coordinate (gpppsu) at (\gxxxs, \gyyyu);
\coordinate (gpppsv) at (\gxxxs, \gyyyv);
\coordinate (gpppsw) at (\gxxxs, \gyyyw);
\coordinate (gpppsx) at (\gxxxs, \gyyyx);
\coordinate (gpppsy) at (\gxxxs, \gyyyy);
\coordinate (gpppsz) at (\gxxxs, \gyyyz);
\coordinate (gpppta) at (\gxxxt, \gyyya);
\coordinate (gppptb) at (\gxxxt, \gyyyb);
\coordinate (gppptc) at (\gxxxt, \gyyyc);
\coordinate (gppptd) at (\gxxxt, \gyyyd);
\coordinate (gpppte) at (\gxxxt, \gyyye);
\coordinate (gppptf) at (\gxxxt, \gyyyf);
\coordinate (gppptg) at (\gxxxt, \gyyyg);
\coordinate (gpppth) at (\gxxxt, \gyyyh);
\coordinate (gpppti) at (\gxxxt, \gyyyi);
\coordinate (gppptj) at (\gxxxt, \gyyyj);
\coordinate (gppptk) at (\gxxxt, \gyyyk);
\coordinate (gppptl) at (\gxxxt, \gyyyl);
\coordinate (gppptm) at (\gxxxt, \gyyym);
\coordinate (gppptn) at (\gxxxt, \gyyyn);
\coordinate (gpppto) at (\gxxxt, \gyyyo);
\coordinate (gppptp) at (\gxxxt, \gyyyp);
\coordinate (gppptq) at (\gxxxt, \gyyyq);
\coordinate (gppptr) at (\gxxxt, \gyyyr);
\coordinate (gpppts) at (\gxxxt, \gyyys);
\coordinate (gppptt) at (\gxxxt, \gyyyt);
\coordinate (gppptu) at (\gxxxt, \gyyyu);
\coordinate (gppptv) at (\gxxxt, \gyyyv);
\coordinate (gppptw) at (\gxxxt, \gyyyw);
\coordinate (gppptx) at (\gxxxt, \gyyyx);
\coordinate (gpppty) at (\gxxxt, \gyyyy);
\coordinate (gppptz) at (\gxxxt, \gyyyz);
\coordinate (gpppua) at (\gxxxu, \gyyya);
\coordinate (gpppub) at (\gxxxu, \gyyyb);
\coordinate (gpppuc) at (\gxxxu, \gyyyc);
\coordinate (gpppud) at (\gxxxu, \gyyyd);
\coordinate (gpppue) at (\gxxxu, \gyyye);
\coordinate (gpppuf) at (\gxxxu, \gyyyf);
\coordinate (gpppug) at (\gxxxu, \gyyyg);
\coordinate (gpppuh) at (\gxxxu, \gyyyh);
\coordinate (gpppui) at (\gxxxu, \gyyyi);
\coordinate (gpppuj) at (\gxxxu, \gyyyj);
\coordinate (gpppuk) at (\gxxxu, \gyyyk);
\coordinate (gpppul) at (\gxxxu, \gyyyl);
\coordinate (gpppum) at (\gxxxu, \gyyym);
\coordinate (gpppun) at (\gxxxu, \gyyyn);
\coordinate (gpppuo) at (\gxxxu, \gyyyo);
\coordinate (gpppup) at (\gxxxu, \gyyyp);
\coordinate (gpppuq) at (\gxxxu, \gyyyq);
\coordinate (gpppur) at (\gxxxu, \gyyyr);
\coordinate (gpppus) at (\gxxxu, \gyyys);
\coordinate (gppput) at (\gxxxu, \gyyyt);
\coordinate (gpppuu) at (\gxxxu, \gyyyu);
\coordinate (gpppuv) at (\gxxxu, \gyyyv);
\coordinate (gpppuw) at (\gxxxu, \gyyyw);
\coordinate (gpppux) at (\gxxxu, \gyyyx);
\coordinate (gpppuy) at (\gxxxu, \gyyyy);
\coordinate (gpppuz) at (\gxxxu, \gyyyz);
\coordinate (gpppva) at (\gxxxv, \gyyya);
\coordinate (gpppvb) at (\gxxxv, \gyyyb);
\coordinate (gpppvc) at (\gxxxv, \gyyyc);
\coordinate (gpppvd) at (\gxxxv, \gyyyd);
\coordinate (gpppve) at (\gxxxv, \gyyye);
\coordinate (gpppvf) at (\gxxxv, \gyyyf);
\coordinate (gpppvg) at (\gxxxv, \gyyyg);
\coordinate (gpppvh) at (\gxxxv, \gyyyh);
\coordinate (gpppvi) at (\gxxxv, \gyyyi);
\coordinate (gpppvj) at (\gxxxv, \gyyyj);
\coordinate (gpppvk) at (\gxxxv, \gyyyk);
\coordinate (gpppvl) at (\gxxxv, \gyyyl);
\coordinate (gpppvm) at (\gxxxv, \gyyym);
\coordinate (gpppvn) at (\gxxxv, \gyyyn);
\coordinate (gpppvo) at (\gxxxv, \gyyyo);
\coordinate (gpppvp) at (\gxxxv, \gyyyp);
\coordinate (gpppvq) at (\gxxxv, \gyyyq);
\coordinate (gpppvr) at (\gxxxv, \gyyyr);
\coordinate (gpppvs) at (\gxxxv, \gyyys);
\coordinate (gpppvt) at (\gxxxv, \gyyyt);
\coordinate (gpppvu) at (\gxxxv, \gyyyu);
\coordinate (gpppvv) at (\gxxxv, \gyyyv);
\coordinate (gpppvw) at (\gxxxv, \gyyyw);
\coordinate (gpppvx) at (\gxxxv, \gyyyx);
\coordinate (gpppvy) at (\gxxxv, \gyyyy);
\coordinate (gpppvz) at (\gxxxv, \gyyyz);
\coordinate (gpppwa) at (\gxxxw, \gyyya);
\coordinate (gpppwb) at (\gxxxw, \gyyyb);
\coordinate (gpppwc) at (\gxxxw, \gyyyc);
\coordinate (gpppwd) at (\gxxxw, \gyyyd);
\coordinate (gpppwe) at (\gxxxw, \gyyye);
\coordinate (gpppwf) at (\gxxxw, \gyyyf);
\coordinate (gpppwg) at (\gxxxw, \gyyyg);
\coordinate (gpppwh) at (\gxxxw, \gyyyh);
\coordinate (gpppwi) at (\gxxxw, \gyyyi);
\coordinate (gpppwj) at (\gxxxw, \gyyyj);
\coordinate (gpppwk) at (\gxxxw, \gyyyk);
\coordinate (gpppwl) at (\gxxxw, \gyyyl);
\coordinate (gpppwm) at (\gxxxw, \gyyym);
\coordinate (gpppwn) at (\gxxxw, \gyyyn);
\coordinate (gpppwo) at (\gxxxw, \gyyyo);
\coordinate (gpppwp) at (\gxxxw, \gyyyp);
\coordinate (gpppwq) at (\gxxxw, \gyyyq);
\coordinate (gpppwr) at (\gxxxw, \gyyyr);
\coordinate (gpppws) at (\gxxxw, \gyyys);
\coordinate (gpppwt) at (\gxxxw, \gyyyt);
\coordinate (gpppwu) at (\gxxxw, \gyyyu);
\coordinate (gpppwv) at (\gxxxw, \gyyyv);
\coordinate (gpppww) at (\gxxxw, \gyyyw);
\coordinate (gpppwx) at (\gxxxw, \gyyyx);
\coordinate (gpppwy) at (\gxxxw, \gyyyy);
\coordinate (gpppwz) at (\gxxxw, \gyyyz);
\coordinate (gpppxa) at (\gxxxx, \gyyya);
\coordinate (gpppxb) at (\gxxxx, \gyyyb);
\coordinate (gpppxc) at (\gxxxx, \gyyyc);
\coordinate (gpppxd) at (\gxxxx, \gyyyd);
\coordinate (gpppxe) at (\gxxxx, \gyyye);
\coordinate (gpppxf) at (\gxxxx, \gyyyf);
\coordinate (gpppxg) at (\gxxxx, \gyyyg);
\coordinate (gpppxh) at (\gxxxx, \gyyyh);
\coordinate (gpppxi) at (\gxxxx, \gyyyi);
\coordinate (gpppxj) at (\gxxxx, \gyyyj);
\coordinate (gpppxk) at (\gxxxx, \gyyyk);
\coordinate (gpppxl) at (\gxxxx, \gyyyl);
\coordinate (gpppxm) at (\gxxxx, \gyyym);
\coordinate (gpppxn) at (\gxxxx, \gyyyn);
\coordinate (gpppxo) at (\gxxxx, \gyyyo);
\coordinate (gpppxp) at (\gxxxx, \gyyyp);
\coordinate (gpppxq) at (\gxxxx, \gyyyq);
\coordinate (gpppxr) at (\gxxxx, \gyyyr);
\coordinate (gpppxs) at (\gxxxx, \gyyys);
\coordinate (gpppxt) at (\gxxxx, \gyyyt);
\coordinate (gpppxu) at (\gxxxx, \gyyyu);
\coordinate (gpppxv) at (\gxxxx, \gyyyv);
\coordinate (gpppxw) at (\gxxxx, \gyyyw);
\coordinate (gpppxx) at (\gxxxx, \gyyyx);
\coordinate (gpppxy) at (\gxxxx, \gyyyy);
\coordinate (gpppxz) at (\gxxxx, \gyyyz);
\coordinate (gpppya) at (\gxxxy, \gyyya);
\coordinate (gpppyb) at (\gxxxy, \gyyyb);
\coordinate (gpppyc) at (\gxxxy, \gyyyc);
\coordinate (gpppyd) at (\gxxxy, \gyyyd);
\coordinate (gpppye) at (\gxxxy, \gyyye);
\coordinate (gpppyf) at (\gxxxy, \gyyyf);
\coordinate (gpppyg) at (\gxxxy, \gyyyg);
\coordinate (gpppyh) at (\gxxxy, \gyyyh);
\coordinate (gpppyi) at (\gxxxy, \gyyyi);
\coordinate (gpppyj) at (\gxxxy, \gyyyj);
\coordinate (gpppyk) at (\gxxxy, \gyyyk);
\coordinate (gpppyl) at (\gxxxy, \gyyyl);
\coordinate (gpppym) at (\gxxxy, \gyyym);
\coordinate (gpppyn) at (\gxxxy, \gyyyn);
\coordinate (gpppyo) at (\gxxxy, \gyyyo);
\coordinate (gpppyp) at (\gxxxy, \gyyyp);
\coordinate (gpppyq) at (\gxxxy, \gyyyq);
\coordinate (gpppyr) at (\gxxxy, \gyyyr);
\coordinate (gpppys) at (\gxxxy, \gyyys);
\coordinate (gpppyt) at (\gxxxy, \gyyyt);
\coordinate (gpppyu) at (\gxxxy, \gyyyu);
\coordinate (gpppyv) at (\gxxxy, \gyyyv);
\coordinate (gpppyw) at (\gxxxy, \gyyyw);
\coordinate (gpppyx) at (\gxxxy, \gyyyx);
\coordinate (gpppyy) at (\gxxxy, \gyyyy);
\coordinate (gpppyz) at (\gxxxy, \gyyyz);
\coordinate (gpppza) at (\gxxxz, \gyyya);
\coordinate (gpppzb) at (\gxxxz, \gyyyb);
\coordinate (gpppzc) at (\gxxxz, \gyyyc);
\coordinate (gpppzd) at (\gxxxz, \gyyyd);
\coordinate (gpppze) at (\gxxxz, \gyyye);
\coordinate (gpppzf) at (\gxxxz, \gyyyf);
\coordinate (gpppzg) at (\gxxxz, \gyyyg);
\coordinate (gpppzh) at (\gxxxz, \gyyyh);
\coordinate (gpppzi) at (\gxxxz, \gyyyi);
\coordinate (gpppzj) at (\gxxxz, \gyyyj);
\coordinate (gpppzk) at (\gxxxz, \gyyyk);
\coordinate (gpppzl) at (\gxxxz, \gyyyl);
\coordinate (gpppzm) at (\gxxxz, \gyyym);
\coordinate (gpppzn) at (\gxxxz, \gyyyn);
\coordinate (gpppzo) at (\gxxxz, \gyyyo);
\coordinate (gpppzp) at (\gxxxz, \gyyyp);
\coordinate (gpppzq) at (\gxxxz, \gyyyq);
\coordinate (gpppzr) at (\gxxxz, \gyyyr);
\coordinate (gpppzs) at (\gxxxz, \gyyys);
\coordinate (gpppzt) at (\gxxxz, \gyyyt);
\coordinate (gpppzu) at (\gxxxz, \gyyyu);
\coordinate (gpppzv) at (\gxxxz, \gyyyv);
\coordinate (gpppzw) at (\gxxxz, \gyyyw);
\coordinate (gpppzx) at (\gxxxz, \gyyyx);
\coordinate (gpppzy) at (\gxxxz, \gyyyy);
\coordinate (gpppzz) at (\gxxxz, \gyyyz);

%\gangprintcoordinateat{(0,0)}{The last coordinate values: }{($(gpppzz)$)}; 


% Draw related part of the coordinate system with dashed helplines (centered at (gpppii)) with letters as background, which would help to determine all coordinates. 
%\coordinatebackground{g}{c}{d}{o};

% Step 2, draw key devices, their accessories, and take related coordinates of their pins, and may define more coordinates. 

% Draw the Opamp at the coordinate (gpppii) and name it as "swopamp".
\draw (gpppii) node [op amp, yscale=-1] (swopamp) {\ctikzflipy{Opamp}} ; 

% Its accessories and lables. 
\draw [-*](swopamp.down) -- ($(swopamp.down)+(0,1)$) node[right]{$V_+$}; 
\node at ($(swopamp.down)+(0.3,0.2)$) {7};  
\draw [-*](swopamp.up) -- ($(swopamp.up)+(0,-1)$) node[right]{$V_-$}; 
\node at ($(swopamp.up)+(0.3,-0.2)$) {4};

% Get the x- and y-components of the coordinates of the "+" and "-" pins. 
\getxyingivenunit{cm}{(swopamp.+)}{\swopampzx}{\swopampzy};
\getxyingivenunit{cm}{(swopamp.-)}{\swopampfx}{\swopampfy};

% Then define a few more coordinates, at least for keeping in mind.
\coordinate (plusshort) at ($(\gxxxg,\swopampzy)$);
%\fill  (plusshort) circle (2pt);  % May be commented later.
\coordinate (minusshort) at ($(\gxxxg,\swopampfy)$);
%\fill  (minusshort) circle (2pt); % May be commented later.
\coordinate (leftinter) at ($(\gxxxe,\swopampzy)$);
\fill  (leftinter) circle (2pt);

% Draw an "npn" at (gpppmi) and name it as "swQ".
\draw (gpppmi) node[npn](swQ){};

% Get the x- and y-components of the needed pins of it for later usage.
\getxyingivenunit{cm}{(swQ.C)}{\swQCx}{\swQCy};
\getxyingivenunit{cm}{(swQ.E)}{\swQEx}{\swQEy};

% Then define more coordinate(s).
\coordinate (Qcshort) at ($(\swQEx,\gyyyj)$);
%\fill  (Qcshort) circle (2pt); % May be commented later.
\coordinate (Qeshort) at ($(\swQEx,\gyyyf)$);
\fill  (Qeshort) circle (2pt) node [right] {$V_0$};

% Then the rectangle by the points (gpppef) -- (gpppej) -- (Qcshort) -- (Qeshort) forms a clear area for the key devices. 

% Connect the two devices.
\draw (swopamp.out) to [short, l=$I_B$, above] (swQ.B);

% Step 3, draw other little devices. For tidiness, better to give two units in length for each new device and align them up.

% For this specific circuit, let us attach the four bi-pole devices (maybe with their accessories) to each corner of the above mentioned rectangle area for the key devices, separately. 

\draw  (gppped) node [ground] {} to [empty ZZener diode] (gpppef) -- (leftinter);
% The Latex system can not work properly when I put the following label into the above "[empty ZZener diode]" in the form of an additional "l= ..." option, then I have to employ the following "\node ..." command. Then the label should be aligned with the "ZZener" as much as possible even the coordinate system is modified later. Since the "\gyyyd" and "\gyyyf" micros are used to position the "ZZener", then better to use the center between them to locate the label, rather than using "\gyyye" directly. This idea is also applied in later "\node ..." commands. 
\node at ($(\gxxxe-1.3, \gyyyd*0.5+\gyyyf*0.5)$) {$V_Z = 5\textnormal{V}$};

\draw (gpppej) to [generic] (gpppel) -| (swQ.C);
\node at ($(\gxxxe-1.1, \gyyyj*0.5+\gyyyl*0.5)$) {$R_{1}=47k\Omega$};
\node [right] at (\swQCx,\gyyyj*0.5+\gyyyl*0.5) {$I_C \approx \beta I_B$};

\draw  (\swQEx, \gyyyd) node [ground] {} to [generic] (\swQEx, \gyyyf) -- (swQ.E);
\node at ($(\swQEx+1.4, \gyyyd*0.5+\gyyyf*0.5)$) {$R_{E}=100k\Omega$};

% Draw the top area. 
\draw  (\gxxxi-0.2,\gyyyn) --  (\gxxxi+0.2,\gyyyn) node [right] {$V_{cc}=15\textnormal{V}$} ;
\draw [->] (gpppin) -- (gpppim) node [right] {$I$};  
\draw  (gpppim) -- (gpppil);
\fill  (gpppil) circle (2pt);

% Step 4, other shorts.
\draw  (swopamp.+)  to [short, l_=$I_+ \approx 0 $, above] (plusshort) -- (leftinter) -- (gpppej);

\draw  (swopamp.-)  to [short, l_=$I_- \approx 0 $, above] (minusshort) |- (Qeshort);

%Step 5, all the rest, especially labels. May also clean unnecessary staff, like the system background and dark points for showing newly defined coordinates previously. 
\draw [->] ($(\swQEx-0.4, \gyyyf - 0.4)$) -- node [left] {$I_E$} ($(\swQEx-0.4, \gyyyd + 0.4)$);

\draw [->] ($(\gxxxe+0.4, \gyyyl*0.5+\gyyyj*0.5 + 0.6)$) -- node [right] {$I_1$} ($(\gxxxe+0.4, \gyyyl*0.5+\gyyyj*0.5 - 0.6)$);


\pgfmathsetmacro{\bigswQCx}{\swQCx}


















% The following is the first circuit, which uses "h" coordinate system, to be inserted into and connected to the above circuit.
% It is modified based on Figure 2.

% Circuits can be drawn by the following five major steps, as shown in the following example. 

% Step 1, preparations. 

% "Install" the coordinate system with keyword "h".
\pgfmathsetmacro{\totalhxxx}{26}
\pgfmathsetmacro{\totalhyyy}{26}
\pgfmathsetmacro{\hxxxspacing}{1}
\pgfmathsetmacro{\hyyyspacing}{1}
\pgfmathsetmacro{\hxxxa}{-2.5}
\pgfmathsetmacro{\hyyya}{06.5}

\pgfmathsetmacro{\hxxxb}{\hxxxa + \hxxxspacing + 0.0 }
\pgfmathsetmacro{\hxxxc}{\hxxxb + \hxxxspacing + 0.0 }
\pgfmathsetmacro{\hxxxd}{\hxxxc + \hxxxspacing + 0.0 }
\pgfmathsetmacro{\hxxxe}{\hxxxd + \hxxxspacing + 0.0 }
\pgfmathsetmacro{\hxxxf}{\hxxxe + \hxxxspacing + 0.0 }
\pgfmathsetmacro{\hxxxg}{\hxxxf + \hxxxspacing + 0.0 }
\pgfmathsetmacro{\hxxxh}{\hxxxg + \hxxxspacing + 0.0 }
\pgfmathsetmacro{\hxxxi}{\hxxxh + \hxxxspacing + 0.0 }
\pgfmathsetmacro{\hxxxj}{\hxxxi + \hxxxspacing + 0.0 }
\pgfmathsetmacro{\hxxxk}{\hxxxj + \hxxxspacing + 0.0 }
\pgfmathsetmacro{\hxxxl}{\hxxxk + \hxxxspacing + 0.0 }
\pgfmathsetmacro{\hxxxm}{\hxxxl + \hxxxspacing + 0.0 }
\pgfmathsetmacro{\hxxxn}{\hxxxm + \hxxxspacing + 0.0 }
\pgfmathsetmacro{\hxxxo}{\hxxxn + \hxxxspacing + 0.0 }
\pgfmathsetmacro{\hxxxp}{\hxxxo + \hxxxspacing + 0.0 }
\pgfmathsetmacro{\hxxxq}{\hxxxp + \hxxxspacing + 0.0 }
\pgfmathsetmacro{\hxxxr}{\hxxxq + \hxxxspacing + 0.0 }
\pgfmathsetmacro{\hxxxs}{\hxxxr + \hxxxspacing + 0.0 }
\pgfmathsetmacro{\hxxxt}{\hxxxs + \hxxxspacing + 0.0 }
\pgfmathsetmacro{\hxxxu}{\hxxxt + \hxxxspacing + 0.0 }
\pgfmathsetmacro{\hxxxv}{\hxxxu + \hxxxspacing + 0.0 }
\pgfmathsetmacro{\hxxxw}{\hxxxv + \hxxxspacing + 0.0 }
\pgfmathsetmacro{\hxxxx}{\hxxxw + \hxxxspacing + 0.0 }
\pgfmathsetmacro{\hxxxy}{\hxxxx + \hxxxspacing + 0.0 }
\pgfmathsetmacro{\hxxxz}{\hxxxy + \hxxxspacing + 0.0 }

\pgfmathsetmacro{\hyyyb}{\hyyya + \hyyyspacing + 0.0 }
\pgfmathsetmacro{\hyyyc}{\hyyyb + \hyyyspacing + 0.0 }
\pgfmathsetmacro{\hyyyd}{\hyyyc + \hyyyspacing + 0.0 }
\pgfmathsetmacro{\hyyye}{\hyyyd + \hyyyspacing + 0.0 }
\pgfmathsetmacro{\hyyyf}{\hyyye + \hyyyspacing + 0.0 }
\pgfmathsetmacro{\hyyyg}{\hyyyf + \hyyyspacing + 0.0 }
\pgfmathsetmacro{\hyyyh}{\hyyyg + \hyyyspacing + 0.0 }
\pgfmathsetmacro{\hyyyi}{\hyyyh + \hyyyspacing  -1.0 }
\pgfmathsetmacro{\hyyyj}{\hyyyi + \hyyyspacing + 0.0 }
\pgfmathsetmacro{\hyyyk}{\hyyyj + \hyyyspacing + 0.0 }
\pgfmathsetmacro{\hyyyl}{\hyyyk + \hyyyspacing + 0.0 }
\pgfmathsetmacro{\hyyym}{\hyyyl + \hyyyspacing + 0.0 }
\pgfmathsetmacro{\hyyyn}{\hyyym + \hyyyspacing + 0.0 }
\pgfmathsetmacro{\hyyyo}{\hyyyn + \hyyyspacing + 0.0 }
\pgfmathsetmacro{\hyyyp}{\hyyyo + \hyyyspacing + 0.0 }
\pgfmathsetmacro{\hyyyq}{\hyyyp + \hyyyspacing + 0.0 }
\pgfmathsetmacro{\hyyyr}{\hyyyq + \hyyyspacing + 0.0 }
\pgfmathsetmacro{\hyyys}{\hyyyr + \hyyyspacing + 0.0 }
\pgfmathsetmacro{\hyyyt}{\hyyys + \hyyyspacing + 0.0 }
\pgfmathsetmacro{\hyyyu}{\hyyyt + \hyyyspacing + 0.0 }
\pgfmathsetmacro{\hyyyv}{\hyyyu + \hyyyspacing + 0.0 }
\pgfmathsetmacro{\hyyyw}{\hyyyv + \hyyyspacing + 0.0 }
\pgfmathsetmacro{\hyyyx}{\hyyyw + \hyyyspacing + 0.0 }
\pgfmathsetmacro{\hyyyy}{\hyyyx + \hyyyspacing + 0.0 }
\pgfmathsetmacro{\hyyyz}{\hyyyy + \hyyyspacing + 0.0 }

\coordinate (hpppaa) at (\hxxxa, \hyyya);
\coordinate (hpppab) at (\hxxxa, \hyyyb);
\coordinate (hpppac) at (\hxxxa, \hyyyc);
\coordinate (hpppad) at (\hxxxa, \hyyyd);
\coordinate (hpppae) at (\hxxxa, \hyyye);
\coordinate (hpppaf) at (\hxxxa, \hyyyf);
\coordinate (hpppag) at (\hxxxa, \hyyyg);
\coordinate (hpppah) at (\hxxxa, \hyyyh);
\coordinate (hpppai) at (\hxxxa, \hyyyi);
\coordinate (hpppaj) at (\hxxxa, \hyyyj);
\coordinate (hpppak) at (\hxxxa, \hyyyk);
\coordinate (hpppal) at (\hxxxa, \hyyyl);
\coordinate (hpppam) at (\hxxxa, \hyyym);
\coordinate (hpppan) at (\hxxxa, \hyyyn);
\coordinate (hpppao) at (\hxxxa, \hyyyo);
\coordinate (hpppap) at (\hxxxa, \hyyyp);
\coordinate (hpppaq) at (\hxxxa, \hyyyq);
\coordinate (hpppar) at (\hxxxa, \hyyyr);
\coordinate (hpppas) at (\hxxxa, \hyyys);
\coordinate (hpppat) at (\hxxxa, \hyyyt);
\coordinate (hpppau) at (\hxxxa, \hyyyu);
\coordinate (hpppav) at (\hxxxa, \hyyyv);
\coordinate (hpppaw) at (\hxxxa, \hyyyw);
\coordinate (hpppax) at (\hxxxa, \hyyyx);
\coordinate (hpppay) at (\hxxxa, \hyyyy);
\coordinate (hpppaz) at (\hxxxa, \hyyyz);
\coordinate (hpppba) at (\hxxxb, \hyyya);
\coordinate (hpppbb) at (\hxxxb, \hyyyb);
\coordinate (hpppbc) at (\hxxxb, \hyyyc);
\coordinate (hpppbd) at (\hxxxb, \hyyyd);
\coordinate (hpppbe) at (\hxxxb, \hyyye);
\coordinate (hpppbf) at (\hxxxb, \hyyyf);
\coordinate (hpppbg) at (\hxxxb, \hyyyg);
\coordinate (hpppbh) at (\hxxxb, \hyyyh);
\coordinate (hpppbi) at (\hxxxb, \hyyyi);
\coordinate (hpppbj) at (\hxxxb, \hyyyj);
\coordinate (hpppbk) at (\hxxxb, \hyyyk);
\coordinate (hpppbl) at (\hxxxb, \hyyyl);
\coordinate (hpppbm) at (\hxxxb, \hyyym);
\coordinate (hpppbn) at (\hxxxb, \hyyyn);
\coordinate (hpppbo) at (\hxxxb, \hyyyo);
\coordinate (hpppbp) at (\hxxxb, \hyyyp);
\coordinate (hpppbq) at (\hxxxb, \hyyyq);
\coordinate (hpppbr) at (\hxxxb, \hyyyr);
\coordinate (hpppbs) at (\hxxxb, \hyyys);
\coordinate (hpppbt) at (\hxxxb, \hyyyt);
\coordinate (hpppbu) at (\hxxxb, \hyyyu);
\coordinate (hpppbv) at (\hxxxb, \hyyyv);
\coordinate (hpppbw) at (\hxxxb, \hyyyw);
\coordinate (hpppbx) at (\hxxxb, \hyyyx);
\coordinate (hpppby) at (\hxxxb, \hyyyy);
\coordinate (hpppbz) at (\hxxxb, \hyyyz);
\coordinate (hpppca) at (\hxxxc, \hyyya);
\coordinate (hpppcb) at (\hxxxc, \hyyyb);
\coordinate (hpppcc) at (\hxxxc, \hyyyc);
\coordinate (hpppcd) at (\hxxxc, \hyyyd);
\coordinate (hpppce) at (\hxxxc, \hyyye);
\coordinate (hpppcf) at (\hxxxc, \hyyyf);
\coordinate (hpppcg) at (\hxxxc, \hyyyg);
\coordinate (hpppch) at (\hxxxc, \hyyyh);
\coordinate (hpppci) at (\hxxxc, \hyyyi);
\coordinate (hpppcj) at (\hxxxc, \hyyyj);
\coordinate (hpppck) at (\hxxxc, \hyyyk);
\coordinate (hpppcl) at (\hxxxc, \hyyyl);
\coordinate (hpppcm) at (\hxxxc, \hyyym);
\coordinate (hpppcn) at (\hxxxc, \hyyyn);
\coordinate (hpppco) at (\hxxxc, \hyyyo);
\coordinate (hpppcp) at (\hxxxc, \hyyyp);
\coordinate (hpppcq) at (\hxxxc, \hyyyq);
\coordinate (hpppcr) at (\hxxxc, \hyyyr);
\coordinate (hpppcs) at (\hxxxc, \hyyys);
\coordinate (hpppct) at (\hxxxc, \hyyyt);
\coordinate (hpppcu) at (\hxxxc, \hyyyu);
\coordinate (hpppcv) at (\hxxxc, \hyyyv);
\coordinate (hpppcw) at (\hxxxc, \hyyyw);
\coordinate (hpppcx) at (\hxxxc, \hyyyx);
\coordinate (hpppcy) at (\hxxxc, \hyyyy);
\coordinate (hpppcz) at (\hxxxc, \hyyyz);
\coordinate (hpppda) at (\hxxxd, \hyyya);
\coordinate (hpppdb) at (\hxxxd, \hyyyb);
\coordinate (hpppdc) at (\hxxxd, \hyyyc);
\coordinate (hpppdd) at (\hxxxd, \hyyyd);
\coordinate (hpppde) at (\hxxxd, \hyyye);
\coordinate (hpppdf) at (\hxxxd, \hyyyf);
\coordinate (hpppdg) at (\hxxxd, \hyyyg);
\coordinate (hpppdh) at (\hxxxd, \hyyyh);
\coordinate (hpppdi) at (\hxxxd, \hyyyi);
\coordinate (hpppdj) at (\hxxxd, \hyyyj);
\coordinate (hpppdk) at (\hxxxd, \hyyyk);
\coordinate (hpppdl) at (\hxxxd, \hyyyl);
\coordinate (hpppdm) at (\hxxxd, \hyyym);
\coordinate (hpppdn) at (\hxxxd, \hyyyn);
\coordinate (hpppdo) at (\hxxxd, \hyyyo);
\coordinate (hpppdp) at (\hxxxd, \hyyyp);
\coordinate (hpppdq) at (\hxxxd, \hyyyq);
\coordinate (hpppdr) at (\hxxxd, \hyyyr);
\coordinate (hpppds) at (\hxxxd, \hyyys);
\coordinate (hpppdt) at (\hxxxd, \hyyyt);
\coordinate (hpppdu) at (\hxxxd, \hyyyu);
\coordinate (hpppdv) at (\hxxxd, \hyyyv);
\coordinate (hpppdw) at (\hxxxd, \hyyyw);
\coordinate (hpppdx) at (\hxxxd, \hyyyx);
\coordinate (hpppdy) at (\hxxxd, \hyyyy);
\coordinate (hpppdz) at (\hxxxd, \hyyyz);
\coordinate (hpppea) at (\hxxxe, \hyyya);
\coordinate (hpppeb) at (\hxxxe, \hyyyb);
\coordinate (hpppec) at (\hxxxe, \hyyyc);
\coordinate (hppped) at (\hxxxe, \hyyyd);
\coordinate (hpppee) at (\hxxxe, \hyyye);
\coordinate (hpppef) at (\hxxxe, \hyyyf);
\coordinate (hpppeg) at (\hxxxe, \hyyyg);
\coordinate (hpppeh) at (\hxxxe, \hyyyh);
\coordinate (hpppei) at (\hxxxe, \hyyyi);
\coordinate (hpppej) at (\hxxxe, \hyyyj);
\coordinate (hpppek) at (\hxxxe, \hyyyk);
\coordinate (hpppel) at (\hxxxe, \hyyyl);
\coordinate (hpppem) at (\hxxxe, \hyyym);
\coordinate (hpppen) at (\hxxxe, \hyyyn);
\coordinate (hpppeo) at (\hxxxe, \hyyyo);
\coordinate (hpppep) at (\hxxxe, \hyyyp);
\coordinate (hpppeq) at (\hxxxe, \hyyyq);
\coordinate (hppper) at (\hxxxe, \hyyyr);
\coordinate (hpppes) at (\hxxxe, \hyyys);
\coordinate (hpppet) at (\hxxxe, \hyyyt);
\coordinate (hpppeu) at (\hxxxe, \hyyyu);
\coordinate (hpppev) at (\hxxxe, \hyyyv);
\coordinate (hpppew) at (\hxxxe, \hyyyw);
\coordinate (hpppex) at (\hxxxe, \hyyyx);
\coordinate (hpppey) at (\hxxxe, \hyyyy);
\coordinate (hpppez) at (\hxxxe, \hyyyz);
\coordinate (hpppfa) at (\hxxxf, \hyyya);
\coordinate (hpppfb) at (\hxxxf, \hyyyb);
\coordinate (hpppfc) at (\hxxxf, \hyyyc);
\coordinate (hpppfd) at (\hxxxf, \hyyyd);
\coordinate (hpppfe) at (\hxxxf, \hyyye);
\coordinate (hpppff) at (\hxxxf, \hyyyf);
\coordinate (hpppfg) at (\hxxxf, \hyyyg);
\coordinate (hpppfh) at (\hxxxf, \hyyyh);
\coordinate (hpppfi) at (\hxxxf, \hyyyi);
\coordinate (hpppfj) at (\hxxxf, \hyyyj);
\coordinate (hpppfk) at (\hxxxf, \hyyyk);
\coordinate (hpppfl) at (\hxxxf, \hyyyl);
\coordinate (hpppfm) at (\hxxxf, \hyyym);
\coordinate (hpppfn) at (\hxxxf, \hyyyn);
\coordinate (hpppfo) at (\hxxxf, \hyyyo);
\coordinate (hpppfp) at (\hxxxf, \hyyyp);
\coordinate (hpppfq) at (\hxxxf, \hyyyq);
\coordinate (hpppfr) at (\hxxxf, \hyyyr);
\coordinate (hpppfs) at (\hxxxf, \hyyys);
\coordinate (hpppft) at (\hxxxf, \hyyyt);
\coordinate (hpppfu) at (\hxxxf, \hyyyu);
\coordinate (hpppfv) at (\hxxxf, \hyyyv);
\coordinate (hpppfw) at (\hxxxf, \hyyyw);
\coordinate (hpppfx) at (\hxxxf, \hyyyx);
\coordinate (hpppfy) at (\hxxxf, \hyyyy);
\coordinate (hpppfz) at (\hxxxf, \hyyyz);
\coordinate (hpppga) at (\hxxxg, \hyyya);
\coordinate (hpppgb) at (\hxxxg, \hyyyb);
\coordinate (hpppgc) at (\hxxxg, \hyyyc);
\coordinate (hpppgd) at (\hxxxg, \hyyyd);
\coordinate (hpppge) at (\hxxxg, \hyyye);
\coordinate (hpppgf) at (\hxxxg, \hyyyf);
\coordinate (hpppgg) at (\hxxxg, \hyyyg);
\coordinate (hpppgh) at (\hxxxg, \hyyyh);
\coordinate (hpppgi) at (\hxxxg, \hyyyi);
\coordinate (hpppgj) at (\hxxxg, \hyyyj);
\coordinate (hpppgk) at (\hxxxg, \hyyyk);
\coordinate (hpppgl) at (\hxxxg, \hyyyl);
\coordinate (hpppgm) at (\hxxxg, \hyyym);
\coordinate (hpppgn) at (\hxxxg, \hyyyn);
\coordinate (hpppgo) at (\hxxxg, \hyyyo);
\coordinate (hpppgp) at (\hxxxg, \hyyyp);
\coordinate (hpppgq) at (\hxxxg, \hyyyq);
\coordinate (hpppgr) at (\hxxxg, \hyyyr);
\coordinate (hpppgs) at (\hxxxg, \hyyys);
\coordinate (hpppgt) at (\hxxxg, \hyyyt);
\coordinate (hpppgu) at (\hxxxg, \hyyyu);
\coordinate (hpppgv) at (\hxxxg, \hyyyv);
\coordinate (hpppgw) at (\hxxxg, \hyyyw);
\coordinate (hpppgx) at (\hxxxg, \hyyyx);
\coordinate (hpppgy) at (\hxxxg, \hyyyy);
\coordinate (hpppgz) at (\hxxxg, \hyyyz);
\coordinate (hpppha) at (\hxxxh, \hyyya);
\coordinate (hppphb) at (\hxxxh, \hyyyb);
\coordinate (hppphc) at (\hxxxh, \hyyyc);
\coordinate (hppphd) at (\hxxxh, \hyyyd);
\coordinate (hppphe) at (\hxxxh, \hyyye);
\coordinate (hppphf) at (\hxxxh, \hyyyf);
\coordinate (hppphg) at (\hxxxh, \hyyyg);
\coordinate (hppphh) at (\hxxxh, \hyyyh);
\coordinate (hppphi) at (\hxxxh, \hyyyi);
\coordinate (hppphj) at (\hxxxh, \hyyyj);
\coordinate (hppphk) at (\hxxxh, \hyyyk);
\coordinate (hppphl) at (\hxxxh, \hyyyl);
\coordinate (hppphm) at (\hxxxh, \hyyym);
\coordinate (hppphn) at (\hxxxh, \hyyyn);
\coordinate (hpppho) at (\hxxxh, \hyyyo);
\coordinate (hppphp) at (\hxxxh, \hyyyp);
\coordinate (hppphq) at (\hxxxh, \hyyyq);
\coordinate (hppphr) at (\hxxxh, \hyyyr);
\coordinate (hppphs) at (\hxxxh, \hyyys);
\coordinate (hpppht) at (\hxxxh, \hyyyt);
\coordinate (hppphu) at (\hxxxh, \hyyyu);
\coordinate (hppphv) at (\hxxxh, \hyyyv);
\coordinate (hppphw) at (\hxxxh, \hyyyw);
\coordinate (hppphx) at (\hxxxh, \hyyyx);
\coordinate (hppphy) at (\hxxxh, \hyyyy);
\coordinate (hppphz) at (\hxxxh, \hyyyz);
\coordinate (hpppia) at (\hxxxi, \hyyya);
\coordinate (hpppib) at (\hxxxi, \hyyyb);
\coordinate (hpppic) at (\hxxxi, \hyyyc);
\coordinate (hpppid) at (\hxxxi, \hyyyd);
\coordinate (hpppie) at (\hxxxi, \hyyye);
\coordinate (hpppif) at (\hxxxi, \hyyyf);
\coordinate (hpppig) at (\hxxxi, \hyyyg);
\coordinate (hpppih) at (\hxxxi, \hyyyh);
\coordinate (hpppii) at (\hxxxi, \hyyyi);
\coordinate (hpppij) at (\hxxxi, \hyyyj);
\coordinate (hpppik) at (\hxxxi, \hyyyk);
\coordinate (hpppil) at (\hxxxi, \hyyyl);
\coordinate (hpppim) at (\hxxxi, \hyyym);
\coordinate (hpppin) at (\hxxxi, \hyyyn);
\coordinate (hpppio) at (\hxxxi, \hyyyo);
\coordinate (hpppip) at (\hxxxi, \hyyyp);
\coordinate (hpppiq) at (\hxxxi, \hyyyq);
\coordinate (hpppir) at (\hxxxi, \hyyyr);
\coordinate (hpppis) at (\hxxxi, \hyyys);
\coordinate (hpppit) at (\hxxxi, \hyyyt);
\coordinate (hpppiu) at (\hxxxi, \hyyyu);
\coordinate (hpppiv) at (\hxxxi, \hyyyv);
\coordinate (hpppiw) at (\hxxxi, \hyyyw);
\coordinate (hpppix) at (\hxxxi, \hyyyx);
\coordinate (hpppiy) at (\hxxxi, \hyyyy);
\coordinate (hpppiz) at (\hxxxi, \hyyyz);
\coordinate (hpppja) at (\hxxxj, \hyyya);
\coordinate (hpppjb) at (\hxxxj, \hyyyb);
\coordinate (hpppjc) at (\hxxxj, \hyyyc);
\coordinate (hpppjd) at (\hxxxj, \hyyyd);
\coordinate (hpppje) at (\hxxxj, \hyyye);
\coordinate (hpppjf) at (\hxxxj, \hyyyf);
\coordinate (hpppjg) at (\hxxxj, \hyyyg);
\coordinate (hpppjh) at (\hxxxj, \hyyyh);
\coordinate (hpppji) at (\hxxxj, \hyyyi);
\coordinate (hpppjj) at (\hxxxj, \hyyyj);
\coordinate (hpppjk) at (\hxxxj, \hyyyk);
\coordinate (hpppjl) at (\hxxxj, \hyyyl);
\coordinate (hpppjm) at (\hxxxj, \hyyym);
\coordinate (hpppjn) at (\hxxxj, \hyyyn);
\coordinate (hpppjo) at (\hxxxj, \hyyyo);
\coordinate (hpppjp) at (\hxxxj, \hyyyp);
\coordinate (hpppjq) at (\hxxxj, \hyyyq);
\coordinate (hpppjr) at (\hxxxj, \hyyyr);
\coordinate (hpppjs) at (\hxxxj, \hyyys);
\coordinate (hpppjt) at (\hxxxj, \hyyyt);
\coordinate (hpppju) at (\hxxxj, \hyyyu);
\coordinate (hpppjv) at (\hxxxj, \hyyyv);
\coordinate (hpppjw) at (\hxxxj, \hyyyw);
\coordinate (hpppjx) at (\hxxxj, \hyyyx);
\coordinate (hpppjy) at (\hxxxj, \hyyyy);
\coordinate (hpppjz) at (\hxxxj, \hyyyz);
\coordinate (hpppka) at (\hxxxk, \hyyya);
\coordinate (hpppkb) at (\hxxxk, \hyyyb);
\coordinate (hpppkc) at (\hxxxk, \hyyyc);
\coordinate (hpppkd) at (\hxxxk, \hyyyd);
\coordinate (hpppke) at (\hxxxk, \hyyye);
\coordinate (hpppkf) at (\hxxxk, \hyyyf);
\coordinate (hpppkg) at (\hxxxk, \hyyyg);
\coordinate (hpppkh) at (\hxxxk, \hyyyh);
\coordinate (hpppki) at (\hxxxk, \hyyyi);
\coordinate (hpppkj) at (\hxxxk, \hyyyj);
\coordinate (hpppkk) at (\hxxxk, \hyyyk);
\coordinate (hpppkl) at (\hxxxk, \hyyyl);
\coordinate (hpppkm) at (\hxxxk, \hyyym);
\coordinate (hpppkn) at (\hxxxk, \hyyyn);
\coordinate (hpppko) at (\hxxxk, \hyyyo);
\coordinate (hpppkp) at (\hxxxk, \hyyyp);
\coordinate (hpppkq) at (\hxxxk, \hyyyq);
\coordinate (hpppkr) at (\hxxxk, \hyyyr);
\coordinate (hpppks) at (\hxxxk, \hyyys);
\coordinate (hpppkt) at (\hxxxk, \hyyyt);
\coordinate (hpppku) at (\hxxxk, \hyyyu);
\coordinate (hpppkv) at (\hxxxk, \hyyyv);
\coordinate (hpppkw) at (\hxxxk, \hyyyw);
\coordinate (hpppkx) at (\hxxxk, \hyyyx);
\coordinate (hpppky) at (\hxxxk, \hyyyy);
\coordinate (hpppkz) at (\hxxxk, \hyyyz);
\coordinate (hpppla) at (\hxxxl, \hyyya);
\coordinate (hppplb) at (\hxxxl, \hyyyb);
\coordinate (hppplc) at (\hxxxl, \hyyyc);
\coordinate (hpppld) at (\hxxxl, \hyyyd);
\coordinate (hppple) at (\hxxxl, \hyyye);
\coordinate (hppplf) at (\hxxxl, \hyyyf);
\coordinate (hppplg) at (\hxxxl, \hyyyg);
\coordinate (hppplh) at (\hxxxl, \hyyyh);
\coordinate (hpppli) at (\hxxxl, \hyyyi);
\coordinate (hppplj) at (\hxxxl, \hyyyj);
\coordinate (hppplk) at (\hxxxl, \hyyyk);
\coordinate (hpppll) at (\hxxxl, \hyyyl);
\coordinate (hppplm) at (\hxxxl, \hyyym);
\coordinate (hpppln) at (\hxxxl, \hyyyn);
\coordinate (hppplo) at (\hxxxl, \hyyyo);
\coordinate (hppplp) at (\hxxxl, \hyyyp);
\coordinate (hppplq) at (\hxxxl, \hyyyq);
\coordinate (hppplr) at (\hxxxl, \hyyyr);
\coordinate (hpppls) at (\hxxxl, \hyyys);
\coordinate (hppplt) at (\hxxxl, \hyyyt);
\coordinate (hppplu) at (\hxxxl, \hyyyu);
\coordinate (hppplv) at (\hxxxl, \hyyyv);
\coordinate (hppplw) at (\hxxxl, \hyyyw);
\coordinate (hppplx) at (\hxxxl, \hyyyx);
\coordinate (hppply) at (\hxxxl, \hyyyy);
\coordinate (hppplz) at (\hxxxl, \hyyyz);
\coordinate (hpppma) at (\hxxxm, \hyyya);
\coordinate (hpppmb) at (\hxxxm, \hyyyb);
\coordinate (hpppmc) at (\hxxxm, \hyyyc);
\coordinate (hpppmd) at (\hxxxm, \hyyyd);
\coordinate (hpppme) at (\hxxxm, \hyyye);
\coordinate (hpppmf) at (\hxxxm, \hyyyf);
\coordinate (hpppmg) at (\hxxxm, \hyyyg);
\coordinate (hpppmh) at (\hxxxm, \hyyyh);
\coordinate (hpppmi) at (\hxxxm, \hyyyi);
\coordinate (hpppmj) at (\hxxxm, \hyyyj);
\coordinate (hpppmk) at (\hxxxm, \hyyyk);
\coordinate (hpppml) at (\hxxxm, \hyyyl);
\coordinate (hpppmm) at (\hxxxm, \hyyym);
\coordinate (hpppmn) at (\hxxxm, \hyyyn);
\coordinate (hpppmo) at (\hxxxm, \hyyyo);
\coordinate (hpppmp) at (\hxxxm, \hyyyp);
\coordinate (hpppmq) at (\hxxxm, \hyyyq);
\coordinate (hpppmr) at (\hxxxm, \hyyyr);
\coordinate (hpppms) at (\hxxxm, \hyyys);
\coordinate (hpppmt) at (\hxxxm, \hyyyt);
\coordinate (hpppmu) at (\hxxxm, \hyyyu);
\coordinate (hpppmv) at (\hxxxm, \hyyyv);
\coordinate (hpppmw) at (\hxxxm, \hyyyw);
\coordinate (hpppmx) at (\hxxxm, \hyyyx);
\coordinate (hpppmy) at (\hxxxm, \hyyyy);
\coordinate (hpppmz) at (\hxxxm, \hyyyz);
\coordinate (hpppna) at (\hxxxn, \hyyya);
\coordinate (hpppnb) at (\hxxxn, \hyyyb);
\coordinate (hpppnc) at (\hxxxn, \hyyyc);
\coordinate (hpppnd) at (\hxxxn, \hyyyd);
\coordinate (hpppne) at (\hxxxn, \hyyye);
\coordinate (hpppnf) at (\hxxxn, \hyyyf);
\coordinate (hpppng) at (\hxxxn, \hyyyg);
\coordinate (hpppnh) at (\hxxxn, \hyyyh);
\coordinate (hpppni) at (\hxxxn, \hyyyi);
\coordinate (hpppnj) at (\hxxxn, \hyyyj);
\coordinate (hpppnk) at (\hxxxn, \hyyyk);
\coordinate (hpppnl) at (\hxxxn, \hyyyl);
\coordinate (hpppnm) at (\hxxxn, \hyyym);
\coordinate (hpppnn) at (\hxxxn, \hyyyn);
\coordinate (hpppno) at (\hxxxn, \hyyyo);
\coordinate (hpppnp) at (\hxxxn, \hyyyp);
\coordinate (hpppnq) at (\hxxxn, \hyyyq);
\coordinate (hpppnr) at (\hxxxn, \hyyyr);
\coordinate (hpppns) at (\hxxxn, \hyyys);
\coordinate (hpppnt) at (\hxxxn, \hyyyt);
\coordinate (hpppnu) at (\hxxxn, \hyyyu);
\coordinate (hpppnv) at (\hxxxn, \hyyyv);
\coordinate (hpppnw) at (\hxxxn, \hyyyw);
\coordinate (hpppnx) at (\hxxxn, \hyyyx);
\coordinate (hpppny) at (\hxxxn, \hyyyy);
\coordinate (hpppnz) at (\hxxxn, \hyyyz);
\coordinate (hpppoa) at (\hxxxo, \hyyya);
\coordinate (hpppob) at (\hxxxo, \hyyyb);
\coordinate (hpppoc) at (\hxxxo, \hyyyc);
\coordinate (hpppod) at (\hxxxo, \hyyyd);
\coordinate (hpppoe) at (\hxxxo, \hyyye);
\coordinate (hpppof) at (\hxxxo, \hyyyf);
\coordinate (hpppog) at (\hxxxo, \hyyyg);
\coordinate (hpppoh) at (\hxxxo, \hyyyh);
\coordinate (hpppoi) at (\hxxxo, \hyyyi);
\coordinate (hpppoj) at (\hxxxo, \hyyyj);
\coordinate (hpppok) at (\hxxxo, \hyyyk);
\coordinate (hpppol) at (\hxxxo, \hyyyl);
\coordinate (hpppom) at (\hxxxo, \hyyym);
\coordinate (hpppon) at (\hxxxo, \hyyyn);
\coordinate (hpppoo) at (\hxxxo, \hyyyo);
\coordinate (hpppop) at (\hxxxo, \hyyyp);
\coordinate (hpppoq) at (\hxxxo, \hyyyq);
\coordinate (hpppor) at (\hxxxo, \hyyyr);
\coordinate (hpppos) at (\hxxxo, \hyyys);
\coordinate (hpppot) at (\hxxxo, \hyyyt);
\coordinate (hpppou) at (\hxxxo, \hyyyu);
\coordinate (hpppov) at (\hxxxo, \hyyyv);
\coordinate (hpppow) at (\hxxxo, \hyyyw);
\coordinate (hpppox) at (\hxxxo, \hyyyx);
\coordinate (hpppoy) at (\hxxxo, \hyyyy);
\coordinate (hpppoz) at (\hxxxo, \hyyyz);
\coordinate (hppppa) at (\hxxxp, \hyyya);
\coordinate (hppppb) at (\hxxxp, \hyyyb);
\coordinate (hppppc) at (\hxxxp, \hyyyc);
\coordinate (hppppd) at (\hxxxp, \hyyyd);
\coordinate (hppppe) at (\hxxxp, \hyyye);
\coordinate (hppppf) at (\hxxxp, \hyyyf);
\coordinate (hppppg) at (\hxxxp, \hyyyg);
\coordinate (hpppph) at (\hxxxp, \hyyyh);
\coordinate (hppppi) at (\hxxxp, \hyyyi);
\coordinate (hppppj) at (\hxxxp, \hyyyj);
\coordinate (hppppk) at (\hxxxp, \hyyyk);
\coordinate (hppppl) at (\hxxxp, \hyyyl);
\coordinate (hppppm) at (\hxxxp, \hyyym);
\coordinate (hppppn) at (\hxxxp, \hyyyn);
\coordinate (hppppo) at (\hxxxp, \hyyyo);
\coordinate (hppppp) at (\hxxxp, \hyyyp);
\coordinate (hppppq) at (\hxxxp, \hyyyq);
\coordinate (hppppr) at (\hxxxp, \hyyyr);
\coordinate (hpppps) at (\hxxxp, \hyyys);
\coordinate (hppppt) at (\hxxxp, \hyyyt);
\coordinate (hppppu) at (\hxxxp, \hyyyu);
\coordinate (hppppv) at (\hxxxp, \hyyyv);
\coordinate (hppppw) at (\hxxxp, \hyyyw);
\coordinate (hppppx) at (\hxxxp, \hyyyx);
\coordinate (hppppy) at (\hxxxp, \hyyyy);
\coordinate (hppppz) at (\hxxxp, \hyyyz);
\coordinate (hpppqa) at (\hxxxq, \hyyya);
\coordinate (hpppqb) at (\hxxxq, \hyyyb);
\coordinate (hpppqc) at (\hxxxq, \hyyyc);
\coordinate (hpppqd) at (\hxxxq, \hyyyd);
\coordinate (hpppqe) at (\hxxxq, \hyyye);
\coordinate (hpppqf) at (\hxxxq, \hyyyf);
\coordinate (hpppqg) at (\hxxxq, \hyyyg);
\coordinate (hpppqh) at (\hxxxq, \hyyyh);
\coordinate (hpppqi) at (\hxxxq, \hyyyi);
\coordinate (hpppqj) at (\hxxxq, \hyyyj);
\coordinate (hpppqk) at (\hxxxq, \hyyyk);
\coordinate (hpppql) at (\hxxxq, \hyyyl);
\coordinate (hpppqm) at (\hxxxq, \hyyym);
\coordinate (hpppqn) at (\hxxxq, \hyyyn);
\coordinate (hpppqo) at (\hxxxq, \hyyyo);
\coordinate (hpppqp) at (\hxxxq, \hyyyp);
\coordinate (hpppqq) at (\hxxxq, \hyyyq);
\coordinate (hpppqr) at (\hxxxq, \hyyyr);
\coordinate (hpppqs) at (\hxxxq, \hyyys);
\coordinate (hpppqt) at (\hxxxq, \hyyyt);
\coordinate (hpppqu) at (\hxxxq, \hyyyu);
\coordinate (hpppqv) at (\hxxxq, \hyyyv);
\coordinate (hpppqw) at (\hxxxq, \hyyyw);
\coordinate (hpppqx) at (\hxxxq, \hyyyx);
\coordinate (hpppqy) at (\hxxxq, \hyyyy);
\coordinate (hpppqz) at (\hxxxq, \hyyyz);
\coordinate (hpppra) at (\hxxxr, \hyyya);
\coordinate (hppprb) at (\hxxxr, \hyyyb);
\coordinate (hppprc) at (\hxxxr, \hyyyc);
\coordinate (hppprd) at (\hxxxr, \hyyyd);
\coordinate (hpppre) at (\hxxxr, \hyyye);
\coordinate (hppprf) at (\hxxxr, \hyyyf);
\coordinate (hppprg) at (\hxxxr, \hyyyg);
\coordinate (hppprh) at (\hxxxr, \hyyyh);
\coordinate (hpppri) at (\hxxxr, \hyyyi);
\coordinate (hppprj) at (\hxxxr, \hyyyj);
\coordinate (hppprk) at (\hxxxr, \hyyyk);
\coordinate (hppprl) at (\hxxxr, \hyyyl);
\coordinate (hppprm) at (\hxxxr, \hyyym);
\coordinate (hppprn) at (\hxxxr, \hyyyn);
\coordinate (hpppro) at (\hxxxr, \hyyyo);
\coordinate (hppprp) at (\hxxxr, \hyyyp);
\coordinate (hppprq) at (\hxxxr, \hyyyq);
\coordinate (hppprr) at (\hxxxr, \hyyyr);
\coordinate (hppprs) at (\hxxxr, \hyyys);
\coordinate (hppprt) at (\hxxxr, \hyyyt);
\coordinate (hpppru) at (\hxxxr, \hyyyu);
\coordinate (hppprv) at (\hxxxr, \hyyyv);
\coordinate (hppprw) at (\hxxxr, \hyyyw);
\coordinate (hppprx) at (\hxxxr, \hyyyx);
\coordinate (hpppry) at (\hxxxr, \hyyyy);
\coordinate (hppprz) at (\hxxxr, \hyyyz);
\coordinate (hpppsa) at (\hxxxs, \hyyya);
\coordinate (hpppsb) at (\hxxxs, \hyyyb);
\coordinate (hpppsc) at (\hxxxs, \hyyyc);
\coordinate (hpppsd) at (\hxxxs, \hyyyd);
\coordinate (hpppse) at (\hxxxs, \hyyye);
\coordinate (hpppsf) at (\hxxxs, \hyyyf);
\coordinate (hpppsg) at (\hxxxs, \hyyyg);
\coordinate (hpppsh) at (\hxxxs, \hyyyh);
\coordinate (hpppsi) at (\hxxxs, \hyyyi);
\coordinate (hpppsj) at (\hxxxs, \hyyyj);
\coordinate (hpppsk) at (\hxxxs, \hyyyk);
\coordinate (hpppsl) at (\hxxxs, \hyyyl);
\coordinate (hpppsm) at (\hxxxs, \hyyym);
\coordinate (hpppsn) at (\hxxxs, \hyyyn);
\coordinate (hpppso) at (\hxxxs, \hyyyo);
\coordinate (hpppsp) at (\hxxxs, \hyyyp);
\coordinate (hpppsq) at (\hxxxs, \hyyyq);
\coordinate (hpppsr) at (\hxxxs, \hyyyr);
\coordinate (hpppss) at (\hxxxs, \hyyys);
\coordinate (hpppst) at (\hxxxs, \hyyyt);
\coordinate (hpppsu) at (\hxxxs, \hyyyu);
\coordinate (hpppsv) at (\hxxxs, \hyyyv);
\coordinate (hpppsw) at (\hxxxs, \hyyyw);
\coordinate (hpppsx) at (\hxxxs, \hyyyx);
\coordinate (hpppsy) at (\hxxxs, \hyyyy);
\coordinate (hpppsz) at (\hxxxs, \hyyyz);
\coordinate (hpppta) at (\hxxxt, \hyyya);
\coordinate (hppptb) at (\hxxxt, \hyyyb);
\coordinate (hppptc) at (\hxxxt, \hyyyc);
\coordinate (hppptd) at (\hxxxt, \hyyyd);
\coordinate (hpppte) at (\hxxxt, \hyyye);
\coordinate (hppptf) at (\hxxxt, \hyyyf);
\coordinate (hppptg) at (\hxxxt, \hyyyg);
\coordinate (hpppth) at (\hxxxt, \hyyyh);
\coordinate (hpppti) at (\hxxxt, \hyyyi);
\coordinate (hppptj) at (\hxxxt, \hyyyj);
\coordinate (hppptk) at (\hxxxt, \hyyyk);
\coordinate (hppptl) at (\hxxxt, \hyyyl);
\coordinate (hppptm) at (\hxxxt, \hyyym);
\coordinate (hppptn) at (\hxxxt, \hyyyn);
\coordinate (hpppto) at (\hxxxt, \hyyyo);
\coordinate (hppptp) at (\hxxxt, \hyyyp);
\coordinate (hppptq) at (\hxxxt, \hyyyq);
\coordinate (hppptr) at (\hxxxt, \hyyyr);
\coordinate (hpppts) at (\hxxxt, \hyyys);
\coordinate (hppptt) at (\hxxxt, \hyyyt);
\coordinate (hppptu) at (\hxxxt, \hyyyu);
\coordinate (hppptv) at (\hxxxt, \hyyyv);
\coordinate (hppptw) at (\hxxxt, \hyyyw);
\coordinate (hppptx) at (\hxxxt, \hyyyx);
\coordinate (hpppty) at (\hxxxt, \hyyyy);
\coordinate (hppptz) at (\hxxxt, \hyyyz);
\coordinate (hpppua) at (\hxxxu, \hyyya);
\coordinate (hpppub) at (\hxxxu, \hyyyb);
\coordinate (hpppuc) at (\hxxxu, \hyyyc);
\coordinate (hpppud) at (\hxxxu, \hyyyd);
\coordinate (hpppue) at (\hxxxu, \hyyye);
\coordinate (hpppuf) at (\hxxxu, \hyyyf);
\coordinate (hpppug) at (\hxxxu, \hyyyg);
\coordinate (hpppuh) at (\hxxxu, \hyyyh);
\coordinate (hpppui) at (\hxxxu, \hyyyi);
\coordinate (hpppuj) at (\hxxxu, \hyyyj);
\coordinate (hpppuk) at (\hxxxu, \hyyyk);
\coordinate (hpppul) at (\hxxxu, \hyyyl);
\coordinate (hpppum) at (\hxxxu, \hyyym);
\coordinate (hpppun) at (\hxxxu, \hyyyn);
\coordinate (hpppuo) at (\hxxxu, \hyyyo);
\coordinate (hpppup) at (\hxxxu, \hyyyp);
\coordinate (hpppuq) at (\hxxxu, \hyyyq);
\coordinate (hpppur) at (\hxxxu, \hyyyr);
\coordinate (hpppus) at (\hxxxu, \hyyys);
\coordinate (hppput) at (\hxxxu, \hyyyt);
\coordinate (hpppuu) at (\hxxxu, \hyyyu);
\coordinate (hpppuv) at (\hxxxu, \hyyyv);
\coordinate (hpppuw) at (\hxxxu, \hyyyw);
\coordinate (hpppux) at (\hxxxu, \hyyyx);
\coordinate (hpppuy) at (\hxxxu, \hyyyy);
\coordinate (hpppuz) at (\hxxxu, \hyyyz);
\coordinate (hpppva) at (\hxxxv, \hyyya);
\coordinate (hpppvb) at (\hxxxv, \hyyyb);
\coordinate (hpppvc) at (\hxxxv, \hyyyc);
\coordinate (hpppvd) at (\hxxxv, \hyyyd);
\coordinate (hpppve) at (\hxxxv, \hyyye);
\coordinate (hpppvf) at (\hxxxv, \hyyyf);
\coordinate (hpppvg) at (\hxxxv, \hyyyg);
\coordinate (hpppvh) at (\hxxxv, \hyyyh);
\coordinate (hpppvi) at (\hxxxv, \hyyyi);
\coordinate (hpppvj) at (\hxxxv, \hyyyj);
\coordinate (hpppvk) at (\hxxxv, \hyyyk);
\coordinate (hpppvl) at (\hxxxv, \hyyyl);
\coordinate (hpppvm) at (\hxxxv, \hyyym);
\coordinate (hpppvn) at (\hxxxv, \hyyyn);
\coordinate (hpppvo) at (\hxxxv, \hyyyo);
\coordinate (hpppvp) at (\hxxxv, \hyyyp);
\coordinate (hpppvq) at (\hxxxv, \hyyyq);
\coordinate (hpppvr) at (\hxxxv, \hyyyr);
\coordinate (hpppvs) at (\hxxxv, \hyyys);
\coordinate (hpppvt) at (\hxxxv, \hyyyt);
\coordinate (hpppvu) at (\hxxxv, \hyyyu);
\coordinate (hpppvv) at (\hxxxv, \hyyyv);
\coordinate (hpppvw) at (\hxxxv, \hyyyw);
\coordinate (hpppvx) at (\hxxxv, \hyyyx);
\coordinate (hpppvy) at (\hxxxv, \hyyyy);
\coordinate (hpppvz) at (\hxxxv, \hyyyz);
\coordinate (hpppwa) at (\hxxxw, \hyyya);
\coordinate (hpppwb) at (\hxxxw, \hyyyb);
\coordinate (hpppwc) at (\hxxxw, \hyyyc);
\coordinate (hpppwd) at (\hxxxw, \hyyyd);
\coordinate (hpppwe) at (\hxxxw, \hyyye);
\coordinate (hpppwf) at (\hxxxw, \hyyyf);
\coordinate (hpppwg) at (\hxxxw, \hyyyg);
\coordinate (hpppwh) at (\hxxxw, \hyyyh);
\coordinate (hpppwi) at (\hxxxw, \hyyyi);
\coordinate (hpppwj) at (\hxxxw, \hyyyj);
\coordinate (hpppwk) at (\hxxxw, \hyyyk);
\coordinate (hpppwl) at (\hxxxw, \hyyyl);
\coordinate (hpppwm) at (\hxxxw, \hyyym);
\coordinate (hpppwn) at (\hxxxw, \hyyyn);
\coordinate (hpppwo) at (\hxxxw, \hyyyo);
\coordinate (hpppwp) at (\hxxxw, \hyyyp);
\coordinate (hpppwq) at (\hxxxw, \hyyyq);
\coordinate (hpppwr) at (\hxxxw, \hyyyr);
\coordinate (hpppws) at (\hxxxw, \hyyys);
\coordinate (hpppwt) at (\hxxxw, \hyyyt);
\coordinate (hpppwu) at (\hxxxw, \hyyyu);
\coordinate (hpppwv) at (\hxxxw, \hyyyv);
\coordinate (hpppww) at (\hxxxw, \hyyyw);
\coordinate (hpppwx) at (\hxxxw, \hyyyx);
\coordinate (hpppwy) at (\hxxxw, \hyyyy);
\coordinate (hpppwz) at (\hxxxw, \hyyyz);
\coordinate (hpppxa) at (\hxxxx, \hyyya);
\coordinate (hpppxb) at (\hxxxx, \hyyyb);
\coordinate (hpppxc) at (\hxxxx, \hyyyc);
\coordinate (hpppxd) at (\hxxxx, \hyyyd);
\coordinate (hpppxe) at (\hxxxx, \hyyye);
\coordinate (hpppxf) at (\hxxxx, \hyyyf);
\coordinate (hpppxg) at (\hxxxx, \hyyyg);
\coordinate (hpppxh) at (\hxxxx, \hyyyh);
\coordinate (hpppxi) at (\hxxxx, \hyyyi);
\coordinate (hpppxj) at (\hxxxx, \hyyyj);
\coordinate (hpppxk) at (\hxxxx, \hyyyk);
\coordinate (hpppxl) at (\hxxxx, \hyyyl);
\coordinate (hpppxm) at (\hxxxx, \hyyym);
\coordinate (hpppxn) at (\hxxxx, \hyyyn);
\coordinate (hpppxo) at (\hxxxx, \hyyyo);
\coordinate (hpppxp) at (\hxxxx, \hyyyp);
\coordinate (hpppxq) at (\hxxxx, \hyyyq);
\coordinate (hpppxr) at (\hxxxx, \hyyyr);
\coordinate (hpppxs) at (\hxxxx, \hyyys);
\coordinate (hpppxt) at (\hxxxx, \hyyyt);
\coordinate (hpppxu) at (\hxxxx, \hyyyu);
\coordinate (hpppxv) at (\hxxxx, \hyyyv);
\coordinate (hpppxw) at (\hxxxx, \hyyyw);
\coordinate (hpppxx) at (\hxxxx, \hyyyx);
\coordinate (hpppxy) at (\hxxxx, \hyyyy);
\coordinate (hpppxz) at (\hxxxx, \hyyyz);
\coordinate (hpppya) at (\hxxxy, \hyyya);
\coordinate (hpppyb) at (\hxxxy, \hyyyb);
\coordinate (hpppyc) at (\hxxxy, \hyyyc);
\coordinate (hpppyd) at (\hxxxy, \hyyyd);
\coordinate (hpppye) at (\hxxxy, \hyyye);
\coordinate (hpppyf) at (\hxxxy, \hyyyf);
\coordinate (hpppyg) at (\hxxxy, \hyyyg);
\coordinate (hpppyh) at (\hxxxy, \hyyyh);
\coordinate (hpppyi) at (\hxxxy, \hyyyi);
\coordinate (hpppyj) at (\hxxxy, \hyyyj);
\coordinate (hpppyk) at (\hxxxy, \hyyyk);
\coordinate (hpppyl) at (\hxxxy, \hyyyl);
\coordinate (hpppym) at (\hxxxy, \hyyym);
\coordinate (hpppyn) at (\hxxxy, \hyyyn);
\coordinate (hpppyo) at (\hxxxy, \hyyyo);
\coordinate (hpppyp) at (\hxxxy, \hyyyp);
\coordinate (hpppyq) at (\hxxxy, \hyyyq);
\coordinate (hpppyr) at (\hxxxy, \hyyyr);
\coordinate (hpppys) at (\hxxxy, \hyyys);
\coordinate (hpppyt) at (\hxxxy, \hyyyt);
\coordinate (hpppyu) at (\hxxxy, \hyyyu);
\coordinate (hpppyv) at (\hxxxy, \hyyyv);
\coordinate (hpppyw) at (\hxxxy, \hyyyw);
\coordinate (hpppyx) at (\hxxxy, \hyyyx);
\coordinate (hpppyy) at (\hxxxy, \hyyyy);
\coordinate (hpppyz) at (\hxxxy, \hyyyz);
\coordinate (hpppza) at (\hxxxz, \hyyya);
\coordinate (hpppzb) at (\hxxxz, \hyyyb);
\coordinate (hpppzc) at (\hxxxz, \hyyyc);
\coordinate (hpppzd) at (\hxxxz, \hyyyd);
\coordinate (hpppze) at (\hxxxz, \hyyye);
\coordinate (hpppzf) at (\hxxxz, \hyyyf);
\coordinate (hpppzg) at (\hxxxz, \hyyyg);
\coordinate (hpppzh) at (\hxxxz, \hyyyh);
\coordinate (hpppzi) at (\hxxxz, \hyyyi);
\coordinate (hpppzj) at (\hxxxz, \hyyyj);
\coordinate (hpppzk) at (\hxxxz, \hyyyk);
\coordinate (hpppzl) at (\hxxxz, \hyyyl);
\coordinate (hpppzm) at (\hxxxz, \hyyym);
\coordinate (hpppzn) at (\hxxxz, \hyyyn);
\coordinate (hpppzo) at (\hxxxz, \hyyyo);
\coordinate (hpppzp) at (\hxxxz, \hyyyp);
\coordinate (hpppzq) at (\hxxxz, \hyyyq);
\coordinate (hpppzr) at (\hxxxz, \hyyyr);
\coordinate (hpppzs) at (\hxxxz, \hyyys);
\coordinate (hpppzt) at (\hxxxz, \hyyyt);
\coordinate (hpppzu) at (\hxxxz, \hyyyu);
\coordinate (hpppzv) at (\hxxxz, \hyyyv);
\coordinate (hpppzw) at (\hxxxz, \hyyyw);
\coordinate (hpppzx) at (\hxxxz, \hyyyx);
\coordinate (hpppzy) at (\hxxxz, \hyyyy);
\coordinate (hpppzz) at (\hxxxz, \hyyyz);

%\gangprintcoordinateat{(0,0)}{The last coordinate values: }{($(hpppzz)$)}; 


% Draw related part of the coordinate system with dashed helplines (centered at (hpppii)) with letters as background, which would help to determine all coordinates. 
%\coordinatebackground{h}{c}{d}{o};

% Step 2, draw key devices, their accessories, and take related coordinates of their pins, and may define more coordinates. 

% Draw the Opamp at the coordinate (hpppii) and name it as "swopamp".
\draw (hpppii) node [op amp, yscale=-1] (swopamp) {\ctikzflipy{Opamp}} ; 

% Its accessories and lables. 
\draw [-*](swopamp.down) -- ($(swopamp.down)+(0,1)$) node[right]{$V_+$}; 
\node at ($(swopamp.down)+(0.3,0.2)$) {7};  
\draw [-*](swopamp.up) -- ($(swopamp.up)+(0,-1)$) node[right]{$V_-$}; 
\node at ($(swopamp.up)+(0.3,-0.2)$) {4};

% Get the x- and y-components of the coordinates of the "+" and "-" pins. 
\getxyingivenunit{cm}{(swopamp.+)}{\swopampzx}{\swopampzy};
\getxyingivenunit{cm}{(swopamp.-)}{\swopampfx}{\swopampfy};

% Then define a few more coordinates, at least for keeping in mind.
\coordinate (plusshort) at ($(\hxxxg,\swopampzy)$);
%\fill  (plusshort) circle (2pt);  % May be commented later.
\coordinate (minusshort) at ($(\hxxxg,\swopampfy)$);
%\fill  (minusshort) circle (2pt); % May be commented later.
\coordinate (leftinter) at ($(\hxxxe,\swopampzy)$);
\fill  (leftinter) circle (2pt);

% Draw an "npn" at (hpppmi) and name it as "swQ".
\draw (hpppmi) node[npn](swQ){};

% Get the x- and y-components of the needed pins of it for later usage.
\getxyingivenunit{cm}{(swQ.C)}{\swQCx}{\swQCy};
\getxyingivenunit{cm}{(swQ.E)}{\swQEx}{\swQEy};

% Then define more coordinate(s).
\coordinate (Qcshort) at ($(\swQEx,\hyyyj)$);
%\fill  (Qcshort) circle (2pt); % May be commented later.
\coordinate (Qeshort) at ($(\swQEx,\hyyyf)$);
\fill  (Qeshort) circle (2pt) node [right] {$V_0$};

% Then the rectangle by the points (hpppef) -- (hpppej) -- (Qcshort) -- (Qeshort) forms a clear area for the key devices. 

% Connect the two devices.
\draw (swopamp.out) to [short, l=$I_B$, above] (swQ.B);

% Step 3, draw other little devices. For tidiness, better to give two units in length for each new device and align them up.

% For this specific circuit, let us attach the four bi-pole devices (maybe with their accessories) to each corner of the above mentioned rectangle area for the key devices, separately. 

\fill  (\hxxxe,\gyyyi) circle (2pt);
\draw  (\hxxxe,\gyyyi) -- (hppped) to [empty ZZener diode] (hpppef) -- (leftinter);
% The Latex system can not work properly when I put the following label into the above "[empty ZZener diode]" in the form of an additional "l= ..." option, then I have to employ the following "\node ..." command. Then the label should be aligned with the "ZZener" as much as possible even the coordinate system is modified later. Since the "\hyyyd" and "\hyyyf" micros are used to position the "ZZener", then better to use the center between them to locate the label, rather than using "\hyyye" directly. This idea is also applied in later "\node ..." commands. 
\node at ($(\hxxxe-1.3, \hyyyd*0.5+\hyyyf*0.5)$) {$V_Z = 5\textnormal{V}$};

\draw (hpppej) to [generic] (hpppel) -| (swQ.C);
\node at ($(\hxxxe-1.1, \hyyyj*0.5+\hyyyl*0.5)$) {$R_{1}=47k\Omega$};
\node [right] at (\swQCx,\hyyyj*0.5+\hyyyl*0.5) {$I_C \approx \beta I_B$};

\fill  (\bigswQCx, \hyyyd) circle (2pt);
\draw  (\bigswQCx, \hyyyd) -- (\swQEx, \hyyyd) to [generic] (\swQEx, \hyyyf) -- (swQ.E);
\node at ($(\swQEx+1.4, \hyyyd*0.5+\hyyyf*0.5)$) {$R_{E}=100k\Omega$};

% Draw the top area. 
%\draw  (\hxxxi-0.2,\hyyyn) --  (\hxxxi+0.2,\hyyyn) node [right] {$V_{cc}=15\textnormal{V}$} ;
%\draw [->] (hpppin) -- (hpppim) node [right] {$I$};  
%\draw  (hpppim) -- (hpppil);
%\fill  (hpppil) circle (2pt);

% Step 4, other shorts.
\draw  (swopamp.+)  to [short, l_=$I_+ \approx 0 $, above] (plusshort) -- (leftinter) -- (hpppej);

\draw  (swopamp.-)  to [short, l_=$I_- \approx 0 $, above] (minusshort) |- (Qeshort);

%Step 5, all the rest, especially labels. May also clean unnecessary staff, like the system background and dark points for showing newly defined coordinates previously. 
\draw [->] ($(\swQEx-0.4, \hyyyf - 0.4)$) -- node [left] {$I_E$} ($(\swQEx-0.4, \hyyyd + 0.4)$);

\draw [->] ($(\hxxxe+0.4, \hyyyl*0.5+\hyyyj*0.5 + 0.6)$) -- node [right] {$I_1$} ($(\hxxxe+0.4, \hyyyl*0.5+\hyyyj*0.5 - 0.6)$);

\pgfmathsetmacro{\secondopampzy}{\swopampzy}





















% The following is the second circuit, which uses "s" coordinate system, to be inserted into and connected the above circuit as well. It is modified based on Figure 6. 

\pgfmathsetmacro{\totalsxxx}{26}
\pgfmathsetmacro{\totalsyyy}{26}
\pgfmathsetmacro{\sxxxspacing}{1}
\pgfmathsetmacro{\syyyspacing}{1}
\pgfmathsetmacro{\sxxxa}{-3}
\pgfmathsetmacro{\syyya}{-2}

\pgfmathsetmacro{\sxxxb}{\sxxxa + \sxxxspacing + 0.0 }
\pgfmathsetmacro{\sxxxc}{\sxxxb + \sxxxspacing + 0.0 }
\pgfmathsetmacro{\sxxxd}{\sxxxc + \sxxxspacing + 0.0 }
\pgfmathsetmacro{\sxxxe}{\sxxxd + \sxxxspacing + 0.0 }
\pgfmathsetmacro{\sxxxf}{\sxxxe + \sxxxspacing + 0.0 }
\pgfmathsetmacro{\sxxxg}{\sxxxf + \sxxxspacing + 0.0 }
\pgfmathsetmacro{\sxxxh}{\sxxxg + \sxxxspacing + 0.0 }
\pgfmathsetmacro{\sxxxi}{\sxxxh + \sxxxspacing + 0.0 }
\pgfmathsetmacro{\sxxxj}{\sxxxi + \sxxxspacing + 0.0 }
\pgfmathsetmacro{\sxxxk}{\sxxxj + \sxxxspacing + 0.0 }
\pgfmathsetmacro{\sxxxl}{\sxxxk + \sxxxspacing + 0.0 }
\pgfmathsetmacro{\sxxxm}{\sxxxl + \sxxxspacing + 0.0 }
\pgfmathsetmacro{\sxxxn}{\sxxxm + \sxxxspacing + 0.0 }
\pgfmathsetmacro{\sxxxo}{\sxxxn + \sxxxspacing + 0.0 }
\pgfmathsetmacro{\sxxxp}{\sxxxo + \sxxxspacing + 0.0 }
\pgfmathsetmacro{\sxxxq}{\sxxxp + \sxxxspacing + 0.0 }
\pgfmathsetmacro{\sxxxr}{\sxxxq + \sxxxspacing + 0.0 }
\pgfmathsetmacro{\sxxxs}{\sxxxr + \sxxxspacing + 0.0 }
\pgfmathsetmacro{\sxxxt}{\sxxxs + \sxxxspacing + 0.0 }
\pgfmathsetmacro{\sxxxu}{\sxxxt + \sxxxspacing + 0.0 }
\pgfmathsetmacro{\sxxxv}{\sxxxu + \sxxxspacing + 0.0 }
\pgfmathsetmacro{\sxxxw}{\sxxxv + \sxxxspacing + 0.0 }
\pgfmathsetmacro{\sxxxx}{\sxxxw + \sxxxspacing + 0.0 }
\pgfmathsetmacro{\sxxxy}{\sxxxx + \sxxxspacing + 0.0 }
\pgfmathsetmacro{\sxxxz}{\sxxxy + \sxxxspacing + 0.0 }

\pgfmathsetmacro{\syyyb}{\syyya + \syyyspacing + 0.0 }
\pgfmathsetmacro{\syyyc}{\syyyb + \syyyspacing + 0.0 }
\pgfmathsetmacro{\syyyd}{\syyyc + \syyyspacing + 0.0 }
\pgfmathsetmacro{\syyye}{\syyyd + \syyyspacing + 0.0 }
\pgfmathsetmacro{\syyyf}{\syyye + \syyyspacing + 0.0 }
\pgfmathsetmacro{\syyyg}{\syyyf + \syyyspacing + 0.0 }
\pgfmathsetmacro{\syyyh}{\syyyg + \syyyspacing + 0.0 }
\pgfmathsetmacro{\syyyi}{\syyyh + \syyyspacing + 0.0 }
\pgfmathsetmacro{\syyyj}{\syyyi + \syyyspacing + 0.0 }
\pgfmathsetmacro{\syyyk}{\syyyj + \syyyspacing + 0.0 }
\pgfmathsetmacro{\syyyl}{\syyyk + \syyyspacing + 0.0 }
\pgfmathsetmacro{\syyym}{\syyyl + \syyyspacing + 0.0 }
\pgfmathsetmacro{\syyyn}{\syyym + \syyyspacing + 0.0 }
\pgfmathsetmacro{\syyyo}{\syyyn + \syyyspacing + 0.0 }
\pgfmathsetmacro{\syyyp}{\syyyo + \syyyspacing + 0.0 }
\pgfmathsetmacro{\syyyq}{\syyyp + \syyyspacing + 0.0 }
\pgfmathsetmacro{\syyyr}{\syyyq + \syyyspacing + 0.0 }
\pgfmathsetmacro{\syyys}{\syyyr + \syyyspacing + 0.0 }
\pgfmathsetmacro{\syyyt}{\syyys + \syyyspacing + 0.0 }
\pgfmathsetmacro{\syyyu}{\syyyt + \syyyspacing + 0.0 }
\pgfmathsetmacro{\syyyv}{\syyyu + \syyyspacing + 0.0 }
\pgfmathsetmacro{\syyyw}{\syyyv + \syyyspacing + 0.0 }
\pgfmathsetmacro{\syyyx}{\syyyw + \syyyspacing + 0.0 }
\pgfmathsetmacro{\syyyy}{\syyyx + \syyyspacing + 0.0 }
\pgfmathsetmacro{\syyyz}{\syyyy + \syyyspacing + 0.0 }

\coordinate (spppaa) at (\sxxxa, \syyya);
\coordinate (spppab) at (\sxxxa, \syyyb);
\coordinate (spppac) at (\sxxxa, \syyyc);
\coordinate (spppad) at (\sxxxa, \syyyd);
\coordinate (spppae) at (\sxxxa, \syyye);
\coordinate (spppaf) at (\sxxxa, \syyyf);
\coordinate (spppag) at (\sxxxa, \syyyg);
\coordinate (spppah) at (\sxxxa, \syyyh);
\coordinate (spppai) at (\sxxxa, \syyyi);
\coordinate (spppaj) at (\sxxxa, \syyyj);
\coordinate (spppak) at (\sxxxa, \syyyk);
\coordinate (spppal) at (\sxxxa, \syyyl);
\coordinate (spppam) at (\sxxxa, \syyym);
\coordinate (spppan) at (\sxxxa, \syyyn);
\coordinate (spppao) at (\sxxxa, \syyyo);
\coordinate (spppap) at (\sxxxa, \syyyp);
\coordinate (spppaq) at (\sxxxa, \syyyq);
\coordinate (spppar) at (\sxxxa, \syyyr);
\coordinate (spppas) at (\sxxxa, \syyys);
\coordinate (spppat) at (\sxxxa, \syyyt);
\coordinate (spppau) at (\sxxxa, \syyyu);
\coordinate (spppav) at (\sxxxa, \syyyv);
\coordinate (spppaw) at (\sxxxa, \syyyw);
\coordinate (spppax) at (\sxxxa, \syyyx);
\coordinate (spppay) at (\sxxxa, \syyyy);
\coordinate (spppaz) at (\sxxxa, \syyyz);
\coordinate (spppba) at (\sxxxb, \syyya);
\coordinate (spppbb) at (\sxxxb, \syyyb);
\coordinate (spppbc) at (\sxxxb, \syyyc);
\coordinate (spppbd) at (\sxxxb, \syyyd);
\coordinate (spppbe) at (\sxxxb, \syyye);
\coordinate (spppbf) at (\sxxxb, \syyyf);
\coordinate (spppbg) at (\sxxxb, \syyyg);
\coordinate (spppbh) at (\sxxxb, \syyyh);
\coordinate (spppbi) at (\sxxxb, \syyyi);
\coordinate (spppbj) at (\sxxxb, \syyyj);
\coordinate (spppbk) at (\sxxxb, \syyyk);
\coordinate (spppbl) at (\sxxxb, \syyyl);
\coordinate (spppbm) at (\sxxxb, \syyym);
\coordinate (spppbn) at (\sxxxb, \syyyn);
\coordinate (spppbo) at (\sxxxb, \syyyo);
\coordinate (spppbp) at (\sxxxb, \syyyp);
\coordinate (spppbq) at (\sxxxb, \syyyq);
\coordinate (spppbr) at (\sxxxb, \syyyr);
\coordinate (spppbs) at (\sxxxb, \syyys);
\coordinate (spppbt) at (\sxxxb, \syyyt);
\coordinate (spppbu) at (\sxxxb, \syyyu);
\coordinate (spppbv) at (\sxxxb, \syyyv);
\coordinate (spppbw) at (\sxxxb, \syyyw);
\coordinate (spppbx) at (\sxxxb, \syyyx);
\coordinate (spppby) at (\sxxxb, \syyyy);
\coordinate (spppbz) at (\sxxxb, \syyyz);
\coordinate (spppca) at (\sxxxc, \syyya);
\coordinate (spppcb) at (\sxxxc, \syyyb);
\coordinate (spppcc) at (\sxxxc, \syyyc);
\coordinate (spppcd) at (\sxxxc, \syyyd);
\coordinate (spppce) at (\sxxxc, \syyye);
\coordinate (spppcf) at (\sxxxc, \syyyf);
\coordinate (spppcg) at (\sxxxc, \syyyg);
\coordinate (spppch) at (\sxxxc, \syyyh);
\coordinate (spppci) at (\sxxxc, \syyyi);
\coordinate (spppcj) at (\sxxxc, \syyyj);
\coordinate (spppck) at (\sxxxc, \syyyk);
\coordinate (spppcl) at (\sxxxc, \syyyl);
\coordinate (spppcm) at (\sxxxc, \syyym);
\coordinate (spppcn) at (\sxxxc, \syyyn);
\coordinate (spppco) at (\sxxxc, \syyyo);
\coordinate (spppcp) at (\sxxxc, \syyyp);
\coordinate (spppcq) at (\sxxxc, \syyyq);
\coordinate (spppcr) at (\sxxxc, \syyyr);
\coordinate (spppcs) at (\sxxxc, \syyys);
\coordinate (spppct) at (\sxxxc, \syyyt);
\coordinate (spppcu) at (\sxxxc, \syyyu);
\coordinate (spppcv) at (\sxxxc, \syyyv);
\coordinate (spppcw) at (\sxxxc, \syyyw);
\coordinate (spppcx) at (\sxxxc, \syyyx);
\coordinate (spppcy) at (\sxxxc, \syyyy);
\coordinate (spppcz) at (\sxxxc, \syyyz);
\coordinate (spppda) at (\sxxxd, \syyya);
\coordinate (spppdb) at (\sxxxd, \syyyb);
\coordinate (spppdc) at (\sxxxd, \syyyc);
\coordinate (spppdd) at (\sxxxd, \syyyd);
\coordinate (spppde) at (\sxxxd, \syyye);
\coordinate (spppdf) at (\sxxxd, \syyyf);
\coordinate (spppdg) at (\sxxxd, \syyyg);
\coordinate (spppdh) at (\sxxxd, \syyyh);
\coordinate (spppdi) at (\sxxxd, \syyyi);
\coordinate (spppdj) at (\sxxxd, \syyyj);
\coordinate (spppdk) at (\sxxxd, \syyyk);
\coordinate (spppdl) at (\sxxxd, \syyyl);
\coordinate (spppdm) at (\sxxxd, \syyym);
\coordinate (spppdn) at (\sxxxd, \syyyn);
\coordinate (spppdo) at (\sxxxd, \syyyo);
\coordinate (spppdp) at (\sxxxd, \syyyp);
\coordinate (spppdq) at (\sxxxd, \syyyq);
\coordinate (spppdr) at (\sxxxd, \syyyr);
\coordinate (spppds) at (\sxxxd, \syyys);
\coordinate (spppdt) at (\sxxxd, \syyyt);
\coordinate (spppdu) at (\sxxxd, \syyyu);
\coordinate (spppdv) at (\sxxxd, \syyyv);
\coordinate (spppdw) at (\sxxxd, \syyyw);
\coordinate (spppdx) at (\sxxxd, \syyyx);
\coordinate (spppdy) at (\sxxxd, \syyyy);
\coordinate (spppdz) at (\sxxxd, \syyyz);
\coordinate (spppea) at (\sxxxe, \syyya);
\coordinate (spppeb) at (\sxxxe, \syyyb);
\coordinate (spppec) at (\sxxxe, \syyyc);
\coordinate (sppped) at (\sxxxe, \syyyd);
\coordinate (spppee) at (\sxxxe, \syyye);
\coordinate (spppef) at (\sxxxe, \syyyf);
\coordinate (spppeg) at (\sxxxe, \syyyg);
\coordinate (spppeh) at (\sxxxe, \syyyh);
\coordinate (spppei) at (\sxxxe, \syyyi);
\coordinate (spppej) at (\sxxxe, \syyyj);
\coordinate (spppek) at (\sxxxe, \syyyk);
\coordinate (spppel) at (\sxxxe, \syyyl);
\coordinate (spppem) at (\sxxxe, \syyym);
\coordinate (spppen) at (\sxxxe, \syyyn);
\coordinate (spppeo) at (\sxxxe, \syyyo);
\coordinate (spppep) at (\sxxxe, \syyyp);
\coordinate (spppeq) at (\sxxxe, \syyyq);
\coordinate (sppper) at (\sxxxe, \syyyr);
\coordinate (spppes) at (\sxxxe, \syyys);
\coordinate (spppet) at (\sxxxe, \syyyt);
\coordinate (spppeu) at (\sxxxe, \syyyu);
\coordinate (spppev) at (\sxxxe, \syyyv);
\coordinate (spppew) at (\sxxxe, \syyyw);
\coordinate (spppex) at (\sxxxe, \syyyx);
\coordinate (spppey) at (\sxxxe, \syyyy);
\coordinate (spppez) at (\sxxxe, \syyyz);
\coordinate (spppfa) at (\sxxxf, \syyya);
\coordinate (spppfb) at (\sxxxf, \syyyb);
\coordinate (spppfc) at (\sxxxf, \syyyc);
\coordinate (spppfd) at (\sxxxf, \syyyd);
\coordinate (spppfe) at (\sxxxf, \syyye);
\coordinate (spppff) at (\sxxxf, \syyyf);
\coordinate (spppfg) at (\sxxxf, \syyyg);
\coordinate (spppfh) at (\sxxxf, \syyyh);
\coordinate (spppfi) at (\sxxxf, \syyyi);
\coordinate (spppfj) at (\sxxxf, \syyyj);
\coordinate (spppfk) at (\sxxxf, \syyyk);
\coordinate (spppfl) at (\sxxxf, \syyyl);
\coordinate (spppfm) at (\sxxxf, \syyym);
\coordinate (spppfn) at (\sxxxf, \syyyn);
\coordinate (spppfo) at (\sxxxf, \syyyo);
\coordinate (spppfp) at (\sxxxf, \syyyp);
\coordinate (spppfq) at (\sxxxf, \syyyq);
\coordinate (spppfr) at (\sxxxf, \syyyr);
\coordinate (spppfs) at (\sxxxf, \syyys);
\coordinate (spppft) at (\sxxxf, \syyyt);
\coordinate (spppfu) at (\sxxxf, \syyyu);
\coordinate (spppfv) at (\sxxxf, \syyyv);
\coordinate (spppfw) at (\sxxxf, \syyyw);
\coordinate (spppfx) at (\sxxxf, \syyyx);
\coordinate (spppfy) at (\sxxxf, \syyyy);
\coordinate (spppfz) at (\sxxxf, \syyyz);
\coordinate (spppga) at (\sxxxg, \syyya);
\coordinate (spppgb) at (\sxxxg, \syyyb);
\coordinate (spppgc) at (\sxxxg, \syyyc);
\coordinate (spppgd) at (\sxxxg, \syyyd);
\coordinate (spppge) at (\sxxxg, \syyye);
\coordinate (spppgf) at (\sxxxg, \syyyf);
\coordinate (spppgg) at (\sxxxg, \syyyg);
\coordinate (spppgh) at (\sxxxg, \syyyh);
\coordinate (spppgi) at (\sxxxg, \syyyi);
\coordinate (spppgj) at (\sxxxg, \syyyj);
\coordinate (spppgk) at (\sxxxg, \syyyk);
\coordinate (spppgl) at (\sxxxg, \syyyl);
\coordinate (spppgm) at (\sxxxg, \syyym);
\coordinate (spppgn) at (\sxxxg, \syyyn);
\coordinate (spppgo) at (\sxxxg, \syyyo);
\coordinate (spppgp) at (\sxxxg, \syyyp);
\coordinate (spppgq) at (\sxxxg, \syyyq);
\coordinate (spppgr) at (\sxxxg, \syyyr);
\coordinate (spppgs) at (\sxxxg, \syyys);
\coordinate (spppgt) at (\sxxxg, \syyyt);
\coordinate (spppgu) at (\sxxxg, \syyyu);
\coordinate (spppgv) at (\sxxxg, \syyyv);
\coordinate (spppgw) at (\sxxxg, \syyyw);
\coordinate (spppgx) at (\sxxxg, \syyyx);
\coordinate (spppgy) at (\sxxxg, \syyyy);
\coordinate (spppgz) at (\sxxxg, \syyyz);
\coordinate (spppha) at (\sxxxh, \syyya);
\coordinate (sppphb) at (\sxxxh, \syyyb);
\coordinate (sppphc) at (\sxxxh, \syyyc);
\coordinate (sppphd) at (\sxxxh, \syyyd);
\coordinate (sppphe) at (\sxxxh, \syyye);
\coordinate (sppphf) at (\sxxxh, \syyyf);
\coordinate (sppphg) at (\sxxxh, \syyyg);
\coordinate (sppphh) at (\sxxxh, \syyyh);
\coordinate (sppphi) at (\sxxxh, \syyyi);
\coordinate (sppphj) at (\sxxxh, \syyyj);
\coordinate (sppphk) at (\sxxxh, \syyyk);
\coordinate (sppphl) at (\sxxxh, \syyyl);
\coordinate (sppphm) at (\sxxxh, \syyym);
\coordinate (sppphn) at (\sxxxh, \syyyn);
\coordinate (spppho) at (\sxxxh, \syyyo);
\coordinate (sppphp) at (\sxxxh, \syyyp);
\coordinate (sppphq) at (\sxxxh, \syyyq);
\coordinate (sppphr) at (\sxxxh, \syyyr);
\coordinate (sppphs) at (\sxxxh, \syyys);
\coordinate (spppht) at (\sxxxh, \syyyt);
\coordinate (sppphu) at (\sxxxh, \syyyu);
\coordinate (sppphv) at (\sxxxh, \syyyv);
\coordinate (sppphw) at (\sxxxh, \syyyw);
\coordinate (sppphx) at (\sxxxh, \syyyx);
\coordinate (sppphy) at (\sxxxh, \syyyy);
\coordinate (sppphz) at (\sxxxh, \syyyz);
\coordinate (spppia) at (\sxxxi, \syyya);
\coordinate (spppib) at (\sxxxi, \syyyb);
\coordinate (spppic) at (\sxxxi, \syyyc);
\coordinate (spppid) at (\sxxxi, \syyyd);
\coordinate (spppie) at (\sxxxi, \syyye);
\coordinate (spppif) at (\sxxxi, \syyyf);
\coordinate (spppig) at (\sxxxi, \syyyg);
\coordinate (spppih) at (\sxxxi, \syyyh);
\coordinate (spppii) at (\sxxxi, \syyyi);
\coordinate (spppij) at (\sxxxi, \syyyj);
\coordinate (spppik) at (\sxxxi, \syyyk);
\coordinate (spppil) at (\sxxxi, \syyyl);
\coordinate (spppim) at (\sxxxi, \syyym);
\coordinate (spppin) at (\sxxxi, \syyyn);
\coordinate (spppio) at (\sxxxi, \syyyo);
\coordinate (spppip) at (\sxxxi, \syyyp);
\coordinate (spppiq) at (\sxxxi, \syyyq);
\coordinate (spppir) at (\sxxxi, \syyyr);
\coordinate (spppis) at (\sxxxi, \syyys);
\coordinate (spppit) at (\sxxxi, \syyyt);
\coordinate (spppiu) at (\sxxxi, \syyyu);
\coordinate (spppiv) at (\sxxxi, \syyyv);
\coordinate (spppiw) at (\sxxxi, \syyyw);
\coordinate (spppix) at (\sxxxi, \syyyx);
\coordinate (spppiy) at (\sxxxi, \syyyy);
\coordinate (spppiz) at (\sxxxi, \syyyz);
\coordinate (spppja) at (\sxxxj, \syyya);
\coordinate (spppjb) at (\sxxxj, \syyyb);
\coordinate (spppjc) at (\sxxxj, \syyyc);
\coordinate (spppjd) at (\sxxxj, \syyyd);
\coordinate (spppje) at (\sxxxj, \syyye);
\coordinate (spppjf) at (\sxxxj, \syyyf);
\coordinate (spppjg) at (\sxxxj, \syyyg);
\coordinate (spppjh) at (\sxxxj, \syyyh);
\coordinate (spppji) at (\sxxxj, \syyyi);
\coordinate (spppjj) at (\sxxxj, \syyyj);
\coordinate (spppjk) at (\sxxxj, \syyyk);
\coordinate (spppjl) at (\sxxxj, \syyyl);
\coordinate (spppjm) at (\sxxxj, \syyym);
\coordinate (spppjn) at (\sxxxj, \syyyn);
\coordinate (spppjo) at (\sxxxj, \syyyo);
\coordinate (spppjp) at (\sxxxj, \syyyp);
\coordinate (spppjq) at (\sxxxj, \syyyq);
\coordinate (spppjr) at (\sxxxj, \syyyr);
\coordinate (spppjs) at (\sxxxj, \syyys);
\coordinate (spppjt) at (\sxxxj, \syyyt);
\coordinate (spppju) at (\sxxxj, \syyyu);
\coordinate (spppjv) at (\sxxxj, \syyyv);
\coordinate (spppjw) at (\sxxxj, \syyyw);
\coordinate (spppjx) at (\sxxxj, \syyyx);
\coordinate (spppjy) at (\sxxxj, \syyyy);
\coordinate (spppjz) at (\sxxxj, \syyyz);
\coordinate (spppka) at (\sxxxk, \syyya);
\coordinate (spppkb) at (\sxxxk, \syyyb);
\coordinate (spppkc) at (\sxxxk, \syyyc);
\coordinate (spppkd) at (\sxxxk, \syyyd);
\coordinate (spppke) at (\sxxxk, \syyye);
\coordinate (spppkf) at (\sxxxk, \syyyf);
\coordinate (spppkg) at (\sxxxk, \syyyg);
\coordinate (spppkh) at (\sxxxk, \syyyh);
\coordinate (spppki) at (\sxxxk, \syyyi);
\coordinate (spppkj) at (\sxxxk, \syyyj);
\coordinate (spppkk) at (\sxxxk, \syyyk);
\coordinate (spppkl) at (\sxxxk, \syyyl);
\coordinate (spppkm) at (\sxxxk, \syyym);
\coordinate (spppkn) at (\sxxxk, \syyyn);
\coordinate (spppko) at (\sxxxk, \syyyo);
\coordinate (spppkp) at (\sxxxk, \syyyp);
\coordinate (spppkq) at (\sxxxk, \syyyq);
\coordinate (spppkr) at (\sxxxk, \syyyr);
\coordinate (spppks) at (\sxxxk, \syyys);
\coordinate (spppkt) at (\sxxxk, \syyyt);
\coordinate (spppku) at (\sxxxk, \syyyu);
\coordinate (spppkv) at (\sxxxk, \syyyv);
\coordinate (spppkw) at (\sxxxk, \syyyw);
\coordinate (spppkx) at (\sxxxk, \syyyx);
\coordinate (spppky) at (\sxxxk, \syyyy);
\coordinate (spppkz) at (\sxxxk, \syyyz);
\coordinate (spppla) at (\sxxxl, \syyya);
\coordinate (sppplb) at (\sxxxl, \syyyb);
\coordinate (sppplc) at (\sxxxl, \syyyc);
\coordinate (spppld) at (\sxxxl, \syyyd);
\coordinate (sppple) at (\sxxxl, \syyye);
\coordinate (sppplf) at (\sxxxl, \syyyf);
\coordinate (sppplg) at (\sxxxl, \syyyg);
\coordinate (sppplh) at (\sxxxl, \syyyh);
\coordinate (spppli) at (\sxxxl, \syyyi);
\coordinate (sppplj) at (\sxxxl, \syyyj);
\coordinate (sppplk) at (\sxxxl, \syyyk);
\coordinate (spppll) at (\sxxxl, \syyyl);
\coordinate (sppplm) at (\sxxxl, \syyym);
\coordinate (spppln) at (\sxxxl, \syyyn);
\coordinate (sppplo) at (\sxxxl, \syyyo);
\coordinate (sppplp) at (\sxxxl, \syyyp);
\coordinate (sppplq) at (\sxxxl, \syyyq);
\coordinate (sppplr) at (\sxxxl, \syyyr);
\coordinate (spppls) at (\sxxxl, \syyys);
\coordinate (sppplt) at (\sxxxl, \syyyt);
\coordinate (sppplu) at (\sxxxl, \syyyu);
\coordinate (sppplv) at (\sxxxl, \syyyv);
\coordinate (sppplw) at (\sxxxl, \syyyw);
\coordinate (sppplx) at (\sxxxl, \syyyx);
\coordinate (sppply) at (\sxxxl, \syyyy);
\coordinate (sppplz) at (\sxxxl, \syyyz);
\coordinate (spppma) at (\sxxxm, \syyya);
\coordinate (spppmb) at (\sxxxm, \syyyb);
\coordinate (spppmc) at (\sxxxm, \syyyc);
\coordinate (spppmd) at (\sxxxm, \syyyd);
\coordinate (spppme) at (\sxxxm, \syyye);
\coordinate (spppmf) at (\sxxxm, \syyyf);
\coordinate (spppmg) at (\sxxxm, \syyyg);
\coordinate (spppmh) at (\sxxxm, \syyyh);
\coordinate (spppmi) at (\sxxxm, \syyyi);
\coordinate (spppmj) at (\sxxxm, \syyyj);
\coordinate (spppmk) at (\sxxxm, \syyyk);
\coordinate (spppml) at (\sxxxm, \syyyl);
\coordinate (spppmm) at (\sxxxm, \syyym);
\coordinate (spppmn) at (\sxxxm, \syyyn);
\coordinate (spppmo) at (\sxxxm, \syyyo);
\coordinate (spppmp) at (\sxxxm, \syyyp);
\coordinate (spppmq) at (\sxxxm, \syyyq);
\coordinate (spppmr) at (\sxxxm, \syyyr);
\coordinate (spppms) at (\sxxxm, \syyys);
\coordinate (spppmt) at (\sxxxm, \syyyt);
\coordinate (spppmu) at (\sxxxm, \syyyu);
\coordinate (spppmv) at (\sxxxm, \syyyv);
\coordinate (spppmw) at (\sxxxm, \syyyw);
\coordinate (spppmx) at (\sxxxm, \syyyx);
\coordinate (spppmy) at (\sxxxm, \syyyy);
\coordinate (spppmz) at (\sxxxm, \syyyz);
\coordinate (spppna) at (\sxxxn, \syyya);
\coordinate (spppnb) at (\sxxxn, \syyyb);
\coordinate (spppnc) at (\sxxxn, \syyyc);
\coordinate (spppnd) at (\sxxxn, \syyyd);
\coordinate (spppne) at (\sxxxn, \syyye);
\coordinate (spppnf) at (\sxxxn, \syyyf);
\coordinate (spppng) at (\sxxxn, \syyyg);
\coordinate (spppnh) at (\sxxxn, \syyyh);
\coordinate (spppni) at (\sxxxn, \syyyi);
\coordinate (spppnj) at (\sxxxn, \syyyj);
\coordinate (spppnk) at (\sxxxn, \syyyk);
\coordinate (spppnl) at (\sxxxn, \syyyl);
\coordinate (spppnm) at (\sxxxn, \syyym);
\coordinate (spppnn) at (\sxxxn, \syyyn);
\coordinate (spppno) at (\sxxxn, \syyyo);
\coordinate (spppnp) at (\sxxxn, \syyyp);
\coordinate (spppnq) at (\sxxxn, \syyyq);
\coordinate (spppnr) at (\sxxxn, \syyyr);
\coordinate (spppns) at (\sxxxn, \syyys);
\coordinate (spppnt) at (\sxxxn, \syyyt);
\coordinate (spppnu) at (\sxxxn, \syyyu);
\coordinate (spppnv) at (\sxxxn, \syyyv);
\coordinate (spppnw) at (\sxxxn, \syyyw);
\coordinate (spppnx) at (\sxxxn, \syyyx);
\coordinate (spppny) at (\sxxxn, \syyyy);
\coordinate (spppnz) at (\sxxxn, \syyyz);
\coordinate (spppoa) at (\sxxxo, \syyya);
\coordinate (spppob) at (\sxxxo, \syyyb);
\coordinate (spppoc) at (\sxxxo, \syyyc);
\coordinate (spppod) at (\sxxxo, \syyyd);
\coordinate (spppoe) at (\sxxxo, \syyye);
\coordinate (spppof) at (\sxxxo, \syyyf);
\coordinate (spppog) at (\sxxxo, \syyyg);
\coordinate (spppoh) at (\sxxxo, \syyyh);
\coordinate (spppoi) at (\sxxxo, \syyyi);
\coordinate (spppoj) at (\sxxxo, \syyyj);
\coordinate (spppok) at (\sxxxo, \syyyk);
\coordinate (spppol) at (\sxxxo, \syyyl);
\coordinate (spppom) at (\sxxxo, \syyym);
\coordinate (spppon) at (\sxxxo, \syyyn);
\coordinate (spppoo) at (\sxxxo, \syyyo);
\coordinate (spppop) at (\sxxxo, \syyyp);
\coordinate (spppoq) at (\sxxxo, \syyyq);
\coordinate (spppor) at (\sxxxo, \syyyr);
\coordinate (spppos) at (\sxxxo, \syyys);
\coordinate (spppot) at (\sxxxo, \syyyt);
\coordinate (spppou) at (\sxxxo, \syyyu);
\coordinate (spppov) at (\sxxxo, \syyyv);
\coordinate (spppow) at (\sxxxo, \syyyw);
\coordinate (spppox) at (\sxxxo, \syyyx);
\coordinate (spppoy) at (\sxxxo, \syyyy);
\coordinate (spppoz) at (\sxxxo, \syyyz);
\coordinate (sppppa) at (\sxxxp, \syyya);
\coordinate (sppppb) at (\sxxxp, \syyyb);
\coordinate (sppppc) at (\sxxxp, \syyyc);
\coordinate (sppppd) at (\sxxxp, \syyyd);
\coordinate (sppppe) at (\sxxxp, \syyye);
\coordinate (sppppf) at (\sxxxp, \syyyf);
\coordinate (sppppg) at (\sxxxp, \syyyg);
\coordinate (spppph) at (\sxxxp, \syyyh);
\coordinate (sppppi) at (\sxxxp, \syyyi);
\coordinate (sppppj) at (\sxxxp, \syyyj);
\coordinate (sppppk) at (\sxxxp, \syyyk);
\coordinate (sppppl) at (\sxxxp, \syyyl);
\coordinate (sppppm) at (\sxxxp, \syyym);
\coordinate (sppppn) at (\sxxxp, \syyyn);
\coordinate (sppppo) at (\sxxxp, \syyyo);
\coordinate (sppppp) at (\sxxxp, \syyyp);
\coordinate (sppppq) at (\sxxxp, \syyyq);
\coordinate (sppppr) at (\sxxxp, \syyyr);
\coordinate (spppps) at (\sxxxp, \syyys);
\coordinate (sppppt) at (\sxxxp, \syyyt);
\coordinate (sppppu) at (\sxxxp, \syyyu);
\coordinate (sppppv) at (\sxxxp, \syyyv);
\coordinate (sppppw) at (\sxxxp, \syyyw);
\coordinate (sppppx) at (\sxxxp, \syyyx);
\coordinate (sppppy) at (\sxxxp, \syyyy);
\coordinate (sppppz) at (\sxxxp, \syyyz);
\coordinate (spppqa) at (\sxxxq, \syyya);
\coordinate (spppqb) at (\sxxxq, \syyyb);
\coordinate (spppqc) at (\sxxxq, \syyyc);
\coordinate (spppqd) at (\sxxxq, \syyyd);
\coordinate (spppqe) at (\sxxxq, \syyye);
\coordinate (spppqf) at (\sxxxq, \syyyf);
\coordinate (spppqg) at (\sxxxq, \syyyg);
\coordinate (spppqh) at (\sxxxq, \syyyh);
\coordinate (spppqi) at (\sxxxq, \syyyi);
\coordinate (spppqj) at (\sxxxq, \syyyj);
\coordinate (spppqk) at (\sxxxq, \syyyk);
\coordinate (spppql) at (\sxxxq, \syyyl);
\coordinate (spppqm) at (\sxxxq, \syyym);
\coordinate (spppqn) at (\sxxxq, \syyyn);
\coordinate (spppqo) at (\sxxxq, \syyyo);
\coordinate (spppqp) at (\sxxxq, \syyyp);
\coordinate (spppqq) at (\sxxxq, \syyyq);
\coordinate (spppqr) at (\sxxxq, \syyyr);
\coordinate (spppqs) at (\sxxxq, \syyys);
\coordinate (spppqt) at (\sxxxq, \syyyt);
\coordinate (spppqu) at (\sxxxq, \syyyu);
\coordinate (spppqv) at (\sxxxq, \syyyv);
\coordinate (spppqw) at (\sxxxq, \syyyw);
\coordinate (spppqx) at (\sxxxq, \syyyx);
\coordinate (spppqy) at (\sxxxq, \syyyy);
\coordinate (spppqz) at (\sxxxq, \syyyz);
\coordinate (spppra) at (\sxxxr, \syyya);
\coordinate (sppprb) at (\sxxxr, \syyyb);
\coordinate (sppprc) at (\sxxxr, \syyyc);
\coordinate (sppprd) at (\sxxxr, \syyyd);
\coordinate (spppre) at (\sxxxr, \syyye);
\coordinate (sppprf) at (\sxxxr, \syyyf);
\coordinate (sppprg) at (\sxxxr, \syyyg);
\coordinate (sppprh) at (\sxxxr, \syyyh);
\coordinate (spppri) at (\sxxxr, \syyyi);
\coordinate (sppprj) at (\sxxxr, \syyyj);
\coordinate (sppprk) at (\sxxxr, \syyyk);
\coordinate (sppprl) at (\sxxxr, \syyyl);
\coordinate (sppprm) at (\sxxxr, \syyym);
\coordinate (sppprn) at (\sxxxr, \syyyn);
\coordinate (spppro) at (\sxxxr, \syyyo);
\coordinate (sppprp) at (\sxxxr, \syyyp);
\coordinate (sppprq) at (\sxxxr, \syyyq);
\coordinate (sppprr) at (\sxxxr, \syyyr);
\coordinate (sppprs) at (\sxxxr, \syyys);
\coordinate (sppprt) at (\sxxxr, \syyyt);
\coordinate (spppru) at (\sxxxr, \syyyu);
\coordinate (sppprv) at (\sxxxr, \syyyv);
\coordinate (sppprw) at (\sxxxr, \syyyw);
\coordinate (sppprx) at (\sxxxr, \syyyx);
\coordinate (spppry) at (\sxxxr, \syyyy);
\coordinate (sppprz) at (\sxxxr, \syyyz);
\coordinate (spppsa) at (\sxxxs, \syyya);
\coordinate (spppsb) at (\sxxxs, \syyyb);
\coordinate (spppsc) at (\sxxxs, \syyyc);
\coordinate (spppsd) at (\sxxxs, \syyyd);
\coordinate (spppse) at (\sxxxs, \syyye);
\coordinate (spppsf) at (\sxxxs, \syyyf);
\coordinate (spppsg) at (\sxxxs, \syyyg);
\coordinate (spppsh) at (\sxxxs, \syyyh);
\coordinate (spppsi) at (\sxxxs, \syyyi);
\coordinate (spppsj) at (\sxxxs, \syyyj);
\coordinate (spppsk) at (\sxxxs, \syyyk);
\coordinate (spppsl) at (\sxxxs, \syyyl);
\coordinate (spppsm) at (\sxxxs, \syyym);
\coordinate (spppsn) at (\sxxxs, \syyyn);
\coordinate (spppso) at (\sxxxs, \syyyo);
\coordinate (spppsp) at (\sxxxs, \syyyp);
\coordinate (spppsq) at (\sxxxs, \syyyq);
\coordinate (spppsr) at (\sxxxs, \syyyr);
\coordinate (spppss) at (\sxxxs, \syyys);
\coordinate (spppst) at (\sxxxs, \syyyt);
\coordinate (spppsu) at (\sxxxs, \syyyu);
\coordinate (spppsv) at (\sxxxs, \syyyv);
\coordinate (spppsw) at (\sxxxs, \syyyw);
\coordinate (spppsx) at (\sxxxs, \syyyx);
\coordinate (spppsy) at (\sxxxs, \syyyy);
\coordinate (spppsz) at (\sxxxs, \syyyz);
\coordinate (spppta) at (\sxxxt, \syyya);
\coordinate (sppptb) at (\sxxxt, \syyyb);
\coordinate (sppptc) at (\sxxxt, \syyyc);
\coordinate (sppptd) at (\sxxxt, \syyyd);
\coordinate (spppte) at (\sxxxt, \syyye);
\coordinate (sppptf) at (\sxxxt, \syyyf);
\coordinate (sppptg) at (\sxxxt, \syyyg);
\coordinate (spppth) at (\sxxxt, \syyyh);
\coordinate (spppti) at (\sxxxt, \syyyi);
\coordinate (sppptj) at (\sxxxt, \syyyj);
\coordinate (sppptk) at (\sxxxt, \syyyk);
\coordinate (sppptl) at (\sxxxt, \syyyl);
\coordinate (sppptm) at (\sxxxt, \syyym);
\coordinate (sppptn) at (\sxxxt, \syyyn);
\coordinate (spppto) at (\sxxxt, \syyyo);
\coordinate (sppptp) at (\sxxxt, \syyyp);
\coordinate (sppptq) at (\sxxxt, \syyyq);
\coordinate (sppptr) at (\sxxxt, \syyyr);
\coordinate (spppts) at (\sxxxt, \syyys);
\coordinate (sppptt) at (\sxxxt, \syyyt);
\coordinate (sppptu) at (\sxxxt, \syyyu);
\coordinate (sppptv) at (\sxxxt, \syyyv);
\coordinate (sppptw) at (\sxxxt, \syyyw);
\coordinate (sppptx) at (\sxxxt, \syyyx);
\coordinate (spppty) at (\sxxxt, \syyyy);
\coordinate (sppptz) at (\sxxxt, \syyyz);
\coordinate (spppua) at (\sxxxu, \syyya);
\coordinate (spppub) at (\sxxxu, \syyyb);
\coordinate (spppuc) at (\sxxxu, \syyyc);
\coordinate (spppud) at (\sxxxu, \syyyd);
\coordinate (spppue) at (\sxxxu, \syyye);
\coordinate (spppuf) at (\sxxxu, \syyyf);
\coordinate (spppug) at (\sxxxu, \syyyg);
\coordinate (spppuh) at (\sxxxu, \syyyh);
\coordinate (spppui) at (\sxxxu, \syyyi);
\coordinate (spppuj) at (\sxxxu, \syyyj);
\coordinate (spppuk) at (\sxxxu, \syyyk);
\coordinate (spppul) at (\sxxxu, \syyyl);
\coordinate (spppum) at (\sxxxu, \syyym);
\coordinate (spppun) at (\sxxxu, \syyyn);
\coordinate (spppuo) at (\sxxxu, \syyyo);
\coordinate (spppup) at (\sxxxu, \syyyp);
\coordinate (spppuq) at (\sxxxu, \syyyq);
\coordinate (spppur) at (\sxxxu, \syyyr);
\coordinate (spppus) at (\sxxxu, \syyys);
\coordinate (sppput) at (\sxxxu, \syyyt);
\coordinate (spppuu) at (\sxxxu, \syyyu);
\coordinate (spppuv) at (\sxxxu, \syyyv);
\coordinate (spppuw) at (\sxxxu, \syyyw);
\coordinate (spppux) at (\sxxxu, \syyyx);
\coordinate (spppuy) at (\sxxxu, \syyyy);
\coordinate (spppuz) at (\sxxxu, \syyyz);
\coordinate (spppva) at (\sxxxv, \syyya);
\coordinate (spppvb) at (\sxxxv, \syyyb);
\coordinate (spppvc) at (\sxxxv, \syyyc);
\coordinate (spppvd) at (\sxxxv, \syyyd);
\coordinate (spppve) at (\sxxxv, \syyye);
\coordinate (spppvf) at (\sxxxv, \syyyf);
\coordinate (spppvg) at (\sxxxv, \syyyg);
\coordinate (spppvh) at (\sxxxv, \syyyh);
\coordinate (spppvi) at (\sxxxv, \syyyi);
\coordinate (spppvj) at (\sxxxv, \syyyj);
\coordinate (spppvk) at (\sxxxv, \syyyk);
\coordinate (spppvl) at (\sxxxv, \syyyl);
\coordinate (spppvm) at (\sxxxv, \syyym);
\coordinate (spppvn) at (\sxxxv, \syyyn);
\coordinate (spppvo) at (\sxxxv, \syyyo);
\coordinate (spppvp) at (\sxxxv, \syyyp);
\coordinate (spppvq) at (\sxxxv, \syyyq);
\coordinate (spppvr) at (\sxxxv, \syyyr);
\coordinate (spppvs) at (\sxxxv, \syyys);
\coordinate (spppvt) at (\sxxxv, \syyyt);
\coordinate (spppvu) at (\sxxxv, \syyyu);
\coordinate (spppvv) at (\sxxxv, \syyyv);
\coordinate (spppvw) at (\sxxxv, \syyyw);
\coordinate (spppvx) at (\sxxxv, \syyyx);
\coordinate (spppvy) at (\sxxxv, \syyyy);
\coordinate (spppvz) at (\sxxxv, \syyyz);
\coordinate (spppwa) at (\sxxxw, \syyya);
\coordinate (spppwb) at (\sxxxw, \syyyb);
\coordinate (spppwc) at (\sxxxw, \syyyc);
\coordinate (spppwd) at (\sxxxw, \syyyd);
\coordinate (spppwe) at (\sxxxw, \syyye);
\coordinate (spppwf) at (\sxxxw, \syyyf);
\coordinate (spppwg) at (\sxxxw, \syyyg);
\coordinate (spppwh) at (\sxxxw, \syyyh);
\coordinate (spppwi) at (\sxxxw, \syyyi);
\coordinate (spppwj) at (\sxxxw, \syyyj);
\coordinate (spppwk) at (\sxxxw, \syyyk);
\coordinate (spppwl) at (\sxxxw, \syyyl);
\coordinate (spppwm) at (\sxxxw, \syyym);
\coordinate (spppwn) at (\sxxxw, \syyyn);
\coordinate (spppwo) at (\sxxxw, \syyyo);
\coordinate (spppwp) at (\sxxxw, \syyyp);
\coordinate (spppwq) at (\sxxxw, \syyyq);
\coordinate (spppwr) at (\sxxxw, \syyyr);
\coordinate (spppws) at (\sxxxw, \syyys);
\coordinate (spppwt) at (\sxxxw, \syyyt);
\coordinate (spppwu) at (\sxxxw, \syyyu);
\coordinate (spppwv) at (\sxxxw, \syyyv);
\coordinate (spppww) at (\sxxxw, \syyyw);
\coordinate (spppwx) at (\sxxxw, \syyyx);
\coordinate (spppwy) at (\sxxxw, \syyyy);
\coordinate (spppwz) at (\sxxxw, \syyyz);
\coordinate (spppxa) at (\sxxxx, \syyya);
\coordinate (spppxb) at (\sxxxx, \syyyb);
\coordinate (spppxc) at (\sxxxx, \syyyc);
\coordinate (spppxd) at (\sxxxx, \syyyd);
\coordinate (spppxe) at (\sxxxx, \syyye);
\coordinate (spppxf) at (\sxxxx, \syyyf);
\coordinate (spppxg) at (\sxxxx, \syyyg);
\coordinate (spppxh) at (\sxxxx, \syyyh);
\coordinate (spppxi) at (\sxxxx, \syyyi);
\coordinate (spppxj) at (\sxxxx, \syyyj);
\coordinate (spppxk) at (\sxxxx, \syyyk);
\coordinate (spppxl) at (\sxxxx, \syyyl);
\coordinate (spppxm) at (\sxxxx, \syyym);
\coordinate (spppxn) at (\sxxxx, \syyyn);
\coordinate (spppxo) at (\sxxxx, \syyyo);
\coordinate (spppxp) at (\sxxxx, \syyyp);
\coordinate (spppxq) at (\sxxxx, \syyyq);
\coordinate (spppxr) at (\sxxxx, \syyyr);
\coordinate (spppxs) at (\sxxxx, \syyys);
\coordinate (spppxt) at (\sxxxx, \syyyt);
\coordinate (spppxu) at (\sxxxx, \syyyu);
\coordinate (spppxv) at (\sxxxx, \syyyv);
\coordinate (spppxw) at (\sxxxx, \syyyw);
\coordinate (spppxx) at (\sxxxx, \syyyx);
\coordinate (spppxy) at (\sxxxx, \syyyy);
\coordinate (spppxz) at (\sxxxx, \syyyz);
\coordinate (spppya) at (\sxxxy, \syyya);
\coordinate (spppyb) at (\sxxxy, \syyyb);
\coordinate (spppyc) at (\sxxxy, \syyyc);
\coordinate (spppyd) at (\sxxxy, \syyyd);
\coordinate (spppye) at (\sxxxy, \syyye);
\coordinate (spppyf) at (\sxxxy, \syyyf);
\coordinate (spppyg) at (\sxxxy, \syyyg);
\coordinate (spppyh) at (\sxxxy, \syyyh);
\coordinate (spppyi) at (\sxxxy, \syyyi);
\coordinate (spppyj) at (\sxxxy, \syyyj);
\coordinate (spppyk) at (\sxxxy, \syyyk);
\coordinate (spppyl) at (\sxxxy, \syyyl);
\coordinate (spppym) at (\sxxxy, \syyym);
\coordinate (spppyn) at (\sxxxy, \syyyn);
\coordinate (spppyo) at (\sxxxy, \syyyo);
\coordinate (spppyp) at (\sxxxy, \syyyp);
\coordinate (spppyq) at (\sxxxy, \syyyq);
\coordinate (spppyr) at (\sxxxy, \syyyr);
\coordinate (spppys) at (\sxxxy, \syyys);
\coordinate (spppyt) at (\sxxxy, \syyyt);
\coordinate (spppyu) at (\sxxxy, \syyyu);
\coordinate (spppyv) at (\sxxxy, \syyyv);
\coordinate (spppyw) at (\sxxxy, \syyyw);
\coordinate (spppyx) at (\sxxxy, \syyyx);
\coordinate (spppyy) at (\sxxxy, \syyyy);
\coordinate (spppyz) at (\sxxxy, \syyyz);
\coordinate (spppza) at (\sxxxz, \syyya);
\coordinate (spppzb) at (\sxxxz, \syyyb);
\coordinate (spppzc) at (\sxxxz, \syyyc);
\coordinate (spppzd) at (\sxxxz, \syyyd);
\coordinate (spppze) at (\sxxxz, \syyye);
\coordinate (spppzf) at (\sxxxz, \syyyf);
\coordinate (spppzg) at (\sxxxz, \syyyg);
\coordinate (spppzh) at (\sxxxz, \syyyh);
\coordinate (spppzi) at (\sxxxz, \syyyi);
\coordinate (spppzj) at (\sxxxz, \syyyj);
\coordinate (spppzk) at (\sxxxz, \syyyk);
\coordinate (spppzl) at (\sxxxz, \syyyl);
\coordinate (spppzm) at (\sxxxz, \syyym);
\coordinate (spppzn) at (\sxxxz, \syyyn);
\coordinate (spppzo) at (\sxxxz, \syyyo);
\coordinate (spppzp) at (\sxxxz, \syyyp);
\coordinate (spppzq) at (\sxxxz, \syyyq);
\coordinate (spppzr) at (\sxxxz, \syyyr);
\coordinate (spppzs) at (\sxxxz, \syyys);
\coordinate (spppzt) at (\sxxxz, \syyyt);
\coordinate (spppzu) at (\sxxxz, \syyyu);
\coordinate (spppzv) at (\sxxxz, \syyyv);
\coordinate (spppzw) at (\sxxxz, \syyyw);
\coordinate (spppzx) at (\sxxxz, \syyyx);
\coordinate (spppzy) at (\sxxxz, \syyyy);
\coordinate (spppzz) at (\sxxxz, \syyyz);

%\gangprintcoordinateat{(0,0)}{The last coordinate values: }{($(spppzz)$)}; 

%\coordinatebackground{s}{f}{g}{q};
\draw (spppni) node [op amp] (opamp) {};
\getxyingivenunit{cm}{(opamp.+)}{\opampzx}{\opampzy};
\getxyingivenunit{cm}{(opamp.-)}{\opampfx}{\opampfy};


\fill  (\sxxxg, \secondopampzy) circle (2pt);
\draw (\sxxxg, \secondopampzy) -- (\sxxxg,\opampzy) node [left] {$U_{we}$} to [R, l=$R_d$, -*]  (\sxxxj,\opampzy) 
to [R, l=$R_d$, *-*] (opamp.+)
to [C, l_=$C_{d2}$, *-] (\opampzx, \syyyf) -- (\opampzx, \gyyyi);
\fill (\opampzx, \gyyyi) circle (2pt);


\draw (opamp.out) |- (\sxxxl,\syyyk) to [C, l_=$C_{d1}$, *-] (\sxxxj,\syyyk) to [short](\sxxxj,\opampzy);

\draw (opamp.-) -|  (\sxxxl,\syyyk);

\getxyingivenunit{cm}{(opamp.out)}{\opampoutx}{\opampouty};
\draw (opamp.out) to [short, *-*] (\bigswQCx,\opampouty);

\end{circuitikz}








\end{document}

